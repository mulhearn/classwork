
\documentclass[12pt]{article}
\usepackage[margin=1in]{geometry}
\usepackage[pdftex]{graphicx}
\usepackage{multirow}
\usepackage{setspace}
\usepackage{enumitem}
\pagestyle{plain}
\setlength\parindent{0pt}

\begin{document}

% Course information
\begin{tabular*}{\textwidth}{l @{\extracolsep{\fill}} r}
  & \multirow{3}{*}{\includegraphics[height=1.0in]{logo.jpg}} \\
  \large Experimental Techniques & \\
  \large Winter Quarter 2019 & \\
  \large Physics 80 & \\
\end{tabular*}
\vspace{10mm}

% Professor information
\begin{tabular}{ l l }
  \multirow{6}{*}{\includegraphics[height=1.25in]{mike.jpg}} & \\
  & \\
  & \large Michael Mulhearn \\
  & \large mulhearn@physics.ucdavis.edu \\
  & \large Physics 317 \\
  & \\
\end{tabular}
\vskip 0.5cm
\noindent
\textbf {Lectures:} M,W 11:00-11:50 PM in Rm. 285 Physics
\begin{tabbing}
\hspace*{3em}\= \hspace*{5em} \= \kill % set the tabbings
\textbf {Lab:}    \> Section 1: \>  M 12:10-2:40 PM in Rm. 152 Roessler \\
%                        \> Section 2: \> W 3:10-6:00 PM in Rm. 152 Roessler \\
\end{tabbing}

\noindent
\textbf {Text:} \emph{The Art of Electronics}, 3\textsuperscript{rd} Edition, Horowitz and Hill\\
\noindent
\textbf{Office Hours:} W 2:00-3:00 PM in 152 Roessler, and also often available during lab.\\
\noindent
\textbf{Lab Instructor:} Christopher Brainerd, cbbrainerd@ucdavis.edu \\
\noindent
\textbf{Quizzes:}  There were be occassional low-stakes single-problem quizzes during lecture.
\textbf{Final Exam:} Wed, March 20 at 1:00 pm in 285 Physics \\
\textbf{Homework:}  There will be approximately X homework assignments.\\

\noindent
\textbf {Course Description:}  This course is an introduction to experimental laboratory techniques and data analysis.  Laboratory techniques include electronics circuits and optical systems and related test equipment.  Data analysis based on scientific python includes statistical and systematic analysis, curve fitting, and noise.\\

\noindent
\textbf {Lab Safety:} 
You should complete the online course for Electrical Safety at \\
{\tt http://safetyservices.ucdavis.edu/training/electrical-safety}.\\

\noindent
\textbf {Lab Reports:} 
Most scientist employ a mixture of handwritten and digital logbooks.  Quick notes and sketches about procedures, calculations, and the results of simple measurements are often most conveniently handwritten.  But data collection and detailed analysis are done entirely on a computer.

You'll preform pre-lab calculations, take notes of your procedure, and record simple measurements in a handwritten logbook, which will remain in the lab to be graded periodically.  Your more extensive analysis and final plots will be submitted online.

\vskip 0.5cm
\noindent
\textbf {Tentative Course Outline}:

This is the first time this course has been offered, so the topics and schedule may be adjusted while the course is in progress.

\begin{table}[h!]
\normalsize % The size of the table text can be changed depending on content. Remove if desired.
\begin{tabular}{ lllll }
\hline
\textbf{Week} & \textbf{Dates} & \textbf{Lecture} & \textbf{Lab} \\
\hline
1 & 7 Jan & Scientific Python & Plotting \\
   & 9 Jan & RLC Circuits & Plotting (catch up)\\
\hline
2 & 14 Jan & & DC Circuits \\
  & 16 Jan & & Thevenin Equivalent Circuits \\
\hline
3 & 23 Jan & & Time Varying Signals \\
\hline
4 & 28 Jan & Distributions & RC amd RL Transient Signals \\
   & 30 Jan & &  Passive Filters and Resonance \\
\hline
5 & 4 Feb & & Histograms and Distributions\\
   & 6 Feb & & Geiger Counter\\
\hline
6 & 11 Feb & The Diode & The Diode \\
   & 13 Feb & Uncertainties & Plank's Constant \\
\hline
7 & 20 Feb & & The Central Limit Theorem \\
\hline
8 & 25 Feb & Analysis & Error Propagation \\
   & 27 Feb & & Monte Carlo and Fitting \\
\hline
9 & 4 Mar & & Speed of Light \\
   & 6 Mar & & Speed of Light  (catch up) \\
\hline
10 & 11 Mar & & Muon Lifetime\\
   & 13 Mar & & \\
\hline
\end{tabular} 
\end{table}

\end{document}

