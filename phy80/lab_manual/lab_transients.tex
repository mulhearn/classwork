\chapter{RC and RL Transient Signals}

\section{Pre-lab Calculation}

1) Show that for an exponential decay with time constant $\tau$, the rise-time, when defined as the time interval between $10\%$ and $90\%$ values, is given by:
\begin{displaymath}
t_{90} = {\rm ln}(9) \; \tau \sim 2.2 \tau
\end{displaymath}

2) Calculate the inductance of a solenoid with N=20 turns, length $\ell=4~\rm cm$, a radius of $1~\rm cm^2$ using the formula:
\begin{displaymath}
L = \frac{\mu N^2 A}{\ell}
\end{displaymath}
where $A$ is the cross-sectional area and $\mu = 1.257 \times 10^{-6}$.


\begin{figure}[htbp]
\begin{center}
\begin{tabular}{cc}
\begin{circuitikz}[line width=1pt]
\draw (0,0) to[sinusoidal voltage source,bipoles/length=1.5cm] ++(0,4.0) to[short] ++(2.0,0)
to[R,-*] ++(0,-2.0) coordinate(X) to[short,*-o] ++(1.0,0) node[right]{A};
\draw (X) to[C,-*] ++(0,-2.0) coordinate(X) to[short,-o] ++(1.0,0) node[right]{B};
\draw (X) to[short,-*] ++(-2.0,0) node[ground,yscale=2.0]{};
\end{circuitikz}  &
\begin{circuitikz}[line width=1pt]
\draw (0,0) to[sinusoidal voltage source,bipoles/length=1.5cm] ++(0,4.0) to[short] ++(2.0,0)
to[R,-*] ++(0,-2.0) coordinate(X) to[short,*-o] ++(1.0,0) node[right]{A};
\draw (X) to[L,-*] ++(0,-2.0) coordinate(X) to[short,-o] ++(1.0,0) node[right]{B};
\draw (X) to[short,-*] ++(-2.0,0) node[ground,yscale=2.0]{};
\end{circuitikz}  \\
(a) & (b) \\
\end{tabular}

\caption{A function generator driving an RC circuit.}
\label{fig:rlc-circuits}
\end{center}
\end{figure}

\section{Transient response of an RC circuit}

Build the circuit in Fig.~\ref{fig:rlc-circuits}a using a precision $R=10~\rm k\Omega$ resistor and an $1~\rm nF$ capacitor.
Window appropriately.
Measure time dependence, rise time according to procedure, and using automated measure function.
Determine time constant.
(Plot exponential.)

% measured lifetime around 22 microseconds.

\section{Transient response of an RL circuit}


Wrap an inductor around the provided dowel, and estimate it's inductance by modifying your pre-lab calculation accordingly.

Turn down the supply to $2.5~\rm V$ peak-to-peak.  Build the circuit in Fig.~\ref{fig:rlc-circuits}b using your homemade inductor and a resistor of $R=47~\rm Ohms.$

Determine the inductance of your coil and compare to your theoretical estimate.


\section{Analysis}

Plot your collected data and compare with an exponential function using the measured life-time.

% measure lifetime values value around 120 ns

