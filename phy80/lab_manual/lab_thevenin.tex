
\chapter{Equivalent Circuits}

\section{Pre-lab Calculation}

\noindent

1) Determine an equation for the Thevenin equivalent voltage $V_{\rm
  th}$ and resistance $R_{\rm th}$ from the values $V_1, V_2, R_1,
R_2, R_3$ for the circuit shown in Fig.~\ref{fig:thevenin}.  Hint: Use
the superposition principle.  Find the equivalent resistance by
setting the voltage $V_1$ and $V_2$ to zero, i.e. shorting them in the
circuit.  Then calculate two contributions to the Thevenin voltage,
one with $V_1$ set to zero and one with $V_2$ set to zero.  The actual
Thevenin voltage is the sum of these two contributions.  Play close
attention to the polarity of $V_2$ as drawn, i.e. that a positive
value of $V_2$ tends to make the voltage $V_{\rm a b}$ negative.

2) Compute $V_{\rm th}$, $R_{\rm th}$, and the short-circuit current
$I_{\rm sc}$ for the particular values of $R_1$,$R_2$,$R_3$,$V_1$, and
$V_2$ you will be using in the lab.

\begin{figure}[htbp]
\begin{center}
\begin{tabular}{c@{\hskip 2cm}c}
\begin{circuitikz}[line width=1pt]
\draw (0,0) to[voltage source,bipoles/length=1.5cm,l=$V_1$] ++(0,2.0) coordinate(X) to[R,l=$R_3$,-*] ++(2.0,0);
\draw (X) to[R,l=$R_1$] ++(0,2.0) to[short,-*] ++ (2.0,0) coordinate(X) to[short,-o] ++ (1.0,0) node[right]{B};
\draw (X) to[voltage source,bipoles/length=1.5cm, l=$V_2$] ++(0,-2.0) to[R,l=$R_2$] ++(0,-2.0) coordinate(X)
to[short,*-o] ++ (1.0,0) node[right]{A};
\draw(X) to[short] ++(-2.0,0);
\end{circuitikz} &
\begin{circuitikz}[line width=1pt]
\draw (0,0) coordinate(X) to[voltage source,bipoles/length=1.5cm,l=$V_{\rm th}$] ++(0,2.0) to[R,l=$R_{\rm th}$,-*] ++(0,2.0)
to[short,-o] ++ (3.0,0) node[right]{B};
\draw(X) to[short,-o] ++ (3.0,0) node[right]{A};
\end{circuitikz} \\
(a) & (b) \\
\end{tabular}
\caption{The circuit (a) you will be building in lab and it's (b) Thevenin Equilvalent.}
\label{fig:thevenin}
\end{center}
\end{figure}

\section{Thevenin Equivalent Circuit}

Build the circuit in Fig.~\ref{fig:thevenin} using $R_1 = 3.3~\rm
k\Omega$, $R_2 = 3.9~\rm k\Omega$, and $R_3 = 4.7~\rm k\Omega$.
Supply $V_1 = 10~\rm V$ and $V_2 = 5~\rm V$ using your two channel
bench-top power supply.  In the diagram, the supplies are not
referenced to ground or each other, so make certain that your supply
is set to provide independent outputs and do not add any jumpers to
ground.  Take careful note of the polarity of the supplies, so
e.g. the negative (black) output of $V_1$ is connected to point (b)
whereas the negative (black) output of $V_2$ is connected to point
(a).

Use your Triplett 9007 as a voltmeter and the Mastech MS8624 as a
current meter.  First measure the open circuit voltage $V_{\rm ab}$.
Next short the points (a) and (b) through your current meter.  These
values should closely match the Thevenin voltage and short-circuit
current which you have already calculated.  If not, you should check
your work and find the discrepancy before proceeding.

Next you will measure the voltage across and current through a load
resistor connected between the terminals at (a) and (b) to
experimentally determine the IV curve for your circuit.  Recall from
the previous lab that you measure the current by connecting your meter
in series and the voltage by connecting your meter in parallel.  As
before, use your Triplett 9007 as a voltmeter and the Mastech MS8624
as a current meter.

Make simultaneous current and voltage measurements for three different
values of the load resistance $R = 470~{\rm \Omega}, 1.2~\rm k\Omega,
4.7~\rm k\Omega.$


\section{Analysis}

{\bf Plot 1:} To present your analysis you should produce a part like
that of Fig.~\ref{fig:egthev}.  Your plot should show the Thevenin
equivalent source IV curve for the circuit you built in lab.  You
should also draw theoretical load IV curves for the three resistor
values you used to make current and voltage.  Finally, you include
data points for the five measurements you made.

\begin{figure}[htbp]
\begin{center}
\includegraphics[width=0.75\textwidth]{figs/labs/thevenin/final.pdf} 
\caption{Using boolean masks to cut on variable $y$.}
\label{fig:egthev}
\end{center}
\end{figure}


\section{$\Delta-Y$ Transformation}

Consider the two different networks shown in Fig.~\ref{fig:deltay}.
If there are no external connections to the central node in the
left-hand circuit, the two networks are equivalent if:
\begin{displaymath}
R_{A} = \frac{R_{AC} R_{AB}}{R_{AB} + R_{AC} + R_{BC}}
\end{displaymath}
as well as two similar equations for $R_{B}$ and $R_{C}$.  Going in the other direction we have:
\begin{displaymath}
R_{AB} = \frac{R_{A}R_{B} + R_{A}R_{C} + R_{B}R_{C}}{R_{C}}.
\end{displaymath}
These transformations are more general than the series and parallel
laws, which you can derive by considering the case that $R_{BC}=0$ for
parallel resistors, and $R_{C} \to \infty$ for series resistors.  They
allow one to simplify more complicated networks for which the series and
parallel equivalence relations are insufficient.

In the special case that $R_{A} = R_{B} = R_{C} = R$ it follows that 
\begin{displaymath}
R_{AB} = R_{AC} = R_{BC} = 3 R.
\end{displaymath}

If time permits, use your soldering iron to construct the left-hand
network using $R_{A} = R_{B} = R_{C} = 1~\rm k\Omega$.  Then construct
the equivalent right-hand network using $R_{AB} = R_{AC} = R_{BC} =
3.0~\rm k\Omega$.  Since $3~\rm k\Omega$ is not a standard sized
$10\%$ resistor, you can construct one by using a $33~\rm k\Omega$ in
parallel with a $3.3~\rm k\Omega$ resistor.

Make sure the soldering iron is on, and the sponge is moist.  Twist
the leads of the resistor together to make initial connections, then
hold the arrangement securely in the clamp.  Wipe the tip of the hot
iron on the sponge to clean it, then apply a small amount of solder to
the tip by touching the hot iron to the solder wire.

Heat the connection by holding the soldering iron against it, then
bring the solder wire in contact with the heated connection (not the
soldering iron) You want the iron to heat the connection, and then the
connection to melt and draw in the solder.  The little bit of solder
on the tip is only there to ensure good thermal conduct between the
tip and the connection: don't ``paint'' the solder onto the
connection.

{\bf Measurement 1:}  Check the resistance between pairs of terminals on your creations, and
compare with your expectation.  You can bring your creations home if
you like.

\begin{figure}[htbp]
\begin{center}
\begin{tabular}{c@{\hskip 2cm}c}
\begin{circuitikz}[line width=1pt]
\draw (0,0) coordinate(A);
\draw (A) to[R,l_=$R_A$,*-*] ++(0,2.0) node[above]{A};
\draw (A) to[R,*-*] ++(-1.73,-1) node[left]{B};
\draw (A) to[R,*-*] ++(1.73,-1) node[right]{C};
\draw (-0.5,0) node[left]{$R_B$};
\draw (1.25,0) node[left]{$R_C$};

\end{circuitikz} &
\begin{circuitikz}[line width=1pt]
\draw (0,2) to [R] (1.73,-1) node[right]{C} to [R,*-,l_=$R_{BC}$] (-1.73,-1) node[left]{B} to [R,*-*] (0,2) node[above]{A};
\draw (-0.75,1) node[left]{$R_{AB}$};
\draw (0.8,1) node[right]{$R_{AC}$};
\end{circuitikz} \\
(a) & (b) \\
\end{tabular}
\caption{Equivalent three-node circuits.}
\label{fig:deltay}
\end{center}
\end{figure}

