\documentclass[12pt]{article}


\usepackage[dvips,letterpaper,margin=0.75in,bottom=0.75in]{geometry}
\usepackage{cite}
\usepackage{slashed}
\usepackage{graphicx}
\usepackage{amsmath}

\usepackage[american,fulldiode]{circuitikz}
\usetikzlibrary{calc}

\begin{document}
\ctikzset{bipoles/thickness=1}
\ctikzset{bipoles/length=.6cm}

\title{AM Radio}

\maketitle

\section{Introduction}

In this lab, you will build an AM radio receiver and audio power amplifier.

\section{RF Amplifier}

\begin{figure}[htbp]
\begin{center}
\begin{circuitikz}[american,line width=1pt]
\draw
(0,0) node[npn](npn1){}
++(0,2) node[npn](npn2){} 
(npn2.E) to[short] (npn1.C)
(npn2.C) to[short,*-o] ++(0.5,0) node[right]{$v_{\rm out}$}
(npn2.C) to[R,l=$R_5$,-*] ++(0,1.5) coordinate(X) to[short,*-o] ++(0,0.5) node[right]{$V_{\rm CC}$}
(X) to[short] ++(-3.0,0) to[R,l_=$R_1$] ++(0,-1.5) coordinate(X)
let \p1 = (npn2.B), \p2=(X) in coordinate(Y) at (\x2,\y1)
(X) to[short] (Y) coordinate(X)
(npn2.B) to[short,-*] (X) to[C,l=$C_1$] ++(-1.5,0) node[ground,yscale=2.0]{}
let \p1 = (npn1.B), \p2=(X) in coordinate(Y) at (\x2,\y1)
(X) to[R,l=$R_2$] (Y)  coordinate(X)
;
\path (X) ++ (1.5,0) coordinate(A);
\draw
(A) to[short] (npn1.B)
(A) to[short] (X) to[C,l=$C_2$,-o] ++(-1.5,0) node[left]{$v_{\rm in}$}
(A) to[R,l=$R_4$,*-*] ++(0,-1.5) coordinate(B)
let \p1 = (npn1.E), \p2=(B) in coordinate(C) at (\x1,\y2)
(B) to[C,l_=$C_3$,-*] (C) to[short] (npn1.E)
let \p1 = (B), \p2=(X) in coordinate(Y) at (\x2,\y1)
(B) to[short,-*] (Y)
(X) to[short] (Y) to[R,l_=$R_3$] ++(0,-1.5) coordinate(X)
(C) to[R,l=$R_6$] ++(0,-1.5) coordinate(D) to[R,l=$R_7$,-*] ++(0,-1.5) coordinate(E) node[ground,yscale=2.0]{} 
(X) |- (E)
(D) to[short,*-] ++(-1.5,0) to[C,l_=$C_4$,-*] ++(0,-1.5)
;
\end{circuitikz} 
\caption{Cascode amplifier with bootstrap.}
\label{fig:beta}
\end{center}
\end{figure}

$R_4$ is boot-strapped so $R_4+R_3||R_2$ does not decrease input impedance for the HF signal.  But still hard to get high gain without lower transistor being high beta:  $R_6$ needs to be about 
$4.6~\rm k\Omega$ but then $R_5 \sim 46~\rm k\Omega$ limits $I_C^Q \sim 0.1~\rm mA$, which is on low end for the 2N3904.

Maybe use a Darlington pair?  With high beta, probably a single emitter resistor suffices with no capacitor.

\section{RF Amplifier 2}

\begin{figure}[htbp]
\begin{center}
\begin{circuitikz}[american,line width=1pt]
\draw
(0,0) node[npn](npn1){}
++(0,2) node[npn](npn2){} 
(npn2.E) to[short] (npn1.C)
(npn2.C) to[short,*-o] ++(0.5,0) node[right]{$v_{\rm out}$}
(npn2.C) to[R,l=$R_5$,-*] ++(0,1.5) coordinate(X) to[short,*-o] ++(0,0.5) node[right]{$V_{\rm CC}$}
(X) to[short] ++(-3.0,0) to[R,l_=$R_1$] ++(0,-1.5) coordinate(X)
let \p1 = (npn2.B), \p2=(X) in coordinate(Y) at (\x2,\y1)
(X) to[short] (Y) coordinate(X)
(npn2.B) to[short,-*] (X) to[C,l=$C_1$] ++(-1.5,0) node[ground,yscale=2.0]{}
let \p1 = (npn1.B), \p2=(X) in coordinate(Y) at (\x2,\y1)
(X) to[R,l=$R_2$] (Y)  coordinate(X)
;
\path (X) ++ (1.5,0) coordinate(A);
\draw
(A) to[short] (npn1.B)
(A) to[short] (X) to[C,l=$C_2$,-o] ++(-1.5,0) node[left]{$v_{\rm in}$}
(A) to[R,l=$R_4$,*-*] ++(0,-1.5) coordinate(B)
let \p1 = (npn1.E), \p2=(B) in coordinate(C) at (\x1,\y2)
(B) to[C,l_=$C_3$,-*] (C) to[short] (npn1.E)
let \p1 = (B), \p2=(X) in coordinate(Y) at (\x2,\y1)
(B) to[short,-*] (Y)
(X) to[short] (Y) to[R,l_=$R_3$] ++(0,-1.5) coordinate(X)
(C) to[R,l=$R_6$] ++(0,-1.5) coordinate(D) to[R,l=$R_7$,-*] ++(0,-1.5) coordinate(E) node[ground,yscale=2.0]{} 
(X) |- (E)
(D) to[short,*-] ++(-1.5,0) to[C,l_=$C_4$,-*] ++(0,-1.5)
;
\end{circuitikz} 
\caption{Cascode amplifier with bootstrap.}
\label{fig:beta}
\end{center}
\end{figure}

\newpage

\section{RF Amplifier 2}

\begin{figure}[htbp]
\begin{center}
\begin{circuitikz}[american,line width=1pt]
\draw
(0,0) node[npn](npn){}
(2,0) node[pnp,xscale=-1](pnp){} 
(-2,0.75) node[njfet](fet){} 
(npn.C) to[short] ++(1,0) coordinate(X) to[short] (pnp.E)
(X) to[short,*-] ++(0,0.75) to[R,l=$R_4$] ++(0,0.75) to[short,-*] ++(0,0.5) coordinate(A) 
let \p1 = (pnp.B), \p2=(A) in coordinate(X) at (\x1,\y2)
(A) to[short] (X) to[short,-o] ++(0.5,0) node[right]{$V_{\rm CC}$}
(X) to[short,*-] ++(0,-0.5) to[R,l_=$R_{8}$] ++(0,-0.75) to[short,-*] (pnp.B)
(npn.E) to[R,l=$R_{5}$] ++(0,-1.25) coordinate(X) to[R,l=$R_{6}$] ++(0,-1.25) coordinate(B)
(X) to[short,*-] ++(-0.75,0) to[C,l_=$C_2$,-*] ++(0,-1.25)
(pnp.C) to[R,l_=$R_{7}$] ++(0,-2.5) coordinate(X)
(X) to[short,*-*] (B)
let \p1 = (pnp.B), \p2=(X) in coordinate(Y) at (\x1,\y2)
(X) to[short,-*] (Y) to[short,-o] ++(0.5,0) node[right]{$V_{\rm EE}$}
(Y) to[R,l_=$R_{9}$] ++(0,2.5) to[short] (pnp.B)
;
\path (pnp.B) ++ (0,0.5) coordinate(X);
\draw
(X) to[C,l_=$C_3$,*-] ++(1.5,0) node[ground,yscale=2.0]{}
(pnp.C) to[short,*-o] ++(3,0) node[right]{$v_{\rm out}$}
% tie in the JFET at (A) and (B)
(fet.D) to[R,l=$R_2$] ++(0,0.75) to[short] ++(0,0.5) coordinate(X)
(A) to[short] (X) coordinate (A)
(fet.S) to[short] ++(0,-0.75) to[R,l=$R_3$] ++(0,-1.25) to[short] ++(0,-1.25) coordinate(X)
(fet.S) to[short,*-] (npn.B)
(B) to[short] (X) coordinate(B)
let \p1 = (fet.G), \p2=(npn.E) in coordinate(X) at (\x1,\y2)
(fet.G) to[short] (X) to[R,l_=$R_1$] ++(0,-1.25) to[short] ++(0,-1.25) to[short,-*] (B)
(fet.G)  to[C,l=$C_1$,*-o] ++(-1.25,0) node[left]{$v_{\rm in}$}
;
\end{circuitikz} 
\caption{Folded cascode amplifier with FET input buffer.}
\label{fig:beta}
\end{center}
\end{figure}


\section{Lab Report}

Your report should include all of your measurements and a comparison with your calculation.
 
\end{document}
