\documentclass[12pt]{article}
\usepackage{lmodern}
\usepackage[T1]{fontenc}
\usepackage[dvips,letterpaper,margin=0.75in,bottom=0.75in]{geometry}
\usepackage{cancel}
\usepackage{graphicx}
\usepackage{braket}
\usepackage{latexsym,amssymb,amsmath}
\usepackage{pdfpages}
\usepackage{xcolor}
\usepackage{capt-of}
\usepackage{amsmath}
\usepackage{cite}

\usepackage[american,fulldiode]{circuitikz}
\tikzset{component/.style={draw,thick,circle,fill=white,minimum size =0.75cm,inner sep=0pt}}

\begin{document}
\ctikzset{bipoles/thickness=1}
\ctikzset{bipoles/length=.6cm}

\title{Proposal for Revising the \\ Undergraduate Physics Curriculum \\ Version 1.0}
\author{Department of Physics}

\maketitle


\section{Charge}

Our initial charge from Mark Foncannon and FEC Chair Pablo Ortiz (18 May 2022) was as follows:
\begin{verbatim}

On behalf of the Undergraduate Council (UGC) and Chair Katheryn Russ,
the committee reviewed the proposed revisions to the Applied Physics
and Physics major at their recent meeting on April 21. The committee
respectfully requests that program respond to the following items:

1.  UGC requests that the program include and specify the target
 timing of the proposed changes and how many students will be impacted
 per the UGC Policy for Establishment or Revision of Academic Degree
 Programs (see page 5). Please also include the department faculty
 vote on proposed revisions.

2.  Include consultation with Mathematics regarding the addition of
 MAT 27A, MAT 27B and MAT 67.

3.  Include consultation with Atmospheric Science, Chemistry, Computer
 Science and Engineering with addition of these courses.

a.  GEL 50/50L/55/150B

b.  ATM 60/105/115/116/124/158

c.  CHE 105/115/124L/129A/129B/129C

d.  ECS 132/171

e.  EEC 150A/150B

f.  ENG 045

g.  EMS 160/162/162L/164/170/170L/172/172L/174

4.  Since Physics courses are important inputs for students in many
 majors, UGC wondered if there is a plan for faculty to review syllabi
 of existing courses that may have been updated incrementally over
 time perhaps outside the formalized process of the dual major
 revamping, to assure that their syllabi all still are in accordance
 with the course descriptions in ICMS. Where syllabi have diverged
 from course descriptions in ICMS, will their descriptions be updated?
  Any plan to encourage that future updates to course content be
 systematically reported within the programs so that updates can be
 made to ICMS if needed would be welcome.
\end{verbatim}

From here:
\begin{verbatim}
https://academicaffairs.ucdavis.edu/deans-directors-and-department-chairs-list
\end{verbatim}
I have taken the following lists of department chairs to contact and request a vote:

\begin{itemize}
\item Mathematics, Chair Anne Schilling (MAT) 
\item Earth and Planetary Sciences, Chair Kari Cooper (GEL) 
\item Land, Air \& Water Resources, Chair William Horwath (ATM)
\item Chemistry, Chair David Goodin (CHE)
\item Computer Science, Chair Dipak Ghosal (ECS)
\item Electrical \& Computer Engineering, Chair Andre Knoesen (EEC)
\item Materials Science \& Engineering, Chair Yayoi Takamura (EMS, ENG 45)
\end{itemize}

\newpage

\section{Boilerplate Email for Department Chairs:}

\begin{verbatim}
 Dear Chair XXX,

 We are required by the Undergraduate Council to consult your
 department regarding proposed changes to our physics undergraduate
 major requirements.  Despite the fact that we expect the impact on
 your department to be quite minimal, we will explain it here fully,
 and we hereby request (A) a vote by your department on whether or not
 to support our changes to the physics major requirements, and (2) a
 letter of support from your department that includes the outcome of
 that vote.  You can find the UCD requirements for program revisions,
 including this required consultation, here:
 https://ucdavis.box.com/s/v1682jq4dp9sgcrhyr3oog1wp3pa36fd

 Our proposal has several aims, the most important of which I consider
 to be: (1) improving the experience for students that transfer to UCD
 from a community college in their junior year, (2) increasing the
 role of computational physics in our program, (3) improving our
 laboratory offerings, and (4) increasing flexibility for students in
 schedule and electives.  You can find the most recent version of our
 complete proposal here:
 
 https://github.com/mulhearn/classwork/blob/master/curriculum/curriculum.pdf

 However, the full proposal is quite long.  Therefore, I have included with this
 e-mail a much shorter summary of the courses taught in your department which
 are impacted by this proposal, and our best estimate of the size of that
 impact.

 The approval process for physics has proven long and windy, with many
 unexpected delays.  I would be happy to share my insights with you
 some day if your department decides to update your own program!  As
 we enter year five of this process for me personally, I am pushing
 hard to return the proposal to undergradudate council, including your
 department's letter, as early in the spring quarter as
 possible. These required consultations are the last item that we
 need, and I would be very much obliged for your department's rapid
 response to help our department finish our proposal this year.

 Sincerely,
 Michael Mulhearn
 Professor and Vice Chair for Undergraduate Program and Administration
 Department of Physics
\end{verbatim}


\newpage

\section{Math Department}

\subsection{Changes to MAT courses in all Physics and Applied Physics majors}

All of our physics majors (BS in Physics, BS in Applied Physics, and
AB in Physics) share a common set of math requirements that cover
calculus, linear algebra, and differential equations.

As part of the upgrade of our undergraduate physics curriculum, we
have updated the math requirements for all physics majors to include
additional options now offered by the math department, specifically
linear algebra and differential equations with applications to biology
(MAT 27A and MAT 27B) and Modern Linear Algebra (MAT 67).  The updated
requirements are now:
\captionof{table}{\label{tbl:prep}Preparatory Subject Matter (MAT)}
\vskip 0.25cm
\noindent
\begin{tabular}{|llllll|}
\hline
Course & & Units & Offered & Prereqs & Name \\
\hline
MAT & 21A & 4 & FWS & & Differential Calculus\\ 
    & 21B & 4 & FWS &  & Integral Calculus \\ 
    & 21C & 4 & FWS &  & Partial Derivatives and Series\\ 
    & 21D & 4 & FWS &  & Vector Analysis\\
\hline
    & one of:  & & & & \\
MAT & 22A & 3 & FWS &  & Linear Algebra\\
    & 27A & 3 & FWS &  & Linear Algebra\\
    & 67  & 4 & FWS &  & Mod. Linear Algebra\\
\hline
MAT & 22B & 3 & FWS &  & Differential Equations\\
    & or  & & & & \\
MAT & 27B & 3 & FWS &  & Differential Equations\\ 
\hline
\end{tabular}\\

\subsection{Expected impact of these changes on the Math department}

In principle, these changes impact all of our 250 physics majors, with
an average annual throughput in any math class of approximately 50.
However, the actual impact will be much smaller.  We have already been
accepting these courses as equivalent on a case by case basis whenever
requested by a student.  These changes simply advertise this
equivalence.  We expect that this change will have very little impact:
even if a surprising large $10\%$ of physics majors decide to use an
alternative class, the impact to the math department would be about
five students per year per course.

We also understand this update to be fully consistent with the math
departments intentions in offering these alternative courses.  Please
note that {\bf the inclusion of MAT 27A and 27B was specifically
requested by the math department}, as shown in the enclosed letter.


\newpage
\subsection{Update to math prerequisites of physics courses}

As part of our curriculum overhaul, we have also updated all of the
math prerequisites to be consistent with the equivalencies used by the
math department, as shown here:
\captionof{table}{
  Math prerequisite for lower division courses related to physics
  courses.  Where indicated, course grades are the minimum accepted to
  meet prerequisites.  {\bf The prerequisites for the math courses are set
  by the math department and are only reported here}, including some
  simplifications.  Where possible, math prerequisites for physics
  courses will adopt the same choices made by the math department with
  respect to equivalent math coursework.}
\label{tbl:math-prereqs}
\begin{center}
\begin{tabular}{|llll|}
\hline
Course & & Math Prereqs (Minimum Grade)& Name \\
\hline
MAT & 16A & & Short Calc.\\
    & 16B & 21A/21AH/17A/16A(C-) & Short Calc.\\
    & 16C & 21B/21BH/17B/16B(C-) & Short Calc.\\
    & 17A &                      & Calc (Bio\&Med)\\
    & 17B & 21A/21AH/17A/16A(C-) & Calc (Bio\&Med)\\
    & 17C & 17B(C-)              & Calc (Bio\&Med)\\
    & 21A &                      & Calculus \\
    & 21B & 21A/21AH(C-) or 17A(B) & Calculus \\
    & 21C & 21B/21BH/17C/16C(C-) or 17B(B) & Calculus\\
    & 21D & 21C/21CH(C-) or 17C(B)  & Calculus \\
    & 22A & 21C/21CH/17C/16C(C-)    & Linear Algebra \\
    & 22B & 67/22A(C-)              & Diff. Eqs. \\
    & 27A & 21C/21CH/17C(C-)        & Linear Algebra w/ App. Bio. \\
    & 27B & 27A/22A(C-)             & Diff. Eqs. w/ App. Bio.\\
    & 67  & 21C/21CH(C-)            & Mod. Lin. Alg. \\
\hline
\hline
PHY & 9A & 21B/21BH/17C/16C(C-) or 17B(B) & Class. Physics\\
    & 9B & 21C/21CH(C-) or 17C(B) & Class. Physics\\
    & 9C & 21D(C-) & Class. Physics\\
    & 9D & 22A(C-) or 27A(C-) or 67(C-) & Class. Physics\\
    &    & $\parallel$22B or $\parallel$27B & \\
\hline
\hline
PHY & 9HA & $\parallel$21B/$\parallel$21BH & Hon. Physics\\
    & 9HB    & 21B/21BH(C-) & Hon. Physics \\
    & 9HC    & 21C/21CH & Hon. Physics \\
    & 9HD    & 21D  & Hon. Physics \\
\hline
PHY & 45     & $\parallel$22B or $\parallel$27B & Computational Physics \\
\hline
\end{tabular}
\end{center}
As physics majors are required to take the MAT 21 sequence, these
changes only impact non-physics majors taking the PHY 9 sequence.
Therefore, {\bf these changes are outside the scope of the vote on the
  proposed changes to physics majors}.  They have been included here
only for completeness, and we certainly welcome any comments on the
appropriateness of these changes.

\newpage
\section{Earth and Planetary Sciences (GEL)}
Chair Kari Cooper

\subsection{No changes to GEL courses for Physics AB and BS majors}

There are no changes to Geology coursework to the Physics majors (AB
and BS) including the Astrophysics specialization.  Astrophysics
majors will still be required to take GEL 163 in the new program.

\subsection{Changes to GEL courses for Applied Physics majors}

The proposed physics curriculum includes changes to the Applied
Physics major with specializations in Atmospheric Physics, Physical
Oceanography, and Geological Physics related to GEL coursework, as
detailed in Tables~\ref{tbl:gelatmospheric}-\ref{tbl:gelgeophysics}.
All three of these majors were designed before significant changes to
the prerequisite structure of GEL courses.

In the proposed Atmospheric Physics major, GEL 50 is required (was
only recommended).  In the current requirements, GEL 150A was a
required course, but due to the increased prerequisite requirements,
this is no longer possible.  Instead, GEL 116N, 150A, and 150B are
listed along with PHY and ATM coursework as possible electives in the
proposed requirements.

In the current requirements for the Physical Oceanography major, GEL
116N and 150A were required.  Due to prerequisites, this now
implicitly requires GEL 50 as well, which pushes this major based the
unit cap guidelines for required coursework.  In the proposed major,
GEL 50, 55, 116N, and 150A are all required courses.  Students may
also include 150A as one of two electives that include other choices
from PHY and ATM courses.

In the current requirements for the Geological Physics major, GEL 161
and 162 are required.  Due to prerequisites, this now implicitly
requires GEL 50 as well, which pushes this major based the unit cap
guidelines for required coursework.  In the proposed major, GEL 50 and
50L are all required courses.  Students can choose five electives from
GEL 116N, 146, 150A, 150B, 161, 162, 163, and other PHY and ATM
courses.  They are required to select at least two from GEL 161,162,
and 163.

To summarize, all three of these specializations have been changed to
properly account for prerequisite coursework for required courses and
to afford students additional choices for elective coursework in PHY,
ATM, and GEL.

\subsection{Expected impact of these changes on the Earth and Planetary Sciences department}

Blah blah blah...

\newpage

\captionof{table}{\label{tbl:gelatmospheric}Required GEL Coursework for Atmospheric Physics}
\vskip 0.25cm
\noindent
*: recommended, $^\#$: not offered every year, $\parallel$: concurrently.\\
\begin{tabular}{|llllll|}
\hline
Course & & Units & Offered & Prereqs & Name \\
\hline
GEL & 50     & 3 & FWS & & Physical Geology \\
\hline
\hline
    & Choose two of & & & & \\
\hline
PHY  & 105B   & 4 & & & \\
     & 105C   & 4 & $^\#$  &             & Continuum Mechanics\\
GEL  & 116N   & 3 & S  & GEL 50      & Oceanography\\
     & 150A   & 4 & S$^\#$ & GEL 116N,55 & Physical and Chemical Oceanography\\
     & 150B   & 3 & W  & GEL 50      & Geological Oceanography\\
ATM  & 124    & 3 & F  & ATM 60      & Meteorological Instruments \& Observations \\
     & 128    & 4 & W  & ATM 60      & Radiation and Satellite Meteorology \\
     & 158    & 4 & S  & ATM 121A    & Boundary-Layer Meteorology \\
\hline
\end{tabular}\\

\captionof{table}{\label{tbl:geloceanography} Required GEL coursework for Physical Oceanography}
\vskip 0.25cm
\noindent
*: recommended, $^\#$: not offered every year, $\parallel$: concurrently.\\
\begin{tabular}{|llllll|}
\hline
Course & & Units & Offered & Prereqs & Name \\
\hline
GEL  & 50     & 3 & FWS & & Physical Geology \\
     & 55     & 3 & F      & CHE 2A & Intro. to Geochemistry\\
\hline
\hline
Course & & Units & Offered & Prereqs & Name \\
\hline
GEL  & 116N   & 3 & S      & GEL 50          & Oceanography\\
     & 150A   & 4 & S$^\#$ & GEL 116N,55 & Physical and Chemical Oceanography\\
\hline
\hline
    & Choose two of & & & & \\
\hline
PHY  & 105B  & 4 & & & \\
     & 105C  & 4 & S$^\#$ & & Continuum Mechanics \\
ATM  & 115   & 3 & S   & ATM 60 & Hydroclimatology \\
     & 116   & 3 & F   & & Modern Climate Change \\
     & 120   & 4 & F   & ATM 60$\parallel$ & Atmos. Thermo. and Cloud Physics \\
GEL  & 150B  & 3 & W & GEL 50 & Geological Oceanography\\
\hline
\end{tabular}\\

\newpage

\captionof{table}{\label{tbl:gelgeophysics} Required GEL coursework for Geological Physics}
\vskip 0.25cm
\noindent
*: recommended, $^\#$: not offered every year, $\parallel$: concurrently.\\
\begin{tabular}{|llllll|}
\hline
Course & & Units & Offered & Prereqs & Name \\
\hline
GEL & 50     & 3 & FWS & & Physical Geology \\
    & 50L    & 2 &     & & Physical Geology Laboratory\\
\hline
\hline
    & Choose five of & & & & \\
\hline
PHY  & 105B   & 4 & &  & \\
     & 105C   & 4 & &  & Continuum Mechanics\\
ATM  & 120    & 4 & F  & ATM 60    & Atmos. Thermodynamics \& Cloud Phys. \\
     & 121A   & 4 & F  & ATM 120   & Atmospheric Dynamics \\
     & 121B   & 4 & W  & ATM 121A  & Atmospheric Dynamics \\
GEL  & 116N   & 3 &    & & Oceanography\\
     & 146    & 3 &    & & Radiogenic Isotope Geochem. \& Cosmochem. \\
     & 150A   & 4 & S$^\#$ & GEL 116N,55 & Physical and Chemical Oceanography\\
     & 150B   & 3 & W  & GEL 50      & Geological Oceanography\\
\hline
     & incl at least two:  & & & & \\
\hline
GEL  & 161    & 3 & S$^\#$ & GEL 50 & Geophysical Field Methods \\
     & 162    & 3 & W$^\#$ & GEL 50 & Geophysics of the Solid Earth \\
     & 163    & 3 & F$^\#$ & GEL 50 & Planetary Geology \\
\hline
\end{tabular}\\

To summarize, all three of these specializations have been changed to
properly account for prerequisite coursework for required courses and
to afford students additional choices for elective coursework in PHY,
ATM, and GEL.

\newpage
\section{Land, Air \& Water Resources}
Chair: William Horwath (ATM)

\subsection{No changes to ATM courses for Physics AB and BS majors}

There are no changes to Geology coursework to the Physics majors (AB
and BS).

\subsection{Changes to ATM courses for Applied Physics majors}

The proposed physics curriculum includes changes to the Applied
Physics major with specializations in Atmospheric Physics, Physical
Oceanography, and Geological Physics related to ATM courework, as
detailed in Tables~\ref{tbl:atmatmospheric}-\ref{tbl:atmgeophysics}.

In the current requirements for the Atmospheric Physics
specialization, ATM 120,121A, and 121B are required courses.  Due to
prerequisites, this now implicitly requires ATM 60 as well, which
pushes this major past the unit cap for required coureswork.  In the
proposed requirements, ATM 60,120,121A and 121B are now all explicitly
required while the major still remains below the unit cap.  In the
current requirements, ATM 128 is a possible elective.  In the proposed
requirements, ATM 124,128, and 158 are included amongst PHY and GEL
courses as possible electives.

In the current requirements for the Physical Oceanography
specialization, ATM 120 is a required course.  This implicitly
requires ATM 60, which pushes this major past the unit cap for
required coursework.  In the proposed requirements, ATM 60 is a
recommended course, and ATM 115,116, and 120 are included as possible
electives alongside PHY and GEL courses.

In the current requirements for the Geophysics major, students must
choose one course from ATM 120,121A, and 121B.  Due to prequisites,
this effectively requires students to take both ATM 60 and 120, 
pushing this major past the unit cap for required coursework.  In the
proposed requirements, ATM 60 is required and students choose five
electives from ATM 120,121A, 121B, and other courses in GEL and PHY.

To summarize, all three of these specializations have been changed to
properly account for prerequisite coursework for required courses and
to afford students additional choices for elective coursework in PHY,
ATM, and GEL.

\subsection{Expected impact of these changes on the Land, Air \& Water Resources department}

Blah blah blah.

\newpage
\captionof{table}{\label{tbl:atmatmospheric}Required ATM coursework for Atmospheric Physics}
\vskip 0.25cm
\noindent
*: recommended, $^\#$: not offered every year, $\parallel$: concurrently.\\
\begin{tabular}{|llllll|}
\hline
Course & & Units & Offered & Prereqs & Name \\
\hline
ATM & 60     & 4 & F   & & Intro. to Atmospheric Sci. \\
    & 120    & 4 & F   & ATM 60$\parallel$ & Atmos. Thermo. and Cloud Physics \\
    & 121A   & 4 & W   & ATM 120 & Atmospheric Dynamics \\
    & 121B   & 4 & S   & ATM 121A & Atmospheric Dynamics \\
\hline
\hline
    & Choose two of & & & & \\
\hline
PHY  & 105B   & 4 & & & \\
     & 105C   & 4 & $^\#$  &             & Continuum Mechanics\\
GEL  & 116N   & 3 & S  & GEL 50      & Oceanography\\
     & 150A   & 4 & S$^\#$ & GEL 116N,55 & Physical and Chemical Oceanography\\
     & 150B   & 3 & W  & GEL 50      & Geological Oceanography\\
ATM  & 124    & 3 & F  & ATM 60      & Meteorological Instruments \& Observations \\
     & 128    & 4 & W  & ATM 60      & Radiation and Satellite Meteorology \\
     & 158    & 4 & S  & ATM 121A    & Boundary-Layer Meteorology \\
\hline
\end{tabular}\\

\captionof{table}{\label{tbl:atmoceanography}Required ATM coursework for Physical Oceanography}
\vskip 0.25cm
\noindent
*: recommended, $^\#$: not offered every year, $\parallel$: concurrently.\\
\begin{tabular}{|llllll|}
\hline
Course & & Units & Offered & Prereqs & Name \\
\hline
ATM  & 60*    & 4 & F   & & Intro. to Atmospheric Sci. \\
\hline
\hline
    & Choose two of & & & & \\
\hline
PHY  & 105B  & 4 & & & \\
     & 105C  & 4 & S$^\#$ & & Continuum Mechanics \\
ATM  & 115   & 3 & S   & ATM 60 & Hydroclimatology \\
     & 116   & 3 & F   & & Modern Climate Change \\
     & 120   & 4 & F   & ATM 60$\parallel$ & Atmos. Thermo. and Cloud Physics \\
GEL  & 150B  & 3 & W & GEL 50 & Geological Oceanography\\
\hline
\end{tabular}\\

\newpage
\captionof{table}{\label{tbl:atmgeophysics} Required ATM coursework Geological Physics}
\vskip 0.25cm
\noindent
*: recommended, $^\#$: not offered every year, $\parallel$: concurrently.\\
\begin{tabular}{|llllll|}
\hline
Course & & Units & Offered & Prereqs & Name \\
\hline
ATM & 60     & 4 & F   & & Intro. to Atmospheric Sci. \\
\hline
\hline
    & Choose five of & & & & \\
\hline
PHY  & 105B   & 4 & &  & \\
     & 105C   & 4 & &  & Continuum Mechanics\\
ATM  & 120    & 4 & F  & ATM 60    & Atmospheric Thermo. \& Cloud Phys. \\
     & 121A   & 4 & F  & ATM 120   & Atmospheric Dynamics \\
     & 121B   & 4 & W  & ATM 121A  & Atmospheric Dynamics \\
GEL  & 116N   & 3 &    & & Oceanography\\
     & 146    & 3 &    & & Radiogenic Isotope Geochem. \& Cosmochem. \\
     & 150A   & 4 & S$^\#$ & GEL 116N,55 & Physical and Chemical Oceanography\\
     & 150B   & 3 & W  & GEL 50      & Geological Oceanography\\
\hline
     & incl at least two:  & & & & \\
\hline
GEL  & 161    & 3 & S$^\#$ & GEL 50 & Geophysical Field Methods \\
     & 162    & 3 & W$^\#$ & GEL 50 & Geophysics of the Solid Earth \\
     & 163    & 3 & F$^\#$ & GEL 50 & Planetary Geology \\
\hline
\end{tabular}\\



\newpage
\section{Chemistry}
Chair: David Goodin (CHE)

\subsection{No changes to CHE courses for Physics AB and BS majors}

There are no changes to Chemistry coursework to the Physics majors (AB
and BS).

\subsection{Changes to CHE courses for Applied Physics majors}

The proposed physics curriculum includes changes to the Applied
Physics major with specializations in Physical Oceanography and Chemical Physics.

In the proposed requirements for the Physical Oceanography
sepcialization, CHE 2A is now a required course.  This is to satisfy
the prequisite for GEL 55 which is also a required course for this
major.

The current requirements for the Chemical Physics specialization
include CHE 2A,2B,2C and 124A as required courses.  There are no
chemistry electives, only a long list of recommended upper division
chemistry courses.  This is because both Chemistry and Physics are
highly hierachical, and it is time consuming to complete the
prerequisites for upper division coursework in both subjects.  In the
proposed requirements, CHE 2A,2B,2C and 124A are still required.
However, students are also given the flexibility to take the sequence
CHE 110A,110B,110C, and 128A in the place of related coursework in
physics.  In additional, they may satisfy their lab requirements with
coursework in either physics or CHE 105,115,124L,129A,129B,129C.

These changes to the CHE requirements for the Chemical Physics
specialization are intended to encourage students to explore more
chemistry coursework, making this a more compelling and interesting
major.

\subsection{Expected impact of these changes on the Chemistry department}

Because CHE 2A was already implicitly required for the Physical
Oceanography specialization, making this requirement explicit should
have a negligible impact on the Chemistry department.

We have had very low enrollment in the Chemical Physics major, perhaps
because the current requirements are so uncompelling (Physics BS plus
CHE 2 and 124A).  While we hope the revised requirements will create
more interest, it is unlikely that these changes would impact
enrollment in chemistry by more than four students per year.

\newpage
\captionof{table}{\label{tbl:cheoceanography}Required CHE coursework for Physical Oceanography}
\vskip 0.25cm
\noindent
*: recommended, $^\#$: not offered every year, $\parallel$: concurrently.\\
\begin{tabular}{|llllll|}
\hline
Course & & Units & Offered & Prereqs & Name \\
\hline
CHE  & 2A     & 5 & FW  & & \\
\hline
\end{tabular}\\

\captionof{table}{\label{tbl:chechemical}Required CHE coursework for Chemical Physics}
\vskip 0.25cm
\noindent
*: recommended, $^\#$: not offered every year, $\parallel$: concurrently.\\
\begin{tabular}{|llllll|}
\hline
Course & & Units & Offered & Prereqs & Name \\
\hline
\hline
CHE  & 2A      & 5 & FW & & General Chemistry \\
     & 2B      & 5 & WS & & General Chemistry \\
     & 2C      & 5 & FS & & General Chemistry \\
CHE  & 124A    & 3 & & & Inorganic Chemistry\\
\hline
\hline
PHY  & 115A   & 4 & & & \\
     & 115B   & 4 & & & \\
     & 112    & 4 & & & \\
     & 140A   & 4 & & & \\
     & 140B*   & 4 & & & \\
\hline
& Or: &&&& \\
\hline
CHE  & 110A   & 4 & & & Physical Chemistry: Intro to QM \\
     & 110B   & 4 & & & Physical Chemistry: Atoms and Molecules \\
     & 110C   & 4 & & & Physical Chemistry: Thermodynamics \\
     & 128A   & 4 & & & \\
     & 128B*   & 4 & & & \\
PHY  & 115A*  & 4 & & & \\
     & 112*   & 4 & & & \\
\hline
\hline
& 6 or more units: & & & & \\
PHY & 122A/B & 4 & & & \\
    & 117   & 4 & & & \\
    & 118   & 4 & & & \\
CHE & 105    & 4 & & & Analytical \& Physical Chemical Methods \\
    & 115    & 4 & & & Instrumental Analysis \\
    & 124L   & 2 & & & Laboratory Methods in Inorganic Chemistry \\
    & 129A   & 2 & & & Organic Chemistry Laboratory \\
    & 129B   & 2 & & & Organic Chemistry Laboratory \\
    & 129C   & 2 & & & Organic Chemistry Laboratory \\
\hline
\end{tabular}\\




\newpage
\section{Computer Science}
Chair: Dipak Ghosal (ECS)

\subsection{No changes to ECS courses for Physics AB and BS majors}

There are no changes to Computer Science coursework to the Physics
majors (AB and BS).

\subsection{Changes to ECS courses for Applied Physics majors}

The proposed physics curriculum includes changes to the Applied
Physics major with specialization in Computational Physics that impact
ECS coursework, as detailed in Table~ ref{tbl:ecscomputational}.

In both the current and proposed major, students are required to take
ECS 36ABC and 122A.  In the proposed requirements, students choose at
least one elective from ECS 120,122B,132, and 171 (in the current
requirements, students choose from ECS 120, 122B, and 130).

It is worth noting that a major objective of the overall curriculum
update is to increase the role of computational physics in training
our students.  The specialization in Computational Physics benefits
from these improvements, including the new course PHY 40 (Introduction
to Computational Physics) and new one unit computational physics lab
courses (110L, 112L, and 115L).  Computational physics students are
expected to satisfy the prerequisites for 110L,112L, and 115L via ECS
36ABC, whereas other majors satisfy this requirement via the less
ambitious (computationally) PHY 45 (Computational Physics).

In summary, the changes to the Computational Physics specialization
are aimed at providing flexibility for students to study programming
and computational techniques outside of physics, and then bring those
skills back to the physics department to apply them specifically to
physics problems.

\subsection{Expected impact of these changes on the department}

\subsection{Additional Request: Pass 1 access to ECS 36ABC for Applied Physics majors}

\newpage
\captionof{table}{\label{tbl:ecscomputational}Required ECS coursework in Computational Physics}
\vskip 0.25cm
\noindent
*: recommended, $^\#$: not offered every year, $\parallel$: concurrently.\\
\begin{tabular}{|llllll|}
\hline
Course & & Units & Offered & Prereqs & Name \\
\hline
ECS & 36A  & 4 & FWS & & Programming and Problem Solving\\
    & 36B  & 4 & FWS & & Software Dev. and OOP in C++\\
    & 36C  & 4 & FWS & & Data Structures, Algorithms, and Programming\\
    & 122A & 4 & FWS & & Algorithm Design \& Analysis\\
\hline
\hline
    & Choose one of & & & & \\
\hline
ECS & 120  & 4 & FWS & & Theory of Computation \\
    & 122B & 4 & WS  & & Algorithm Design \& Analysis \\
    & 132  & 4 & FWS & & Probability \& Stat Modeling \\
    & 171  & 4 & F   & & Machine Learning \\
\hline
\end{tabular}\\

\newpage
\section{Electrical \& Computer Engineering}
Chair: Andre Knoesen (EEC)

\subsection{No changes to EEC courses for Physics AB and BS majors}

There are no changes to Electrical \& Computer Engineering coursework
to the Physics majors (AB and BS).

\subsection{Changes to EEC courses and ENG 17 for Applied Physics majors}

The proposed physics curriculum includes changes to the Applied
Physics major with specializations in Physical Electronics that impact
ENG and EEC coursework, as detailed in Table~ ref{tbl:ecselectronics}.

In both the current and proposed requirements, students are required
to take EEC 17 and 100.  In the proposed requirements, students choose
four elective courses from EEC 110A, 110B, 140A, 140B, 150A, and 150B.
(In the curent requirements, 150A and 150B are not included)

\subsection{Expected impact of these changes on the Electrical \& Computer Engineering department}

We expect the impact of these changes to be minimal.  In particular,
we have been accepting EEC 150A and 150B as electives on a
case-by-case basis when requested by students.  These changes merely
make this flexibility more explicit.

\captionof{table}{\label{tbl:eecelectronics}Required EEC coursework for Physical Electronics }
\vskip 0.25cm
\noindent
Units:  8 units. *: recommended, $^\#$: not offered every year, $\parallel$: concurrently.\\
\begin{tabular}{|llllll|}
\hline
Course & & Units & Offered & Prereqs & Name \\
\hline
ENG & 17     & 4 & FWS & & Circuits I\\
EEC & 100    & 4 & FW & & Circuits II\\
\hline
\hline
    & Choose four of & & & & \\
\hline
EEC & 110A  & 4 & WS & & Electronic Circuits I \\
    & 110B  & 4 & S  & & Electronic Circuits II\\
    & 140A  & 4 & FW & & Principles of Device Physics I\\
    & 140B  & 4 & S  & & Principles of Device Physics II\\
    & 150A  & 4 & WS & & Intro. to Signals \& Systems I\\
    & 150B  & 4 & F  & & Intro. to Signals \& Systems II\\
\hline
\end{tabular}\\



\newpage
\section{Materials Science \& Engineering}
Chair: Yayoi Takamura (EMS, ENG 45)

\subsection{No changes to EMS and ENG courses for Physics AB and BS majors}

There are no changes to Materials Science \& Engineering coursework
to the Physics majors (AB and BS).

\subsection{Changes to EMS courses and ENG 45 for Applied Physics majors}

The proposed physics curriculum includes changes to the Applied
Physics major with specializations in Materials Physics that impact
ENG and EMS coursework, as detailed in Table~ \ref{tbl:emsmaterials}.

In the current requirements, Materials Physics students are required
to take EMS 174 and 180, which implicitly requires them to take ENG 45
as well, pushing this degree past the unit cap on required coursework.
In the proposed requirements, students are required to take ENG 45,
and choose two electives from EMS 162, 160+164, 170, 172, 174, and
180.  Furthermore, they may satisfy their lab work requirements from
several choices within both PHY and EMS.

\subsection{Expected impact of these changes on the Materials Science \& Engineering department}

\newpage
\captionof{table}{\label{tbl:emsmaterials}Required ENG and EMS coursework for Materials Physics}
\vskip 0.25cm
\noindent
*: recommended, $^\#$: not offered every year, $\parallel$: concurrently.\\
\begin{tabular}{|llllll|}
\hline
Course & & Units & Offered & Prereqs & Name \\
\hline
ENG & 45     & 4 & FWS & & Properties of Materials \\
\hline
\hline
    & Choose two of & & & & \\
\hline
EMS & 162     & 4 & W & & Structure \& Characterization \\
    & 160+164 & 7 & F+W & & Thermo+Kinetics \\
    & 170     & 4 & S & & Sustainable Energy \\ 
    & 172     & 4 & F & & Smart Materials \\
    & 174     & 4 & S & & Mech. Behavior of Materials\\
    & 180     & 4 & F & & Materials in Eng. Design\\
\hline
\hline
    & Choose two of & & & & \\
\hline
PHY & 122A/B & 4 & & & \\
    & 117   & 4 & & & \\
    & 118   & 4 & & & \\
    & at most one of & & & & \\
EMS & 162L   & 3 & W & & Structure \& Characterization Lab\\
    & 170L   & 3 & S & & Sustainable Energy Lab \\
    & 172L   & 3 & F & & Smart Materials Lab \\
    & 174L   & 3 & S & & Mech. Behavior of Materials Lab \\
\hline
\end{tabular}\\







\newpage

\end{document}

\section{Proposed BS with Specialization in Astrophysics}

The proposed required courses for the astrophysics specialization are presented in
Tables~\ref{tbl:prep-astro-applied} and \ref{tbl:depth-astro}.  To avoid
version conflicts, some information in Tables \ref{tbl:prep} and
\ref{tbl:depth} is not duplicated in these tables.

\captionof{table}{\label{tbl:prep-astro-applied}Preparatory Subject Matter:  Astrophysics, Applied Physics, and AB.}
\noindent
\vskip 0.25cm
\begin{center}
Units:  48-50. *: recommended.\\
\begin{tabular}{|lll|}
\hline
Course & & Units \\
\hline
MAT & 21A & 4 \\  
    & 21B & 4 \\ 
    & 21C & 4 \\ 
    & 21D & 4 \\
\hline
\hline
    & one of: & \\
\hline
MAT & 22A & 3 \\
    & 27A & 3 \\
    & 67  & 4 \\
\hline
\hline
MAT & 22B & 3 \\ 
    & or: & \\
MAT & 27B & 3 \\ 
\hline
\hline
PHY & 9A & 5 \\  
    & 9B & 5 \\  
    & 9C & 5 \\  
    & 9D & 4 \\  
\hline
&or&\\
\hline
PHY & 9HA & 5 \\  
    & 9HB & 5 \\  
    & 9HC & 5 \\  
    & 9HD & 5 \\  
\hline
\hline
PHY & 40   & 3 \\
    & 80   & 4 \\ 
    & 185* & 1 \\ 
    & 190* & 1 \\ 
\hline
\end{tabular}
\end{center}

\newpage
\captionof{table}{\label{tbl:depth-astro}Depth Subject Matter: Astrophysics}
\noindent
\vskip 0.25cm
Units:  60-64. *: recommended, $^\#$: not offered every year, $\parallel$: concurrently.\\
\begin{tabular}{|llllll|}
\hline
Course & & Units & Offered & Prereqs & Name \\
\hline
AST & 25* & 4 & & & \\ 
PHY & 104A & 4 & & & \\ 
    & 105A & 4 & & & \\
    & 108  & 3 & S & 9D/9HD & Optics \\  
    & 108L & 1 & S & $\parallel$108 & Optics Laboratory \\  
    & 110A & 4 & & & \\
    & 110B & 4 & & & \\
    & 112  & 4 & & & \\    
    & 115A & 4 & & & \\
    & 115B & 4 & & & \\
\hline
\hline
PHY & 157 & 4 & S$^\#$ & (Same as 122A/B) & Astronomy Instrumentation and \\  
    &     &   &     & & Data Analysis Lab\\  
\hline
    & or & & & & \\
\hline
PHY & 122A/B & 4 & & & Advanced Physics Laboratory \\  
\hline
\hline
 & Choose four of & & & \\
\hline
PHY & 151 & 4 & F$^\#$ & $\parallel$40,$\parallel$104A & Stellar Structure and Evolution \\ 
    & 152 & 4 & F$^\#$ & $\parallel$40,$\parallel$104A & Galactic Structure and \\
    &     &   &     &           & the Interstellar Medium\\
    & 153 & 4 & W$^\#$ & 40,104A,$\parallel$105A & Extragalactic Astrophysics\\  
    & 156 & 4 & W$^\#$ & 40,104A,$\parallel$105A & Introduction to Cosmology\\ 
    & 158  & 4 & S$^\#$ & 40,104A,105A & Galaxy Formation \\ 
\hline
 & Any two of & & & & : \\
\hline 
PHY & 105B & 4 &  &  & \\ 
    & 117 & 4 &  &  & \\  
    & 118 & 4 &  &  & \\  
    & 129A & 4 &  &  & \\  
    & 130A & 4 &  &  & \\  
    & 130B & 4 &  &  & \\  
    & 150  & 4 &  &  & Special Topics\\  
    & 154  & 4 & S$^\#$ & 40,105B,110B,115A & Astro. Appl. of Phys. \\
    & 155  & 4 & W   & 104A,105B,110A & General Relativity \\ 
GEL & 163  & 4 &  &  & Planetary Geology and Geophysics\\ 
 & At most one of &  &  &  & \\ 
PHY & 194HAB & 8 &  &  & Special Study for Honors Students\\ 
    & 195  & 5 &  &  & Senior Thesis\\
    & 198  & 3+ &  &  & Directed Group Study\\ 
    & 199  & 3+ &  &  & Special Study for Adv. Undergrads.\\ 
\hline
\end{tabular}\\
\vskip 0.25cm
\noindent
PHY 108 has alternative prerequisites (PHY 7C and 21D)
intended for non-majors.  PHY 150 must be an astro topic and requires
prior department approval.\\
\noindent
{\bf Total Units:} 108-114 \\


\newpage
\section{Proposed Applied Physics Majors}

The applied physics majors (excluding chemical physics) require the
preparatory courses in Table~\ref{tbl:prep-astro-applied} and the
depth courses listed in Table~\ref{tbl:depth-applied}.  Together,
these required courses account for 72-74 units, leaving 38 units for
concentration courses as detailed in the following tables.  This
proposal includes modifications to all of the applied physics majors:
\begin{itemize}

  \item The Computational Physics major does not require PHY 45 and ECS
  36B will satisfy the prerequisite for the computing labs.  The
  elective choices from the Computer Science and Engineering
  department have been extended.  See Table~\ref{tbl:computational}.

\item The Physical Electronics major does not include 117 and 118 as
  part of the upper division lab requirement, only 122A/B is required.
  The major instead includes a large amount of electronics from the
  Electrical and Computer Engineering department.  Due to unit limits,
  only one computational lab is required.  See
  Table~\ref{tbl:electronics}.

\item The Materials Science major is renamed Materials Physics, and
  the elective choices have been extended.  Students may now use a
  Material Science and Engineering lab course as one of their required
  upper division lab courses.  See Table~\ref{tbl:materials}.

\item In the past, the Atmospheric Physics major required the Physical
  and Chemical Oceanography course (GEL 150A), which now has extensive
  prerequisites that would exceed the unit limit.  The Oceanography
  thrust of this course has been relocated to the electives, including
  the more easily obtained (GEL 116N).  The list of Atmospheric
  Physics electives has been extended.  See Table~\ref{tbl:atmospheric}.

\item The Physical and Chemical Oceanography course (GEL 150A) is not
  offered every year and now has extensive prerequisites.  Therefore,
  the Physical Oceanography major might not be achievable in two years
  for students that have not already satisfied these prerequisites.
  The prerequisites for reaching GEL 150A are now included in the
  concentration courses.  See Table~\ref{tbl:oceanography}.
  
\item The Chemical Physics major requires the courses listed in
  Tables~\ref{tbl:prep-astro-applied} and~\ref{tbl:chemical}.  
  Both chemistry and physics have significant lower division
  coursework as prerequisite for upper division coursework.  The
  additional 15 units of lower division general chemistry course
  work required by this major limits the number of upper division
  courses that can be required.  This major features two paths, one
  featuring PHY 112, 115AB, 140AB and the other featuring CHE 110ABC
  128AB.  The latter choice relies on the treatment of quantum
  mechanics and thermodynamics provided by our colleagues in the
  chemistry department.  This major requires 6 units of upper division
  lab work in chemistry or physics.
\end{itemize}
Some of the majors might benefit from further revisions, which we plan
to pursue once we have experience with the significant changes already
proposed here.  For example, it would be helpful if PHY 115AB were
listed as an alternative prerequisite for CHE 110B.

\captionof{table}{\label{tbl:depth-applied} Depth subject matter for
  all applied physics majors excluding chemical physics.}
\noindent
\vskip 0.25cm
\begin{center}
\begin{tabular}{|lll|}
\hline
Course & & Units \\
\hline
PHY & 104A   & 4 \\  
    & 105A   & 4 \\ 
    & 110A   & 4 \\ 
    & 110B   & 4 \\
    & 112    & 4 \\
    & 115A   & 4 \\ 
\hline
\end{tabular}
\end{center}

\newpage
\captionof{table}{\label{tbl:computational}Computational Physics}
\vskip 0.25cm
\noindent
{\bf Additional Preparatory Courses:  }\\
Units:  12 units. *: recommended, $^\#$: not offered every year, $\parallel$: concurrently.\\
\begin{tabular}{|llllll|}
\hline
Course & & Units & Offered & Prereqs & Name \\
\hline
ECS & 36A  & 4 & FWS & & Programming and Problem Solving\\
    & 36B  & 4 & FWS & & Software Dev. and OOP in C++\\
    & 36C  & 4 & FWS & & Data Structures, Algorithms, and Programming\\
\hline
\end{tabular}\\
\vskip 0.25cm
\noindent
{\bf Concentration Courses:  }\\
Units:  18 units. *: recommended, $^\#$: not offered every year, $\parallel$: concurrently.\\
\begin{tabular}{|llllll|}
\hline
Course & & Units & Offered & Prereqs & Name \\
\hline
ECS & 122A & 4 & FWS & & Algorithm Design \& Analysis\\
\hline
\hline
    & Choose two of & & & & \\
\hline
PHY    & 110L & 1 & & & \\
    & 115L & 1 & & & \\
    & 112L & 1 & & & \\
\hline
\hline
    & Choose one of & & & & \\
\hline
ECS & 120  & 4 & FWS & & Theory of Computation \\
    & 122B & 4 & WS  & & Algorithm Design \& Analysis \\
    & 132  & 4 & FWS & & Probability \& Stat Modeling \\
    & 171  & 4 & F   & & Machine Learning \\
\hline
\hline
    & Choose two of & & & & \\
\hline
PHY & 122A/B & 4 & & & \\
    & 117   & 4 & & & \\
    & 118   & 4 & & & \\
\hline
\end{tabular}\\
\vskip 0.25cm
\noindent
{\bf Additional Electives: (8 units)} Two courses for a total of at least 8 units of additional upper division coursework from ECS, MAT, or PHY.\\
{\bf Total Units:} 110-112\\



\newpage
\captionof{table}{\label{tbl:electronics}Physical Electronics }
\vskip 0.25cm
\noindent
{\bf Additional Preparatory Courses:  }\\
Units:  8 units. *: recommended, $^\#$: not offered every year, $\parallel$: concurrently.\\
\begin{tabular}{|llllll|}
\hline
Course & & Units & Offered & Prereqs & Name \\
\hline
ENG & 17     & 4 & FWS & & Circuits I\\
PHY & 45     & 4 & & & \\
\hline
\end{tabular}\\
\vskip 0.25cm
\noindent
{\bf Concentration Courses:  }\\
Units:  30 units. *: recommended, $^\#$: not offered every year, $\parallel$: concurrently.\\
\begin{tabular}{|llllll|}
\hline
Course & & Units & Offered & Prereqs & Name \\
\hline
EEC & 100    & 4 & FW & & Circuits II\\
PHY & 115B*  & 4 & & & \\
    & 140A   & 4 & & & \\
    & 140B*  & 4 & & & \\
    & 122A/B & 4 & & & \\
\hline
\hline
    & Choose two of & & & & \\
\hline
PHY & 110L & 1 & & & \\
    & 115L & 1 & & & \\
    & 112L & 1 & & & \\
\hline
\hline
    & Choose four of & & & & \\
\hline
EEC & 110A  & 4 & WS & & Electronic Circuits I \\
    & 110B  & 4 & S  & & Electronic Circuits II\\
    & 140A  & 4 & FW & & Principles of Device Physics I\\
    & 140B  & 4 & S  & & Principles of Device Physics II\\
    & 150A  & 4 & WS & & Intro. to Signals \& Systems I\\
    & 150B  & 4 & F  & & Intro. to Signals \& Systems II\\
\hline
\end{tabular}\\
\vskip 0.25cm
\noindent
{\bf Total Units:} 110-112\\

\newpage
\captionof{table}{\label{tbl:materials}Materials Physics}
\vskip 0.25cm
\noindent
{\bf Additional Preparatory Courses:  }\\
Units:  8 units. *: recommended, $^\#$: not offered every year, $\parallel$: concurrently.\\
\begin{tabular}{|llllll|}
\hline
Course & & Units & Offered & Prereqs & Name \\
\hline
PHY & 45     & 4 & & & \\
ENG & 45     & 4 & FWS & & Properties of Materials \\
\hline
\end{tabular}\\
\vskip 0.25cm
\noindent
{\bf Concentration Courses:  }\\
Units:  30-34 units. *: recommended, $^\#$: not offered every year, $\parallel$: concurrently.\\
\begin{tabular}{|llllll|}
\hline
Course & & Units & Offered & Prereqs & Name \\
\hline
PHY & 115B   & 4 & & & \\
    & 140A   & 4 & & & \\
    & 140B   & 4 & & & \\
    & 110L & 1 & & & \\
    & 115L & 1 & & & \\
    & 112L & 1 & & & \\
\hline
\hline
    & Choose two of & & & & \\
\hline
EMS & 162     & 4 & W & & Structure \& Characterization \\
    & 160+164 & 7 & F+W & & Thermo+Kinetics \\
    & 170     & 4 & S & & Sustainable Energy \\ 
    & 172     & 4 & F & & Smart Materials \\
    & 174     & 4 & S & & Mech. Behavior of Materials\\
    & 180     & 4 & F & & Materials in Eng. Design\\
\hline
\hline
    & Choose two of & & & & \\
\hline
PHY & 122A/B & 4 & & & \\
    & 117   & 4 & & & \\
    & 118   & 4 & & & \\
    & at most one of & & & & \\
EMS & 162L   & 3 & W & & Structure \& Characterization Lab\\
    & 170L   & 3 & S & & Sustainable Energy Lab \\
    & 172L   & 3 & F & & Smart Materials Lab \\
    & 174L   & 3 & S & & Mech. Behavior of Materials Lab \\
\hline
\end{tabular}\\
\noindent
{\bf Total Units:} 110-116 \\

\newpage
\captionof{table}{\label{tbl:atmospheric}Atmospheric Physics}
\vskip 0.25cm
\noindent
{\bf Additional Preparatory Courses:  }\\
Units: 11 units. *: recommended, $^\#$: not offered every year, $\parallel$: concurrently.\\
\begin{tabular}{|llllll|}
\hline
Course & & Units & Offered & Prereqs & Name \\
\hline
PHY & 45     & 4 &     & & \\
GEL & 50     & 3 & FWS & & Physical Geology \\
ATM & 60     & 4 & F   & & Intro. to Atmospheric Sci. \\
\hline
\end{tabular}\\
\vskip 0.25cm
\noindent
{\bf Concentration Courses:  }\\
Units:  27-29 units. *: recommended, $^\#$: not offered every year, $\parallel$: concurrently.\\
\begin{tabular}{|llllll|}
\hline
Course & & Units & Offered & Prereqs & Name \\
\hline
ATM & 120    & 4 & F   & & Atmos. Thermo. and Cloud Physics \\
    & 121A   & 4 & W   & & Atmospheric Dynamics \\
    & 121B   & 4 & S   & & Atmospheric Dynamics \\
\hline
\hline
    & Choose one of & & & & \\
\hline
PHY & 110L   & 1 & & & \\
    & 115L   & 1 & & & \\
    & 112L   & 1 & & & \\
\hline
\hline
    & Choose two of & & & & \\
\hline
PHY & 122A/B & 4 & & & \\
    & 117   & 4 & & & \\
    & 118   & 4 & & & \\
\hline
\hline
    & Choose two of & & & & \\
\hline
PHY  & 105B   & 4 & & & \\
     & 105C   & 4 & $^\#$  &             & Continuum Mechanics\\
GEL  & 116N   & 3 & S  & GEL 50      & Oceanography\\
     & 150A   & 4 & S$^\#$ & GEL 116N,55 & Physical and Chemical Oceanography\\
     & 150B   & 3 & W  & GEL 50      & Geological Oceanography\\
ATM  & 124    & 3 & F  & ATM 60      & Meteorological Instruments \& Observations \\
     & 128    & 4 & W  & ATM 60      & Radiation and Satellite Meteorology \\
     & 158    & 4 & S  & ATM 121A    & Boundary-Layer Meteorology \\
\hline
\end{tabular}\\
\vskip 0.25cm
\noindent
{\bf Total Units:} 110-114 \\


\newpage
\captionof{table}{\label{tbl:oceanography}Physical Oceanography}
\vskip 0.25cm
\noindent
{\bf Additional Preparatory Courses:  }\\
Units:  15 units. *: recommended, $^\#$: not offered every year, $\parallel$: concurrently.\\
\begin{tabular}{|llllll|}
\hline
Course & & Units & Offered & Prereqs & Name \\
\hline
CHE  & 2A     & 5 & FW  & & \\
PHY  & 45     & 4 &     & & \\
GEL  & 50     & 3 & FWS & & Physical Geology \\
     & 55     & 3 & F      & CHE 2A & Intro. to Geochemistry\\
ATM  & 60*    & 4 & F   & & Intro. to Atmospheric Sci. \\
\hline
\end{tabular}\\
\vskip 0.25cm
\noindent
{\bf Concentration Courses:  }\\
Units:  23-25 units. *: recommended, $^\#$: not offered every year, $\parallel$: concurrently.\\
\begin{tabular}{|llllll|}
\hline
Course & & Units & Offered & Prereqs & Name \\
\hline
GEL  & 116N   & 3 & S      & GEL 50          & Oceanography\\
     & 150A   & 4 & S$^\#$ & GEL 116N,55 & Physical and Chemical Oceanography\\
\hline
    & Choose two of & & & & \\
\hline
PHY  & 110L & 1 & & & \\
     & 112L & 1 & & & \\
     & 115L & 1 & & & \\
\hline
\hline
    & Choose two of & & & & \\
\hline
PHY & 122A/B & 4 & & & \\
    & 117   & 4 & & & \\
    & 118   & 4 & & & \\
\hline
\hline
    & Choose two of & & & & \\
\hline
PHY  & 105B  & 4 & & & \\
     & 105C  & 4 & S$^\#$ & & Continuum Mechanics \\
ATM  & 115   & 3 & S   & ATM 60 & Hydroclimatology \\
     & 116   & 3 & F   & & Modern Climate Change \\
     & 120   & 4 & F   & ATM 60$\parallel$ & Atmos. Thermo. and Cloud Physics \\
GEL  & 150B  & 3 & W & GEL 50 & Geological Oceanography\\
\hline
\end{tabular}\\
\vskip 0.25cm
\noindent
{\bf Total Units:} 110-114\\
{\bf Note: } For transfer students arriving in their junior year, this
degree may not be possible to complete in two years.  Students should
seek instructor permission to take GEL 150A alongside 116N prerequisite if
it is offered in their junior year.\\

\newpage
\captionof{table}{\label{tbl:chemical}Chemical Physics}
\vskip 0.25cm
\noindent
{\bf Additional Preparatory Courses:  }\\
Units:  19 units. *: recommended, $^\#$: not offered every year, $\parallel$: concurrently.\\
\begin{tabular}{|llllll|}
\hline
Course & & Units & Offered & Prereqs & Name \\
\hline
\hline
CHE  & 2A      & 5 & FW & & General Chemistry \\
     & 2B      & 5 & WS & & General Chemistry \\
     & 2C      & 5 & FS & & General Chemistry \\
PHY  & 45      & 4 & & & \\
\hline
\end{tabular}\\
\noindent
{\bf Concentration Courses:  }\\
Units:  43-45 units. *: recommended, $^\#$: not offered every year, $\parallel$: concurrently.\\
\begin{tabular}{|llllll|}
\hline
Course & & Units & Offered & Prereqs & Name \\
\hline
PHY & 104A     & 4 & & & \\  
    & 105A     & 4 & & & \\ 
    & 110A     & 4 & & & \\ 
    & 110B     & 4 & & &\\    
CHE & 124A     & 3 & & & Inorganic Chemistry\\
\hline
\hline
PHY  & 115A   & 4 & & & \\
     & 115B   & 4 & & & \\
     & 112    & 4 & & & \\
     & 140A   & 4 & & & \\
     & 140B*   & 4 & & & \\
\hline
& Or: &&&& \\
\hline
CHE  & 110A   & 4 & & & Physical Chemistry: Intro to QM \\
     & 110B   & 4 & & & Physical Chemistry: Atoms and Molecules \\
     & 110C   & 4 & & & Physical Chemistry: Thermodynamics \\
     & 128A   & 4 & & & \\
     & 128B*   & 4 & & & \\
PHY  & 115A*  & 4 & & & \\
     & 112*   & 4 & & & \\
\hline
\hline
& Choose two of: &&&& \\
\hline
PHY  & 110L  & 1 & & & \\
     & 112L  & 1 & & & \\
     & 115L  & 1 & & & \\
\hline
& 6 or more units: & & & & \\
PHY & 122A/B & 4 & & & \\
    & 117   & 4 & & & \\
    & 118   & 4 & & & \\
CHE & 105    & 4 & & & Analytical \& Physical Chemical Methods \\
    & 115    & 4 & & & Instrumental Analysis \\
    & 124L   & 2 & & & Laboratory Methods in Inorganic Chemistry \\
    & 129A   & 2 & & & Organic Chemistry Laboratory \\
    & 129B   & 2 & & & Organic Chemistry Laboratory \\
    & 129C   & 2 & & & Organic Chemistry Laboratory \\
\hline
\end{tabular}\\
\noindent
{\bf Total Units:} 110-114 \\

\newpage
\captionof{table}{\label{tbl:geophysics} Geological Physics}
\vskip 0.25cm
\noindent
{\bf Additional Preparatory Courses:  }\\
Units: 13 units. *: recommended, $^\#$: not offered every year, $\parallel$: concurrently.\\
\begin{tabular}{|llllll|}
\hline
Course & & Units & Offered & Prereqs & Name \\
\hline
PHY & 45     & 4 &     & & \\
GEL & 50     & 3 & FWS & & Physical Geology \\
    & 50L    & 2 &     & & Physical Geology Laboratory\\
ATM & 60     & 4 & F   & & Intro. to Atmospheric Sci. \\
\hline
\end{tabular}\\
\vskip 0.25cm
\noindent
{\bf Concentration Courses:  }\\
Units:  25-30 units. *: recommended, $^\#$: not offered every year, $\parallel$: concurrently.\\
\begin{tabular}{|llllll|}
\hline
Course & & Units & Offered & Prereqs & Name \\
\hline
    & Choose two of & & & & \\
\hline
PHY & 110L   & 1 & & & \\
    & 115L   & 1 & & & \\
    & 112L   & 1 & & & \\
\hline
\hline
    & Choose two of & & & & \\
\hline
PHY & 122A/B & 4 & & & \\
    & 117   & 4 & & & \\
    & 118   & 4 & & & \\
\hline
\hline
    & Choose five of & & & & \\
\hline
PHY  & 105B   & 4 & &  & \\
     & 105C   & 4 & &  & Continuum Mechanics\\
ATM  & 120    & 4 & F  & ATM 60    & Atmospheric Thermodynamics \& Cloud Physics \\
     & 121A   & 4 & F  & ATM 120   & Atmospheric Dynamics \\
     & 121B   & 4 & W  & ATM 121A  & Atmospheric Dynamics \\
GEL  & 116N   & 3 &    & & Oceanography\\
     & 146    & 3 &    & & Radiogenic Isotope Geochemistry \& Cosmochemistry \\
     & 150A   & 4 & S$^\#$ & GEL 116N,55 & Physical and Chemical Oceanography\\
     & 150B   & 3 & W  & GEL 50      & Geological Oceanography\\
\hline
     & incl at least two:  & & & & \\
\hline
GEL  & 161    & 3 & S$^\#$ & GEL 50 & Geophysical Field Methods \\
     & 162    & 3 & W$^\#$ & GEL 50 & Geophysics of the Solid Earth \\
     & 163    & 3 & F$^\#$ & GEL 50 & Planetary Geology \\
\hline
\end{tabular}\\
\vskip 0.25cm
\noindent
{\bf Total Units:} 110-117 \\



\newpage
\section{Proposed AB Physics Major}
\label{sec:ab}

The proposed requirements for the AB Physics major are listed in
Tables~\ref{tbl:prep-astro-applied} and \ref{tbl:ab-depth}.  To avoid
version conflicts, some information in Tables \ref{tbl:prep} and
\ref{tbl:depth} is not duplicated in these tables.

\captionof{table}{\label{tbl:ab-depth}Depth Subject Matter}
\noindent
\vskip 0.25cm
\begin{center}
Units:  32. *: recommended, $\parallel$: concurrently.\\
\begin{tabular}{|lll|}
\hline
Course & & Units \\
\hline
PHY & 104A & 4 \\
    & 105A & 4 \\
    & 110A & 4 \\
    & 110B & 4 \\
    & 112  & 4 \\
    & 115A & 4 \\
\hline
\hline
PHY & 122A/B & 4 \\
\hline
    & At least one of & \\ 
\hline
PHY & 129A & 4 \\
    & 130A & 4 \\
    & 140A & 4 \\
    & 151 & 4 \\
    & 152 & 4 \\
    & 153 & 4 \\
    & 156 & 4 \\
    & 158 & 4 \\
\hline
\end{tabular}\\ \vskip 0.25cm
\end{center}
\noindent
{\bf Foreign Language:} The AB degree requires proficiency in a language other than English (15 units)\\
\noindent
{\bf Total Units:} 80-82 units, excluding language requirement.




\end{document}

