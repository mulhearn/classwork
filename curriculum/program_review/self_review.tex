\documentclass[12pt]{article}
\usepackage{lmodern}
\usepackage[T1]{fontenc}
\usepackage[dvips,letterpaper,margin=0.75in,bottom=0.75in]{geometry}
\usepackage{cancel}
\usepackage{graphicx}
\usepackage{braket}
\usepackage{latexsym,amssymb,amsmath}
\usepackage{pdfpages}
\usepackage{xcolor}
\usepackage{capt-of}
\usepackage{amsmath}
\usepackage{cite}
\usepackage{lineno}
\usepackage[hyperfootnotes=false,hidelinks]{hyperref}
\usepackage[shortlabels]{enumitem}
\newcommand{\tcr}{\textcolor{red}}
\newcommand{\tcb}{\textcolor{blue}}

\usepackage[american,fulldiode]{circuitikz}
\tikzset{component/.style={draw,thick,circle,fill=white,minimum size =0.75cm,inner sep=0pt}}

\begin{document}
\ctikzset{bipoles/thickness=1}
\ctikzset{bipoles/length=.6cm}

% DRAFT ONLY:
\linenumbers



%%%%%%%%%%%%%%%%%%%%%%%%%%%%%%%%%%%%%%%%%%%%%%%%%%%%%%%%%%%%%%%%
%%%%%%%%%%%%%%%%%%%%%%%%%%%%%%%%%%%%%%%%%%%%%%%%%%%%%%%%%%%%%%%%
\title{Physics and Applied Physics \\ Program Self-Review (Draft)}
%%%%%%%%%%%%%%%%%%%%%%%%%%%%%%%%%%%%%%%%%%%%%%%%%%%%%%%%%%%%%%%%
%%%%%%%%%%%%%%%%%%%%%%%%%%%%%%%%%%%%%%%%%%%%%%%%%%%%%%%%%%%%%%%%

\maketitle

\newpage
\section{Overview of the major}
\label{sec:overview}

{\bf Questions: What are the Program Learning Outcomes identified for this major? What is the 
role of this major in undergraduate education on the campus, i.e., how does the major 
contribute to the undergraduate educational mission of the campus? Is the major clearly 
distinguished from other similar majors on campus?}\\[3pt]

%\noindent
%{\it {\color{red} Refer to the catalog description of the major and the other majors reviewed 
%in the same cluster (Appendi%x A). Describe any inaccuracies in the catalog description and 
%explain plans for correcting them. Identify the other majors% in the cluster that are most 
%similar to yours and explain how your major differs from them.}}\\[3pt]

\noindent
The Department of Physics and Astronomy, and our majors, play a vital
role in the undergraduate education mission of the campus.  Ours is an
ancient discipline, and much of the undergraduate program is spent
learning theoretical and experimental physics from a hundred years ago
or more.  Yet these ideas and concepts remain highly potent in the
modern world, and equally challenging and rewarding to master.  While
studying fundamental physics concepts, our students are exposed to the
latest cutting-edge research and concepts concerning the physics of
the small (nuclear processes, atomic structure, particle searches, and
cellular processes) and the large (dark matter, dark energy,
cosmology).  Our majors are trained in the techniques of experimental
physics, and we have recently dramatically expanded our training in
computational physics.

We are in the process of updating our undergraduate curriculum, as
described below.  The catalog description of the major provides an
accurate overview of the program, and the proposed undergraduate
curriculum update includes changes to the catalog to maintain its
accuracy under the new program.

Our department is distinguished from other departments in the cluster
by the requirement of the most vigorous introductory physics courses,
a diverse upper-level curriculum that develops physics concepts which
appear nowhere else on campus, and a vigorous and sustained emphasis
on fundamental concepts and analytic thinking.  The program learning
objectives specific to our major are described in detail below.\\[3pt]

\noindent
{\bf Undergraduate Curriculum Update (UCU):}
The physics department, led by the undergraduate curriculum committee,
has been developing over the past five years an update to the
undergraduate physics curriculum.  The proposal was subjected to an
extensive vetting process that involved assigned department readers
who were not involved in formulating the initial plan.  The proposal
was unanimously supported by the department in a December 2021 vote.
All new and modified courses have been approved in the UC Davis
Integrated Curriculum Management System (ICMS).  We are working toward
final approval, by campus, of the proposal during AY 2023-2024.

The proposed changes to our program reflect three main observations about our current program: 
\begin{itemize}
 \item Workload over four years of study was imbalanced, with too few physics
courses in the sophomore year, and too many in the junior year. 
 \item Integration between transfer students and
four-year students, who have somewhat different backgrounds, needed to be
improved. Our recent departmental Climate Survey confirmed this by showing a
huge satisfaction gap between the groups. 
 \item  The curriculum did not reflect the explosive growth
in computational methods and their applications.
\end{itemize}
There are many other aims of the update, but these are the primary motivations.

{\bf The proposed undergraduate curriculum update will be widely
  referenced throughout this document by the acronym UCU.}  The
proposal itself is far too ponderous of a document to include as an
appendix here.  It will be provided as a stand-alone file.
The latest version of the proposal is also available for download
online\footnote{See
  \href{https://github.com/mulhearn/classwork/blob/main/curriculum/curriculum.pdf}{https://
github.com/mulhearn/classwork/blob/main/curriculum/curriculum.pdf (click for link)}
  and use the download option from the pull down menu to view the entire document.}.\\[3pt]

\noindent
{\bf Program Learning Objectives (PLOs)}: Although we did not use the
specific term PLO\footnote{We prefer to use the word objectives over
outcomes, as outcomes could be accident al, whereas objectives are
intentional.  Fortunately, they both have have the same acronym.} in
that document, these objectives were carefully considered as part of
the development of the UCU (see Sections 4-6 in particular).

There are a number of objectives that are both widely applicable and
central to the discipline of physics, most of which are reinforced in
nearly every physics course that our majors take:
\begin{itemize}
 \item Using logic and analytic reasoning to make predictions.
 \item Applying general principles (e.g. conservation of energy, symmetry) in specific 
situations. 
 \item Testing results using dimensional analysis and limiting cases. 
 \item Dividing complex problems into manageable steps.
 \item Establishing feedback to determine if something is working or not.
\end{itemize}  
The next set of objectives is related to mathematical preparation:
\begin{itemize}
\item Understand the theory and practical application of differential, integral, and vector 
calculus, linear algebra, ordinary and partial differential equations.  (MAT 21ABCD, 22AB, and 
PHY 104A.)
\end{itemize}
While these concepts are first introduced in those courses, they are continuously reinforced 
throughout the major.  The second set of objectives are related to theoretical physics, 
students are expected to acquire a working knowledge of the theory and practical application 
of the following core topics:
\begin{itemize}
 \item Classical Mechanics (9A/9HA,105AB): the fundamental principles of physical laws
  (e.g. least action and symmetries) are taught in a familiar and intuitive context. 
\item Electromagnetism (9C/9HD,110AB): a remarkable special case of classical
  phenomena that anticipate non-Newtonian physics (e.g. special
  relativity, gauge theories).  No other force in nature can be understood so
  completely in such a straightforward fashion.  
\item Quantum Mechanics (9HC/9D,115AB): the rules governing the microscopic world are
  different from those governing our familiar macroscopic world.  The
  rules are not intuitive but they can be codified and used to make
  quantitative predictions which can be experimentally verified.
\item Statistical Mechanics (9HB/9B,9D,112): the crucial statistical explanation for how
  microscopic laws ultimately produce the macroscopic world which we inhabit.
\end{itemize}
A major focus of the curriculum update has been to devote more coursework to computational 
physics.
\begin{itemize}
 \item Develop programming skills sufficient for tackling problems from computational physics 
(PHY 40,45)
 \item Apply the techniques of computational physics to problems from theoretical physics (PHY 
110L, 112L, 115L).
\end{itemize}
Our current programming toolset includes Python (first encountered in
PHY 40) and C/C++ (first encountered in PHY 45) but these may tools
may evolve over time.  The next set of objectives are related to
experimental physics:
\begin{itemize}
 \item Learn how to conduct and report scientific experiments (PHY 80, 117, 118, 122A/B, 157).
 \item Gain practical hands-on knowledge of lab equipment, electronics, and technical trouble-
shooting (PHY 80, 117, 118, 122A/B, 157).
\end{itemize}
The objective is for students to apply what they have learned previously to advanced 
specialized topics such as nuclear physics, particle physics, condensed matter, and astronomy.  
We refer to these courses as capstone courses:
\begin{itemize}
 \item Demonstrate mastery of physics by applying it to advanced topics (PHY 129AB, 130AB, 
140AB, 151-158).
\end{itemize}

\newpage
\section{Outcome of Previous Program Review}
\label{sec:previous}

{\bf Please list the recommendations made at the conclusion of the previous review (these may 
have been made by the review committee, Executive Committee and/or Dean) and comment briefly 
on the current status of the matters noted in the recommendations. Discuss any other 
significant changes in the major since the last review.}\\

\noindent
The committee report of the Undergraduate Instruction Program Review
(UIPR) and the Review Team Report from the previous review are
included in the appendix.  These reports were a source of motivation
for the UCU described above, particularly in the specific cases noted
below.  The reports identified strengths in our program, which we
appreciate, but in this overview, we will address only the identified
weaknesses:
\begin{itemize}

\item {\bf Space, Lab Conditions and Maintenance:} the review team was
  concerned that the undergraduate lab space was ``shabby and in need
  of renovation''.  Within the UC, the cost of building renovations is
  shockingly expensive\footnote{Even by the standards of
    cost-of-living-numbed California residents}, and the university
  funding available for renovations is limited.  Fortunately, we have
  received vigorous support from campus with several sizable grants
  for both equipment and renovations, for which we are deeply
  appreciative.  This issue is discussed in more detail below (Section~\ref{sec:instruction}).
  
\item {\bf Transfer Student Readiness:} the review team was concerned
  that many incoming transfer students were not prepared for a physics
  major, and were dropping out or changing major at a high rate.
  Following the recommendations of the review team and committee, we
  began selective major review in time for academic year (AY)
  2021-2022.  The department has since seen a substantial drop in
  transfer enrollment at a level that we find potentially troubling,
  if it persists.  This is discussed in more detail below
  (Section~\ref{sec:students}: Students in the major).  {\bf Improving
    the experience for incoming transfer students was a major focus of
    the UCU.}  By ensuring transfer students are retained, we hope to
  mitigate the loss of initial enrollment.

\item {\bf Computational Instruction:} The review team noted that
  programming coursework students were taking outside of the
  department was not adequately training our students in computational
  physics.  {\bf Expanding the role of computational physics in our
    major was a central focus of the UCU} and we have added to our
  curriculum several new required courses focused on computation
  physics.  This is described in more detail below
  (Section~\ref{sec:snws}: Major strengths and weaknesses).
  
\item {\bf Academic Advising:} The review team noted low satisfaction
  with campus and department advising.  In this report, we note that
  student satisfaction with advising has improved.  The likely
  contributions to this improvement, as well as some lingering
  concerns, are described below (Section~\ref{sec:perceptions}:
  Student perceptions of the major).  The size of our undergraduate
  program has increased by $50\%$, but we still have only one
  undergraduate advisor.

\item {\bf PHY 122: Advanced Physics Laboratory} the review team was
  concerned that students were generally poorly prepared for this
  upper division lab course.  Following the recommendations of the
  committee, we have introduced a new course (PHY 80: Experimental
  Techniques) as a prerequisite for PHY 122.  {\bf As part of the UCU,
    PHY 80 is now a required course for all physics majors (not just
    those taking PHY 122).}  This is discussed in more detail below
  (Section~\ref{sec:snws}: Major strengths and weaknesses)
  
\item {\bf PHY 157:} The review team noted that demand for PHY 157 is
  greater than it's capacity.  Regrettably, and despite diligent
  effort by Prof. Tucker Jones in particular, we have struggled to
  maintain the same level of throughput in this course, due to the
  retirement of the previous instructor.  The practical challenges
  here and future plans are discussed below (Section~\ref{sec:snws}:
  Major strengths and weaknesses)

\item {\bf Communication Skills:} The review team reported anecdotal
  evidence of student dissatisfaction with the level of development of
  skills in writing and oral presentation.  As currently taught, upper
  division physics lab courses (PHY 157 and 122A/B) satisfy the upper
  division English composition requirement (WR104E) for which we have
  recently sought approval English Language and Literacy Committee.
  We discuss this issue, along with related concerns about creativity
  and qualitative reasoning, in more detail below
  (Section~\ref{sec:perceptions}: Student perceptions about the
  major.)

\end{itemize}

We have taken the concerns raised during the previous review quite
seriously, and have made as much progress as we could manage, to the
clear benefit of our department, especially our students.  We therefore
look forward to a fruitful collaboration during this review as well.

\newpage
\section{Faculty in the major}
\label{sec:faculty}

{\bf Questions: Who does the bulk of teaching in the major? What are
  the demographics of instructors in the major? Will the program be
  affected by substantial changes in the faculty (e.g. anticipated
  retirements) in the next review period?}\\[3pt]

%Refer to the Tableau data concerning faculty in your department and the other departments 
reviewed in the same cluster (Appendix B, Tables 1-5). Based on those data and any additional 
information you wish to include, comment on each of the following for your major over the 
review period, referring when appropriate, to differences between your major and others in the 
cluster:
%    a) Table 1.  Instructional Faculty – FTE and Percent by Rank 
%    b) Table 2.  Age of Ladder Faculty – Percent by Age Group 
%    c) Table 3.  Gender of Ladder Faculty – Number and Percent by Rank 
%    d) Table 4.  Under-represented Ladder Faculty – Number and Percent by Ran 
%    e) Table 5.  New Faculty Hires and Separations – Number by Rank 

\noindent
{\it This section refers to tables which are available in the Appendix.  We
use a compact notation, where, for example, [B3] refers to Table 3 in
Appendix B.  The program was provided with an explicit list of items
to discuss.  To increase readability, we discuss the salient issues in
the most natural order first.  At the end of this section, we include a
``Discussion Checklist'' which indicates where each discussion point
was covered, using the original enumeration (a-e) of topics provided
in the original prompt. We have omitted the original prompt to
increase readability.}  {\color{red} TODO: copy to other sections as
      needed}.\\[3pt]

\noindent
Our teaching responsibilities are evenly shared by the faculty, and
the demographics of our course instructors largely reflects the
demographics of our faculty.  A long period of limited hiring has put
the department in an uncomfortable position: new hiring is not
keeping up with retirements and the size of the faculty is shrinking.
This contraction is occurring when the number of physics majors, and
demand for introductory physics courses by non-majors, are both
increasing.\\[3pt]

\noindent {\bf Rank and Age of Faculty:} There are clear trends
revealed in [B1-2].  The Physics department faculty is significantly
older than the the average for the College of Letters and Science
(L\&S), and the size of the faculty is shrinking.  These trends are
correlated: retirements are out-pacing new hires.  While these numbers
do not reflect our most recent hires (Profs. Matthew Citron and Nancy
Arggawal) this has not been fast enough to avoid a shrinking
department.  Retirements have also left us with no LSOE (i.e. Teaching Professors
with ladder-rank) faculty at present.

A significant obstacle to hiring at an even faster rate is the
availability of startup funding.  Under the current budget model, the
physics department will struggle to maintain a pace of one hire per
year, a pace which would lead to size of the physics faculty shrinking
further.\\[3pt]

\noindent
{\bf Diversity, Inclusion, and Equity:} The data provided in [B3-5] is
incomplete, so we will address the topic of diversity of the faculty
in purely qualitative terms.  Most importantly, our faculty is
overwhelming supportive of taking strong measures to improve the
diversity of our faculty within the limits of what we are legally
allowed to do.  We are not looking for quick and easy fixes.  Instead,
we have been studying and adopting best-practices in hiring.  For
example, our two most recent faculty searches were intentionally
broadened in scope, as this has been shown to increase the diversity
of the applicant pool.  As another example, we have started providing
zoom interview questions in advance, as well as more details about the
interview process in general, because evidence shows that members of
URGs are systematically disadvantaged when such details are assumed to
be already known.  As is often the case when adopting best practices,
we found that these steps have also made the interview process better
overall. For example, when provided with the questions in advance,
applicants gave better, more thoughtful, and more useful responses.
This remains an urgent issue, and there remains much to be done here.\\[3pt]

\noindent
{\bf Discussion Checklist:} (a-e) Discussed above [B1-5].

\newpage
\section{Instruction, advising, and resources in the major}
\label{sec:instruction}

{\bf Questions: How effective is the delivery of instruction in the major? Are faculty engaged 
in the major? Is advising adequate? Is there adequate staff support? Are adequate space and 
facilities available? Is the program keeping pace with developments in the field? Are grading 
standards appropriate? What is the role of virtual and hybrid courses in this major? Please 
attach or include here a sample 4 year graduation plan for your program.}\\[3pt]

%Refer to the Tableau data concerning instruction in the major and the other majors reviewed 
in the same cluster (Appendix B, Tables 6 -12). Based on those data and any additional 
information you wish to include, comment on each of the following for your major over the 
review period, referring, when appropriate to differences between your major and others in the 
cluster:
%\begin{enumerate}[a)]
% \item Table 6.  Majors per Instructional Faculty FTE
% \item Table 7.  Students in Major Enrolled in Upper Division Courses – Percent of Total 
Course Enrollment 
% \item Table 8.  TAs Assigned to Upper Division Courses – Number By TA Role 
% \item Table 9.  Student Faculty Ratio – By Instructor Type 
% \item Table 10.  Courses Taught – Percent By Instructor Type and Course Leve 
% \item Table 11.  Assigned Space – I\&R Assignable Square Feet (ASF) – By Department
% \item Table 12.  Distribution of Grades in Upper Division Courses – Percent of Total 
Enrolled and Average GPA 
%\end{enumerate}

%\noindent
%Please also address the following issues, for which no data are provided:\\
%\begin{enumerate}[a)]
% \setcounter{enumi}{7}
% \item Comment on the degree of interest and engagement of the faculty in the major.
% \item Comment on the adequacy of staff support for the major.
% \item Comment on the adequacy of staff advising for the major.
% \item Comment on the adequacy of instructional equipment and facilities for the major.
% \item Comment on the program’s record of keeping pace with advances in the field.
% \item Comment on any academic programs that share or compete for instructional, advising, or 
other resources with this major (e.g., a similar major, a masters program, or a doctoral 
program)
% \item Comment on the number of hybrid and virtual courses are offered by the department/
program and the average number of hybrid and virtual courses students take to complete their 
major requirements. Comment on any establish departmental resources and best practices 
regarding hybrid and virtual instruction.
%\end{enumerate}

\noindent
Our highly-engaged faculty is dedicated to delivering effective
instruction to our students, and we adhere to rigorous academic
standards with respect to grading.  While we do face challenges from
limited resources, overall we find that advising, staff support, space
and facilities are all adequate for our purposes.  In the sections
that follow, we attempt to provide evidence to support this bold
assessment, as well as call out areas where we hope to do better than
``adequate''.  Undergraduate physics is largely focused on the
cutting-edge theoretical physics of a hundred years ago or more.  Yet
we do stay relevant in the modern world: those old ideas remain
challenging to master and highly potent to cutting-edge applications.
Our multi-year continuing commitment to the UCU is strong evidence
that we continue to innovate.  One exception: outside of emergencies,
we have no plans to include virtual or hybrid courses in our program,
as we have found, for our discipline, that no reasonable amount of
effort can produce such a course which is as effective as learning in
person (See Section~\ref{sec:remote}).  Sample four-year graduation
plans (as well as two year plans for transfer students) are provided
for every major and specialization, in the UCU.  See Table 6 in that
document, for one example\footnote{We can update this report with one
  example included here if requested.}.\\[3pt]

\noindent
{\bf Grading Policy:} The grade distribution in [B12] shows that
physics (and chemistry) are grading significantly more strictly than
L\&S overall, which is a conscious decision on our part.  Our
adherence to rigorous academic standards adds significant value to the
physics degree, and to physics coursework in general.  For example,
our top students are regularly accepted into top graduate programs,
likely in part because the A's they receive in our courses represent a
meaningful accomplishment.  We are concerned about the ability of UCD
to continue as a forefront research institution and engine of social
mobility if the grade inflation that we see in L\&S overall is allowed
to continue or accelerate.  Our students are at a disadvantage for
receiving graduation honors because the thresholds are set
unachievably high due to grade inflation elsewhere.  In the College of 
L\&S seniors are permitted to do an Honors thesis if their GPA  falls 
in the top 16\%  (Currently 3.846).  Less than 8\% of our seniors typically 
fall above this overall GPA, or less than half the overall number.\\[3pt]

\noindent
{\bf Instructors:} Our department prioritizes providing physics majors
with ladder faculty instructors, particularly for upper-division
coursework.  The student FTE per faculty FTE is increasing [B6].  This
is the result of both an increase in the number of physics majors and
a decrease in the number of faculty.  Despite the challenges from a
shrinking faculty, physics majors are being taught by ladder faculty
[B9-10] for $90\%$ of their upper division coursework, a larger
fraction than any other L\&S department considered this cycle.  They
are being taught by ladder faculty for $58.5\%$ of their
lower-division physics coursework, also above average for L\&S.

For introductory physics coursework, we also utilize the
highly-effective pedagogy of Continuing Lecturers.  Dr. Dina
Zhabinskaya manages the PHY 7 series (introductory physics aimed at
bioscience) and Dr. Weideman manages the PHY 9 series (introductory
physics aimed at natural science and engineering).  They are both
deeply committed to innovative pedagogy, are great assets to our
department, and regularly share their wisdom with the rest of the
faculty.  We regularly hire Unit-18 pre-Six Instructors (temporary
lecturers) as well.  Some of our graduate students are interested in
more extensive teaching experience than typical TA assignments afford,
and, when we believe they are up to the challenge, we also hire
Associate Instructors to teach in PHY 7 and 9.\\[3pt]

\noindent
{\bf Staff Support:}
Physics is a unique discipline with its own unique culture, and we are profoundly grateful to 
the staff members that work alongside us to maintain it.
({\color{red} TODO:  Discuss with Tracy how we would like to describe the adequacy of staff 
support overall, Mike can meanwhile describe the key players in instructional support}  
Richards comment about same number of advisors.)\\[3pt]


\noindent
{\bf Instructional Space, Equipment, and Facilities:} While total
assignable space has increased [B11], it has not been faster than the
number of student FTEs has increase.  The overall amount of assignable
space is above average for L\&S, but this reflects our need for
dedicated laboratory space, and large lecture halls adjacent to
demonstration support.  As discussed in Section~\ref{sec:previous},
the previous review noted a deficiency in the condition of the lab
space used for undergraduate instruction.  Fortunately, we have
received vigorous support for acquiring new instructional equipment
and renovation of lab space.  ({\color{red} TODO: get detailed list
  from Tracy, e.g. 122 renovations, new computing room, equipment
  grants, computing lab, etc}.)\\[3pt]

\noindent
{\bf Collaboration with other programs: } Physics (like Mathematics)
is a foundational discipline for most Science, Technology,
Engineering, and Math (STEM) disciplines.  Nearly every STEM
discipline trains its students in our introductory physics courses.
Physics itself is a rather unique discipline, and we have little
competition from other fields to consider.  Instead, we focus on
collaboration, which is much more useful.  We have a wide range of
applied physics majors which afford students the opportunities to
study complementary specialized topics (e.g. computer science,
atmospheric science) outside of the department.  The UCU strengthens
those majors by affording more flexibility with a more expansive list
of specialized courses to choose from.  Our new coursework in
computational physics (particularly PHY 40: Introduction to
Computational Physics, a course with no college-level prerequisites)
focuses on the bare minimum of programming techniques necessary to
tackle challenging physics problems (See Section~\ref{sec:snws}).  It
moves quickly from ``introductory programming'' to
``computational physics''.  This makes the course highly complementary
to coursework in the Computer Science and Engineering department, and the markedly different 
emphasis may be of interest to other disciplines as well.\\[3pt]

\noindent
{\bf Discussion Checklist:} (a) Discussed above [B6].  (b) According
to table [B7] only about $70\%$ of students enrolled in upper division
physics courses arephysics majors.  We suspect this is innaccurate,
most likely double majors are not being handled correctly.  But we
would certainly be delighted if 30\% of our upper-division students
were from other majors! (c) There is an expected increase [B8] in the
number of TAs that resulted from new courses, particularly PHY 80, PHY
40 and PHY 45, which require significant TA support for lab sections,
as well as the addition of discussion sections to some core upper
division courses.  (d-e) Discussed above [B9-10]. (f) Discussed above
(Instructional Space, Equipment, and Facilities) (g) Discussed above
[B12] (h) Discussed above (introductory remarks) (i) Discussed above
(Staff Support) (j) Staff advising is discussed in
Section~\ref{sec:perceptions} (Academic Advising).  (k) Discussed
above (Instructional Space, Equipment, and Facilities) (l) Discussed
above (introductory remarks) (m) Discussed above (Collaboration with
other programs) (n) Discussed above (introductory remarks).

\newpage
\section{Students in the major}
\label{sec:students}

{\bf Questions: This section is intended to characterize the students in this major. How have 
enrollments in the major varied over the period of the review, in terms of both the numbers 
and quality of the students? Are students succeeding in the major both in terms of qualitative 
and quantitative academic standards? Are students graduating on time? Are there impacted 
classes (e.g., with limited offerings or long wait-lists) or other bottlenecks that 
unnecessarily impede student success? How do students find out about the major?  Is the major 
reaching a wide and diverse spectrum of students? Are students who enter the major retained in 
the major, and if not, why not?}\\[3pt]

%Refer to the Tableau data concerning enrollments in the major and the other majors reviewed 
in the same cluster (Appendix B, Tables 13-23). Based on those data and any additional 
information you wish to include, comment on each of the following for your major over the 
review period, referring, when appropriate to differences between your major and others in the 
cluster:
%\begin{enumerate}[a)]
%    \item Table 13.  Number of Students - Duplicated Count and Percent Change 
%    \item Table 14.  Students in Multiple Majors - Percent of Total in Major 
%    \item Table 15.  Gender of Students – Percent of Total in Major and Percent Change 
%    \item Table 16.  Under-represented Students – Percent of Total in Major and Percent 
Change 
%    \item Table 17.  New Freshman Students Number and Percent Change 
%    \item Table 18.  New Transfer Students Number and Percent Change 
%    \item Table 19.  Average Cumulative UC Davis GPA 
%    \item Table 20.  Students in Good Standing – Percent of Total by Level 
%    \item Table 21.  Degrees Conferred – Duplicated Count and Percent Change 
%    \item Table 22.  Time to Degree for Freshman and Transfer Students – All Students 
%    \item Table 23.  Time to Degree for Freshman and Transfer Students – In Same Major 
%    \item Table 24.  Graduating Students with a Minor
%    \item Table 25.  Number of International Students
%    \item In light of the information presented in Tables 13-23, describe and evaluate the 
effectiveness of any efforts by the program’s faculty and staff to retain students in the 
major.
%\end{enumerate}
%Please also address the following issue, for which no data are provided:
%\begin{enumerate}[o)]
%    \item Describe and evaluate how students find information about the major (websites, 
course catalog, etc.).
%\end{enumerate}

\noindent

As described in detail below, during this review period, the
undergraduate program has grown in both size and diversity.  We
believe there is strong demand by students for excellent physics
programs, and by employers for those we train.  With sufficient
resources, our department could grow further to meet that demand.  We
hope that we continue to grow in Diversity as well.  By most of the
measures of student academic achievement which we consider below,
physics students are performing just slightly below the average for
L\&S, which is a notable accomplishment given that physics courses are
graded significantly more stringently than L\&S on average.  The most
concerning metrics are a dramatic drop in transfer student admissions
to eight total in AY 202-2023, and a four-year time-to-graduation for
$73\%$ of physics majors, which is significantly lower than for L\&S
overall (85\%).  The UCU includes a number of changes to our program,
including the removal of bottlenecks, that we expect to impact these
and other metrics.  However, it is too soon to judge their efficacy.
Looking at the available metrics all together, we see that our program
is successful at attracting, retaining, and graduating students.

The primary source of information about our major is the department
website, the UC Davis course catalog, and the undergraduate academic
advisor.  We do considerable outreach with our students to keep them
on track.  This is all discussed in more detail in
Section~\ref{sec:perceptions} under ``Academic advising''.\\[3pt]

\noindent
{\bf Size of the undergraduate program:} The number of physics and
applied majors has been growing, as the result of a mutual agreement
between the university and department to begin admitting about $50\%$
more students into the program starting in AY 2019-2020.  This amount
was targeted as the maximum number of additional students that physics
could handle without the need for additional lecture sections in core
physics courses.  This increase is the main trend evident in [B13] and
[B17]. Taking applied physics and physics together, the number of
international students in physics [B25] has increased by roughly
$25\%$, consistent with the increase in L\&S.  To accommodate the
increase in the number of students, we added TA-led discussion
sections to some of the upper-division physics course courses, which
is something physics majors had been requesting even before the class
size increase.  These tables also put the ratio of applied physics to
physics majors historically at 1:2, but of late the share of applied
physics majors appears to be growing.\\[3pt]

\noindent
{\bf Diversity of physics majors:}
Combining physics and applied physics majors, the fraction [B15] who
are women has increased just slightly during the review period, while
the fraction of women in L\&S overall has decreased by about $3\%$.
Again combining physics and applied physics majors, the fraction [B16]
who are from underrepresented groups (URGs) has increased from $14\%$
to $19\%$.  When the number of majors is accounted for, the increase
in the number of women is not statistically significant, but the
increase in the number of students from URGs is statistically
significant.\\[3pt]

\noindent
{\bf Double majors and minors:} There has been a noticeable increase
in the number of physics majors who are double majors [B14] but the
rate is still lower than the average for L\&S.  This lower rate is the
combination of unit caps on the amount of total coursework a student
can take (imposed by L\&S) coupled to the hierarchical structure of
most science disciplines.  In our experience, double majors in physics
have little flexibility left in their schedule to pursue other
interests and electives, so we generally do not advise them.  We do
heartily encourage minors, but yet few physics majors complete one
[B24].\\[3pt]

\noindent
{\bf Transfer students and their experience:}
As discussed in Section~\ref{sec:previous}, the previous review was
concerned that a large number of incoming transfer students were not
prepared for a physics major.  Following the advice of the review
committee, the department started selective review of transfer
students in time for AY 2021-2022.  Our hope is that by selecting
students that are more likely to succeed as physics majors, we will
continue to graduate transfer students as physics majors at
approximately the same rate, while dramatically reducing the number of
students who drop out of the program.  A major focus of the UCU was
improving the experience of transfer students, by leveling out their
coursework in the first year to avoid a ``brick wall'' of intense
coursework that previous students faced in their first quarter.  It is
too early the judge the effectiveness of any of these changes.

Omitting AY 2020-2021 from consideration due to the pandemic, we see
that since AY 2021-2022 there has been a significant drop [B18] in the
number of new transfer students, the result of a drop in applications
and the tighter admission requirements of selective major review.
Prior to selective major review, in a typical year we would have
120-130 applicants.  Of those accepted, typically 30-35 students would
start as physcics majors, but typically only 10-12 would remain as
physics majors past the first year.  In recent years, applications
have fallen to as low as 90 students, and typically about 75\% of
student pass selective major review.  In AY 2022-2023, we had about
eight new transfer students [B18] but in AY 2023-2024 (not in the
appendix data set), we had 13 new transfer students.  Assuming that
selective major review allows us to retain those students, we should
not reduce the number of transfer students that graduate as physics
majors.

A significant number of transfer students in applied physics and
physics require an additional year to graduate: only $22\%$ of applied
physics majors and $41\%$ of physics majors graduate in two years as
intended [B22].  This is significantly lower than $64\%$ for L\&S overall.
Improving this experience for transfer students was a major focus of
the UCU.

It is too soon to draw any conclusions about the changes to our
program that we have made which impact transfer students.  However, we
have made a request for reports of additional metrics pertaining to
transfer students.  We plan to scrutinize these statistics to judge
the impact and efficacy of these changes to our program.  We are also
considering outreach to community colleges to boost the application
rate.  These considerations fall within the scope of our departments
DEI efforts as well, as many students from URGs access the university
through their community colleges.  It is vital that we maintain a
viable path for transfer students through our program.\\[3pt]

\noindent
{\bf Student Academic Performance:}
The average GPA [B19] of physics and applied physics majors is around
3.2 but varies as lows as 2.8 and as high as 3.3 depending on the
class and major.  This is slightly lower than the average for L\&S
(3.3).  This likely reflects the fact that physics majors take more
physics courses, which we have already established are generally
graded more stringently than courses from other departments in L\&S.
In [B20] we see qualitatively similar results with respect to the
fraction of students in good standing, which ranges from $79\%$ to
$94\%$ depending on the class and major, and is slightly lower than
L\&S overall.  We see a slight increase in degrees conferred [B21], for both physics
and applied physics majors.  When the data for AY 2023-2024 are
available, we should start to see an increase due to the larger number
of physics majors.

Looking at time to graduation [B23] for students that remain physics
majors the entire time, we see that $83.3\%$ of applied physics majors
and $72.7\%$ of physics majors graduate within four years as planned.
That rate for physics majors is significantly smaller than the average
for L\&S (84.5\%).  This is integrated from 2016 to 2022, which makes
it somewhat difficult to interpret.  We should monitor this metric
over the next few years to see if there is a persistent problem here
to address.  We do expect that time-to-graduation for nominal
four-year students should decrease as a result of the changes in the
UCU, which includes changes that remove stalling out of four-year
students in their sophomore year.  But it is too soon to judge the
efficacy of these changes.\\[3pt]

\noindent
{\bf Discussion Checklist:}
(a-m) Discussed above [B13-25] (n-o) Discussed above (introductory remarks).

\newpage
\section{Student perceptions of the major}
\label{sec:perceptions}

{\bf Question: What are current students' and recent graduates' opinions of the major? }\\

Our students' opinion of the physics and applied physics majors are
for the most part consistent with college averages, within the limited
statistics available from survey responses.  In the discussion that follows,
we note some areas of encouragement and concern, and include anecdotal
data where pertinent.\\[3pt]

%\noindent
%Refer to the Tableau data obtained from surveys of current students and alumni concerning 
%their perceptions of the quality of the major and the other majors reviewed in the same 
%cluster (Appendix C, Figures 1-53). Based on those data and any additional information you 
%wish to include (e.g., results of departmentally administered course evaluations), comment on 
%each of the following for your major over the review period, referring, when appropriate to 
%differences between your major and others in the cluster:
%\begin{enumerate}[a)]
%    \item overall understanding of the major (Figures 1-4)
%    \item overall satisfaction with the major (Figures 5-22)
%    \item satisfaction with instruction in the major (Figures 23-36)
%    \item satisfaction with academic advising in the major (Figures 37-43)
%    \item satisfaction with courses offered in the major (Figures 44-53)
%\end{enumerate}

\noindent
{\bf Interpretation of student feedback:} Soliciting feedback from
students is a useful and important diagnostic tool, but the results
must be carefully interpreted.  We know that student evaluations
reflect significant implicit biases\footnote{See,
for example: Kreitzer, Rebecca J. \& Sweet-Cushman, Jennie. Evaluating Student Evaluations of 
Teaching: a Review of
Measurement and Equity Bias in SETs and Recommendations for Ethical
Reform. Journal of Academic Ethics 20 (1):73-84 (2021).}  and are strongly
correlated with student perceptions of their course grades.  For
example (referring [B12] and [C29]) we are not surprised to see that a department which is an 
outlier with respect to grading ($85\%$ of
letter grades are A's) is also an outlier with respect to student satisfaction with faculty 
instruction ($88\%$).  In the physics department, $33\%$ of letter
grades are A's and student satisfaction (amongst junior and senior physics majors) with 
faculty
instruction is a more modest $67\%$.  We should not conclude that the other department is 
performing any better (or worse, for that matter)
than the physics department, only that our approaches are different,
likely reflecting the different needs of our students.\\[3pt]

\noindent
{\bf Limited Statistics from Alumni Responses:}
We also need to be careful not to draw sharp conclusions from samples
with limited statistics.  There were  few responses collected from
physics alumni: three from applied physics majors and 17 from physics
majors.  We have carefully checked all of the alumni responses
[C12-22], [C31-36], [C41-43], and [C49-54].  Amongst the questions,
the only ones which yielded a statistically significant deviation from
the college average were:
\begin{itemize}
  \item Questions [C18], [C21] and [C22] none of which we found to be illuminating.
  \item Questions [C31] and [C32] where we do see (borderline) statistically significant 
dissatisfaction with the grading system and use of information technology.   
  \item Question [C43] which references peer advisors, which we do not have in physics.
\end{itemize}

\begin{table}[htbp]
\caption{\label{tbl:appc} Survey results from Appendix C, using results from freshman through 
seniors, combining results from physics and applied physics majors. The sample size is N=61 
from which a binomial statistical uncertainty has been calculated.  Results as a percent are 
organized by question (Q) and the results for physics and applied physics majors (Physics) are 
compared to the College (L\&S).}
\begin{center}
\begin{tabular}{|lll|lll|lll|lll|}
\hline
Q & Physics & L\&S & Q & Physics & L\&S & Q & Physics & L\&S & Q & Physics & L\&S \\
\hline
% Binomial Uncertainty:
C1 & 90 $\pm$ 4 & 91 & C5  & 55 $\pm$ 6 & 64 & C23 & 53 $\pm$ 6 & 76 & C38 & 55 $\pm$ 6 & 52 \\
C2 & 91 $\pm$ 4 & 90 & C6  & 71 $\pm$ 6 & 74 & C24 & 84 $\pm$ 5 & 68 & C39 & 56 $\pm$ 6 & 47 \\
C3 & 79 $\pm$ 5 & 85 & C7  & 57 $\pm$ 6 & 60 & C25 & 84 $\pm$ 5 & 73 &     &            &    \\                  
C4 & 89 $\pm$ 4 & 93 & C8  & 78 $\pm$ 5 & 79 & C26 & 82 $\pm$ 5 & 76 & C44 & 46 $\pm$ 6 & 37 \\
~  &            &    & C9  & 30 $\pm$ 6 & 46 & C27 & 48 $\pm$ 6 & 61 & C45 & 52 $\pm$ 6 & 56 \\
~  &            &    & C10 & 68 $\pm$ 6 & 69 & C28 & 39 $\pm$ 6 & 49 & C46 & 41 $\pm$ 6 & 54 \\
~  &            &    & C11 & 69 $\pm$ 6 & 69 & C29 & 52 $\pm$ 6 & 66 & C47 & 55 $\pm$ 6 & 59 \\
~  &            &    &     &            &    & C30 & 75 $\pm$ 5 & 61 &     &            &    \\
\hline 
\end{tabular}
\end{center}
\end{table}

\noindent
{\bf Maximizing Statistical Uncertainty from Student Responses:}
To maximize the sample size from current students, we have included
all years in the survey result (the default results include only
juniors and seniors).  We have also computed a weighted average of the
responses from both applied physics and physics majors.  This combined
sample has a size of N=61 (with a slight variation from question to
question that has been neglected from our analysis) from which we
estimated a naive binomial statistical uncertainty.  The results are
reported in Table~\ref{tbl:appc}.\\[3pt]

\noindent
{\bf Understanding of the major:} Using the combined survey results in
Table~\ref{tbl:appc} we see that understanding of the major [C1],
program requirements [C2], program policy [C3], and accuracy of the
catalog [C4] were all comparable to the L\&S average, with the last
two being low by about one standard deviation.  Even though there is
therefore no statistically significant evidence to support the claim,
we do suspect there may be some student confusion in the context of
the UCU.  For example, we have begun teaching newly approved courses,
even though they are not yet required by the major, as the UCU has not
yet reached final approval.  We have done our best to communicate the
situation to our students, and we are encouraging them to take the new
courses even though they are not required yet.  We would not be
surprised if some confusion exists in this context.  We plan to
complete the approval process and make the necessary catalog updates
as soon as possible, at which point we are confident that any
confusion will begin to dissipate.\\[3pt]

\noindent
{\bf Satisfaction with major:} Using the combined survey results in
Table~\ref{tbl:appc}, we see that the responses from our majors regarding
fair treatment [C6], faculty access [C7], ability to get into major
[C8], library resources [C10], and overall satisfaction [C11] were all
consistent with college averages within statistical uncertainty of the
survey sample.  As discussed above, there are no statistically
significant results from the alumni survey that warrant further discussion.

Our majors were less satisfied ($30\% \pm 6\%$) than the college
average ($46\%$) with respect to [C9] ``Education enrichment
programs''.  In the current program, it is nearly impossible for our
majors to study abroad, which likely contributes to the observed
student dissatisfaction.  In the UCU, four-year students have
sufficient flexibility that more students may find it feasible to
study abroad.  The department does provide other sources of education
enrichment, such as student research, and it would be illuminating to
probe student satisfaction with specific avenues of enrichment.

Our majors were less satisfied ($55\%\pm6\%$) than the college
average ($64\%$) with respect to [C5] the ``faculty being open to
discuss students' needs, concerns, and suggestions''.  Even though
this is just barely more than one standard deviation from the college
average, it is disappointing to learn that so many students feel this
way.  This is something we should consider as a faculty.\\[3pt]

\noindent
{\bf Satisfaction with Instruction:} Using the combined survey results
in Table~\ref{tbl:appc} we see that our majors believe plagiarism is
not being adequately explained [C23].  We encounter plagiarism most
frequently in physics courses through the use of online websites, such
as Chegg, to provide answers to homework questions.  There is growing
evidence that using sites such as Chegg for cheating expanded during
the pandemic and hasn't receded.  We wonder if the low score here is
reflecting student and faculty disagreement as to what constitutes
plagiarism, or, conversely, if it indicates students' concerns that we
are not doing enough to discourage and condemn this relatively new
form of cheating.  Clearly more discussion between students and
faculty is needed about cheating.

The results of [C24-26] show that students believe they are well
practiced in recalling, explaining, and analyzing concepts, methods,
and ideas.  However, when it comes to qualitative judgment [C27] and
creativity [C28] physics majors reported lower than average practice,
which we discuss further below.

Our majors are quite well satisfied with their TAs ($75\% \pm 5\%$)
compared to the college average ($61\%$).  This is heartening, if not
surprising.  TA's are most often cast in roles which directly support
students, helping them work through their homework problems in
discussion sessions, for example.

Satisfaction with faculty is lower ($52\% \pm 6\%$) than the college
average ($66\%$).  This is concerning and warrants a closer look at the
data.  If we consider physics majors only, and restrict ourselves to
junior and seniors. we we see 67\% are satisfied with faculty
instruction, just a bit higher than the college average.  Further
investigation confirms that the heightened dissatisfaction comes from
two contributions: current freshman and sophomores, and all applied
physics majors.

One major difference between applied physics and physics majors is the
amount of coursework that is taken outside of the department. We
are well aware of the frustration that applied physics majors are
facing in completing their coursework outside of physics.  Perhaps
part of the dissatisfaction stems from this.  The UCU aims to give
applied physics majors more flexibility which we hope will improve
this situation.  In any event, it is clear that additional outreach
with applied physics majors and physics majors in lower division
courses is needed to determine the root causes of this elevated
dissatisfaction.\\[3pt]

\noindent
{\bf Academic Advising:} The satisfaction of alumni with advising was
all consistent with the college average, within the statistical
uncertainty of the sample.  The only exception was a question
regarding peer advising, which the physics department has just
recently begun providing.  Amongst our majors, using the combined
sample of Table~\ref{tbl:appc}, we see that students are satisfied
with the quality of academic advising (55\% $\pm$ 6\%) slightly higher
than the college average (52\%) and with access to academic advising
(56\% $\pm$ 6\% ) higher than the college average (47\%).  Only the
latter is (borderline) statistically significant.

These results are encouraging.  As discussed in
Section~\ref{sec:previous}, low satisfaction with advising was noted
by the previous review.  We believe we understand the source of the
improvement, based on our own anecdotal data from individual student
interactions.  Most importantly, Prof. Boeshaar runs a physics career
seminar and Prof. Knox runs an alumni speaker seminar.  These are not
required courses, but the scope of discussion typically extends to
academic advising, particularly for subjects that students are
concerned about.  Secondly, the staff undergraduate advisor, Amy Folz
and the vice-chair for the undergraduate program, Prof. Mulhearn, have
devised a highly effective triage system, whereby typical issues are
handled by Folz, and more complicated problems are immediately kicked
up to Prof. Mulhearn.  We have found this arrangement to be quite
effective.  Folz also arranges multiple annual outreach events (pizza
nights) which provide a forum for academic advising.  Prof. Mulhearn
is an annual speaker in the physics career seminar.  Lastly, a
significant number of students participate in undergraduate research,
and research advisors are a natural source of academic advice as well.
Our attempts to impose a more top-down assigned faculty advisor had
lackluster results, with a clear lack of student and faculty interest,
which was compounded by the pandemic.  Fortunately, it seems that our
more organic, fully optional avenues for academic advising are proving
to be effective.  Prof. Boeshaar is continuing to play her pivotal role past
her recent retirement, but we cannot expect this to continue
indefinitely.\\[3pt]

\noindent
{\bf Course Offerings:} Using the combined data sample from
Table~\ref{tbl:appc}, we note that our students rated access to small
class sizes [C44] higher (46\% $\pm$ 6\%) than the college average
($37\%$).  Our students rated the availability of courses needed to
graduate [C46] at lower (41\% $\pm$ 6\%) than the college average
(54\%), and this lower level of satisfaction is present for both
physics and applied physics majors.  Anecdotally, we are aware that
applied physics majors struggle to schedule outside coursework.
We are addressing this problem in the UCU by increasing the number of
choices available to applied physics majors for their coursework
outside of the department.  We are also discussing with the computer
science department the possibility to give higher priority to our
applied physics (computational physics) majors in their impacted
courses.  We are also aware that physics majors struggle to schedule
advanced physics labs, and we are attempting to improve throughput in
these courses as well.  We offered a fall section of advanced physics
lab for the first time in AY 2023-2024.  We have also begun
renovations to provide more space, and are adding additional lab
modules, toward increasing the number of students in each section.
Our students rated the availability of general education courses [C45]
and the variety or courses [C47] at levels consistent with the college
average within the statistical uncertainty of the sample.  The alumni
responses were all consistent with the college average within the
statistical uncertainty of the sample.\\[3pt]

\noindent
{\bf Communication, Creativity, and Qualitative Judgment:} Oral and
written communication was mentioned as a concern in the previous
program review.  Our students gain experience in scientific writing by
writing lab reports in their experimental physics courses.  One area
where we could probably add more practice with writing is in student
homework.  We generally grade homework looking for just enough steps
to show the student understood the problem.  We could easily assign
students longer write-ups for specific problems, in which they would
need to explain what they are doing in a clear and concise manner.
If done correctly, this might actually make homework easier to grade!

When it comes to oral communication, this immediately runs into
several practical challenges.  With a throughput of order 60 students
per year, even having each student provide a 15 minute presentation
would be a commitment of 15 hours: 50\% of a standard lecture course.
That's a big tine commitment, and one we cannot afford in our core
required course work.  However, many instructors of upper division
specialty courses include oral presentations as part of their
curriculum.  We also note that students are free to choose about 50\%
of their undergraduate coursework, so they could take ``CMN 001:
Introduction to Public Speaking'' as just one of many possible avenues
for gaining additional practice in oral communication.

We saw above that when it comes to qualitative judgment [C27] and
creativity [C28] physics majors reported lower than average practice.
This is disappointing, if not entirely surprising.  A large
fraction of the undergraduate physics degree is invested toward
understanding well-established theories which more mostly developed a
century ago or more.  Problems for these theories which have exact
analytic solutions are limited and so students invest a lot of time
solving problems that have been being solved by students for a century
or more as well.  This paints perhaps too bleak a picture, as
instructors do labor to breathe life into their subjects every quarter,
but they are facing strong headwinds, and these results most likely
reflect those headwinds.  One area of the undergraduate physics degree
where creativity and qualitative judgment are more naturally exercised
is in computational physics and experimental physics.  Here students
devise and debug their own code or experimental apparatus, face new
problems not constrained to those with exact analytic solutions.  It
will be interesting to see if the improvements and increased focus on
these subjects in the curriculum update will lead to students getting
more practice in these topics.\\[3pt]

\noindent
{\bf Discussion Checklist:} (a-e) Discussed above [C1-52].

\newpage
\section{Post-graduate Preparation}
{\bf Questions: How well does the major prepare students for postgraduate education and careers? 
Do the students have opportunities to meet and work with faculty outside the classroom 
setting? Is there sufficient support for internships or experiential learning opportunities? 
Are there ample opportunities for students to learn about career options?}
  
%%Refer to the Tableau data obtained from surveys of current students and alumni concerning 
%preparation by the major for postgraduate education and careers (Appendix C, Figures 54-80). 
%Based on those data and any additional information you wish to include, comment on each of the 
%following for your major over the review period, referring, when appropriate to differences 
%between your major and others in the cluster:

%    a) quantity and quality of research and creative activities provided by the major (Figures 54-59)
%    b) quality of preparation by the major for postgraduate education (Figures 60-64)
%    c) quality of preparation by the major for the workforce (Figures 65-74)
%    d) the degree to which students have sufficient contact with faculty to help them in 
%their postgraduate education and careers (Figures 75-80).

\begin{table}[htbp]
\caption{\label{tbl:appcii} Survey results from Appendix C, using results from only junior and seniors, combining results from physics and applied physics majors. The sample size is N=37
from which a binomial statistical uncertainty has been calculated.  Results as a percent are 
organized by question (Q) and the results for physics and applied physics majors (Physics) are 
compared to the College (L\&S).}
\begin{center}
\begin{tabular}{|lll|}
\hline
Q & Physics & L\&S \\
\hline
% Binomial Uncertainty:
C54 & 60 $\pm$ 8 & 46 \\ 
C60 & 90 $\pm$ 5 & 74 \\
C75 & 49 $\pm$ 8 & 42 \\ 
C76 & 58 $\pm$ 8 & 65 \\ 
C77 & 62 $\pm$ 8 & 62 \\ 
\hline 
\end{tabular}
\end{center}
\end{table}

\noindent
Our students' rating of their opportunities for undergraduate research [C54] and their propensity for further academic study [C60] are both significantly higher than the college average.  The difference is large enough as to be statistically significant even considering the limiting sample size of the survey.
They rated faculty access (in various ways) [C75-77] as comparable to college averages, within the statistical uncertainty of the survey sample.\\[3pt]

\noindent
{\bf Limited Statistics of Survey Data:}  The limited statistics in the survey data from physics alumni renders [C55-59],[C61-74],[C78-80] of limited use and they are not considered here.  Following the procedure of the previous section, we have computer the weighted average of responses from applied physics and physics majors in Table~\ref{tbl:appcii}.  Because these questions pertain to issues that arise later in a typical students career (such as undergraduate research) we have included only juniors and seniors, which limits the sample size to $N=37$.\\[3pt]

\noindent
{\bf Preparation for Workforce:}  Absent statistically significant results from the alumni surveys, we can consider our own anecdotal data.  Many faculty maintain contact with their former students long after they graduate.  This keeps us well informed about issues related to workforce preparation.  The networks of former students that faculty develop are an extremely valuable tool for helping newly graduated students enter the workforce\footnote{In some cases, we have former students interviewing former students!}.  A recurring theme is that the advanced labs (122, 117, 118, and 157) were the most valuable learning experience the student encountered.  We are optimistic that our investment in computational physics will yield similar feedback in the future.

\noindent
{\bf Discussion Checklist:} (a) Discussed above [C54], (b) Discussed above [C60], (c) Discussed above (Preparation for Workforce), (d) Discussed above [C75-77].

\newpage
\section{Educational Objectives and Assessment}

{\bf Question: How does the program monitor and evaluate its success in achieving its Program  
Learning Outcomes (Section 1)?}\\[3pt]

%Specifically:
%    a) Please confirm that the PLOs are clearly listed in an easily accessible location on the program website and provide the URL for that website.
%    b) Please provide a program curriculum matrix or map which identifies the required  courses in which each PLO is specifically addressed and assessed. Attach sample syllabi for up  to six of the required core courses for the major. An additional resource for completing the  curriculum Map/Matrix is available on the UIPRC website.
%    c) What unit (committee or officer) in your program is responsible for collecting and  analyzing data on student progress toward PLOs?
%    d) Please list the direct evidence of student learning used to assess student achievement  of PLOs. Examples of direct evidence include projects, scores from exams or quizzes, and  written work. For this data, describe the methodology for sample selection and size.  
%    e) Please list sources of indirect data used to contextualize student achievement of  PLOs. Examples of indirect data include student evaluations, peer evaluation of teaching, and  the survey data from current students and graduates provided for this review.
%    f) In what ways do the results of this self-review reveal particular areas of strength or  weakness in student progress toward achieving PLOs?
%    g) What changes will the program make based on the results of the program assessment of  student learning?

%Programs are encouraged to contact the Academic Assessment team
%(assessment@ucdavis.edu) in the Center for Educational Effectiveness
%(Office of Undergraduate Education) to learn more about assessment of
%student learning outcomes.

\noindent
The PLOs for our majors are defined in Section~\ref{sec:overview}
along with specific courses which reinforce those objectives.  We have
not historically hosted these objectives publicly, but as this has
been requested as part of the review we are planning to do
so. {\color{red} Let's get these hosted PLOs hosted somewhere ASAP.}

Responsibility for monitoring the effectiveness of our program at
enabling our student to reach these objectives lies with the
vice-chair for the undergraduate program and the undergraduate
curriculum, in collaboration with the physics faculty.  As part of the
curriculum update, we have developed a new procedure for continuously
monitoring, discussing, and revising the curriculum.  Each spring, the
undergraduate curriculum committee organizes a faculty meeting devoted
entirely to the undergraduate program.  The instructors of core
undergraduate courses prepare short summaries of their courses,
covering how they taught the course, including notable differences
from example syllabi which we now maintain, and any observed
shortcomings in student preparation from prerequisite courses.  If any
changes to the example syllabi are needed, the UGCC prepares new
versions, and the faculty votes on whether or not to adopt the new
versions.

The acquisition of PLOs related to mathematics and theoretical physics
are most directly assessed through in-person final exams in core
courses (e.g. 104A, 105AB, 110AB, 112, 115AB).  The faculty members
that teach core courses are generally able to immediately spot
short-comings in student progress from the final exam performance.
The experimental physics PLOs are assessed mainly through student lab
reports.  Instructors of PHY 80 have also recently introduced a lab
practical examination which we have found to be quite successful.

Professors engaged in teaching our new computational physics courses
are learning new pedagogy appropriate to that discipline.  We have
become quite found of Parsons problems, which have students provide
the correct ordering of (scrambled) lines of code, in order to meet a
given prompt.  Not only are these a fast and practical means for
evaluating programming abilities in person, they also train students
to ``compile in their head'' instead of relying (as many
new students do) on trial and error.  We also evaluate submitted
programs, but we are concerned, in the long run, that students might
get by on simply copying existing solutions, unless we have an
effective means of in-person evaluation.

This self-review is synergistic with the years of prior effort that
our department has invested in understanding the performance of our
program, and proposing improvements.  The strengths and weaknesses we
have identified are summarized in Section~\ref{sec:snws} and our
future plans in Section~\ref{sec:future}.  However, we heartily
encourage the reviewers to read through the UCU proposal document.  It
is rather long, but it represents years of effort on our part.\\[3pt]

\noindent
{\bf Discussion Checklist:} (a-g) Discussed above.

\newpage
\section{Major strengths and weaknesses}
\label{sec:snws}

{\bf Summarize the major overall strengths of the program as well as any current problems that 
you perceive.}\\[3pt]

\noindent
All happy departments are alike; each unhappy department is unhappy in
its own way.  Fortunately, ours is a happy department, and so our
strengths are those fairly universal ones.  We have a collegial
faculty that works well together, allowing each faculty member to
effectively add their unique talents and interests.  Our staff is
dedicated and effective.  Our students are diligent and inquisitive,
and they support one another.  They are remarkably empathetic.
Together we have built an impressive and productive departmental
culture.

Our small class sizes and an active undergraduate research program
promote strong ties between faculty and students.  Ours students
receive a considerable amount of personal attention both in and out of
class.  We have an exceptionally varied offering of advanced
undergraduate specialty courses, the work of a dedicated faculty
directed at its students.\\[3pt]

\noindent
{\bf Computational Physics:} Our curriculum now include a substantial
number of required computational physics courses.  In PHY 40
(Introduction to Computational Physics) students learn the basics of
Python and then move directly into solving numerical and physical
problems using computational methods.  In PHY 45 (Computational
Physics) they are introduced to C++, without much concern for advanced
language features, and tackle more advanced physics problems based on
their coursework in introductory physics.  Later, they take 2-3
one-unit computational lab courses, designed to work alongside their
upper division coursework.  Much of the earlier computational physics
problems focused on mechanics; now they use computational physics to
tackle problems from upper division quantum mechanics, electricity and
magnetism, and statistical mechanics. These courses are just the core.
As the faculty gains confidence in our students computational
abilities, we should see computational physics spread throughout our
program.  We cannot make every physics major a computer scientists any
more than we can make every one a mathematician, but we can make every
physics major as effective at solving physics problems with computer
programs as they are with calculus.\\[3pt]

\noindent
{\bf Experimental Physics:} Following the advice of the previous
committee, we have introduced PHY 80: Experimental Techniques to better
prepare students for the PHY 122A: Advanced Lab in Condensed Matter
Physics and PHY 122B: Advanced Lab in Particle Physics.  We developed
a course that introduces basic laboratory equipment, analog
electronics (up to the diode), introductory statistics, experimental 
uncertainty, and analysis and presentation of experimental data.  After 
learning these techniques, primarily through lab activities, students
apply them to the measurement of fundamental constants like the speed 
of light and Planck's constant.  

As part of the UCU, we have made PHY 80 a required course for every physics major (not just those taking 122A/B).  We have adjusted the other upper division lab courses to take advantage of it:the 116ABC sequence (Electronic Instrumentation) has been replacedwith PHY 80, PHY 117 (Physics Instrumentation with Analog and DigitalElectronics) and PHY 118 (Physics Instrumentation for Data Acquisition).The latter two courses can be taken in any order, and are in a nearly continuous state of renewal by Prof. Prebys, in order to take advantage of new technology (such as systems thathost programmable digital logic and a CPU on a single chip) well suited forstudents.

Unlike the the setups for the lab exercises in PHY 80, 117, and 118, which are easily duplicated, each experiment in PHY 122A, 122B, and 157 generally uses a unique custom designed apparatus.  The complicated nature of these experiments also require a lot of personal attention from the professor teaching the course.  For all of these reasons, the throughput in these advanced lab courses is limited, and requires precise rationing by the undergraduate advisors to ensure students can finish the course in time to graduate.  We are making several changes in an attempt to increase throughput.  We taught our first fall section of 122A in 2023, which was feasible as a result of scheduling and prerequisites changes which are part of the UCU.  If we add a third 122B section, these changes alone will increase throughput by 50\%.  
However, finding professors that are qualified and willing to teach these extremely demanding courses is a challenge.  We have also begun renovations of the first floor of physics, to provide space for additional experimental setups, which should allow us to increase the number of students in each section.  However, we must take care not to reduce the amount of personal attention students receive in this lab, or make it even more demanding for the instructors.  

The throughput in PHY 157 is even more limited, due to other practical considerations such as the time of year during which weather is favorable for observation, safety concerns which limit the number of people
that can be on the roof for observation, and limited faculty qualified to teach the course.  With the retirement of Prof. Tony Tyson, Prof. Tucker Jones has been teaching the course, but we have struggled to maintain throughput, let alone increase it.  {\color{red} Will the new telescope help with throughput, other ideas?}\\[3pt]

\noindent
{\bf Unit Caps:}  A typical course with three hours of lecture per week at UC Davis rates as three units\footnote{Most of our physics courses are actually rated as four unit courses, due to extensive problem solving in the homework.}.  Four-year students typically graduate with around 180 units, and students must receive special permission to register for any more classes once they reach 225 units.
{\bf Within L\&S, we are allowed to require a maximum of 110 units in our majors.} Prerequisite courses, even those outside our department, count toward this maximum.  This unit cap places severe limitation on what we can do with our program.  

The ostensible reason for the unit cap is to allow sufficient time for students to explore areas of interest outside of their major.  We are sensitive to that consideration, but find that this mixture is too lean (on major coursework) for our purposes.  Our students invest a great deal of time in prerequisite coursework, and by the time they are ready to take on advanced coursework in physics we run into the unit cap.  

The situation is even more dire for applied physics majors.  We must make substantial cuts in core physics material (such as second quarter in mechanics and quantum mechanics) simply to make space for required coursework outside of their major.  Offering a degree that combines two scientific disciplines, such as bio-physics or chemical physics, is effectively impossible.  Achieving a double major in two scientific disciplines is also nearly impossible due to the limit of 225 units on total coursework.

We propose a compromise:  lower-division prerequisite coursework {\bf outside the host department} should count at a pro-rated amount of 50\% toward the 110 unit cap.  This would ensure that our students have plenty of exposure to coursework outside of physics, while giving us some breathing room for advanced coursework.  It would enable us to develop applied physics majors that are much more useful to our students.  By limiting the concession to additional lower-division coursework, it should help preserve a feasible two-year path to graduation for transfer students.

\newpage
\section{Future Plans}
{\bf Describe current or proposed plans to strengthen educational
objectives of the program, such as increasing enrollments, improving
student performance, and increasing the contribution of the program to
the campus educational objectives. Comment on the long term strategy
or goals regarding hybrid and virtual instruction in this
program. Describe and justify if new resources are needed to preserve
or strengthen the program.}

Our most immediate future plans for the undergraduate program are
centered on receiving final approval and finishing the implementation
of the UCU.  Following that, we will want to monitor the performance
of the program carefully.  Of particular importance are the impact of
selective major review and the curriculum changes to transfer
students.  Ideally we will see that we continue to graduate at least
10-12 transfer students in physics each year, with hardly any leaving
the program.

We have an annual faculty meeting devoted to a review of the
undergraduate program, and evaluating the impact of the UCU will be a
central topic for the next few years at least.  We will likely want to
make small further adjustments.  This process has given us expertise
in the various tools for updating course descriptions and catalog
copy, and so we would like to invest some time in bringing those up to
date as well.  The scope of the undergraduate curriculum update did
not include introductory physics (PHY 7 and PHY 9) so we plan to
tackle those next.

We see our substantial investment in computational physics as merely a
start.  Our hope is that professors will take further advantage of
student expertise in computational physics as it becomes more reliable.
For when you calculate an analytic solution {\em and} write a computer
program for the same problem, and they agree, then you have
understanding and {\em confidence}.

The biggest challenge facing our department is the decrease in the
size of the faculty, because hiring is not keeping up with
retirements.  While the faculty is shrinking the demands on our
department are increasing.  We have more physics majors, but also more
undergraduate students taking our introductory physics courses.  We
need substantial investments from campus to dramatically increase our
hiring rate.  Here the main hold-up is startup.  And yet these
retirements are saving campus enormous sums of money.  If we direct
some fraction of those savings toward start-up expenses of new hires,
we could likely avert disaster for our department.  We also need more
investment in staff.  We have only one undergraduate student advisor
on the faculty who is severely taxed to keep up with not only our
majors, but the myriad of issues that arise with the {\em thousands}
of students that take introductory physics each year.

\newpage	
\section{Minors}
{\bf Please comment on the minors currently administered by your department.}\\[3pt]

% Include all minors, and for each:
%    a) Briefly explain the history of the minor.
%    b) Briefly explain the rational for the minor, including the importance of the minor to students.
%    c) Provide a brief overview of the minor, especially how students receive academic guidance through the minor.
%    d) Comment on numbers of students who have declared the minor during the review period.
%    e) Describe the faculty participation in, and staff administrative support for, the  minor.
%    f) Describe your approach to ensuring that course listings for the minor are regularly  updated and are realistic for students to complete in four years.
%    g) Describe faculty participation in the minor. e.g., classroom instruction, co- ordination of undergraduate events, supervision of undergraduate theses.
%    h) Describe your approach to assessing the effectiveness of the minor on students’  learning and preparation for the future.
%    i) Describe the current goals of the minor. Have goals changed over time? If so, explain  how the goals have changed.
%    j) What are the strengths of the minor and benefits to students of the minor?
%    k) What are the weaknesses of the minor? Describe any steps taken to address the  weaknesses.
%    l) What are faculty members’ future plans for the minor?

%Attachments:
%    m) If it was not supplied in Appendix A, please attach the current catalog copy for the  minor.
%    n) Please provide a chart indicating the number of students who have declared the minor  for each year in the review period.

%Enter your text here.

\noindent
The department offers a minor in physics which 1-3 students complete each year.  The objective for the
physics minor is to provide an opportunity for students to gain a
meaningful understanding of physics concepts and expertise in putting
them to practical use, but without committing to an entire degree.

The requirements for the minor are quite simple to state: students
must complete six upper division physics courses (excluding a small
set of ineligible courses, e.g. independent study).  However,
achieving this minor is quite challenging due to the hierarchical
nature of our coursework.  Upper division coursework generally
requires completing six courses in mathematics, four courses in
introductory physics, and many courses further require PHY 104A (Math
Methods in Physics).  This large number of prerequisites likely
discourages many students.  And those that have completed these
prerequisites for their own major are likely already science majors,
and therefore see little benefit to a physics minor.

We believe the physics minor is well designed for those relatively few
students dedicated enough to their interest to pursue.  We see no need
to make adjustments and we accept that demand for the minor is small.

\newpage
\section{Emergency Remote Instruction}
\label{sec:remote}

{\bf The review team, UIPRC, and UGC understand that the emergency remote
instructional environment required by the COVID-19 pandemic presented
extraordinary demands on department faculty, staff, and
students. Please address the successes and challenges faced by your
program during emergency remote instruction. Provide information about
any new practices that were beneficial and the program plans to
continue, as well as any practices and outcomes that fell short of
your standards.}\\[3pt]

%Some items you may want to address:
%    • Student experience: Student response to remote instruction; Student engagement with 
%courses and department; Student interactions with faculty and staff (Zoom appointments, online 
%office hours, etc.); Student retention of course information; Student progress toward degree
%    • Faculty experience: Impacts of teaching remotely; Costs and benefits
%    • TA experience: Interactions with faculty; Interactions with students
%    • Staff / advising experience: Costs and benefits
%    • Course equivalency: Degree of equivalency between normal and emergency remote 
%instruction with regards to course learning outcomes and variability by learning activity 
%(lecture courses, discussion courses, writing courses, laboratory courses, etc.)
%    • Assessment of student learning: Strategies used for remote assignments and exams and 
%their efficacy
%    • Lessons learned for future virtual or hybrid instruction or advising in the program

\noindent
Our faculty made a heroic effort during the COVID-19 pandemic, but the
quality of instruction suffered tremendously nonetheless.  In
qualitative terms, we saw that the top 1/3 of students managed
reasonable well with remote instruction, but the bottom fell out for
the remaining 2/3.  We believe we saw a dramatic increase in cheating,
and there is growing evidence that it has not receded.  We also
believe that we are still seeing residual effects of the pandemic on
current students.  Specifically, the level of mathematical preparation
and study habits have not fully recovered to pre-pandemic levels.
These are subjective evaluations, but they have been widely observed
by our faculty.

During emergency remote instruction, student engagement dropped
precipitously.  Early advice to instructors urged us to provide
asynchronous coursework as much as possible, to accommodate students
that might struggle, under the circumstances, to connect at a fixed
time.  We soon discovered, however, that fully asynchronous courses 
were a disaster for student engagement and morale.  Most instructors
quickly reverted back to fully live (via zoom) lectures, a mixture of
{\bf short} asynchronous pre-recorded lectures\footnote{Video hosting
  sites which monitor viewer engagement over time showed that students
  generally drop out of pre-recorded lectures after about ten minutes}
punctuated with live in-person discussions.

For faculty teaching lab courses, the burden was even greater.  These
are crucial experiences for our students, and faculty generally went
to extraordinary lengths to provide kits or other means for students
to gain some hands on experience.  This generally required redesigning
entire courses with little notice.

During live (via zoom) lectures, instructors were generally faced with
a wall of blankness: students nearly universally leave their cameras
off, and deeply resent being forced to turn them on.  With few
exceptions, faculty despised remote instruction.  But we did our best.

And despite all of these efforts, there was no meaningful equivalency
between the online courses we provided and our traditional in-person
courses.  Any instructor has seen that students that do not attend
lecture generally become demoralized, detached, and perform well below
their abilities.  COVID-19 forced something like that onto every single
one of our students.  May we never have to do that again.

\end{document}

