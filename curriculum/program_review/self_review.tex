\documentclass[12pt]{article}
\usepackage{lmodern}
\usepackage[T1]{fontenc}
\usepackage[dvips,letterpaper,margin=0.75in,bottom=0.75in]{geometry}
\usepackage{cancel}
\usepackage{graphicx}
\usepackage{braket}
\usepackage{latexsym,amssymb,amsmath}
\usepackage{pdfpages}
\usepackage{xcolor}
\usepackage{capt-of}
\usepackage{amsmath}
\usepackage{cite}
\usepackage[shortlabels]{enumitem}
\newcommand{\tcr}{\textcolor{red}}
\newcommand{\tcb}{\textcolor{blue}}


\usepackage[american,fulldiode]{circuitikz}
\tikzset{component/.style={draw,thick,circle,fill=white,minimum size =0.75cm,inner sep=0pt}}

\begin{document}
\ctikzset{bipoles/thickness=1}
\ctikzset{bipoles/length=.6cm}


%%%%%%%%%%%%%%%%%%%%%%%%%%%%%%%%%%%%%%%%%%%%%%%%%%%%%%%%%%%%%%%%
%%%%%%%%%%%%%%%%%%%%%%%%%%%%%%%%%%%%%%%%%%%%%%%%%%%%%%%%%%%%%%%%
\title{Draft of Self-Review}
%%%%%%%%%%%%%%%%%%%%%%%%%%%%%%%%%%%%%%%%%%%%%%%%%%%%%%%%%%%%%%%%
%%%%%%%%%%%%%%%%%%%%%%%%%%%%%%%%%%%%%%%%%%%%%%%%%%%%%%%%%%%%%%%%

\maketitle

\section{Overview of the major}

Questions: What are the Program Learning Outcomes identified for this major? What is the role of this major in undergraduate education on the campus, i.e., how does the major contribute to the undergraduate educational mission of the campus? Is the major clearly distinguished from other similar majors on campus?

Refer to the catalog description of the major and the other majors reviewed in the same cluster (Appendix A). Describe any inaccuracies in the catalog description and explain plans for correcting them. Identify the other majors in the cluster that are most similar to yours and explain how your major differs from them.

Enter your text here.

\section{Outcome of Previous Program Review}
Please list the recommendations made at the conclusion of the previous review (these may have been made by the review committee, Executive Committee and/or Dean) and comment briefly on the current status of the matters noted in the recommendations. Discuss any other significant changes in the major since the last review.

Enter your text here.


\section{Faculty in the major}
Questions: Who does the bulk of teaching in the major? What are the demographics of instructors in the major? Will the program be affected by substantial changes in the faculty (e.g. anticipated retirements) in the next review period?

Refer to the Tableau data concerning faculty in your department and the other departments reviewed in the same cluster (Appendix B, Tables 1-5). Based on those data and any additional information you wish to include, comment on each of the following for your major over the review period, referring when appropriate, to differences between your major and others in the cluster:

    a) Table 1.  Instructional Faculty – FTE and Percent by Rank 
    b) Table 2.  Age of Ladder Faculty – Percent by Age Group 
    c) Table 3.  Gender of Ladder Faculty – Number and Percent by Rank 
    d) Table 4.  Under-represented Ladder Faculty – Number and Percent by Rank 
    e) Table 5.  New Faculty Hires and Separations – Number by Rank 

Enter your text here.


\section{Instruction, advising, and resources in the major}
Questions: How effective is the delivery of instruction in the major? Are faculty engaged in the major? Is advising adequate? Is there adequate staff support? Are adequate space and facilities available? Is the program keeping pace with developments in the field? Are grading standards appropriate? What is the role of virtual and hybrid courses in this major? Please attach or include here a sample 4 year graduation plan for your program.

Refer to the Tableau data concerning instruction in the major and the other majors reviewed in the same cluster (Appendix B, Tables 6 -12). Based on those data and any additional information you wish to include, comment on each of the following for your major over the review period, referring, when appropriate to differences between your major and others in the cluster:
\begin{enumerate}[a)]
 \item Table 6.  Majors per Instructional Faculty FTE
 \item Table 7.  Students in Major Enrolled in Upper Division Courses – Percent of Total Course Enrollment 
 \item Table 8.  TAs Assigned to Upper Division Courses – Number By TA Role 
 \item Table 9.  Student Faculty Ratio – By Instructor Type 
 \item Table 10.  Courses Taught – Percent By Instructor Type and Course Level 
 \item Table 11.  Assigned Space – I\&R Assignable Square Feet (ASF) – By Department
 \item Table 12.  Distribution of Grades in Upper Division Courses – Percent of Total Enrolled and Average GPA 
\end{enumerate}
   
\noindent
Please also address the following issues, for which no data are provided:\\

    h) Comment on the degree of interest and engagement of the faculty in the major.\\
    i) Comment on the adequacy of staff support for the major.\\
    j) Comment on the adequacy of staff advising for the major.\\
    k) Comment on the adequacy of instructional equipment and facilities for the major.\\
    l) Comment on the program’s record of keeping pace with advances in the field.\\
    m) Comment on any academic programs that share or compete for instructional, advising, or other resources with this major (e.g., a similar major, a masters program, or a doctoral program)\\
    n) Comment on the number of hybrid and virtual courses are offered by the department/program and the average number of hybrid and virtual courses students take to complete their major requirements. Comment on any establish departmental resources and best practices regarding hybrid and virtual instruction.\\

Enter your text here.


\section{Students in the major}
Questions: This section is intended to characterize the students in this major. How have enrollments in the major varied over the period of the review, in terms of both the numbers and quality of the students? Are students succeeding in the major both in terms of qualitative and quantitative academic standards? Are students graduating on time? Are there impacted classes (e.g., with limited offerings or long waitlists) or other bottlenecks that unnecessarily impede student success? How do students find out about the major?  Is the major reaching a wide and diverse spectrum of students? Are students who enter the major retained in the major, and if not, why not?
Refer to the Tableau data concerning enrollments in the major and the other majors reviewed in the same cluster (Appendix B, Tables 13-23). Based on those data and any additional information you wish to include, comment on each of the following for your major over the review period, referring, when appropriate to differences between your major and others in the cluster:

    a) Table 13.  Number of Students - Duplicated Count and Percent Change 
    b) Table 14.  Students in Multiple Majors - Percent of Total in Major 
    c) Table 15.  Gender of Students – Percent of Total in Major and Percent Change 
    d) Table 16.  Under-represented Students – Percent of Total in Major and Percent Change 
    e) Table 17.  New Freshman Students Number and Percent Change 
    f) Table 18.  New Transfer Students Number and Percent Change 
    g) Table 19.  Average Cumulative UC Davis GPA 
    h) Table 20.  Students in Good Standing – Percent of Total by Level 
    i) Table 21.  Degrees Conferred – Duplicated Count and Percent Change 
    j) Table 22.  Time to Degree for Freshman and Transfer Students – All Students 
    k) Table 23.  Time to Degree for Freshman and Transfer Students – In Same Major 
    l) Table 24.  Graduating Students with a Minor
    m) Table 25.  Number of International Students
    n) In light of the information presented in Tables 13-23, describe and evaluate the effectiveness of any efforts by the program’s faculty and staff to retain students in the major.

Please also address the following issue, for which no data are provided:

    o) Describe and evaluate how students find information about the major (websites, course catalog, etc.).

Enter your text here.


\section{Students’ Perceptions of the Major}
Question: What are current students’ and recent graduates’ opinions of the major?

Refer to the Tableau data obtained from surveys of current students and alumni concerning their perceptions of the quality of the major and the other majors reviewed in the same cluster (Appendix C, Figures 1-53). Based on those data and any additional information you wish to include (e.g., results of departmentally administered course evaluations), comment on each of the following for your major over the review period, referring, when appropriate to differences between your major and others in the cluster:

    a) overall understanding of the major (Figures 1-4)
    b) overall satisfaction with the major (Figures 5-22)
    c) satisfaction with instruction in the major (Figures 23-36)
    d) satisfaction with academic advising in the major (Figures 37-43)
    e) satisfaction with courses offered in the major (Figures 44-53)

Enter your text here.

\section{Post-graduate Preparation}
Questions: How well does the major prepare students for postgraduate education and careers? Do the students have opportunities to meet and work with faculty outside the classroom setting? Is there sufficient support for internships or experiential learning opportunities? Are there ample opportunities for students to learn about career options?  
Refer to the Tableau data obtained from surveys of current students and alumni concerning preparation by the major for postgraduate education and careers (Appendix C, Figures 54-80). Based on those data and any additional information you wish to include, comment on each of the following for your major over the review period, referring, when appropriate to differences between your major and others in the cluster:

    a) quantity and quality of research and creative activities provided by the major (Figures 54-59)
    b) quality of preparation by the major for postgraduate education (Figures 60-64)
    c) quality of preparation by the major for the workforce (Figures 65-74)
    d) the degree to which students have sufficient contact with faculty to help them in their postgraduate education and careers (Figures 75-80).

Enter your text here.

\section{Educational objectives and Assessment}
Question: How does the program monitor and evaluate its success in achieving its Program  Learning Outcomes (section 1)?  

Specifically:
    a) Please confirm that the PLOs are clearly listed in an easily accessible location on the program website and provide the URL for that website.
    b) Please provide a program curriculum matrix or map which identifies the required courses in which each PLO is specifically addressed and assessed. Attach sample syllabi for up to six of the required core courses for the major. An additional resource for completing the curriculum Map/Matrix is available on the UIPRC website.
    c) What unit (committee or officer) in your program is responsible for collecting and analyzing data on student progress toward PLOs?
    d) Please list the direct evidence of student learning used to assess student achievement of PLOs. Examples of direct evidence include projects, scores from exams or quizzes, and written work. For this data, describe the methodology for sample selection and size.  
    e) Please list sources of indirect data used to contextualize student achievement of PLOs. Examples of indirect data include student evaluations, peer evaluation of teaching, and the survey data from current students and graduates provided for this review.
    f) In what ways do the results of this self-review reveal particular areas of strength or weakness in student progress toward achieving PLOs?
    g) What changes will the program make based on the results of the program assessment of student learning?

Programs are encouraged to contact the Academic Assessment team (assessment@ucdavis.edu) in the Center for Educational Effectiveness (Office of Undergraduate Education) to learn more about assessment of student learning outcomes.

Enter your text here.










 

\section{Major strengths and weaknesses/problems}
Summarize the major overall strengths of the program as well as any current problems that you perceive. 

Enter your text here.

\section{Future Plans}
Describe current or proposed plans to strengthen educational objectives of the program, such as increasing enrollments, improving student performance, and increasing the contribution of the program to the campus educational objectives. Comment on the long term strategy or goals regarding hybrid and virtual instruction in this program. Describe and justify if new resources are needed to preserve or strengthen the program.

Enter your text here.

	
\section{Minors}
Please comment on the minors currently administered by your department. Include all minors, and for each:

    a) Briefly explain the history of the minor.
    b) Briefly explain the rational for the minor, including the importance of the minor to students.
    c) Provide a brief overview of the minor, especially how students receive academic
guidance through the minor.
    d) Comment on numbers of students who have declared the minor during the review period.
    e) Describe the faculty participation in, and staff administrative support for, the minor.
    f) Describe your approach to ensuring that course listings for the minor are regularly updated and are realistic for students to complete in four years.
    g) Describe faculty participation in the minor. e.g., classroom instruction, co-ordination of
undergraduate events, supervision of undergraduate theses.
    h) Describe your approach to assessing the effectiveness of the minor on students’ learning and preparation for the future.
    i) Describe the current goals of the minor. Have goals changed over time? If so, explain
how the goals have changed.
    j) What are the strengths of the minor and benefits to students of the minor?
    k) What are the weaknesses of the minor? Describe any steps taken to address the
weaknesses.
    l) What are faculty members’ future plans for the minor?

Attachments:
    m) If it was not supplied in Appendix A, please attach the current catalog copy for the minor.
    n) Please provide a chart indicating the number of students who have declared the minor for each year in the review period.

Enter your text here.


\section{Emergency Remote Instruction}
The review team, UIPRC, and UGC understand that the emergency remote instructional environment required by the COVID-19 pandemic presented extraordinary demands on department faculty, staff, and students. Please address the successes and challenges faced by your program during emergency remote instruction. Provide information about any new practices that were beneficial and the program plans to continue, as well as any practices and outcomes that fell short of your standards.

Some items you may want to address:
    • Student experience: Student response to remote instruction; Student engagement with courses and department; Student interactions with faculty and staff (Zoom appointments, online office hours, etc.); Student retention of course information; Student progress toward degree
    • Faculty experience: Impacts of teaching remotely; Costs and benefits
    • TA experience: Interactions with faculty; Interactions with students
    • Staff / advising experience: Costs and benefits
    • Course equivalency: Degree of equivalency between normal and emergency remote instruction with regards to course learning outcomes and variability by learning activity (lecture courses, discussion courses, writing courses, laboratory courses, etc.)
    • Assessment of student learning: Strategies used for remote assignments and exams and their efficacy
    • Lessons learned for future virtual or hybrid instruction or advising in the program

Enter your text here.



\end{document}

