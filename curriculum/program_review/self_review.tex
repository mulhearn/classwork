\documentclass[12pt]{article}
\usepackage{lmodern}
\usepackage[T1]{fontenc}
\usepackage[dvips,letterpaper,margin=0.75in,bottom=0.75in]{geometry}
\usepackage{cancel}
\usepackage{graphicx}
\usepackage{braket}
\usepackage{latexsym,amssymb,amsmath}
\usepackage{pdfpages}
\usepackage{xcolor}
\usepackage{capt-of}
\usepackage{amsmath}
\usepackage{cite}
\usepackage[hyperfootnotes=false,hidelinks]{hyperref}
\usepackage[shortlabels]{enumitem}
\newcommand{\tcr}{\textcolor{red}}
\newcommand{\tcb}{\textcolor{blue}}

\usepackage[american,fulldiode]{circuitikz}
\tikzset{component/.style={draw,thick,circle,fill=white,minimum size =0.75cm,inner sep=0pt}}

\begin{document}
\ctikzset{bipoles/thickness=1}
\ctikzset{bipoles/length=.6cm}


%%%%%%%%%%%%%%%%%%%%%%%%%%%%%%%%%%%%%%%%%%%%%%%%%%%%%%%%%%%%%%%%
%%%%%%%%%%%%%%%%%%%%%%%%%%%%%%%%%%%%%%%%%%%%%%%%%%%%%%%%%%%%%%%%
\title{Physics and Applied Physics \\ Program Self-Review (Draft)}
%%%%%%%%%%%%%%%%%%%%%%%%%%%%%%%%%%%%%%%%%%%%%%%%%%%%%%%%%%%%%%%%
%%%%%%%%%%%%%%%%%%%%%%%%%%%%%%%%%%%%%%%%%%%%%%%%%%%%%%%%%%%%%%%%

\maketitle

\section{Overview of the major}

{\bf Questions: What are the Program Learning Outcomes identified for this major? What is the role of this major in undergraduate education on the campus, i.e., how does the major contribute to the undergraduate educational mission of the campus? Is the major clearly distinguished from other similar majors on campus?}\\[3pt]

%\noindent
%{\it {\color{red} Refer to the catalog description of the major and the other majors reviewed in the same cluster (Appendi%x A). Describe any inaccuracies in the catalog description and explain plans for correcting them. Identify the other majors% in the cluster that are most similar to yours and explain how your major differs from them.}}\\[3pt]

\noindent
The physics major plays a vital role in the undergraduate education
mission of the campus.  Ours is an ancient discipline, and much of the
undergraduate program is spent learning theoretical physics from a
hundred years ago or more.  Yet these ideas and concepts remain highly
potent in the modern world and are equally challenging and rewarding
to master.  While studying fundamental physics concepts, our students
are exposed to the latest cutting-edge research and concepts
concerning the physics of the small (nuclear processes, atomic
struture, particle searches, and cellular processes) and the large
(dark matter, dark energy, cosmology).  Our majors are trained in the
techniques of experimental physics, and we have recently dramatically
expanded our training in computational physics.

We are in the process of updating our undergraduate curriculum, as
described below.  The catalog description of the major provides an
accurate overview of the program, and the proposed undergraudate
curriculum update includes updates to the catalog to maintain its
accuracy under the new program.

Our department is distinguished from other departments in the cluster
by the requiremenst of the most vigorous introductory physics courses,
a diverse upper-level curriculum that develops physics concepts which
appear nowhere else on campus, and a vigorous and sustained emphasis
on fundamental concepts and anlytic thinking.  The program learning
objectives specific to our major are described in detail below.\\[3pt]

\noindent
{\bf Undergradudate Curriculum Update (UCU):}
The physics department, led by the undergraduate curriculum committee,
has been developing over the past five years on an update to the
undergraduate physics curriculum.  The proposal was subjected to an
extensive vetting process that involved assigned department readers
who were not involved in formulating the initial plan.  The proposal
was unamimously supported by the department in a December 2021 vote.
All new amd modified courses have been approved in the UC Davis
Integrated Curriculum Management System (ICMS).  We are working toward
final approval, by campus, of proposal during AY 2023-2024.

The proposed changes to our program reflect three main observations about our current program: 
\begin{itemize}
 \item Workload over four years of study was imbalanced, with too few physics
courses in the sophomore year, and too many in the junior year. 
 \item Integration between transfer students and
four-year students, who have somewhat different backgrounds, needed to be
improved. Our recent departmental Climate Survey confirmed this by showing a
huge satisfaction gap between the groups. 
 \item  The curriculum did not reflect the explosive growth
in computational methods and their applications.
\end{itemize}
There are many other aims of the update, but these are the primary motivations.

{\bf The proposed undergraduate curriculum update will be widely
  referenced throughout this document by the acronym UCU.}  The
proposal itself is far too ponderous of a document to include as an
appendix to this document.  It will be provided as a stand-alone file.
The latest version of the proposal is also available for download
online\footnote{See
  \href{https://github.com/mulhearn/classwork/blob/main/curriculum/curriculum.pdf}{https://github.com/mulhearn/classwork/blob/main/curriculum/curriculum.pdf}
  and use the download option to view the entire document from the
  pull down menu to view the entire document.}.\\[3pt]

\noindent
{\bf Program Learning Objectives:} Although we did not use the
specific term program learning objectives (PLOs) in that document,
these objectives were carefully considered as part of the development
of the UCU (see Sections 4-6 in particular).

There are a number of objectives that are both widely applicable and
central to the discipline of physics, most of which are reinforced in
nearly every physics course that a physics major  takes:
\begin{itemize}
 \item Using logic and analytic reasoning to make predictions.
 \item Applying general principles (e.g. conservation of energy, symmetry) in specific situations. 
 \item Testing results using dimensional analysis and limiting cases. 
 \item Dividing complex problems into manageable steps.
 \item Establishing feedback to determine if something is working or not.
\end{itemize}  
The next set of objectives are related to mathematical preparation:
\begin{itemize}
\item Understand the theory and practical application of differential, integral, and vector calculus, linear algebra, ordinary and partial differential equations.  (MAT 21ABCD, 22AB, and PHY 104A.)
\end{itemize}
While these concepts are first introduced in those courses, they are continuously reinforced throughout the major.  The second set of objectives are related to theoretical physics, students are expected to acquire a working knowledge of the theory and practical application of the following core topics:
\begin{itemize}
 \item Classical Mechanics (9A/9HA,105AB): the fundamental principles of physical laws
  (e.g. least action and symmetries) are taught in a familiar and intuitive context. 
\item Electromagnetism (9C/9HD,110AB): a remarkable special case of classical
  phenomena that anticipate non-Newtonian physics (e.g. special
  relativity, gauge theories).  No other force in nature can be understood so
  completely in such a straightforward fashion.  
\item Quantum Mechanics (9HC/9D,115AB): the rules governing the microscopic world are
  different from those governing our familiar macroscopic world.  The
  rules are not intuitive but they can be codified and used to make
  quantitative predictions which can be experimentally verified.
\item Statistical Mechanics (9HB/9B,9D,112): the crucial statistical explanation for how
  microscopic laws ultimately produce the macroscopic world which we inhabit.
\end{itemize}
A major focus of the curriculum update has been to devote more coursework to computational physics.
\begin{itemize}
 \item Develop programming skills sufficient for tackling problems from computational physics (PHY 40,45)
 \item Apply the techniques of computational physics to problems from theoretical physics (PHY 110L, PHY 112L, 115L).
\end{itemize}
The next set of objectives are related to experimental physics:
\begin{itemize}
 \item Learn how to conduct and report scientific experiments (PHY 80,117,118,122A/B)
 \item Gain practical hands-on knowledge of lab equipment, electronics, and technical trouble-shooting (PHY 80,117,118,122A/B
\end{itemize}
The objectives is for student to apply what they have learned previously to advanced specialized topics such as nuclear physics, particle physics, condensed matter, and astronomy.  We refer to these courses as capstone courses:
\begin{itemize}
 \item Demonstrate mastery of physics by applying it to advanced topics (PHY 129AB, 130AB, 140AB, 151-155)
\end{itemize}

\section{Outcome of Previous Program Review}
\label{sec:previous}

{\bf Please list the recommendations made at the conclusion of the previous review (these may have been made by the review committee, Executive Committee and/or Dean) and comment briefly on the current status of the matters noted in the recommendations. Discuss any other significant changes in the major since the last review.}\\

\noindent
The committee report of the Undergraduate Instruction Program Review
(UIPR) and the Review Team Report from the previous review are
included in the appendix.  These reports were a source of motivation
for the UCU described above, particularly in the specific cases noted
below.  The reports identified strengths in our program, which we
appreciate, but in this overview, we will address only the identified
weaknesses:
\begin{itemize}

\item {\bf Space, Lab Conditions and Maintenance:} the review team was
  concerned that the undergraduate lab space was ``shabby and in need
  of renovation''.  Within the UC, the cost of building renovations is
  shockingly expensive\footnote{Even by the standards of
    cost-of-living-numbed California residents}, and the university
  funding available for renovations is limited.  Fortunately, we have
  received vigorous support from campus with several sizable grants
  for both equipment and renovations, for which we are deeply
  appreciative.  This issue is discussed in more detail below (Section~\ref{sec:instruction})
  
\item {\bf Transfer Student Readiness:} the review team was concerned
  that a large fraction of incoming transfer students were not
  prepared for a physics major, and were dropping out or changing
  major at a high rate.  Following the recommendations of the review
  team and committee, we began selective major review in time for
  academic year (AY) 2021-2022.  The department has since seen a
  substantial drop in transfer enrollment at a level that we find
  potentially troubling, if it persists.  This is dicussed in more
  detail below (Section~\ref{sec:students}: Students in the major).
  {\bf Improving the experience for incoming transfer students was a major
  focus of the UCU.}

\item {\bf Computational Instruction:} The review team noted that
  programming coursework students were taking outside of the
  department was not adequately training our students in computational
  physics.  {\bf Expanding the role of computational physics in our
    major was a central focus of the UCU} and we have added to our
  curriculum several new required courses focused on computation
  physics.  This is described in more detail below
  (Section~\ref{sec:snws}: Major strengths and weaknesses).
  
\item {\bf Academic Advising:} The review team noted low satisfaction
  with campus and department advising.  In this report, we note that
  student satisfaction with advising has improved, most noticably with
  respect to faculty advising.  The likely contributions to this
  improvement, as well as some lingering concerns, are described below
  (Section~\ref{sec:perceptions}: Student perceptions of the major).

\item {\bf PHY 122: Advanced Physics Laboratory} the review team was
  concerned that students were generally poorly prepared for this
  upper division lab course.  Following the recommendations of the
  committee, we have introduced a new course (PHY 80: Experimental Techniques) as a prerequisite for PHY 122.  {\bf As part of the UCU, PHY 80 is now a required course for all physics majors (not just those taking PHY 122).}  This is ciscussed in more detail below (Section~\ref{sec:snws}: Major strengths and weaknesses)
  
\item {\bf PHY 157:}
  The review team noted that demand for PHY 157 is greater than it's
  capacity.  Regrettably, and despite deligent effort by Prof. Tucker Jones in particular, we have struggled to  maintain the same level of throughput in this course, due to the retirement of the previous instructor.  The pratical challenges here and future plans are discussed below (Section~\ref{sec:snws}: Major strengths and weaknesses)

\item {\bf Communication Skills:} The review team reported anecdotal
  evidence of student dissatisfaction with the level of development of
  skills in writing and oral presentation.  Although this concern has
  not been treated with same level of priority as the concerns above,
  we discuss this issue, along with related concerns about creativity
  and qualitative reasoning, in more detail below
  (Section~\ref{sec:perceptions}: Student perceptions about the
  major.)

\end{itemize}

We have taken the concerns raised during the previous review quite
seriously, and have made as much progress as we could manage, too the
clear benefit of our department, including our students.  We therefore
look forward to a fruitful collaboration during this review as well.


\section{Faculty in the major}
\label{sec:faculty}

{\bf Questions: Who does the bulk of teaching in the major? What are the demographics of instructors in the major? Will the program be affected by substantial changes in the faculty (e.g. anticipated retirements) in the next review period?}\\[3pt]

%Refer to the Tableau data concerning faculty in your department and the other departments reviewed in the same cluster (Appendix B, Tables 1-5). Based on those data and any additional information you wish to include, comment on each of the following for your major over the review period, referring when appropriate, to differences between your major and others in the cluster:
%    a) Table 1.  Instructional Faculty – FTE and Percent by Rank 
%    b) Table 2.  Age of Ladder Faculty – Percent by Age Group 
%    c) Table 3.  Gender of Ladder Faculty – Number and Percent by Rank 
%    d) Table 4.  Under-represented Ladder Faculty – Number and Percent by Ran 
%    e) Table 5.  New Faculty Hires and Separations – Number by Rank 

\noindent
{\it This section refers to tables which are available in the Appendix.  We
use a compact notation, where, for example, [B3] refers to Table 3 in
Appendix B.  The program was provided with an explicit list of items
to discuss.  To increase readibilty, we discuss the salient issues in
the most natural order first.  At the end of this section, we include a
``Discussion Checklist'' which indicates where each discussion point
was covered, using the original enumeration (a-e) of topics provided
in the original prompt. We have omitted the original prompt to
increase readability.}  {\color{red} TODO: copy to other sections as
      needed}.\\[3pt]

\noindent
Our teaching responsibilities are evenly shared by the faculty, and
the demographics of our course instructors largely reflects the
demographics of our faculty.  A long period of limited hiring has put
the department in a very uncomfortable position: new hiring is not
keeping up with retirements and the size of the faculty is shrinking.
This contraction is occuring when the number of physics majors, and
demand for introductory physics courses by non-majors, are both
increasing.\\[3pt]

\noindent {\bf Rank and Age of Faculty:} There are clear trends
revealed in [B1-2].  The Physics department faculty is significantly
older than the the average for the L\&S, and the size of the faculty
is shrinking.  These trends are correlated: retirements are outpacing
new hires.  While these numbers do not reflect our most recent hires
(Profs. Matthew Citron and Nancy Argawal) even this recent brisk pace
of hiring has not been fast enough to avoid a shrinking department.
Retirements have also left us with no LSOE (lecturers with
ladder-rank) faculty at present.

A significant obstacle to hiring at an even faster rate is the
availability of startup funding.  Under the current budget model, the
physics department will struggle to maintain a pace of one hire per
year, a pace which would lead to size of the physics faculty shrinking
further.\\[3pt]

\noindent
{\bf Diversity, Inclusion, and Equity:} The data provided in [B3-5] is
incomplete, so we will address the topic of diversity of the faculty
in purely qualitative terms.  Most importantly, our faculty is
overwhelming supportive of taking strong measures to to improve the
diversity of our faculty within the limits of what we are legally
allowed to do.  We are not looking for quick and easy fixes.  Instead,
we have been studying and adopting best-practices in hiring.  For
example, our two most recent faculty searches were intentionally
broadened in scope, as this has been shown to increase the diversity
of the applicant pool.  As another example, we have started providing
zoom interview questions in advance, as well as more details about the
interview process in general, because evidence shows that members of
URGs are systematically disadvantaged when such details are assumed to
be already known.  As is often the case when adopting best practices,
we found that these steps have also made the interview process better
overall. For example, when provided with the questions in advanced,
applicants gave better, more thoughtful, and more useful responses.
This remains an urgent issue, and their remains much to be done here.\\[3pt]

\noindent
{\bf Discussion Checklist:} (a-e) Discussed above [B1-5].
    
\section{Instruction, advising, and resources in the major}
\label{sec:instruction}

{\bf Questions: How effective is the delivery of instruction in the major? Are faculty engaged in the major? Is advising adequate? Is there adequate staff support? Are adequate space and facilities available? Is the program keeping pace with developments in the field? Are grading standards appropriate? What is the role of virtual and hybrid courses in this major? Please attach or include here a sample 4 year graduation plan for your program.}\\[3pt]

%Refer to the Tableau data concerning instruction in the major and the other majors reviewed in the same cluster (Appendix B, Tables 6 -12). Based on those data and any additional information you wish to include, comment on each of the following for your major over the review period, referring, when appropriate to differences between your major and others in the cluster:
%\begin{enumerate}[a)]
% \item Table 6.  Majors per Instructional Faculty FTE
% \item Table 7.  Students in Major Enrolled in Upper Division Courses – Percent of Total Course Enrollment 
% \item Table 8.  TAs Assigned to Upper Division Courses – Number By TA Role 
% \item Table 9.  Student Faculty Ratio – By Instructor Type 
% \item Table 10.  Courses Taught – Percent By Instructor Type and Course Leve 
% \item Table 11.  Assigned Space – I\&R Assignable Square Feet (ASF) – By Department
% \item Table 12.  Distribution of Grades in Upper Division Courses – Percent of Total Enrolled and Average GPA 
%\end{enumerate}

%\noindent
%Please also address the following issues, for which no data are provided:\\
%\begin{enumerate}[a)]
% \setcounter{enumi}{7}
% \item Comment on the degree of interest and engagement of the faculty in the major.
% \item Comment on the adequacy of staff support for the major.
% \item Comment on the adequacy of staff advising for the major.
% \item Comment on the adequacy of instructional equipment and facilities for the major.
% \item Comment on the program’s record of keeping pace with advances in the field.
% \item Comment on any academic programs that share or compete for instructional, advising, or other resources with this major (e.g., a similar major, a masters program, or a doctoral program)
% \item Comment on the number of hybrid and virtual courses are offered by the department/program and the average number of hybrid and virtual courses students take to complete their major requirements. Comment on any establish departmental resources and best practices regarding hybrid and virtual instruction.
%\end{enumerate}

\noindent
Our highly-engaged faculty is dedicated to delivering effective
instruction to our students, and we adhere to rigourous academic
standards with respect to grading.  While we do face challenges from
limited resources, overall we find that advising, staff support, space
and facilities are all adequate for our purposes.  In the sections
that follow, we attempt to provide evidence to support this bold
assessment, as well as call out areas where we hope to do better than
``adequate''.  Undergraduate physics is largely focused on the
cutting-edge theoretical physics of a hundred years ago or more. But
yet we do stay relevant in the modern world: those old ideas remain
challenging to master and highly potent.  Our multiyear continuing
commitment to the UCU is strong evidence that we continue to innovate
within our ancient discipline.  One exception: outside of emergencies,
we have no plans to include virtual or hybrid courses in our program,
as we have found, for our discipline, that no reasonable amount of
effort can produce such a course which is as effective as learning in
person (See Section~\ref{sec:remote}).  Sample four-year graduation
plans (as well as two year plans for transfer students) are provided
for every major and specialization, in the UCU.  See Table 6 in that
document, for one example\footnote{We can update this report with one
  example included here if requested.}.\\[3pt]

\noindent
{\bf Grading Policy:} The grade distribution in [B12] shows that
physics (and chemistry) are grading significantly more strictly than
L\&S overall, which is a conscious decision on our part.  Our
adherence to rigourous academic standards adds significant value to
the physics degree, and to physics coursework in general.  For
example, our top students are regularly accepted into top graduate
programs, likely in part because the A's they receive in our courses
represent a meaningful accomplishment.  We are concerned about the
ability of UCD to continue as a forefront research institution and
engine of social mobility if the grade inflation that we see
in L\&S overall is allowed to continue or accelerate.\\[3pt]

\noindent
{\bf Instructors:} Our department prioritizes providing physics majors
with ladder faculty instructors, particularly for upper-division
coursework.  The student FTE per faculty FTE is increasing [B6].  This
is the result of both an increase in the number of physics majors and
a decrease in the number of faculty.  Despite the challenges from a
shrinking faculty, physics majors are being taught by ladder faculty
[B9-10] for $90\%$ of their upper division coursework, a larger
fraction than any other L\&S department considered this cycle.
Undergraduate physics majors are being taught by ladder faculty for
$58.5\%$ of their physics coursework, also above average for L\&S.

For introductory physics coursework, we also utilize the
highly-effective pedagogy of Continuing Lecturers.  Dr. Dina
Zhabinskaya manages the PHY 7 series (introductory physics aimed at
bioscience) and Dr. Weideman manages the PHY 9 series (introductory
physics aimed at natural science and engineering).  They are both
deeply committed to innovative pedagogy, are great assets to our
department, and regularly share their wisdom with the rest of the
faculty.  We regularly hire unit-18 pre-six instructors as well.  Some
of our graduate students are interested in more extensive teaching
experience than typical TA assignments afford, and, when we believe
they are up to the challenge, we also hire Associate Instructors to
teach in PHY 7 and 9.\\[3pt]

\noindent
{\bf Staff Suppport:}
Physics is a unique discipline with its own unique culture, and we are profoundly grateful to the staff members that work alongside us to maintain it.
({\color{red} TODO:  Discuss with Tracy how we would like to describe the adequacy of staff support overall, Mike can meanwhile describe the key players in instructional support})\\[3pt]


\noindent
{\bf Instructional Space, Equipment, and Facilities} While total
assignable space has increased [B11], it has not been faster than the
number of student FTEs has increase.  The overall amount of assignable
space is above average for L\&S, but this reflects our need for
dedicated laboratory space, and large lecture halls adjacent to
demonstration support.  As discussed in Section~\ref{sec:previous},
the previous review noted a deficiency in the condition of the lab
space used for undergraduate instruction.  Fortunately, we have
received vigorous support for acquiring new instructional equipment
and renovation of lab space.  ({\color{red} TODO: get detailed list
  from Tracy, e.g. 122 renovations, new computing room, equipment
  grants, computing lab, etc}.)\\[3pt]

\noindent
{\bf Collaboration with other programs: } Physics (like Mathematics)
is a foundational discipline for most Science, Technology,
Engineering, and Math (STEM) disciplines.  Nearly every STEM
discipline trains its students in our introductory physics courses.
Physics itself is a rather unique discipline, and we have little
competition from other fields to consider.  Instead, we focus on
collaboration, which is much more useful.  We have a wide range of
applied physics majors which afford students the opportunities to
study complementary specialized topics (e.g. computer science,
atmospheric science) outside of the department.  The UCU strengthens
those majors by affording more flexibility with a more expansive list
of specialized courses to choose from.  Our new coursework in
computational physics (particularly PHY 40: Introduction to
Computational Physics, a course with no college-level prerequisites)
focuses on the bare minimum of programming techniques necessary to
tackle challenging physics problems (See Section~\ref{sec:snws}).  It
moves very quickly from ``introductory programming'' to
``computational physics''.  This makes the course highly complementary
to coursework in the Computer Science and Engineering department, and the markedly different emphasis may be of interest to other disciplines as well.\\[3pt]

\noindent
{\bf Discussion Checklist:} (a) Discussed above [B6].  (b)
Nearly $70\%$ of students enrolled in upper division physics courses are physics majors [B7].  The other students are from a wide variety of other departments, and their willingness to tackle these challenging courses illustrate  the broad appeal of physics. (c) There is an expected
increase [B8] in the number of TAs that resulted from new courses,
particularly PHY 80, PHY 40 and PHY 45, which require significant TA
support for lab sections, as well the addition of discussion sections
to some core upper division courses.  (d-e) Discussed above
[B9-10]. (f) Discussed above (Instructional Space, Equipment, and
Facilities) (g) Discussed above [B12] (h) Discussed above (introductory remarks) (i) Discussed above (Staff
Support) (j) Staff advising is discussed in
Section~\ref{sec:perceptions} (Academic Advising).  (k) Discussed
above (Instructional Space, Equipment, and Facilities) (l)
Discussed above (introductory remarks) (m) Discussed above
(Collaboration with other programs) (n) Discussed above (introductory remarks).

\section{Students in the major}
\label{sec:students}

{\bf Questions: This section is intended to characterize the students in this major. How have enrollments in the major varied over the period of the review, in terms of both the numbers and quality of the students? Are students succeeding in the major both in terms of qualitative and quantitative academic standards? Are students graduating on time? Are there impacted classes (e.g., with limited offerings or long waitlists) or other bottlenecks that unnecessarily impede student success? How do students find out about the major?  Is the major reaching a wide and diverse spectrum of students? Are students who enter the major retained in the major, and if not, why not?}\\[3pt]

%Refer to the Tableau data concerning enrollments in the major and the other majors reviewed in the same cluster (Appendix B, Tables 13-23). Based on those data and any additional information you wish to include, comment on each of the following for your major over the review period, referring, when appropriate to differences between your major and others in the cluster:
%\begin{enumerate}[a)]
%    \item Table 13.  Number of Students - Duplicated Count and Percent Change 
%    \item Table 14.  Students in Multiple Majors - Percent of Total in Major 
%    \item Table 15.  Gender of Students – Percent of Total in Major and Percent Change 
%    \item Table 16.  Under-represented Students – Percent of Total in Major and Percent Change 
%    \item Table 17.  New Freshman Students Number and Percent Change 
%    \item Table 18.  New Transfer Students Number and Percent Change 
%    \item Table 19.  Average Cumulative UC Davis GPA 
%    \item Table 20.  Students in Good Standing – Percent of Total by Level 
%    \item Table 21.  Degrees Conferred – Duplicated Count and Percent Change 
%    \item Table 22.  Time to Degree for Freshman and Transfer Students – All Students 
%    \item Table 23.  Time to Degree for Freshman and Transfer Students – In Same Major 
%    \item Table 24.  Graduating Students with a Minor
%    \item Table 25.  Number of International Students
%    \item In light of the information presented in Tables 13-23, describe and evaluate the effectiveness of any efforts by the program’s faculty and staff to retain students in the major.
%\end{enumerate}
%Please also address the following issue, for which no data are provided:
%\begin{enumerate}[o)]
%    \item Describe and evaluate how students find information about the major (websites, course catalog, etc.).
%\end{enumerate}

\noindent
As described in detail below, during this review period, the
undergraduate program has grown in both size and diversity.  We do not
plan to increase the size any further, but we do hope that the
diversity will continue to increase.  By most measures are student
academic achievement, physics students are performing just slightly
below the average for L\&S, which is a notable accomplishment given
that we have established physics courses are graded significantly more
stringently then L\&S on average.  The most concerning metrics are a
dramatic drop in transfer student admissions to eight total in AY
202-2023, and a four-year time-to-graduation for $73\%$ of physics
majors, which is significantly lower than for L\&S overall (85\%).
The UCU includes a number of changes to our program, including the
removal of bottlenecks, that we expect to impact these and other
metrics.  However, it is too soon to judge the efficacy of these
changes.  Looking at the available metrics all together, we see that
our program is successful at attracting, retaining, and graduating
students.

The primary source of information about our major is the department
website, the UC Davis course catalog, and the undergraduate academic
advisor.  We do considerable outreach with our students to keep them
on track.  This is all discussed in more detail in
Section~\ref{sec:perceptions} under ``Academic advising''.\\[3pt]

\noindent
{\bf Size of the undergraduate program:} The number of physics and
applied majors has been growing, as the result of a mutual agreement
between the university and department to begin admitting about $50\%$
more students into the program starting in time for AY 2019-2020.
This amount was targeted as the maximum number of additional students
that physics could handle without the need for additional lecture
sections in core physics courses.  This increase is the main trend
evident in [B13] and [B17] Taking applied physics and physics
together, the number of international students in physics [B25] has
increased by roughly $25\%$, consistent with the increase in L\&S.  To
accommodate the increase in the number of students, we added TA-led
discussion sections to some of the upper-division physics course
courses, which is something physics majors had been requesting even
before the class size increase.  These tables also put the ratio of
applied physics to physics majors historically at 1:2, but of late the
share of applied physics majors appears to be growing.\\[3pt]

\noindent
{\bf Diversity of physics majors:}
Combining physics and applied physics majors, the fraction [B15] who
are women has increase just slightly during the review period, while
the fraction of women in L\&S overall has decreased by about $3\%$.
Again combining physics and applied physics majors, the fraction [B16]
who are from underrepresented groups (URGs) has increased from $14\%$
to $19\%$.  When the number of majors is accounted for, the increase
in the number of women is not statistically significant, but the
increase in the number of students from URGs is statistically
significant.\\[3pt]

\noindent
{\bf Double majors and minors:}
There has been a noticable increase in the number of physics majors
who are double majors [B14] but the rate is still lower than the
average for L\&S.  We expect that this reflects the challenging nature
of the physics major.  The number of applied physics majors who are
double majors has remained even lower, and we suspect this reflects
that applied physics majors already complete significant coursework
outside of the physics department.  Physics advisors do not actively
encourage double majors.  In our experience, double majors in physics
have very little flexibility left in their schedule to pursue other
interests and electives.  We do heartily encourage minors, but yet
very few physics complete one [B24].\\[3pt]

\noindent
{\bf Transfer students and their experience:}
As discussed in Section~\ref{sec:previous}, the previous review was
concerned that a large number of incoming transfer students were not
prepared for a physics major.  Following the advice of the review
committee, the department started selective review of transfer
students in time for AY 2021-2022.  Our hope is that by selecting
students that are more likely to succeed as physics majors, we will
continue to graduate transfer students as physics majors at
approximately the same rate, while dramatically reducing the number of
students who drop out of the program.  A major focus of the UCU was
improving the experience of transfer students, by leveling out their
coursework in the first year to avoid a ``brick wall'' of intense
coursework that previous students faced in their first quarter.  It is
too early the judge the effectiveness of any of these majors.

Omitting AY 2020-2021 from consideration due to the pandemic, we see
that since AY 2021-2022 there has been a siginficant drop [B18] in the
number of new transfer students, which reflects the lower acceptance
rate due to selective major review.  However, we are concerned by the
size of this reduction: there were only eight new transfer students in
AY 2022-2023.  This is a much larger drop than the (still sizable)
$35\%$ reduction of transfer student in L\&S overall.

A significant number of transfer students in applied physics and
physics require an additional year to graduate: only $22\%$ of applied
physics majors and $41\%$ of physics majors graduate in two years as
intended [B22].  This is significantly lower than $64\%$ for L\&S overall.
Improving this experience for transfer students was a major focus of
the UCU.

We believe it is too soon to draw any conclusions about the changes to
our program that we have made which impact transfer students.
However, we have made a request for reports of additional metrics
pertaining to transfer students\footnote{Perhaps we will have historic
  data in time for the site visit...}  We plan to scrutinize these
statistics to judge the impact and efficacy of these changes to our
program.\\[3pt]

\noindent
{\bf Student Academic Performance:}
The average GPA [B19] of physics and applied physics majors is around
3.2 but varies as lows as 2.8 and as high as 3.3 depending on the
class and major.  This is slightly lower than the average for L\&S
(3.3).  This likely reflects the fact that physics majors take more
physics courses, which we have already established are generally
graded more stringently than courses from other departments in L\&S.
In [B20] we see qualitatively similar results with respect to the
fraction of students in good standing, which rangest from $79\%$ to
$94\%$ depending on the class and major, and is slightly lower than
L\&S overall.  We see a slight increase in degrees conferred [B21], for both physics
and applied physics majors.  When the data for AY 2023-2024 is
available, we should start to see an increase due to the larger number
of physics majors.

Looking at time to graduation [B23] for students that remain physics
majors the entire time, we see that $83.3\%$ of applied physics majors
and $72.7\%$ of physics majors graduate within four years as planned.
That rate for physics majors is significantly smaller than the average
for L\&S (84.5\%).  This is integrated from 2016 to 2022, which makes
it somewhat difficult to interpret.  We should monitor this metric
over the next few years to see if there is a persistent problem here
to address.  We do expect that time-to-graduation for nominal
four-year students should decrease as a result of the changes in the
UCU, which includes changes that remove stalling out of four-year
students in their sophomore year.  But it is too soon to judge the
efficacy of these changes.\\[3pt]

\noindent
{\bf Discussion Checklist:}
(a-m) Discussed above [B13-25] (n-o) Discussed above (introductory remarks) 

\section{Student perceptions of the major}
\label{sec:perceptions}

{\bf Question: What are current students' and recent graduates' opinions of the major? }\\

Our students' opinion of the physics and applied physics majors are
for the most part consistent with college averages, within the limited
statistics available from survey responses.  In the dicussion that follows,
we note some areas of encouragement and concern, and include anecdotal
data where pertinant.

%\noindent
%Refer to the Tableau data obtained from surveys of current students and alumni concerning their perceptions of the quality of the major and the other majors reviewed in the same cluster (Appendix C, Figures 1-53). Based on those data and any additional information you wish to include (e.g., results of departmentally administered course evaluations), comment on each of the following for your major over the review period, referring, when appropriate to differences between your major and others in the cluster:
%\begin{enumerate}[a)]
%    \item overall understanding of the major (Figures 1-4)
%    \item overall satisfaction with the major (Figures 5-22)
%    \item satisfaction with instruction in the major (Figures 23-36)
%    \item satisfaction with academic advising in the major (Figures 37-43)
%    \item satisfaction with courses offered in the major (Figures 44-53)
%\end{enumerate}

\noindent
{\bf Interpretation of student feedback:} Soliciting feedback from
students is a useful and important diagnostic tool, but the results
must be carefully interpreted.  We know that student evaluations
reflect significant implicit biases\footnote{See,
for example: Kreitzer, Rebecca J. \& Sweet-Cushman, Jennie
(2021). Evaluating Student Evaluations of Teaching: a Review of
Measurement and Equity Bias in SETs and Recommendations for Ethical
Reform. Journal of Academic Ethics 20 (1):73-84.}  and are strongly
correlated with student perceptions of their course grades.  For
example (referring [B12] and [C29]) we are not surprised to see that the
music department is an outlyer with respect to both grading ($85\%$ of
letter grades are A's) and student satisfaction with faculty
instruction ($88\%$).  In the physics department, $33\%$ of letter
grades are A's and student satisfaction (amongst junior and senior physics majors) with faculty
instruction is a more modest $67\%$.  We should not conclude that the
music department is performing any better (or worse, for that matter)
than the physics department, only that our approaches are different,
likely reflecting the different needs of our students.\\[3pt]

\noindent
{\bf Limited Statistical Uncertainty from Alumni Responses}
We also need to be careful not to draw sharp conclusions from samples
with limited statistics.  There were very few responses collected from
physics alumni: three from applied physics majors and 17 from physics
majors.  We have carefully checked all of the alumni responses
[C12-22], [C31-36], [C41-43], and [C49-54].  Amongst the questions,
the only ones which yielded a statistically significant deviation from
the college average were:
\begin{itemize}
  \item Questions [C18], [C21] and [C22] none of which we found to be illuminating.
  \item Questions [C31] and [C32] where we do see (borderline) statistically significant dissatisfaction with the grading system and use of information technology.   
  \item Question [C43] which references peer advisors, which we do not have in physics.
\end{itemize}

\begin{table}[htbp]
\caption{\label{tbl:appc} Survey results from Appendix C, using results from freshman through seniors, combining results from physics and applied physics majors. The sample size is N=61 from which a binomial statistical uncertainty has been calculated.  Results as a percent are organized by question (Q) and the results for physics and applied physics majors (Physics) are compared to the College (L\&S).}
\begin{center}
\begin{tabular}{|lll|lll|lll|lll|}
\hline
Q & Physics & L\&S & Q & Physics & L\&S & Q & Physics & L\&S & Q & Physics & L\&S \\
\hline
% Poisson Uncertainty:
%C1 & 90 $\pm$ 11 & 91 & C5  & 55 $\pm$ 7  & 64 & C23 & 53 $\pm$ 7  & 76 & C38 & 55 $\pm$ 7 & 52 \\
%C2 & 91 $\pm$ 12 & 90 & C6  & 71 $\pm$ 9  & 74 & C24 & 84 $\pm$ 11 & 68 & C39 & 56 $\pm$ 7 & 47 \\
%C3 & 79 $\pm$ 10 & 85 & C7  & 57 $\pm$ 7  & 60 & C25 & 84 $\pm$ 11 & 73 &     &          &    \\                
%C4 & 89 $\pm$ 11 & 93 & C8  & 78 $\pm$ 10 & 79 & C26 & 82 $\pm$ 10 & 76 & C44 & 46 $\pm$ 6 & 37 \\
%~  &           &    & C9  & 30 $\pm$ 4  & 46 & C27 & 48 $\pm$ 6  & 61 & C45 & 52 $\pm$ 7 & 56 \\
%~  &           &    & C10 & 68 $\pm$ 9  & 69 & C28 & 39 $\pm$ 5  & 49 & C46 & 41 $\pm$ 5 & 54 \\
%~  &           &    & C11 & 69 $\pm$ 9  & 69 & C29 & 52 $\pm$ 7  & 66 & C47 & 55 $\pm$ 7 & 59 \\
%~  &           &    &     &           &    & C30 & 75 $\pm$ 10 & 61 &     &          &    \\
% Binomial Uncertainty:
C1 & 90 $\pm$ 4 & 91 & C5  & 55 $\pm$ 6 & 64 & C23 & 53 $\pm$ 6 & 76 & C38 & 55 $\pm$ 6 & 52 \\
C2 & 91 $\pm$ 4 & 90 & C6  & 71 $\pm$ 6 & 74 & C24 & 84 $\pm$ 5 & 68 & C39 & 56 $\pm$ 6 & 47 \\
C3 & 79 $\pm$ 5 & 85 & C7  & 57 $\pm$ 6 & 60 & C25 & 84 $\pm$ 5 & 73 &     &            &    \\                  
C4 & 89 $\pm$ 4 & 93 & C8  & 78 $\pm$ 5 & 79 & C26 & 82 $\pm$ 5 & 76 & C44 & 46 $\pm$ 6 & 37 \\
~  &            &    & C9  & 30 $\pm$ 6 & 46 & C27 & 48 $\pm$ 6 & 61 & C45 & 52 $\pm$ 6 & 56 \\
~  &            &    & C10 & 68 $\pm$ 6 & 69 & C28 & 39 $\pm$ 6 & 49 & C46 & 41 $\pm$ 6 & 54 \\
~  &            &    & C11 & 69 $\pm$ 6 & 69 & C29 & 52 $\pm$ 6 & 66 & C47 & 55 $\pm$ 6 & 59 \\
~  &            &    &     &            &    & C30 & 75 $\pm$ 5 & 61 &     &            &    \\
\hline 
\end{tabular}
\end{center}
\end{table}

\noindent
{\bf Maximizing Statistical Uncertainty from Student Responses:}
To maximize the sample size from current students, we have included
all years in the survey result (the default results include only
juniors and seniors).  We have also computed a weighted average of the
responses from both applied physics and physics majors.  This combined
sample has a size of N=61 (with a slight variation from question to
question that has been neglected from our analysis) from which we
estimated a naive binomial statistical uncertainty.  The results are
reported in Table~\ref{tbl:appc}.\\[3pt]

\noindent
{\bf Understanding of the major:} Using the combined survey results in
Table~\ref{tbl:appc} we see that understanding of the major [C1],
program requirements [C2], program policy [C3], and accuracy of the
catalog [C4] were all comparable to the L\&S average, with the last
two being low by about one standard deviation.  Even though there is
therefore no statistically significant evidence to support the claim,
we do suspect there may be some student confusion in the context of
the UCU.  For example, we have begun teaching newly approved courses,
even though they are not yet required by the major, as the UCU has not
yet reached final approval.  We have done our best to communicate the
situation to our students, and we are encouraging them to take the new
courses even though they are not required yet.  We would not be
surprised if some confusion exists in this context.  We plan to
complete the approval process and make the necessary catalog updates
as soon as possible, at which point we are confident that any
confusion will begin to dissipate.\\[3pt]

\noindent
{\bf Satisfaction with major:} Using the combined survey results in
Table~\ref{tbl:appc}, we see that the responses from our majors regarding
fair treatment [C6], faculty access [C7], ability to get into major
[C8], library resources [C10], and overall satisfaction [C11] were all
consistent with college averages within statistical uncertainty of the
survey sample.  As discussed above, there are no statistically
significant results from the alumni survery that warant further discussion.

Our majors were less satisfied ($30\% \pm 6\%$) than the college
average ($46\%$) with respect to [C9] ``Education enrichment
programs''.  In the current program, it is nearly impossible for our
majors to study abroad, which likely contributes to the observed
student disastisfaction.  In the UCU, four-year students have
sufficient flexibility that more students may find it feasibile to
study abroad.  The department does provide other sources of education
enrichment, such as student research, and it would be illuminating to
probe student satisfaction with specific avenues of enrichment.

Our majors were less satisfied ($55\%\pm6\%$) than the college
average ($64\%$) with respect to [C5] the ``faculty being open to
discuss students' needs, concerns, and suggestions''.  Even though
this is just barely more than one standard deviation from the college
average, it is dissapointing to learn that so many students feel this
way.  This is something we should consider as a faculty.\\[3pt]

\noindent
{\bf Satisfaction with Instruction} Using the combined survey results
in Table~\ref{tbl:appc} we see that our majors believe plagiarism is
not being adequately explained [C23].  We encounter plagiarism most
frequently in physics courses through the use of online websites, such
as Chegg, to provide answers to homework questions.  There is growing
evidence that using sites such as Chegg for cheating expanded during
the pandemic and hasn't receded.  We wonder if the low score here is
reflecting student and faculty disagreement as to what constitutes
plagiarism, or, conversely, if it indicates students concerns that we
are not doing enough to discourage and condemn this relatively new
form of cheating.  Clearly more discussion between students and
faculty is needed about cheating.

The results of [C24-26] show that students believe they are well
practiced in recalling, explaining, and analyzing concepts, methods,
and ideas.  However, when it comes to qualitative judgement [C27] and
creativity [C28] physics majors reported lower than average practice,
which we discuss further below.

Our majors are quite well satisfied with their TAs ($75\% pm 5\%$)
compared to the college average ($61\%$).  This is heartening, if not
surprising.  TA's are most often cast in roles which directly support
students, helping them work through their homework problems in
discussion sessions, for example.

Satisfaction with faculty is lower ($52\% \pm 6\%$) than the college
average ($66\%$).  This is concerning and warants a closer look at the
data.  If we consider physics majors only, and restrict ourselves to
junior and seniors. we we see 67\% are satisfied with faculty
instruction, just a bit higher than the colleage average.  Further
investigation confirms that the heightened dissatisfaction comes from
two contributions: current freshman and sophomores, and all applied
physics majors.

One major difference between applied physics and physics majors is the
amount of coursework that is taken outside of the department. We
are well aware of the frustration that applied physics majors are
facing in completing their coursework outside of physics.  Perhaps
part of the dissatisfaction stems from this.  The UCU aims to give
applied physics majors more flexibility which we hope will improve
this stuation.  In any event, it is clear that additional outreach
with applied physics majors and physics majors in lower division
courses is needed to determine the root causes of this elevated
dissatisfation.\\[3pt]

\noindent
{\bf Academic Advising:}
The satisfaction of alumni with advising was all
consistent with the college average, within the statistical uncertainy
of the sample.  The only exception was a question regarding peer
advising, which physics does not provide.  Amongst our majors, using
the combined sample of Table~\ref{tbl:appc}, we see that students are
satisfied with the quality of academic advising (55\% $\pm$ 6\%)
slightly higher than the college average (52\%) and with access to
academic advising (56\% $\pm$ 6\% ) higher than the college average
(47\%).  Only the latter is (borderline) statistically significant.

These results are encouraging.  As discussed in
Section~\ref{sec:previous}, low satisfaction with advising was noted
by the previous review.  We believe we understand the source of the
improvement, based on our own anecdotal data from individual student
interactions.  Most importantly, Prof. Boeshaar runs a physics career
seminar and Prof. Know runs an alumni speaker seminar.  These are not
required courses, but the scope of discussion typically extends to
academic advising, particularly for subjects that students are
concerned about.  Secondly, the staff undergraduate advisor, Amy Foelz
and the vice-chair for the undergraduate program, Prof. Mulhearn, have
devised a highly effective triage system, whereby typical issues are
handled by Foelz, and more complicated problems are immediately kicked
up to Prof. Mulhearn.  We have found this arrangement to be quite
effective.  Foelz also arranges multiple annual outreach events (pizza
nights) which provide a forum for academic advising.  Prof. Mulhearn
is an annual speaker in the physics career seminar.  Lastly, a
significant number of students participate in undergraduate research,
and research advisors are a natural source of academic advice as well.
Our attempts to impose a more top-down assigned faculty advisor had
lackluster results, with a clear lack of student and faculty interest,
which was compounded by the pandemic.  Fortunately, it seems that our
more organic, fully optional avenues for academic advising are proving
to be effective.  Prof. Boeshaar is continuing to play her pivotal role past
her recent retirement, but we cannot expect this to continue
indefinitely.\\[3pt]

\noindent
{\bf Course Offerings:} Using the combined data sample from
Table~\ref{tbl:appc}, we that our students rated access to small class
sizes [C44] higher (46\% $\pm$ 6\%) than the college average ($37\%$).
Our students rated the availability of courses needed to graduate
[C46] at lower (41\% $\pm$ 6\%) than the college average (54\%), and
this lower level of satisfaction is present for both physics and
applied physics majors.  Anecdotally, we are aware that applied
physics majors stuggle to schedule their coursework outside of the
department.  We are addressing this problem in the UCU by increasing
the number of choices available to applied physics majors for their
coursework outside of the department.  We are also discussing with the
computer science department the possiblity to give higher priority to
our applied physics (computational physics) majors in their impacted
courses.  We are also aware that physics majors struggle to schedule
advanced physics labs, and we are attempting to improve throughput in
these courses as well.  We offered a fall section of advanced physics
lab for the first ime in AT 2023-2024.  We have also begun renovations
to provide more space, and are adding additional lab modules, toward
increasing the number of students in each section.  Our students rated
the availability of general education courses [C45] and the variety or
courses [C47] at levels consistent with the colleage average within
the statistical uncertainty of the sample.  The alumni responses were
all consistent with the college average within the statistical
uncertainty of the sample.\\[3pt]

\noindent
{\bf Communication, Creativity, and Qualitative Judgement:} {\color{red}TODO}
Communication was mentioned as a concern in the previous program
review.  Addressing this deficiency immediately runs into several
major practical challenges.  With a throughput of order 60 students
per year, even having each student provide a 15 minute presentation
would be a commitment of 15 hours: 50\% of standard lecture course.
As we are limited in the number of units which we can require of our
majors, we have already made difficult cuts to required coursework.
We don't see a way to definitely address this problem that doesn't
include either (1) an increase in the number of units we are allowed
to require for our students, or (2) improving the quality of GE
instruction so that students receive adequate training in writing and
oral commmunication as part of the 50\% of the coursework they
complete outside of their major requirements.  However, one
anticipated outcome of the updated curriculum is that four-year
students will have more time for elective offerings in physics.  We
think therefore, that an incremental way forward here is to provide an
elective course designed to provide more opportunities for practicing
scientific communication.

and applied physics major reported significantly lower than average
practice.  This is disappointing, if not entirely surprising.  A large
fraction of the undergraduate physics degree is invested toward
understanding well-established theories which more mostly developed a
century ago or more.  Problems for these theories which have exact
analytic solutions are limited and so students invest a lot of time
solving problems that have been being solved by students for a century
or more as well.  This paints perhaps too bleak a picture, as
instructors do labor to breath life into their subjects every quarter,
but they are facing strong headwinds, and these results most likely
reflect those headwinds.  One are of the undergraduate physics degree
where creativity and qualitative judgement is more naturally exercises
is in computational physics and experimental physics.  Here students
devise and debug their own code or experimental apparatus, face new
problems not constrained to those with exact analytic solutions.  It
will be interesting to see if the improvements and increased focus on
these subjects in the curriculum update will lead to students getting
more practice in these topics.\\[3pt]

\noindent
{\bf Discussion Checklist:} (a-e) Discussed above [C1-52] 


\end{document}

\section{Post-graduate Preparation}
Questions: How well does the major prepare students for postgraduate education and careers? Do the students have opportunities to meet and work with faculty outside the classroom setting? Is there sufficient support for internships or experiential learning opportunities? Are there ample opportunities for students to learn about career options?  
Refer to the Tableau data obtained from surveys of current students and alumni concerning preparation by the major for postgraduate education and careers (Appendix C, Figures 54-80). Based on those data and any additional information you wish to include, comment on each of the following for your major over the review period, referring, when appropriate to differences between your major and others in the cluster:

    a) quantity and quality of research and creative activities provided by the major (Figures 54-59)
    b) quality of preparation by the major for postgraduate education (Figures 60-64)
    c) quality of preparation by the major for the workforce (Figures 65-74)
    d) the degree to which students have sufficient contact with faculty to help them in their postgraduate education and careers (Figures 75-80).

Enter your text here.

\section{Educational objectives and Assessment}

Question: How does the program monitor and evaluate its success in achieving its Program  Learning Outcomes (section 1)?  

Specifically:
    a) Please confirm that the PLOs are clearly listed in an easily accessible location on the program website and provide the URL for that website.
    b) Please provide a program curriculum matrix or map which identifies the required courses in which each PLO is specifically addressed and assessed. Attach sample syllabi for up to six of the required core courses for the major. An additional resource for completing the curriculum Map/Matrix is available on the UIPRC website.
    c) What unit (committee or officer) in your program is responsible for collecting and analyzing data on student progress toward PLOs?
    d) Please list the direct evidence of student learning used to assess student achievement of PLOs. Examples of direct evidence include projects, scores from exams or quizzes, and written work. For this data, describe the methodology for sample selection and size.  
    e) Please list sources of indirect data used to contextualize student achievement of PLOs. Examples of indirect data include student evaluations, peer evaluation of teaching, and the survey data from current students and graduates provided for this review.
    f) In what ways do the results of this self-review reveal particular areas of strength or weakness in student progress toward achieving PLOs?
    g) What changes will the program make based on the results of the program assessment of student learning?

Programs are encouraged to contact the Academic Assessment team (assessment@ucdavis.edu) in the Center for Educational Effectiveness (Office of Undergraduate Education) to learn more about assessment of student learning outcomes.

Enter your text here.

(a) {\color{red} We need to do this (put this on a webpage).}
(b) The required courses which relate to each PLO were already listed in Section 1.
({\color{red} TODO: we can make a little table here to make folks happy.})
(c) The vice-chair for the undergraduate program and the undergraduate curriculum, in collaboration with the physics faculty, are responsible for evaluating the performance of the undergraduate program.
As part of the curriculum update, we have developed a new procedure for maintaining the curriculum.  Each spring, the undergraduate curriculum committee organizes a faculty meeting devoted to reviewing the undergraduate program.  The instructors of core courses prepare short summaries of their courses, covering how they taught the course, including notable differences from the example syllabi, and any observed shortcomings in student preparation from prerequisite courses.  If any changes to the example syllabi are needed, the UGCC prepares new version, upon which the faculty vote.
(d) The acquisition of PLOs related to mathematics and theoretical physics are most directly assessed through final exams in core courses (e.g. 104A, 105AB, 110AB, 112, 115AB).  The experimental physics PLOs are assessed through student lab reports.  Instructors of PHY 80 have also recently introduced a lab practical examination which we have found to be quite successful.  The assessment of PLOs are based on submitted student programs, cross-checked by in-class evalution based on Parsons problems (correctly ordering provided lines of code to answer a prompt) and predicting the output of provide code snippets.
Because physics is quite hierarchical, there are multiple opportunities to assess students acquisition of the PLOs.  For one example, several years ago we learned that PHY 104A was not consistently covering vector calculus because the 110AB instructor reported having to review this material, and we made adjustments to address this deficiency.(e-f) {\color{red} TODO.}
 
\section{Major strengths and weaknesses}
\label{sec:snws}

{\bf Theoretical Physics:}


{\bf Computational Physics:}
Mostly strengths now.


{\bf Experimental Physics:}
Dicuss the role of PHY 80.

Discuss challenges with PHY 157.  This troughput is limited by several
practical concerns: the time of year during which weather is favorable
for observation, safety concerns which limit the numbrer of people
that can be on the room for observation, and limited faculty capable
of teaching the course.  With the retirement of Prof. Tony Tyson,
Prof. Tucker Jones has been teaching the course, but we have struggled
to maintain throughput, let alone increase it.  We recently aquired a
grant for a new telescope, and perhaps this can improve the situation?
Talk with Pat...



\section{Future Plans}
Describe current or proposed plans to strengthen educational objectives of the program, such as increasing enrollments, improving student performance, and increasing the contribution of the program to the campus educational objectives. Comment on the long term strategy or goals regarding hybrid and virtual instruction in this program. Describe and justify if new resources are needed to preserve or strengthen the program.

Enter your text here.

	
\section{Minors}
Please comment on the minors currently administered by your department. Include all minors, and for each:

    a) Briefly explain the history of the minor.
    b) Briefly explain the rational for the minor, including the importance of the minor to students.
    c) Provide a brief overview of the minor, especially how students receive academic
guidance through the minor.
    d) Comment on numbers of students who have declared the minor during the review period.
    e) Describe the faculty participation in, and staff administrative support for, the minor.
    f) Describe your approach to ensuring that course listings for the minor are regularly updated and are realistic for students to complete in four years.
    g) Describe faculty participation in the minor. e.g., classroom instruction, co-ordination of
undergraduate events, supervision of undergraduate theses.
    h) Describe your approach to assessing the effectiveness of the minor on students’ learning and preparation for the future.
    i) Describe the current goals of the minor. Have goals changed over time? If so, explain
how the goals have changed.
    j) What are the strengths of the minor and benefits to students of the minor?
    k) What are the weaknesses of the minor? Describe any steps taken to address the
weaknesses.
    l) What are faculty members’ future plans for the minor?

Attachments:
    m) If it was not supplied in Appendix A, please attach the current catalog copy for the minor.
    n) Please provide a chart indicating the number of students who have declared the minor for each year in the review period.

Enter your text here.


\section{Emergency Remote Instruction}
\label{sec:remote}

The review team, UIPRC, and UGC understand that the emergency remote instructional environment required by the COVID-19 pandemic presented extraordinary demands on department faculty, staff, and students. Please address the successes and challenges faced by your program during emergency remote instruction. Provide information about any new practices that were beneficial and the program plans to continue, as well as any practices and outcomes that fell short of your standards.

Some items you may want to address:
    • Student experience: Student response to remote instruction; Student engagement with courses and department; Student interactions with faculty and staff (Zoom appointments, online office hours, etc.); Student retention of course information; Student progress toward degree
    • Faculty experience: Impacts of teaching remotely; Costs and benefits
    • TA experience: Interactions with faculty; Interactions with students
    • Staff / advising experience: Costs and benefits
    • Course equivalency: Degree of equivalency between normal and emergency remote instruction with regards to course learning outcomes and variability by learning activity (lecture courses, discussion courses, writing courses, laboratory courses, etc.)
    • Assessment of student learning: Strategies used for remote assignments and exams and their efficacy
    • Lessons learned for future virtual or hybrid instruction or advising in the program

Enter your text here.

Our faculty made a heroic effort during the COVID-19 pandemic, but the quality of instruction suffered tremendously, and we believe we still see evidence that students have not fully recovered.  In qualitative terms, we saw that the top 1/3 of students managed reasonable well with remote instruction, but the bottom fell out for the remaining 2/3.  We believe we saw a dramatic increase in cheating, and there is growing evidence that this was a widespread phenomenon.  We also believe that we are still seeing residual effects of the pandemic on current students.  Specifically, the level of mathematical preparation and study habits have not fully recovered to pre-pandemic levels.  These are subjective evaluations, but they have been widely observed by our faculty.

During emergency remote instruction, student engagement dropped precipitously.  Those brave instructors that continued live (via zoom) lectures were faced with a wall of black: students left their cameras off.  As a result, with few exceptions, faculty despised remote instruction.  But we did our best.  An approach that emerged as somewhat effective was providing a mixture of {\bf short} asynchronous pre-recorded lectures and live in-person recap discussions.  For faculty teaching lab courses, the burden was even greater.  These are crucial experiences for our students, and faculty generally went to extraordinary lengths to provide kits or other means for students to get some hands on experience, which required redesigning entire courses with little notice.  

And despite all of these efforts, there was no meaningful equivalency between the online courses we provided and in-person courses.  Any instructor has seen that students that do not attend lecture generally become demoralized, detached, and perform well below their abilities.  COVID forced something like that onto every one of our students.  May we never have to do that again.

\end{document}

