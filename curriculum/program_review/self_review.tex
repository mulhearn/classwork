\documentclass[12pt]{article}
\usepackage{lmodern}
\usepackage[T1]{fontenc}
\usepackage[dvips,letterpaper,margin=0.75in,bottom=0.75in]{geometry}
\usepackage{cancel}
\usepackage{graphicx}
\usepackage{braket}
\usepackage{latexsym,amssymb,amsmath}
\usepackage{pdfpages}
\usepackage{xcolor}
\usepackage{capt-of}
\usepackage{amsmath}
\usepackage{cite}
\usepackage[shortlabels]{enumitem}
\newcommand{\tcr}{\textcolor{red}}
\newcommand{\tcb}{\textcolor{blue}}


\usepackage[american,fulldiode]{circuitikz}
\tikzset{component/.style={draw,thick,circle,fill=white,minimum size =0.75cm,inner sep=0pt}}

\begin{document}
\ctikzset{bipoles/thickness=1}
\ctikzset{bipoles/length=.6cm}


%%%%%%%%%%%%%%%%%%%%%%%%%%%%%%%%%%%%%%%%%%%%%%%%%%%%%%%%%%%%%%%%
%%%%%%%%%%%%%%%%%%%%%%%%%%%%%%%%%%%%%%%%%%%%%%%%%%%%%%%%%%%%%%%%
\title{Draft of Self-Review}
%%%%%%%%%%%%%%%%%%%%%%%%%%%%%%%%%%%%%%%%%%%%%%%%%%%%%%%%%%%%%%%%
%%%%%%%%%%%%%%%%%%%%%%%%%%%%%%%%%%%%%%%%%%%%%%%%%%%%%%%%%%%%%%%%

\maketitle

\section{Overview of the major}

Questions: What are the Program Learning Outcomes identified for this major? {\color{red} What is the role of this major in undergraduate education on the campus, i.e., how does the major contribute to the undergraduate educational mission of the campus? Is the major clearly distinguished from other similar majors on campus?\\

Refer to the catalog description of the major and the other majors reviewed in the same cluster (Appendix A). Describe any inaccuracies in the catalog description and explain plans for correcting them. Identify the other majors in the cluster that are most similar to yours and explain how your major differs from them.}\\[5pt]

\noindent
Although we did not use the specific term program learning objectives (PLO), these were carefully defined as part of the undergraduate curriculum update which is available in the appendix.  The first set of objectives are related to mathematical preparation:
\begin{itemize}
\item Understand the theory and practical application of differential, integral, and vector calculus, linear algebra, ordinary and partial differential equations.  (MAT 21ABCD, 22AB, and PHY 104A.)
\end{itemize}
While these concepts are first introduced in those courses, they are continuously reinforced throughout the major.  The second set of objectives are related to theoretical physics, students are expected to acquire a working knowledge of the theory and practical application of the following core topics:
\begin{itemize}
 \item Classical Mechanics (9A/9HA,105AB): the fundamental principles of physical laws
  (e.g. least action and symmetries) are taught in a familiar and intuitive context. 
\item Electromagnetism (9C/9HD,110AB): a remarkable special case of classical
  phenomena that anticipate non-Newtonian physics (e.g. special
  relativity, gauge theories).  No other force in nature can be understood so
  completely in such a straightforward fashion.  
\item Quantum Mechanics (9HC/9D,115AB): the rules governing the microscopic world are
  different from those governing our familiar macroscopic world.  The
  rules are not intuitive but they can be codified and used to make
  quantitative predictions which can be experimentally verified.
\item Statistical Mechanics (9HB/9B,9D,112): the crucial statistical explanation for how
  microscopic laws ultimately produce the macroscopic world which we inhabit.
\end{itemize}
A major focus of the curriculum update has been to devote more coursework to computational physics.
\begin{itemize}
 \item Develop programming skills sufficient for tackling problems from computational physics (PHY 40,45)
 \item Apply the techniques of computational physics to problems from theoretical physics (PHY 110L, PHY 112L, 115L).
\end{itemize}
The next set of objectives are related to experimental physics:
\begin{itemize}
 \item Learn how to conduct and report scientific experiments (PHY 80,117,118,122A/B)
 \item Gain practical hands-on knowledge of lab equipment, electronics, and technical trouble-shooting (PHY 80,117,118,122A/B
\end{itemize}
The last objectives is for student to apply what they have learned previously to advanced specialized topics such as nuclear physics, particle physics, condensed matter, and astronomy.  We refer to these courses as capstone courses:
\begin{itemize}
 \item Demonstrate mastery of physics by applying it to advanced topics (PHY 129AB, 130AB, 140AB, 151-155)
\end{itemize}






\section{Outcome of Previous Program Review}

Please list the recommendations made at the conclusion of the previous review (these may have been made by the review committee, Executive Committee and/or Dean) and comment briefly on the current status of the matters noted in the recommendations. Discuss any other significant changes in the major since the last review.\\[5pt]


\noindent
The committee report of the Undergraduate Instruction Program Review
(UIPR) and the Review Team Report from the previous review are
included in the appendix.  Spurred by these reports and other
concerns, the department has been revising its undergraduate curriculum
over the past five years, and the proposed revisions are included in
the appendix. While the reports identified strengths in
the program, we will address here the weaknesses that were listed:
\begin{itemize}
\item Space, Lab Conditions and Maintenance:  the review team was concerned that the undergraduate lab space was ``shabby and in need of renovation''.  We are happy to report a number of past and current investments and renovations:  {\color{red}TODO get details from Tracy, 122 updates, new computational physics lab, etc}
\item Transfer Student Readiness: Following the recommendations of the
  committee, we began selective major review in ({\color{red} TODO get date from
  Amy}).  The department has seen a substantial drop in transfer
  enrollment at a level that we find potentially troubling, if it
  persists.  It is difficult to decouple this data from the impact of COVID-19
  and an overall drop in transfer enrollment campus wide ({\color{red}CONFIRM?}).
  It is also too soon to draw conclusions on the impact of this
  change on the average time to degree and the fraction of students
  leaving without a degree.  We are in the process of gathering these
  statistics specifically for transfer students.  This concern was
  also a major focus of the undergraduate curriculum update as
  described in more details below.
\item {\color{red} Advising:  work with Amy on this one. (TODO)}
\item PHY 122: the review was concerned that students were generally
  poorly prepared for this upper division lab course.  Following the
  recommendations of the committee, we introduced a new course, PHY
  80, as a prerequisite for PHY 122.  As described in more detail
  below, this was also a major focus of the undergraduate curriculum
  update, which makes PHY 80 a required course for all majors (not
  just those taking PHY 122).
\item Communication Skills: The review team reported anecdotal
  evidence of student dissatisfaction with the level of development of
  skills in writing and oral presentation.  Addressing this deficiency
  immediately runs into several major practical challenges.  With a
  throughput of order 60 students per year, even having each student
  provide a 15 minute presentation would be a commitment of 15 hours:
  50\% of standard lecture course.  As we are limited in the number of
  units which we can require of our majors, we have already made
  difficult cuts to required coursework.  We don't see a way to
  definitely address this problem that doesn't include either (1) an
  increase in the number of units we are allowed to require for our
  students, or (2) improving the quality of GE instruction so that
  students receive adequate training in writing and oral
  commmunication as part of the 50\% of the coursework they complete
  outside of their major requirements.  However, one anticipated
  outcome of the updated curriculum is that four-year students will
  have more time for elective offerings in physics.  We think
  therefore, that an incremental way forward here is to provide an
  elective course designed to provide more opportunities for
  practicing scientific communication.
\item Computational Instruction: The review team noted that ECS 30 was
  not adequately preparing our students for computational physics.  We
  have since introduced five new computational physics courses: PHY
  40,45,110L,112L, and 115L.  This was a major focus of the curriculum
  update that is described in more detail below.
\item PHY 157: The review team noted that demand for PHY 157 is
  greater than it's capacity.  This troughput is limited by several
  practical concerns: the time of year during which weather is
  favorable for observation, safety concerns which limit the numbrer
  of people that can be on the room for observation, and limited
  faculty capable of teaching the course.  With the retirement of
  Prof. Tony Tyson, Prof. Tucker Jones has been teaching the course,
  but we have struggled to maintain throughput, let alone increase it.
  We recently aquired a grant for a new telescope, and perhaps this
  can improve the situation?  Talk with Pat...
\end{itemize}

{\color{red} TODO:  Describe the undergraduate curriculum update.}

\section{Faculty in the major}
{\color{red} Questions: Who does the bulk of teaching in the major? What are the demographics of instructors in the major? Will the program be affected by substantial changes in the faculty (e.g. anticipated retirements) in the next review period?}

Refer to the Tableau data concerning faculty in your department and the other departments reviewed in the same cluster (Appendix B, Tables 1-5). Based on those data and any additional information you wish to include, comment on each of the following for your major over the review period, referring when appropriate, to differences between your major and others in the cluster:

    a) Table 1.  Instructional Faculty – FTE and Percent by Rank 
    b) Table 2.  Age of Ladder Faculty – Percent by Age Group 
    c) Table 3.  Gender of Ladder Faculty – Number and Percent by Rank 
    d) Table 4.  Under-represented Ladder Faculty – Number and Percent by Rank 
    e) Table 5.  New Faculty Hires and Separations – Number by Rank 

Enter your text here.

(a-b) There are clear trends revealed in Appendix B, Tables 1-2.  The Physics department faculty is significantly older than the the average for the CLAS, and the size of the faculty is shrinking.  These trends are correlated: retirements are outpacing new hires.  While these numbers do not reflect our most recent hires (Profs. Matthew Citron and Nancy Argawal) even this recent brisk pace of hiring has not been fast enough to avoid a shrinking department.  Retirements have also left us with no LSOE faculty.

A significant obstacle to hiring at an even faster rate is the availability of startup funding.  Under the current budget model, the physics department will struggle to maintain a pace of one hire per year, a pace which would lead to size of the physics faculty shrinking further.

(c-d) The data in Appendix in Tables 3-5 is incomplete, so we will address the topic of diversity of the faculty in qualitative terms.  Most importantly, the faculty is overwhelming supportive of taking strong measures to increase diversity of the faculty, and we are not looking for quick and easy fixes.  We have been studying and adopting best practices in hiring toward that end.  For example, our two most recent hires were intentionally broadened in scope as this has been shown to increase the diversity of the applicant pool.  As another example, we have started providing zoom interview questions in advance, as well as more details about the interview process in general, because evidence shows that members of URGs are systematically disadvantaged when such details are assumed to be already known.  As is often the case when adopting best practices, we found that these steps also made the interview process better overall. 


\section{Instruction, advising, and resources in the major}
{\color{red} Questions: How effective is the delivery of instruction in the major? Are faculty engaged in the major? Is advising adequate? Is there adequate staff support? Are adequate space and facilities available? Is the program keeping pace with developments in the field? Are grading standards appropriate? What is the role of virtual and hybrid courses in this major? Please attach or include here a sample 4 year graduation plan for your program.}

Refer to the Tableau data concerning instruction in the major and the other majors reviewed in the same cluster (Appendix B, Tables 6 -12). Based on those data and any additional information you wish to include, comment on each of the following for your major over the review period, referring, when appropriate to differences between your major and others in the cluster:
\begin{enumerate}[a)]
 \item Table 6.  Majors per Instructional Faculty FTE
 \item Table 7.  Students in Major Enrolled in Upper Division Courses – Percent of Total Course Enrollment 
 \item Table 8.  TAs Assigned to Upper Division Courses – Number By TA Role 
 \item Table 9.  Student Faculty Ratio – By Instructor Type 
 \item Table 10.  Courses Taught – Percent By Instructor Type and Course Level 
 \item Table 11.  Assigned Space – I\&R Assignable Square Feet (ASF) – By Department
 \item Table 12.  Distribution of Grades in Upper Division Courses – Percent of Total Enrolled and Average GPA 
\end{enumerate}
   
\noindent
Please also address the following issues, for which no data are provided:\\
\begin{enumerate}[a)]
 \setcounter{enumi}{7}
 \item Comment on the degree of interest and engagement of the faculty in the major.
 \item Comment on the adequacy of staff support for the major.
 \item Comment on the adequacy of staff advising for the major.
 \item Comment on the adequacy of instructional equipment and facilities for the major.
 \item Comment on the program’s record of keeping pace with advances in the field.
 \item Comment on any academic programs that share or compete for instructional, advising, or other resources with this major (e.g., a similar major, a masters program, or a doctoral program)
 \item Comment on the number of hybrid and virtual courses are offered by the department/program and the average number of hybrid and virtual courses students take to complete their major requirements. Comment on any establish departmental resources and best practices regarding hybrid and virtual instruction.
\end{enumerate}

a) Table 6 shows that student FTE per faculty FTE is increasing.  This is the result of both an increase in the number of physics majors and a decrease in the number of faculty.  
b) {\color{red} We don't understand the results in Table 7 and are investigating: we would expect even greater than $70\%$ of students enrolled in upper division coursework to be physics majors.} 
c) Table 8 shows an expected increase in the number of TAs that resulted from new courses like PHY 80 and PHY 40 that include large numbers of TA support for the lab sections, as well as the addition of discussion sections to some core upper division courses. (d-e) Tables 9 and 10 show that, despite the challenges from a shrinking faculty, physics majors are being taught by ladder faculty for $90\%$ of their upper division coursework, a larger fraction than any other CLAS department considered this cycle.  Undergraduate physics majors are being taught by ladder faculty for $58.5\%$ of their physics coursework, above average for CLAS.  (f) Table 11 shows that while total assignable space has increased, it has not been faster than the number of student FTEs has increase.  The overall amount of assignable space is above average for CLAS, but this reflects our need for dedicated laboratory space, and large lecture halls adjacent to demonstration support. (g) The grade distribution in Table 12 shows that physics (and chemistry) are grading significantly more strictly than CLAS overall.  We believe that we are in the right here.  Our top students are regularly accepted into top graduate schools because getting A's in our courses is a meaningful accomplishment.  We believe that we have this grade distribution correct.  Furthermore, we are concerned about the ability of UCD to continue as a forefront research institution and engine of social mobility if the extreme grade inflation evident in CLAS overall is allowed to continue.  

{\color{red}(h-n) TODO}

\section{Students in the major}
{\color{red} Questions: This section is intended to characterize the students in this major. How have enrollments in the major varied over the period of the review, in terms of both the numbers and quality of the students? Are students succeeding in the major both in terms of qualitative and quantitative academic standards? Are students graduating on time? Are there impacted classes (e.g., with limited offerings or long waitlists) or other bottlenecks that unnecessarily impede student success? How do students find out about the major?  Is the major reaching a wide and diverse spectrum of students? Are students who enter the major retained in the major, and if not, why not?}

Refer to the Tableau data concerning enrollments in the major and the other majors reviewed in the same cluster (Appendix B, Tables 13-23). Based on those data and any additional information you wish to include, comment on each of the following for your major over the review period, referring, when appropriate to differences between your major and others in the cluster:
\begin{enumerate}[a)]
    \item Table 13.  Number of Students - Duplicated Count and Percent Change 
    \item Table 14.  Students in Multiple Majors - Percent of Total in Major 
    \item Table 15.  Gender of Students – Percent of Total in Major and Percent Change 
    \item Table 16.  Under-represented Students – Percent of Total in Major and Percent Change 
    \item Table 17.  New Freshman Students Number and Percent Change 
    \item Table 18.  New Transfer Students Number and Percent Change 
    \item Table 19.  Average Cumulative UC Davis GPA 
    \item Table 20.  Students in Good Standing – Percent of Total by Level 
    \item Table 21.  Degrees Conferred – Duplicated Count and Percent Change 
    \item Table 22.  Time to Degree for Freshman and Transfer Students – All Students 
    \item Table 23.  Time to Degree for Freshman and Transfer Students – In Same Major 
    \item Table 24.  Graduating Students with a Minor
    \item Table 25.  Number of International Students
    \item In light of the information presented in Tables 13-23, describe and evaluate the effectiveness of any efforts by the program’s faculty and staff to retain students in the major.
\end{enumerate}

Please also address the following issue, for which no data are provided:
\begin{enumerate}[o)]
    \item Describe and evaluate how students find information about the major (websites, course catalog, etc.).
\end{enumerate}

a+e) In Table 13, we see an increase from 223 physics and applied physics majors to 310.  In Table 17, we see that the number of freshman physics and applied physics majors has increased from 58 to 94, with an even larger incoming class in 2021-2022.  This was a decision by the university, mutually agreed upon with the physics department, to admit more physics majors. 
b) In Table 14 we see that applied physics majors are less apt to take on a second major, which is not surprising as that major includes significant coursework outside of the department which is already difficulty to schedule.  We also see a significant increase in the number of double majors in the past three years for which we do not have an explanation.
c) In Table 15, we see that while the percentage of women in applied physics has increased, the percentage of women in physics has dropped by about 10\%.  We see a larger drop in mathematics and a smaller drop in chemistry.  
d) In Table 16, we that the fraction of physics majors that are members of URGs has increased by $52.9\%$.
f) In Table 18, we see that the number of transfer students in applied physics and physics has dropped dramatically since the introduction of selective major review in ({\color{red} date}).  Although there has been a campus wide drop in new transfer students, this drop is significantly larger.  We plan to watch this statistic for a few more years, as well as graduation rates and time to graduation for transfer students specifically, before considering further revisions to selective major review.





\section{Students’ Perceptions of the Major}
Question: What are current students’ and recent graduates’ opinions of the major?

Refer to the Tableau data obtained from surveys of current students and alumni concerning their perceptions of the quality of the major and the other majors reviewed in the same cluster (Appendix C, Figures 1-53). Based on those data and any additional information you wish to include (e.g., results of departmentally administered course evaluations), comment on each of the following for your major over the review period, referring, when appropriate to differences between your major and others in the cluster:

    a) overall understanding of the major (Figures 1-4)
    b) overall satisfaction with the major (Figures 5-22)
    c) satisfaction with instruction in the major (Figures 23-36)
    d) satisfaction with academic advising in the major (Figures 37-43)
    e) satisfaction with courses offered in the major (Figures 44-53)

Enter your text here.

\section{Post-graduate Preparation}
Questions: How well does the major prepare students for postgraduate education and careers? Do the students have opportunities to meet and work with faculty outside the classroom setting? Is there sufficient support for internships or experiential learning opportunities? Are there ample opportunities for students to learn about career options?  
Refer to the Tableau data obtained from surveys of current students and alumni concerning preparation by the major for postgraduate education and careers (Appendix C, Figures 54-80). Based on those data and any additional information you wish to include, comment on each of the following for your major over the review period, referring, when appropriate to differences between your major and others in the cluster:

    a) quantity and quality of research and creative activities provided by the major (Figures 54-59)
    b) quality of preparation by the major for postgraduate education (Figures 60-64)
    c) quality of preparation by the major for the workforce (Figures 65-74)
    d) the degree to which students have sufficient contact with faculty to help them in their postgraduate education and careers (Figures 75-80).

Enter your text here.

\section{Educational objectives and Assessment}

Question: How does the program monitor and evaluate its success in achieving its Program  Learning Outcomes (section 1)?  

Specifically:
    a) Please confirm that the PLOs are clearly listed in an easily accessible location on the program website and provide the URL for that website.
    b) Please provide a program curriculum matrix or map which identifies the required courses in which each PLO is specifically addressed and assessed. Attach sample syllabi for up to six of the required core courses for the major. An additional resource for completing the curriculum Map/Matrix is available on the UIPRC website.
    c) What unit (committee or officer) in your program is responsible for collecting and analyzing data on student progress toward PLOs?
    d) Please list the direct evidence of student learning used to assess student achievement of PLOs. Examples of direct evidence include projects, scores from exams or quizzes, and written work. For this data, describe the methodology for sample selection and size.  
    e) Please list sources of indirect data used to contextualize student achievement of PLOs. Examples of indirect data include student evaluations, peer evaluation of teaching, and the survey data from current students and graduates provided for this review.
    f) In what ways do the results of this self-review reveal particular areas of strength or weakness in student progress toward achieving PLOs?
    g) What changes will the program make based on the results of the program assessment of student learning?

Programs are encouraged to contact the Academic Assessment team (assessment@ucdavis.edu) in the Center for Educational Effectiveness (Office of Undergraduate Education) to learn more about assessment of student learning outcomes.

Enter your text here.

(a) {\color{red} We need to do this (put this on a webpage).}
(b) The required courses which relate to each PLO were already listed in Section 1.
({\color{red} TODO: we can make a little table here to make folks happy.})
(c) The vice-chair for the undergraduate program and the undergraduate curriculum, in collaboration with the physics faculty, are responsible for evaluating the performance of the undergraduate program.
As part of the curriculum update, we have developed a new procedure for maintaining the curriculum.  Each spring, the undergraduate curriculum committee organizes a faculty meeting devoted to reviewing the undergraduate program.  The instructors of core courses prepare short summaries of their courses, covering how they taught the course, including notable differences from the example syllabi, and any observed shortcomings in student preparation from prerequisite courses.  If any changes to the example syllabi are needed, the UGCC prepares new version, upon which the faculty vote.
(d) The acquisition of PLOs related to mathematics and theoretical physics are most directly assessed through final exams in core courses (e.g. 104A, 105AB, 110AB, 112, 115AB).  The experimental physics PLOs are assessed through student lab reports.  Instructors of PHY 80 have also recently introduced a lab practical examination which we have found to be quite successful.  The assessment of PLOs are based on submitted student programs, cross-checked by in-class evalution based on Parsons problems (correctly ordering provided lines of code to answer a prompt) and predicting the output of provide code snippets.
Because physics is quite hierarchical, there are multiple opportunities to assess students acquisition of the PLOs.  For one example, several years ago we learned that PHY 104A was not consistently covering vector calculus because the 110AB instructor reported having to review this material, and we made adjustments to address this deficiency.(e-f) {\color{red} TODO.}
 
\section{Major strengths and weaknesses/problems}
Summarize the major overall strengths of the program as well as any current problems that you perceive. 

Enter your text here.

\section{Future Plans}
Describe current or proposed plans to strengthen educational objectives of the program, such as increasing enrollments, improving student performance, and increasing the contribution of the program to the campus educational objectives. Comment on the long term strategy or goals regarding hybrid and virtual instruction in this program. Describe and justify if new resources are needed to preserve or strengthen the program.

Enter your text here.

	
\section{Minors}
Please comment on the minors currently administered by your department. Include all minors, and for each:

    a) Briefly explain the history of the minor.
    b) Briefly explain the rational for the minor, including the importance of the minor to students.
    c) Provide a brief overview of the minor, especially how students receive academic
guidance through the minor.
    d) Comment on numbers of students who have declared the minor during the review period.
    e) Describe the faculty participation in, and staff administrative support for, the minor.
    f) Describe your approach to ensuring that course listings for the minor are regularly updated and are realistic for students to complete in four years.
    g) Describe faculty participation in the minor. e.g., classroom instruction, co-ordination of
undergraduate events, supervision of undergraduate theses.
    h) Describe your approach to assessing the effectiveness of the minor on students’ learning and preparation for the future.
    i) Describe the current goals of the minor. Have goals changed over time? If so, explain
how the goals have changed.
    j) What are the strengths of the minor and benefits to students of the minor?
    k) What are the weaknesses of the minor? Describe any steps taken to address the
weaknesses.
    l) What are faculty members’ future plans for the minor?

Attachments:
    m) If it was not supplied in Appendix A, please attach the current catalog copy for the minor.
    n) Please provide a chart indicating the number of students who have declared the minor for each year in the review period.

Enter your text here.


\section{Emergency Remote Instruction}
The review team, UIPRC, and UGC understand that the emergency remote instructional environment required by the COVID-19 pandemic presented extraordinary demands on department faculty, staff, and students. Please address the successes and challenges faced by your program during emergency remote instruction. Provide information about any new practices that were beneficial and the program plans to continue, as well as any practices and outcomes that fell short of your standards.

Some items you may want to address:
    • Student experience: Student response to remote instruction; Student engagement with courses and department; Student interactions with faculty and staff (Zoom appointments, online office hours, etc.); Student retention of course information; Student progress toward degree
    • Faculty experience: Impacts of teaching remotely; Costs and benefits
    • TA experience: Interactions with faculty; Interactions with students
    • Staff / advising experience: Costs and benefits
    • Course equivalency: Degree of equivalency between normal and emergency remote instruction with regards to course learning outcomes and variability by learning activity (lecture courses, discussion courses, writing courses, laboratory courses, etc.)
    • Assessment of student learning: Strategies used for remote assignments and exams and their efficacy
    • Lessons learned for future virtual or hybrid instruction or advising in the program

Enter your text here.

Our faculty made a heroic effort during the COVID-19 pandemic, but the quality of instruction suffered tremendously, and we believe we still see evidence that students have not fully recovered.  In qualitative terms, we saw that the top 1/3 of students managed reasonable well with remote instruction, but the bottom fell out for the remaining 2/3.  We believe we saw a dramatic increase in cheating, and there is growing evidence that this was a widespread phenomenon.  We also believe that we are still seeing residual effects of the pandemic on current students.  Specifically, the level of mathematical preparation and study habits have not fully recovered to pre-pandemic levels.  These are subjective evaluations, but they have been widely observed by our faculty.

During emergency remote instruction, student engagement dropped precipitously.  Those brave instructors that continued live (via zoom) lectures were faced with a wall of black: students left their cameras off.  As a result, with few exceptions, faculty despised remote instruction.  But we did our best.  An approach that emerged as somewhat effective was providing a mixture of {\bf short} asynchronous pre-recorded lectures and live in-person recap discussions.  For faculty teaching lab courses, the burden was even greater.  These are crucial experiences for our students, and faculty generally went to extraordinary lengths to provide kits or other means for students to get some hands on experience, which required redesigning entire courses with little notice.  

And despite all of these efforts, there was no meaningful equivalency between the online courses we provided and in-person courses.  Any instructor has seen that students that do not attend lecture generally become demoralized, detached, and perform well below their abilities.  COVID forced something like that onto every one of our students.  May we never have to do that again.

\end{document}

