\documentclass[12pt]{article}
\usepackage{lmodern}
\usepackage[T1]{fontenc}
\usepackage[dvips,letterpaper,margin=0.75in,bottom=0.75in]{geometry}
\usepackage{cancel}
\usepackage{graphicx}
\usepackage{braket}
\usepackage{latexsym,amssymb,amsmath}
\usepackage{pdfpages}
\usepackage{xcolor}
\usepackage{capt-of}
\usepackage{amsmath}
\usepackage{cite}
\usepackage{hyperref}
\newcommand{\tcr}{\textcolor{red}}
\newcommand{\tcb}{\textcolor{blue}}


\usepackage[american,fulldiode]{circuitikz}
\tikzset{component/.style={draw,thick,circle,fill=white,minimum size =0.75cm,inner sep=0pt}}

\begin{document}
\ctikzset{bipoles/thickness=1}
\ctikzset{bipoles/length=.6cm}

%%%%%%%%%%%%%%%%%%%%%%%%%%%%%%%%%%%%%%%%%%%%%%%%%%%%%%%%%%%%%%%%
%%%%%%%%%%%%%%%%%%%%%%%%%%%%%%%%%%%%%%%%%%%%%%%%%%%%%%%%%%%%%%%%
\title{Physics Department: Undergraduate Instruction and Program Review and GE Assessment}
%%%%%%%%%%%%%%%%%%%%%%%%%%%%%%%%%%%%%%%%%%%%%%%%%%%%%%%%%%%%%%%%
%%%%%%%%%%%%%%%%%%%%%%%%%%%%%%%%%%%%%%%%%%%%%%%%%%%%%%%%%%%%%%%%

\maketitle

\section{Vital Information}

\noindent
{\bf Links:}\\
\href{https://academicsenate.ucdavis.edu/committees/undergraduate-council/uipr}{https://academicsenate.ucdavis.edu/committees/undergraduate-council/uipr (UIPR)}\\
\href{https://aggiedash.ucdavis.edu/\#/projects/5}{
https://aggiedash.ucdavis.edu/\#/projects/5 (Data on aggiedash)}\\
  
\noindent
{\bf General Education Assessment:}\\
Next Deadline: January 2, 2024\\
GE Committee Chair: Marina Ellefson Crowder (mecrowder@ucdavis.edu)\\
GE Committee Analyst: Theresa Costa (tacosta@ucdavis.edu)\\

\noindent
{\bf Undergraduate Instruction and Program Review:}\\
Dates of the review team visit: January 24-25, 2024\\
National Reviewer: David Sokoloff, University of Oregon\\
UC Davis Reviewer: Susan Gentry, Material Science and Engineering\\
Visit Itinerary Due: December 15 (DONE)
Self-Review Due:  January 2, 2024 

\section{Summary}

The Physics Department is undergoing an ``Undergraduate Instruction and Program Review'' and ``General Education Assessment'' as supervised by the Undergraduate Council Committee.  This is {\em not} WASC accreditation but it is crucial to maintaining it as part of our quality assurance process.\\[8pt]

\noindent
{\bf General Education Assessment:}\\
This assessment has two parts, in-depth assessments of specific courses by the review committee, and a self-assessment of all GE literacy courses.

The specific courses for in-depth assessments have been determined as:  PHY 122A (Writing Experience), 
AST 10G (Visual Literacy), AST 10C (Scientific Literacy).  Mulhearn has been working with the instructors of these courses in Spring and Fall quarters to gather the required documentation.  For each of these courses we will deliver a one page statement, an example assignment, example student work (one poor, one average, one above average), and the course syllabus.  Here's a summary of the current status:\\[10pt]
\begin{tabular}{llll}
\hline
Course   & GE Requirement      & Instructor    & Documents Collected \\
\hline
PHY 122A & Writing Experience  & Mathew Citron & Syllabus, Assignment, Student Work \\
AST 10G  & Visual Literacy     & Andrew Wetzel & Syllabus \\
AST 10C  & Scientific Literacy & Lori Lubin    & Syllabus, Assignment, Student Work\\
\hline
\end{tabular}\\[10pt]
All of the collected documents have been reviewed by Mulhearn and look adequate for our purposes.  We are only waiting on AST 10G which is in progress.  Mulhearn will need to prepare the one page statements for each of these courses.  {\bf The deadline for these materials is January 2, 2024.}

The self-assessment deliverable is quite simple:  we'll need to fill out a spreadsheet with the name and e-mail of a recent instructor who completed the self-assessment and the result of their assessment (whether the course meets requirements or will be modified).  I assume the deadline for this is the same as the material for the in-depth self-assessment, but I couldn't see this listed anywhere.
{\bf We are in good shape to finish the preparation for the GE Assessment before the deadline.}

{\bf Undergraduate Instruction and Program Review:}  This consists of a self-review, and a two-day visit by the Review Team.

There is a template for the self-review that includes these sections:
\begin{enumerate} 
 \item Overview of the major
 \item Outcome of Previous Program Review
 \item Faculty in the major
 \item Instruction, advising, and resources in the major
 \item Students in the major
 \item Students’ Perceptions of the Major
 \item Post-graduate Preparation
 \item Educational objectives and Assessment
 \item Major strengths and weaknesses/problems
 \item Future Plans
 \item Minors
 \item Emergency Remote Instruction
\end{enumerate}
Supporting data is available in from




I think the fact that we have had (at least one so far) undergradaute program yearly discussion will be useful as well.

The template also mentioned that the following data may be useful to the Review Team:
\begin{itemize}
\item data on program class sizes and for related or comparable programs on our campus, including courses offered at introductory, intermediate, and advanced levels
\item current and historical faculty-student ratios in your and related/comparable programs, at different levels
\item specific courses that serve large numbers of students in majors outside your program
\item impacted courses and measures of how impacted they are
\item retention rates for cohorts in the major (freshmen, transfers, upper-division)
\item transfers into the major from other majors, and what majors they transfer from
\item changes in the curriculum for the major over that past 10 years or more
\item the role of advisory boards (for example from industry) in shaping the major
\item top 5 employers or fields of employment for your graduates
\item top 5 graduate programs your graduates attend
\item the development of internship programs over the past 10 years or more
\item if your faculty have been recruited from elsewhere in the past seven years, from where have they been recruited, and if you have lost faculty to other institutions, where have they gone?
\end{itemize}
I'd like to see if I can get staff support to gather this data (but note that it is not required.)

Clearly the self-review is the largest bit of work required, but I think it is doable on this timescale.

As for the visit from the Review Team, we need to plan the itinerary for their two day trip which should include meeting with undergraduate students, committee chairs, other unique features of our department.  Here's what I am thinking so far:
\begin{itemize}
\item Undergraduate Students
\item Physics Club?
\item Undergraduate Researchers?
\item Instruction Support Team:  Course Demo Demo?
\item Tours (Get Chris cracking on clean up?)
\item Instructor Panel: 105,110,112,115
\item Capstone Courses Panel?
\item UCC Meeting
\item TA / hbar panel (graduate students)?
\item Computational Physics (40,45,110L,112L,115L)
\item DEI, Team Up, UCD Ignites
\item UPC Meeting?
\end{itemize}
We need to get an intinerary together for December 15.

\section{TODO Lists}

\noindent
For the General Education Assessment:
\begin{itemize}
\item Deadline:  January 2, 2024
\item Collect remaining materials for AST 10G this quarter (Mulhearn)
\item Write the one page department statements for the three courses (Mulhearn)
\item Identify instructors and work with them on self-assessment (Mulhearn)
\end{itemize}

\noindent
For the Self-Review Part of the Undergraduate Instruction and Program Review:
\begin{itemize}
\item Deadline:  January 2, 2024  
\item {\bf Gain access to the BIA data tables (promised in September)}
\item Make a first draft of the self-review (within one month?)
\item Work with staff to gather other data.
\item Complete the self-review and submit by the deadline
\end{itemize}

\noindent
For the Review Team visit:
\begin{itemize}
\item Date of Visit: January 24-25, 2024
\item Deadline for itinerary:  December 15, 2023  
\item Plan itinerary and submit by the deadline
\end{itemize}

I think it will look really bad if our curriculum updates are still stalled by January 24-25, or even January 2... need to get cracking on that too...


\end{document}
