\documentclass[12pt]{article}

\usepackage[dvips,letterpaper,margin=0.75in,bottom=0.5in]{geometry}
\usepackage{cite}
\usepackage{slashed}
\usepackage{graphicx}
\usepackage{amsmath}
\usepackage{braket}
\begin{document}

\newcommand{\ihbar}{\ensuremath{i \hbar}}
\newcommand{\Pss}{\ensuremath{\Psi^*}}
\newcommand{\dPsidt}{\ensuremath{ \frac{\partial \Psi}{\partial t} }}
\newcommand{\dPsidx}{\ensuremath{ \frac{\partial \Psi}{\partial x} }}
\newcommand{\ddPsidx}{\ensuremath{ \frac{\partial^2 \Psi}{\partial x^2} }}
\newcommand{\dPssdt}{\ensuremath{ \frac{\partial \Psi^*}{\partial t} }}
\newcommand{\dPssdx}{\ensuremath{ \frac{\partial \Psi^*}{\partial x} }}
\newcommand{\ddPssdx}{\ensuremath{ \frac{\partial^2 \Psi^*}{\partial x^2} }}

\newcommand{\dphidt}{\ensuremath{ \frac{d \phi}{dt} }}
\newcommand{\dpsidx}{\ensuremath{ \frac{d \psi}{dx} }}
\newcommand{\ddpsidx}{\ensuremath{ \frac{d^2 \psi}{dx^2} }}


\title{PHY 115A \\ Summary\\}
\author{Michael Mulhearn}

\section{Review of Chapter 1}

We are considering a particle of mass $m$, in one dimension $x$, in a potential $V$.

\begin{itemize}
\item The state of a particle is described by its wave function $\Psi(x,t)$.
\item The wave function $\Psi(x,t)$ is a solution to the Schr\"odinger Equation (SE):
\begin{equation}
\label{eqn:se}
\ihbar \dPsidt = - \frac{\hbar^2}{2 m}\ddPsidx + V \Psi
\end{equation}
\item The physical interpretation of the wave function is statistical.  The probability $P_{ab}(t)$ that the particle will be found between points $a$ and $b$ at time $t$ is:
\begin{equation}
\label{eqn:prob}
P_{ab}(t) = \int_a^b |\Psi(x,t)|^2 dx
\end{equation}
\item The wave functions are normalized such that
\begin{equation}
\int_{-\infty}^{+\infty} |\Psi(x,t)|^2 dx = 1
\end{equation}
\item Once a wave function is normalized, it will remain normalized for all time as a consequence of the Schr\"odinger equation.
\item Measurements of observable quantities are associated with operators.  Given an operator $\hat{o}$ we can calculate the expectation value for the observable $o$ as:
\begin{eqnarray}
\braket{o}  &=& \int_{-\infty}^{+\infty} \, \Psi^* \left[ \hat{o} \right] \Psi \,dx 
\end{eqnarray}
The operator acts toward its right.
\item For the position observable we have the position operator 
\begin{equation}
\hat{x} = x.
\end{equation}
\item For the momentum observable we have the momentum operator 
\begin{equation}
\hat{p} = - \ihbar \frac{\partial}{\partial x}.
\end{equation}
\item Any classical dynamical variable can be calculated from the position and momentum, and we can calculate its expectation value as:
\begin{eqnarray}
\braket{f(x,p)}  &=& \int_{-\infty}^{+\infty} \, \Psi^* \left[ f(\hat{x},\hat{p}) \right] \Psi \,dx 
\end{eqnarray}
\item Newtons law is still true, but only in an average sense, by Ehrenfest's Theorem:
\begin{eqnarray}
\frac{d \braket{p}}{dt} &=& - \braket{\frac{\partial V}{\partial x}}
\end{eqnarray}
\item The variance $\sigma^2_o$ of an observable $o$ can be calculated as:
\begin{equation}
\sigma_o^2 \equiv \braket{\,(o-\braket{o})^2\,} = \braket{o^2}-\braket{o}^2
\end{equation}
The square root of the variance is referred to as the uncertainty.  
\item The Heisenberg uncertainty relation states that:
\begin{equation}
\sigma_x \sigma_p \, \geq \, \frac{\hbar}{2}
\end{equation}

\end{itemize}

\section{Review of Chapter 2 Part A}

\begin{itemize}

\item If the potential is a function of position only ($V(x)$), and if we can find solutions to Schr\"odinger Equation (SE) which are separable:
\begin{equation}
\Psi(x,t) \; = \; \psi(x) \,\phi(t)
\end{equation}
then for these solutions:
\begin{equation}
\phi(t) \; = \; \exp\left( -i\frac{Et}{\hbar}\right)
\end{equation}
and $\psi(x)$ satisfies the Time-Independent Schr\"odinger Equation (TISE):
\begin{equation}
- \frac{\hbar^2}{2 m} \; \ddpsidx \, + \, V(x) \, \psi(x) = E \, \psi(x)\\
\end{equation}
The boundary conditions on the solutions $\psi(x)$ are that:
\begin{itemize}
\item $\psi(x)$ is always continuous.
\item $d\psi/dx$ is continuous except where the potential is infinite.
\end{itemize}
Other features:
\begin{itemize}
\item For normalizable solutions, the separation constant $E$ must be real.
\item $\psi(x)$ can always be taken as a real function.
\item If $V(x)$ is an even function, then the general solutions $\psi(x)$ can be constructed as either even or odd solutions.
\item For normalizable solutions, $E$ must be greater than the minimum value of $V(x)$.
\end{itemize}
\item We can write the TISE as:
\begin{equation}
\hat{H} \, \psi(x) = E \, \psi(x)
\end{equation}
where:
\begin{equation}
\hat{H} = H(\hat{x}, \hat{p})= -\frac{\hbar^2}{2m}\,\frac{\partial^2}{\partial x^2} + V(x)
\end{equation}
is the Hamiltonian operator, which is the operator for the total energy observable.  We say that the separable solutions are eigenstates of the Hamiltonian operator.

\item The separable solutions have some very nice properties indeed:
\begin{itemize}
\item They represent {\bf stationary states} for which
$$|\Psi(x,t)|^2 = | \psi(x) |^2$$
and so every expectation value is constant wrt time as well.  Also:
\begin{equation*}
\int_{-\infty}^{+\infty}\, |\psi(x)|^2 dx = 1\\
\end{equation*}
\item They represent {\bf states of definite total energy:} so:
$$\braket{H} = E$$
and:
$$\sigma^2_H = 0$$
every measurement of the total energy gives the same result $E$.
\item They are {\bf orthogonal and complete}, as describe in more detail below.
\end{itemize}

\item The normalizable wave functions are a vector space over the complex numbers, with an inner product:
\begin{equation}
\braket{f|g} \equiv \int_{-\infty}^{\infty} f^*(x) \, g(x) \, dx
\end{equation}
Expectation values are inner products:
\begin{equation}
\braket{O} = \braket{\Psi|\hat{O}\Psi} \equiv \int_{-\infty}^{\infty} \Psi^*(x,t)\hat{O}\Psi(x,t) dx
\end{equation}

\item The stationary states $\psi_n(x)$ with:
$$\hat{H} \psi_n(x) = E_n \psi_n(x)$$
can be arranged to be orthonormal:
\begin{equation}
\braket{\psi_n|\psi_m}=\delta_{mn}
\end{equation}
This is automatic if $E_n \neq E_m$ when $n \neq m$ (the allowed energies are nondegenerate) as in all the cases we studied in this chapter.  The stationary states are also complete:
\begin{equation}
\Psi(x,t) = \sum_n c_n \; \psi_n(x) \; \exp\left(\frac{-E_n t}{\hbar}\right)
\end{equation}
where 
$$\sum_n |c_n|^2 = 1$$
and each $|c_n|^2$ is the probability that a measurement of the total energy gives result $E_n$.  The coefficients can be determined from the initial state $\Psi(x,0)$ using Fourier's trick:
\begin{equation}
c_n = \braket{\psi_n| \Psi(x,0)}
\end{equation}
\item We define the commutator of two operators $\hat{A}$ and $\hat{B}$ as:
\begin{equation}
[\hat{A},\hat{B}]=\hat{A}\hat{B} - \hat{B}\hat{A}
\end{equation}
If the commutator of two operators is zero, we say they commute.  Otherwise, we say they do not commute.  The momentum and position operators do not commute:
\begin{equation}
[\hat{x},\hat{p}]=i\hbar
\end{equation}
which is known as the {\bf canonical commutation relation}.  In the homework, you showed that
for any operators $\hat{A}$, $\hat{B}$, and $\hat{C}$:
$$ [\hat{A}\hat{B}, \hat{C}] = \hat{A}[\hat{B}, \hat{C}] + [\hat{A}, \hat{C}]\hat{B}.$$

\item For an operator $\hat{O}$ we define it's Hermitian adjoint (or conjugate) $\hat{O}^\dagger$ by the property:
\begin{equation}
\braket{f|\hat{O^\dagger}g} = \braket{\hat{O} f| g}
\end{equation}
which implies:
\begin{equation}
\braket{f|\hat{O}g} = \braket{\hat{O^\dagger} f| g}
\end{equation}
When an operator is it's own Hermitian adjoint, we say the operator is Hermitian.  The expectation value of a hermitian operator are real, and so physical observables are always associated with hermitian operators.  In the homework, you showed that:
$$\left(\hat{A}^\dagger\right)^\dagger = \hat{A}$$
and:
$$\left(\hat{A}\hat{B}\right)^\dagger = \hat{B}^\dagger \, \hat{A}^\dagger.$$
for any operators $\hat{A}$ and $\hat{B}$.

\item We studied the solutions to the infinite square well potential:
\begin{equation}
V(x) = 
\begin{cases}    
   0 & 0 \leq x \leq b \\
   +\infty & {\rm otherwise} \\
\end{cases}   
\end{equation}
and found the allowed energies:
\begin{equation}
E_n = \frac{n^2 \pi^2 \hbar^2}{2mb^2}.
\end{equation}
We showed that the eigenfunctions are the sine functions of the Fourier series:
\begin{equation}
\psi_n(x) = 
\begin{cases}
{\displaystyle \sqrt{\frac{2}{b}} \, \sin(k_n x)} & 0 \leq x \leq b\\[8pt]
0 & \rm otherwise \\
\end{cases}
\end{equation}
where
\begin{eqnarray*}
k_n&=&\frac{n\pi}{b}
\end{eqnarray*}

\item We studied the solutions of the harmonic oscillator potential:
$$V(x) = \frac{1}{2}k x^2 = \frac{1}{2} m \omega^2 x^2$$
where:
$$\omega \equiv \sqrt{\frac{k}{m}}$$
and showed that the power series solution diverges if it does not terminate, and it only terminates if:
\begin{equation}
E_n = \hbar \omega \left( n + \frac{1}{2} \right)
\end{equation}
for
$$n=0,1,2,3,\ldots$$
We found explicit solutions in terms of the Hermite polynomials $H_n(u)$, which we can obtain from the generator function:
$$\exp(-z^2+2zu) = \sum_{n=0}^{\infty} \frac{z_n}{n!}H_n(u)$$
and the eigenfunctions are:
$$\psi_n(x) = \left( \frac{m \omega}{\pi \hbar} \right)^{1/4} \frac{1}{\sqrt{2^n n!}}H_n(u) e^{-u^2/2}$$.


\item We studied the ladder operators of the harmonic oscillator potential:
\begin{eqnarray}
\hat{a}_- &\equiv& \frac{1}{\sqrt{2}}\left(i \frac{\hat{p}}{p_0} + \frac{\hat{x}}{x_0}\right) \notag \\
\hat{a}_+ &\equiv& \frac{1}{\sqrt{2}}\left(-i \frac{\hat{p}}{p_0} + \frac{\hat{x}}{x_0} \right) \notag \\
\end{eqnarray}
where
$$x_0 \equiv \sqrt{\frac{\hbar}{m \omega}}$$
and
$$p_0 \equiv \frac{\hbar}{x_0} = \sqrt{\hbar m \omega}.$$
We showed that the Hamiltonian operator can be written as:
$$\hat{H} =  \hbar\omega \left( \hat{a}_+ \hat{a}_- + \frac{1}{2} \right)$$
And that the ladder operators satisfy the following commutation relationships:
\begin{eqnarray}
\left[\hat{a}_-\,,\,\hat{a}_+\right] \; &=& \;  1 \notag \\  
\left[\hat{H}\,,\,\hat{a}_+\right]   \; &=& \;  \hbar \omega \; \hat{a}_+ \notag \\ 
\left[\hat{H}\,,\,\hat{a}_-\right]   \; &=& \; -\hbar \omega \; \hat{a}_- \notag \\
\end{eqnarray}
Using the algebra of these commutators, we showed that the ladder operators act as raising and lowering operators that move up and down the stationary solutions:
\begin{eqnarray}
\hat{a}_+ \psi_n &=& \sqrt{n+1} \; \psi_{n+1} \notag \\
\hat{a}_- \psi_n &=& \sqrt{n} \; \psi_{n-1} \notag \\
\end{eqnarray}

\item We found the eigenstates of the Hamiltonian for the free particle ($V(x)=0$) as:
\begin{equation}
\Psi_k(x,t) = A e^{i (k x-\omega t)}
\end{equation}
where
\begin{equation}
k = \pm\frac{\sqrt{2mE}}{\hbar}
\end{equation}
and the sign of $k$ indicates the direction of the traveling wave:
\begin{eqnarray*}
k > 0 &\implies& {\rm wave~is~traveling~right} \\[5pt]
k < 0 &\implies& {\rm wave~is~traveling~left} \\
\end{eqnarray*}
while these solutions are not normalizable, we found that we can construct wave packets:
\begin{equation}
\Psi(x,t) = \frac{1}{\sqrt{2\pi}} \int_{-\infty}^{\infty} \; \widetilde{\Psi}(k,0) \; 
\exp\left[ik\left(x-\frac{\hbar k}{2m}t\right)\right]\; dk 
\end{equation}
where $\widetilde{\Psi}(k,0)$ is the Fourier Transform of the initial state:
\begin{equation}
\widetilde{\Psi}(k,0) = \frac{1}{\sqrt{2\pi}} \int_{-\infty}^{\infty} \; \Psi(x,0) \; e^{-ikx} \; dx 
\end{equation}
\end{itemize}

\section{Review of Fourier Series and Transform}

The main idea of this Appendix is that functions are (in a
mathematically rigourous sense) vectors in an abstract vector space,
with an inner product.  The precise definitions vary depending on what
type of function we are considering, but the general properties always
remain the same:
\begin{itemize}
\item The sines and cosines, or the closely related complex exponentials, form a complete orthnormal basis for the vector space.
\item That is, any function in the vector space can be written as a sum or integral of the basis functions times coefficients (or transforms). 
\item The coefficient (or transform) of each basis function can be determined using Fourier's trick, that is:  by calculating the inner product of a basis function with the original function.
\end{itemize}
We are left with three cases to consider, real periodic functions, complex-valued periodic functions, and normalizable complex-valued functions:
\begin{itemize}
\item For real periodic functions with period $a$, the inner product is:
\begin{equation}
\braket{f|g} \equiv \int_{-\frac{a}{2}}^{\frac{a}{2}} f(x) \, g(x) \, dx
\end{equation}
and the complete orthonormal basis functions are:
\begin{eqnarray}
s_n(x) &\equiv& \sqrt{\frac{2}{a}}\,\sin\left(\frac{2\pi n}{a} \, x \right)\\
c_n(x) &\equiv& \sqrt{\frac{2}{a}}\,\cos\left(\frac{2\pi n}{a} \, x \right)
\end{eqnarray}
which are defined for
\begin{eqnarray*}
n = 1,2,3,...
\end{eqnarray*}
plus the constant function:
\begin{eqnarray}
c_0(x) \equiv \sqrt{\frac{1}{a}}
\end{eqnarray}
Note that if it existed, $s_0(x) = 0$ would not be normalizable.

The orthonormality conditions are:
\begin{eqnarray}
\braket{s_n | s_m} &=& \delta_{nm}\notag\\[8pt]
\braket{c_n | c_m} &=& \delta_{nm}\notag\\[8pt]
\braket{s_n | c_m} &=& 0\notag\\[8pt]
\end{eqnarray}
for all $n$ and $m$, but take care that $c_0$ exists while $s_0$ does not.  
For compact notation we use the Kronecker delta symbol:
\begin{displaymath}
\delta_{nm} =  
\left\{
        \begin{array}{ll}
                1  & \mbox{if } n=m \\
                0 & \mbox{otherwise}
        \end{array}
\right.
\end{displaymath}

Any function in this vector space (that is, any real periodic function with period $a$) can be written as a summer of the basis vectors:
\begin{equation}
f(x) \; = \; \sum_{n=0}^{\infty}  A_n \, c_n(x)  \; \; + \; \; \sum_{n=1}^{\infty} B_n \, s_n(x) \label{eqn:bfs}
\end{equation}
and the coefficients can be determined with Fourier's trick:
\begin{eqnarray*}
A_n = \braket{c_n| f} \\
B_n = \braket{s_n| f}
\end{eqnarray*}

Explicitly in terms of sine and cosine functions, we have
\begin{equation}
f(x) = A_0 \sqrt{\frac{1}{a}} + \sqrt{\frac{2}{a}} \sum_{n=1}^{\infty}  \left[ A_n \, \cos\left(\frac{2\pi n}{a} \, x \right) + B_n \, \sin\left(\frac{2\pi n}{a} \, x \right)\right]\label{eqn :lfs}
\end{equation}
and:
\begin{eqnarray*}
A_0 &=& \sqrt{\frac{1}{a}} \int_{-\frac{a}{2}}^{\frac{a}{2}} f(x) \, dx \\
A_n &=& \sqrt{\frac{2}{a}} \int_{-\frac{a}{2}}^{\frac{a}{2}} 
\cos\left(\frac{2\pi n}{a} \, x \right) \, f(x) \, dx \\
B_n &=& \sqrt{\frac{2}{a}} \int_{-\frac{a}{2}}^{\frac{a}{2}} 
\sin\left(\frac{2\pi n}{a} \, x \right) \, f(x) \, dx \\
\end{eqnarray*}

\item For complex-valued functions with period $a$, the inner product is:
\begin{equation}
\braket{f|g} = \int_{-\frac{a}{2}}^{\frac{a}{2}} f^*(x) g(x) dx
\end{equation}
and the complete orthonormal basis functions are:
\begin{equation}
e_n(x) \equiv \sqrt{\frac{1}{a}} \exp\left( i \, \frac{2 \pi n x}{a}\right)
\end{equation}
with the orthormality properties:
\begin{equation}
\braket{e_n|e_m} = \delta_{nm}
\end{equation}
Any vector in this vector space (any complex-valued periodic function of period $a$) can be expressed as:
\begin{equation}
\psi(x) = \sum_{n=-\infty}^{\infty}  c_n \, e_n(x)  \\
\end{equation}
where the coefficients can be determined using Fourier's trick:
\begin{equation}
c_n = \braket{e_n|f}
\end{equation}

\item We carefully consider the limiting case of periodic functions with the period $a$ going to infinity, and concluded that any normalizable complex-valued function can be written as an integral over a complex exponential:
\begin{eqnarray}
\psi(x) &=& \frac{1}{\sqrt{2\pi}} 
\int_{-\infty}^{\infty} \; \widetilde{\psi}(k) \; e^{ikx} \; dk
\end{eqnarray}
where
\begin{eqnarray}
\psi(x) &=& \frac{1}{\sqrt{2\pi}} 
\int_{-\infty}^{\infty} \; \widetilde{\psi}(k) \; e^{ikx} \; dk
\end{eqnarray}
is the Fourier transform of $\psi(x)$.

(Later, we'll connect this much more explicitly with completeness and orthonormality, but we'll have to expand our concept of orthogonality a bit first.)
\end{itemize}
  
\end{document}






