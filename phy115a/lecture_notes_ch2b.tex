\documentclass[12pt]{book}

\usepackage[dvips,letterpaper,margin=0.75in,bottom=0.5in]{geometry}
\usepackage{cite}
\usepackage{slashed}
\usepackage{graphicx}
\usepackage{amsmath}
\usepackage{amssymb}
\usepackage{braket}
\begin{document}

\newcommand{\ihbar}{\ensuremath{i \hbar}}
\newcommand{\Pss}{\ensuremath{\Psi^*}}
\newcommand{\dPsidt}{\ensuremath{ \frac{\partial \Psi}{\partial t} }}
\newcommand{\dPsidx}{\ensuremath{ \frac{\partial \Psi}{\partial x} }}
\newcommand{\ddPsidx}{\ensuremath{ \frac{\partial^2 \Psi}{\partial x^2} }}
\newcommand{\dPssdt}{\ensuremath{ \frac{\partial \Psi^*}{\partial t} }}
\newcommand{\dPssdx}{\ensuremath{ \frac{\partial \Psi^*}{\partial x} }}
\newcommand{\ddPssdx}{\ensuremath{ \frac{\partial^2 \Psi^*}{\partial x^2} }}

\newcommand{\dphidt}{\ensuremath{ \frac{d \phi}{dt} }}
\newcommand{\dpsidx}{\ensuremath{ \frac{d \psi}{dx} }}
\newcommand{\ddpsidx}{\ensuremath{ \frac{d^2 \psi}{dx^2} }}


\title{PHY 115A \\ Lecture Notes 2B: \\ 
Tunneling and Scattering \\
(Griffith's 2.5-2.6,9.1-9.2)}
\author{Michael Mulhearn}

\maketitle

\setcounter{chapter}{1}
\chapter{Tunneling and Scattering}
\setcounter{section}{25}
\setcounter{equation}{70}

\section{Continuity of the Wave Function}

Let's start with the case $V(x)$ is finite everywhere, than we start from the TISE:
$$-\frac{\hbar^2}{2m}\frac{d^2 \psi}{d x^2} + V(x) \psi(x) = E \psi(x)$$
Without loss of generality, we'll investigate continuity at $x=0$, by integrating the TISE from $-\epsilon$ to $+\epsilon$:
$$\int_{-\epsilon}^{+\epsilon}\frac{d^2 \psi}{d x^2} dx = \frac{2m}{\hbar^2}\int_{-\epsilon}^{+\epsilon} \left(V(x) - E\right) \; \psi(x) \, dx$$
We'll assume that we keep $\epsilon > 0$ here and everywhere below.  By the fundamental theorem of calculus the LHS is:
\begin{equation}
\label{eqn:psicont}
\left. \frac{d\psi}{d x} \right\rvert_{+\epsilon} 
- \left. \frac{d\psi}{d x} \right\rvert_{-\epsilon}
 = \frac{2m}{\hbar^2}\int_{-\epsilon}^{+\epsilon} \left(V(x) - E\right) \; \psi(x) \, dx
\end{equation}
In the limit $\epsilon \to 0$, the RHS vanishes since $V(x)$ is finite, so:
$$ \lim_{\epsilon \to 0} \left( \left. \frac{d\psi}{d x} \right\rvert_{+\epsilon} 
- \left. \frac{d\psi}{d x} \right\rvert_{-\epsilon} \right) = 0$$
which is to say the derivative of the wave function is continuous, and so the wave function is continuous as well.

But what about infinite (or undefined) $V(x)$?  Here we still insist that the wave function be continuous, as otherwise the state of a particle would be undefined at some point.  But the derivative need not be continuous, as the $V(x)$ term in LHS in Equation~\ref{eqn:psicont} no longer vanishes in the limit $\epsilon \to 0$:
\begin{equation}
\label{eqn:psidiscont}
\lim_{\epsilon \to 0} \left( \left. \frac{d\psi}{d x} \right\rvert_{+\epsilon} 
- \left. \frac{d\psi}{d x} \right\rvert_{-\epsilon} \right) = 
\lim_{\epsilon \to 0}
\frac{2m}{\hbar^2}\int_{-\epsilon}^{+\epsilon} V(x) \; \psi(x) \, dx
\end{equation}

\section{The Dirac Delta Function}

The so-called ``Dirac Delta Function'' $\delta(x)$ is defined by it's behavior in an integral:
\begin{equation}
\int_{-\infty}^{+\infty} f(x) \, \delta(x) \, dx = f(0) 
\end{equation}
where it ``picks out'' the value of $f(x)$ at $x=0$.  It immediately follows (put $f(x)=1$) that:
\begin{equation}
\int_{-\infty}^{+\infty} \delta(x) \, dx = 1 
\end{equation}
Also, changing variables to make the subsitutions clearer:
$$\int_{-\infty}^{+\infty} g(y) \, \delta(y) \, dy = g(0)$$
and putting $y = x - a$, we get:
$$\int_{-\infty}^{+\infty} g(x-a) \, \delta(x-a) \, dy = g(0)$$
and defining $f(x) \equiv g(x-a)$ we have:
\begin{equation}
\int_{-\infty}^{+\infty} f(x) \, \delta(x-a) \, dy = f(a)
\end{equation}
The Dirac Delta Function isn't real a function at all, but it is often described as one:
$$\delta(x) = \begin{cases}
0, & x \neq 0 \\
\infty, & x=0 \\
\end{cases}
$$
but such a definition shouldn't be taken too seriously.  A better way to consider it is as a limit of perfectly reasonable functions with integral one, that get narrower and narrower around 0.  Just as the limit of a series of rational numbers can be an irrational number, the $\delta$-function is the limit of a sequence of integrable functions, but isn't itself square integrable.  We could try:
\begin{equation}
\int_{-\infty}^{+\infty} \delta^2(x) \, dy = \delta(0) 
\end{equation}
but what are we to make of $\delta(0)$? At best, we could say it is in infinity.  Mathematician's call the $\delta$-function a generalized function or distribution.  It only makes sense in the context of its defining integral equation above, and doesn't exist as a function on it's own.  If you think of what we actually do with wave functions (calculate integrals) this isn't really any limitation at all.

For $x \neq 0$, $\delta(x) = 0$ is well defined.  But otherwise, just stick to its well defined properties (the numbered equations here) within integrals, and we will see the $\delta$-function is extremely useful.

\section{Bound State of the Delta Function Potential}

We turn to the very useful example a delta function potential.
\begin{equation}
V(x) = -\alpha \delta(x) 
\end{equation}
Since we've agreed to never discuss the delta function at $x=0$ outside of an integral, we will just say that $V(x)$ does not have a defined minimum, and so we are free to see if there are normalizable solution with $E<0$.  

Away from $x=0$, where $\delta(x)$ is well defined, $V(x)=0$ and the TISE is:
\begin{equation*}
\frac{d^2 \psi}{d x^2} = -\frac{2mE}{\hbar^2}\psi(x) = \kappa^2 \psi(x), \hspace{2cm} \kappa \equiv \frac{\sqrt{-2mE}}{\hbar}.
\end{equation*}
which has general solutions:
$$\psi(x) = A e^{\displaystyle - \kappa x} + B e^{\displaystyle \kappa x}$$
But for the wave function to be well defined only:
$$\psi(x) = A e^{\displaystyle - \kappa x}, \hspace{2cm} x>0$$ 
and
$$\psi(x) = B e^{\displaystyle \kappa x}, \hspace{2cm} x<0$$
are acceptable.  From continuity of the wave function at $x=0$, we conclude:
$$A=B$$
and write $\psi(x)$ as:
$$\psi(x) = \begin{cases}
B e^{\displaystyle \kappa x} &  x\leq0 \\
B e^{\displaystyle -\kappa x} &  x\geq0 \\
\end{cases}
$$
We saw above that the presence of the $\delta$-function means the wave function need not be continuous at $x=0$, and in fact:
$$
\lim_{\epsilon \to 0} \left( \left. \frac{d\psi}{d x} \right\rvert_{+\epsilon} 
- \left. \frac{d\psi}{d x} \right\rvert_{-\epsilon} \right) = 
\lim_{\epsilon \to 0}
\frac{2m}{\hbar^2}\int_{-\epsilon}^{+\epsilon} V(x) \; \left( -\alpha \delta(x) \right) \, dx
$$
The $\delta$-function is well defined in the context of this integral, which can be evaluated as:
$$
\lim_{\epsilon \to 0} \left( \left. \frac{d\psi}{d x} \right\rvert_{+\epsilon} 
- \left. \frac{d\psi}{d x} \right\rvert_{-\epsilon} \right) = 
\lim_{\epsilon \to 0} \left( -\frac{2m\alpha}{\hbar^2} \psi(0) \right) = -\frac{2m\alpha}{\hbar^2} \psi(0) 
$$
In our case:
$$\psi(0) = B$$
and:
$$\frac{d\psi}{dx} = \begin{cases}
\kappa B e^{\displaystyle \kappa x} &  x\leq0 \\
-\kappa B e^{\displaystyle -\kappa x} &  x\geq0 \\
\end{cases}
$$
so:
$$
\lim_{\epsilon \to 0} \left( \left. \frac{d\psi}{d x} \right\rvert_{+\epsilon} 
- \left. \frac{d\psi}{d x} \right\rvert_{-\epsilon} \right) = -\kappa B - \kappa B = 
-\frac{2m\alpha}{\hbar^2} B 
$$
or
$$\kappa = \frac{m\alpha}{\hbar^2}$$
or
$$E = -\frac{\hbar^2 \kappa^2}{2m} = - \frac{m\alpha^2}{2\hbar^2}$$
Normalizing the wave function is left as an exercise, it yields:
$$|B|^2 = \kappa$$

\section{Scattering States of the Delta Function Well}

For the case that $E>0$, we have the free particle TISE for $x<0$:
\begin{equation*}
\frac{d^2 \psi}{d x^2} = -\frac{2mE}{\hbar^2}\psi(x) = -k^2 \psi(x), \hspace{2cm} k \equiv \frac{\sqrt{2mE}}{\hbar}.
\end{equation*}
with general solution:
$$\psi(x) = A e^{\displaystyle ikx} + B e^{\displaystyle -ikx}$$
Similarly, for $x>0$ the general solution is:
$$\psi(x) = F e^{\displaystyle ikx} + G e^{\displaystyle -ikx}$$
and so:
$$\psi(x) = \begin{cases}
A e^{\displaystyle ikx} + B e^{\displaystyle -ikx} &  x\leq0 \\
F e^{\displaystyle ikx} + G e^{\displaystyle -ikx} &  x\geq0 \\
\end{cases}
$$
and:
$$\frac{d\psi}{dx} = \begin{cases}
(iAk) e^{\displaystyle ikx} + (-iBk) e^{\displaystyle -ikx} &  x\leq0 \\
(iFk) e^{\displaystyle ikx} + (-iGk) e^{\displaystyle -ikx} &  x\geq0 \\
\end{cases}
$$
Continuity of $\psi(x)$ at $x=0$ requires:
$$F+G=A+B$$
and from:
$$
\lim_{\epsilon \to 0} \left( \left. \frac{d\psi}{d x} \right\rvert_{+\epsilon} 
- \left. \frac{d\psi}{d x} \right\rvert_{-\epsilon} \right) = 
-\frac{2m\alpha}{\hbar^2} \psi(0) 
$$
so:
\begin{eqnarray*}
ik(F-G-A+B) &=& -\frac{2m\alpha}{\hbar^2} (A+B) \\
F-G &=& (A-B)+i\frac{2m\alpha}{k\hbar^2} (A+B) \\
\end{eqnarray*}
Finally:
$$F-G = A (1+2i\beta) - B (1-2i\beta) $$
where:
$$\beta = \frac{m \alpha}{\hbar^2 k}$$

To measure scattering, let $A$ represent the (known) incident wave and set
$$G=0$$
so now we have two equations and two unknowns:
$$F = A + B$$
and:
$$F = A (1+2i\beta) - B (1-2i\beta) $$
Solving for $F$ in terms of $A$:
\begin{eqnarray*}
F &=& A (1+2i\beta) - (F-A)\,(1-2i\beta)\\
2F(1-i\beta) &=& 2A \\
\end{eqnarray*}
and so:
$$F = \frac{A}{1-i\beta}$$
similarlyy:
$$B = \frac{i\beta}{1-i\beta}$$

The reflection coefficient is:
$$R = \frac{|B|^2}{|A|^2} = \frac{\beta^2}{1+\beta^2}$$
and:
$$T = \frac{|F|^2}{|A|^2} = \frac{1}{1+\beta^2}$$
Notice that:
$$R+T=1$$

Now let's look at what happens for:
$$V(x) = +\alpha \delta(x)$$
Nothing changes until we reach the boundary condition on the derivative, which becomes:
$$
\lim_{\epsilon \to 0} \left( \left. \frac{d\psi}{d x} \right\rvert_{+\epsilon} 
- \left. \frac{d\psi}{d x} \right\rvert_{-\epsilon} \right) = 
+\frac{2m\alpha}{\hbar^2} \psi(0) 
$$
So we can read off the solutions for this case by putting $\beta \to -\beta$ in the solutions:


so the boundary conditions (keeping $G=0$) become:

\begin{eqnarray*}
F &=& A+B \\
F &=& A (1-2i\beta) - B (1+2i\beta)\\
\end{eqnarray*}




\section{The Finite Square Well}

\section{The WKB Approximation and the ``Classical'' Region}

\section{Tunneling in the WKB Approximation }


\section{The Fourier Transform Revisited}

Our inner product now extends between positive and negative infinity:
\begin{equation}
\braket{\Psi, \phi} \equiv \int_{-\infty}^{\infty} \Psi^*(x) \phi(x) \, dx
\end{equation}
Our basis functions, which are now defined for any value of $k$,
\begin{equation}
e_k = \frac{1}{\sqrt{2\pi}} \exp(i k x)
\end{equation}
are still orthonormal, but the condition looks a bit different in the continuum case:
\begin{eqnarray*}
\braket{e_k, e_{k'}} &=& \delta(k-k')
\end{eqnarray*}
See the appendix for more details on the Dirac delta function $\delta(x)$, which is zero everywhere but at $x=0$, where it is infinite.  It is the continuous version of $\delta_{nm}$.

Our basis functions are also still complete.  In the discrete case we have a complex Fourier coefficient for every integer $n$.   Now we have a complex Fourier coefficient for any real value of $k$.  In place of Fourier coefficients, we have instead a function of $k$ which we call the Fourier transform: $\widetilde{\Psi}(k)$.
Instead of a sum over discrete terms, we now have to integrate over all values of $k$:
\begin{equation} \label{eqn:ift}
\Psi(x) = \frac{1}{\sqrt{2\pi}} \int_{-\infty}^{\infty} \widetilde{\Psi}(k) \exp(ikx) \, dk.
\end{equation}
Just as in the discrete case, we determine the Fourier transform from the inner product:
\begin{equation} \label{eqn:ft}
\widetilde{\Psi}(k) = \braket{e_k, \Psi} = \frac{1}{\sqrt{2\pi}} \int_{-\infty}^{\infty} {\Psi}(x) \exp(-ikx) \, dx
\end{equation}
Equation~\ref{eqn:ft} is generally referred to as the {\em Fourier Transform}, while Equation~\ref{eqn:ift} is referred to as the {\em Inverse Fourier Transform}.

\section{The Fourier Transform in Quantum Mechanics}

So far we have been considering the Fourier transform with respect to position $x$ and wave-number $k$.  A much more useful pair of variables for Quantum Mechanics turns out to be momentum $p$ and position $x$.  To relate $p$ to $k$ we need only apply the DeBroglie relation to the wavelength in the definition of the wavenumber:
\begin{displaymath}
k \equiv \frac{2 \pi}{\lambda} = \frac{2 \pi p}{h} = \frac{p}{\hbar}
\end{displaymath}
We could therefore make the substitution $k \to p/\hbar$ (and $dk \to dp / \hbar)$) in Equations~\ref{eqn:ift} and ~\ref{eqn:ft}.  It turns out that a marginally more useful equation results if we make the normalization factors symmetric, by splitting the normalization factor of $1/\hbar$ across both equations with $1/\sqrt{\hbar}$ applied to each:
\begin{eqnarray} 
\Psi(x) &=& \frac{1}{\sqrt{2\pi\hbar}} \int_{-\infty}^{\infty} \widetilde{\Psi}(p) \exp(ipx/\hbar) \, dp \\
\widetilde{\Psi}(p) &=&  \frac{1}{\sqrt{2\pi\hbar}} \int_{-\infty}^{\infty} {\Psi}(x) \exp(-ipx/\hbar) \, dx
\end{eqnarray}
The major benefit of this symmetric form is that the normalization of $\Psi(x)$ and $\widetilde{\Psi}(p)$ in this case turns out to be the same:
\begin{displaymath}
\int_{-\infty}^{\infty} |\Psi(x)|^2 dx = \int_{-\infty}^{\infty} |\widetilde{\Psi}(p)|^2 dp = 1 
\end{displaymath}
Because we can always calculate $\Psi(x)$ from $\widetilde{\Psi}(p)$ either one completely describes the quantum mechanical state.  We call $\widetilde{\Psi}(p)$ the momentum wave function.   Whereas $|\Psi(x)|^2$ gives us the probability density for the quanton to be at position $x$, $|\Psi(p)|^2$ gives us the probability density for the quanton to have momentum $p$.


\end{document}




