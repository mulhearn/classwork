\documentclass[12pt]{article}


\usepackage[dvips,letterpaper,margin=0.75in,bottom=0.5in]{geometry}
\usepackage{cite}
\usepackage{slashed}
\usepackage{graphicx}
\usepackage{amsmath}
\usepackage{amssymb}
\usepackage{braket}
\usepackage[american,fulldiode]{circuitikz}

\begin{document}
\newcommand{\ihbar}{\ensuremath{i \hbar}}
\newcommand{\dPsidt}{\ensuremath{ \frac{\partial \Psi}{\partial t} }}
\newcommand{\dPsidx}{\ensuremath{ \frac{\partial \Psi}{\partial x} }}
\newcommand{\ddPsidx}{\ensuremath{ \frac{\partial^2 \Psi}{\partial x^2} }}
\newcommand{\dPssdt}{\ensuremath{ \frac{\partial \Psi^*}{\partial t} }}
\newcommand{\dPssdx}{\ensuremath{ \frac{\partial \Psi^*}{\partial x} }}
\newcommand{\ddPssdx}{\ensuremath{ \frac{\partial^2 \Psi^*}{\partial x^2} }}

\newcommand{\dphidt}{\ensuremath{ \frac{d \phi}{dt} }}
\newcommand{\dpsidx}{\ensuremath{ \frac{d \psi}{dx} }}
\newcommand{\ddpsidx}{\ensuremath{ \frac{d^2 \psi}{dx^2} }}


\date{\vspace{-5ex}}

\title{Homework Assignment 3 \\ Infinite Square Well Potential}

\maketitle

\section*{Practice Problems}

These problems are graded on effort only.\\

\noindent
{\bf Griffiths: P1.17, P2.1bc, P2.3} \\

\section*{Additional Problems}

\noindent
{\bf Problem 1:} Suppose that $\psi_n(x)$ is a solution to the TISE:
 $$- \frac{\hbar^2}{2 m} \; \frac{\partial^2 \psi_n}{\partial x^2} \, + \, V(x) \, \psi_n(x) = E_n \psi_n(x)\\$$
for 
$$n=1,2,3,\ldots$$  
Show that:
 $$ \Psi(x,t) \equiv \sum_{n=1}^{\infty} c_n \exp\left( -\frac{i \, E_n \, t}{\hbar}\right)\psi_n(x) $$
is a solution to the SE:
$$\ihbar \, \dPsidt \; = \; - \frac{\hbar^2}{2 m} \; \ddPsidx \, + \, V \, \Psi$$

\vskip 0.5cm

\noindent
{\bf Problem 2:} We found the complete set of orthonormal solutions to the TISE for the infinite square well potential to be:
\begin{equation}
\label{eqn:psin-isw}
\psi_n(x) = 
\begin{cases}    
   {\displaystyle \sqrt{\frac{2}{a}}\sin(\frac{n \pi x}{a})} & 0 {\displaystyle \leq x \leq a} \\[8pt]
   0 & {\rm otherwise} \\
\end{cases}   
\hspace{2cm} n=1,2,3,\ldots
\end{equation}
Show that if the particle is in any stationary state $\psi_n(x)$ then the expectation value of the position is:
$$\braket{x} = \frac{a}{2}$$
You may use the definite integral:
$$\int_0^{n\pi} \, x \, \sin^2 \, x \;dx = \frac{n^2 \, \pi^2}{4}$$.

\newpage

\noindent
{\bf Problem 3:} Suppose at time t=0, a particle is in the state
$$\Psi(x,0) = \frac{1}{\sqrt{2}} \psi_\alpha(x) + \frac{1}{\sqrt{2}} \psi_\beta(x)$$
where $\psi_\alpha(x)$ and $\psi_\beta(x)$ are two specific $\psi_n(x)$ from Problem 2 with definite energies $E_\alpha$ and $E_\beta$ and $\alpha \neq \beta$. Remember that $\psi_\alpha(x)$ and $\psi_\beta(x)$ are real functions.\\[5pt]

\noindent
(A) Write down the time-dependent wave function $\Psi(x,t)$  (You verified in Problem 1 that this will be a solution to the SE.)\\[5pt]

\noindent
(B) Calculate $|\Psi|^2$. You should find:
$$ |\Psi|^2 = \frac{|\Psi_\alpha|^2 + |\Psi_\beta|^2}{2} + \ldots $$
where the remaining bit is due to interference.\\[5pt]

\noindent
(C) Calculate the expectation value of the position $\braket{x}$ at any time $t$.  Make sure you use the results from Problem 2 so you should just have one not-so-bad integral to compute.  You may use:
$$\sin a \sin b = \frac{\cos(a-b) - \cos(a+b)}{2}$$
to help compute the integral.

\vskip 1cm
\noindent
{\bf Problem 4:} The axiomatic definition of a vector space $V$ and inner product space $H$ over the real numbers $\mathbb{R}$ is detailed in Table~\ref{tbl:ipspace} on the next page.\\[5pt]
    
\noindent
(A) Show that D1 follows from A1-5 and M1-5.\\  

Hint:  we already know for the scalars that $0+0=0$\\[5pt]

\noindent
(B) Show that D2 follows from A1-5, M1-5, and D1.\\

Hint: you need to show $x + (-1)x = 0$.  And we already know that $1+(-1)=0$.\\[5pt]

\noindent
(C) Show that D3 follows from I1 and I2.\\[5pt] 

\noindent
(D) Show that D4 follows from I1 and I3.\\[5pt]


\newpage

\begin{table}
\caption{ \label{tbl:ipspace} Here we define the properties of a vector space $V$ 
and an inner product space $H$.  Note that no complex conjugation appears in these definitions as the scalar field is the real numbers numbers.}
\begin{center}
{\bf Useful Math Symbols:}\\
\begin{tabular}{ll}
  $\forall \, x \in V$ & for all $x$ in $V$ (for any vector $x$)\\
  $\forall \, \alpha \in \mathbb{R}$ & for all $\alpha$ in $\mathbb{R}$ (for any real number $\alpha$) \\
%  $\exists y$ & there exists $y$ \\
  $\exists ! \, y$ & there exists unique $y$ \\
  s.t.          & such that \\
\end{tabular}\\    
\vskip 0.5cm
{\bf Properties of Addition:}\\
\begin{tabular}{llll}
{\bf A1} & {\bf Closure} & $\forall x,y \in V$ & $(x+y) \in V $\\
{\bf A2} & {\bf Commutative} & $\forall x,y \in V$ & $x+y=y+z$\\
{\bf A3} & {\bf Associative} & $\forall x,y,z \in V$ & $(x+y)+z = x+(y+z)$\\
{\bf A4} & {\bf Zero}        & $\exists !~0$~~s.t.~~ $\forall x \in V$ & $x+0 = x$ \\
{\bf A5} & {\bf Inverse} & $\forall x \in V \exists !\;(-x) \in V$~~s.t.~~& $x+(-x)=0$\\
\end{tabular} \\
\vskip 0.5cm
{\bf Properties of Scalar Multiplication:}\\
\begin{tabular}{llll}
  {\bf M1} & {\bf Closure} & $\forall x \in V$ and $\forall \alpha \in \mathbb{R}$ & $\alpha x \in V$\\
  {\bf M2} & {\bf Identity} & $\forall x \in V$ & $1x=x$\\
  {\bf M3} & {\bf Associative} & $\forall x \in V$
and $\forall \alpha,\beta \in \mathbb{R}$ & $\alpha(\beta x) = (\alpha \beta) x$\\
{\bf M4} & {\bf Distributive} & $\forall x,y \in V$and $\forall \alpha \in \mathbb{R}$ & $\alpha(x+y) = \alpha x + \alpha y$ \\
  {\bf M5} & {\bf Distributive} & $\forall x \in V$and $\forall \alpha,\beta \in \mathbb{R}$ & $ (\alpha + \beta)x = \alpha x + \beta x $ \\
\end{tabular}
\vskip 0.5cm
{\bf Deducible Properties:}\\
\begin{tabular}{lll}
{\bf D1}  & $\forall x \in V $  & $0x = 0$ \\
{\bf D2}  & $\forall x \in V $  & $(-1)x = (-x)$ \\
\end{tabular}
\vskip 0.5cm
{\bf Properties of Inner Products:}\\
\begin{tabular}{lll}
  {\bf I1} & $\forall x,y \in H$ & $\braket{x|y} = \braket{y|x}$\\
{\bf I2} & $\forall x,y,z \in H$and $\forall \alpha \in \mathbb{R}$ &
$\braket{x|\alpha y} = \alpha \braket{x|y}$\\
{\bf I3} & $\forall x,y,z \in H$ & $\braket{x+y|z} = \braket{x|z}+\braket{y|z}$\\
{\bf I4} & $\forall x \in H$ & $\braket{x|x} \geq 0$ \\
{\bf I5} & $\forall x \in H$ & $\braket{x|x}=0$ if and only if $x=0$ \\
\end{tabular}
\vskip 0.5cm
{\bf Deducible Properties:}\\
\begin{tabular}{llll}
{\bf D3} & & $\forall x,y \in H$and $\forall \alpha \in \mathbb{R}$ &
$\braket{\alpha x|y} = \alpha \braket{x|y}$\\
{\bf D4} & & $\forall x,y,z \in H$ &
$\braket{x|y+z} = \braket{x|y}+\braket{x|z}$\\
\end{tabular}
\end{center}
\end{table}

\newpage



       

\end{document}




