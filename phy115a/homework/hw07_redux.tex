\documentclass[12pt]{article}

\usepackage[dvips,letterpaper,margin=0.75in,bottom=0.5in]{geometry}
\usepackage{cite}
\usepackage{slashed}
\usepackage{graphicx}
\usepackage{amsmath}
\usepackage{amssymb}
\usepackage{braket}
\usepackage[american,fulldiode]{circuitikz}

\begin{document}
\newcommand{\ihbar}{\ensuremath{i \hbar}}
\newcommand{\dPsidt}{\ensuremath{ \frac{\partial \Psi}{\partial t} }}
\newcommand{\dPsidx}{\ensuremath{ \frac{\partial \Psi}{\partial x} }}
\newcommand{\ddPsidx}{\ensuremath{ \frac{\partial^2 \Psi}{\partial x^2} }}
\newcommand{\dPssdt}{\ensuremath{ \frac{\partial \Psi^*}{\partial t} }}
\newcommand{\dPssdx}{\ensuremath{ \frac{\partial \Psi^*}{\partial x} }}
\newcommand{\ddPssdx}{\ensuremath{ \frac{\partial^2 \Psi^*}{\partial x^2} }}

\newcommand{\dphidt}{\ensuremath{ \frac{d \phi}{dt} }}
\newcommand{\dpsidx}{\ensuremath{ \frac{d \psi}{dx} }}
\newcommand{\ddpsidx}{\ensuremath{ \frac{d^2 \psi}{dx^2} }}


\date{\vspace{-5ex}}

\title{Homework Assignment 7 \\ ``Just when I thought I was out...''}

\maketitle

\section*{Practice Problems}

These problems are graded on effort only.\\

\noindent
{\bf Griffiths: P2.44} \\[5pt]
{\bf Hint}: evaluate
$$\psi_2\frac{d\psi_1}{dx} - \psi_1\frac{d\psi_2}{dx}$$
at $x \to +\infty$ to determine it's constant value.
  
\section*{Additional Problems}

The first two problems are from the midterm exam.  If you are confident of
your solution there, you may just indicate ``done'' as your solution and we will substitute your exam solution.\\

\noindent
{\bf Problem 1:}  From the midterm exam:\\

A particle is in a certain potential $V(x)$ with corresponding Hamiltonian operator~$\hat{H}$.  Suppose that the properly normalized stationary states $\psi_1(x)$, $\psi_2(x)$, and $\psi_3(x)$ have definite energies:
$$E_1=2\,\epsilon, \hspace{1cm} E_2=3\,\epsilon, \hspace{1cm} E_3=5\,\epsilon$$
for some positive real constant $\epsilon$.  Using positive real constants when possible, construct a state $\Psi(x,t)$ with the properties:
$$\braket{\Psi|\psi_2} = 0$$
and:
$$\braket{\hat{H}} = 3\epsilon$$

\noindent
{\bf Problem 2:} From the midterm exam:\\

\noindent
Consider a particle in a harmonic oscillator potential with allowed energies:
\begin{equation}
E_n = \hbar \omega \left( n + \frac{1}{2} \right)
\end{equation}
for
$$n=0,1,2,3,\ldots$$
Consider a general solution:
$$\Psi(x,t) = \sum_n c_n \, \psi_n(x) \, e^{-i\,n\,\omega\,t}$$
Using the ladder operators $\hat{a}_+$ and $\hat{a}_-$, with the properties recalled in Problem 2, verify that Ehrenfest's Theorem holds in this case:
$$\frac{d}{dt}\braket{p} = -\braket{\frac{dV}{dx}} = -m\omega^2 \braket{x}$$
Suggested approach:\\
(A) Calculate $\braket{\psi_m|\hat{a}_+\psi_n}$ and $\braket{\psi_m|\hat{a}_-\psi_n}$ and express the answer in terms of $\delta_{ij}$.\\[5pt]
(B) Calculate $\braket{\Psi|\hat{a}_+\Psi}$ and $\braket{\Psi|\hat{a}_-\Psi}$ and leave your answer as an infinite series.\\[5pt]
(C) Calculate $\braket{p}$ and $\braket{x}$ from the (B).\\[5pt]
(D) Calculate:
$$\frac{d}{dt}\braket{p}$$
and hope that it all works out!

{\bf Additional Note:} You may have noticed that as we have defined the $c_n$ above, they have absorbed a time-dependent factor $e^{-i\omega t/2}$.  But then note also that $c_n^*c_m$ is still time-independent for any values of $n$ and $m$.  \\[5pt]

\noindent
{\bf Problem 3:}
Recall that the probability current for a wave function $\psi(x)$ is defined as:
$$J(x) \equiv \frac{i\hbar}{2m}\left( \psi \frac{d\psi^*}{dx} - \psi^* \frac{d\psi}{dx}\right)$$
In Problem 3 of the previous homework assignent (HW06) we considered the potential
$$V(x) = \begin{cases}
0   &  x < 0 \\
V_0 &  x \geq 0 \\
\end{cases}
$$
and assumed that $E>V_0$.  You showed that the general solution is:
$$\psi(x) = \begin{cases}
A e^{\displaystyle ikx} + B e^{\displaystyle -ikx} &  x\leq0 \\
F e^{\displaystyle i\eta kx} + G e^{\displaystyle -i \eta kx} &  x\geq0 \\
\end{cases}$$
where:
$$\eta \equiv \sqrt{\frac{E-V_0}{E}}.$$
In this problem, we will assume scattering from the left, i.e. $G=0$.\\[5pt]

\noindent
(A) Calculate the probability current $J_i$ for just the incoming wave:
$$\Psi_i = A e^{ikx}$$

\noindent
(B) Calculate the probability current $J_r$ for just the reflected wave:
$$\Psi_r = B e^{-ikx}$$

\noindent
(C) Define the coefficient of reflection as:
$$R \equiv \left\lvert \frac{J_r}{J_i} \right\rvert $$
and compare to your answer for Problem 3D of the previous homework.

\noindent
(D) Calculate the probability current $J_t$ for just the transmitted wave:
$$\Psi_t = F e^{i\eta kx}$$

\noindent
(E) Define the coefficient of transmission as:
$$T \equiv \left\lvert  \frac{J_t}{J_i} \right\rvert $$
and compare to your answer for Problem 3E of the previous homework.\\[5pt]

\noindent
(F) Calculate the current density $J_L$ for the complete solution for $x<0$.\\[5pt]

\noindent
(G) Calculate the current density $J_R$ for the complete solution for $x>0$. (We still have $G=0$)\\[5pt]

\noindent
(H) Set $J_L = J_R$ and show that:
$$T + R = 1$$

\end{document}




