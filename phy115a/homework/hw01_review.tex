\documentclass[12pt]{article}


\usepackage[dvips,letterpaper,margin=0.75in,bottom=0.5in]{geometry}
\usepackage{cite}
\usepackage{slashed}
\usepackage{graphicx}
\usepackage{amsmath}
\usepackage[american,fulldiode]{circuitikz}

\begin{document}
\ctikzset{bipoles/thickness=1}
\ctikzset{bipoles/length=.6cm}

\date{\vspace{-5ex}}

\title{Homework Assignment 1 \\ Review}

\maketitle

\section*{Practice Problems}

(No Practice Problems)

\section*{Additional Problems}

\vskip 1cm
\noindent
{\bf Problem 1:}  You have already covered\footnote{See Chapter 2, Section 11 of {\bf Mathematical Methods in the Physical Sciences (Third Edition)} by M.L.~Boas if you need a review.} linear algebra in PHY 104A.  Consider the real symmetric matrix:
\begin{displaymath}
  A =
  \begin{pmatrix}
    1 & 3 \\
    3 & 1 \\
  \end{pmatrix}
\end{displaymath}
(a) Find the eigenvalues of the matrix by finding the roots of the characteristic polynomial:
\begin{displaymath}
\det \left(A - \lambda I \right) = 0
\end{displaymath}  
(b) Find the two eigenvectors $\textbf{u}$ and $\textbf{v}$, and normalize them so that:
\begin{displaymath}
\textbf{u} \cdot \textbf{u} = \textbf{v} \cdot \textbf{v} = 1
\end{displaymath}  
(c) Note that $A=A^T$.  Show that the two eigenvectors are orthogonal, i.e.:
\begin{displaymath}
\textbf{u} \cdot \textbf{v} = 0
\end{displaymath}  
(d) For the vector:
\begin{displaymath}
  x =
    \begin{pmatrix}
    2 \\
    3 \\
  \end{pmatrix}
\end{displaymath}  
Find the values $a$ and $b$ such that:
\begin{displaymath}
x = a \textbf{u} + b \textbf{v}
\end{displaymath}  

\vskip 1cm
\begin{center}
Next problem on next page...
\end{center}

\newpage
\noindent
{\bf Problem 2:}  Show that:
\begin{eqnarray*}
\frac{2}{L} \int_{-\frac{L}{2}}^{\frac{L}{2}} 
\sin\left(\frac{2\pi n}{L} \, x \right) \sin\left(\frac{2\pi m}{L} \, x \right) \, dx &=& \delta_{nm} \label{eqn:trigortha} \\
 \frac{2}{L} \int_{-\frac{L}{2}}^{\frac{L}{2}} 
\cos\left(\frac{2\pi n}{L} \, x \right) \cos\left(\frac{2\pi m}{L} \, x \right) \, dx &=& \delta_{nm} \label{eqn:trigorthb}\\
\frac{2}{L} \int_{-\frac{L}{2}}^{\frac{L}{2}} 
\sin\left(\frac{2\pi n}{L} \, x \right) \cos\left(\frac{2\pi m}{L} \, x \right) \, dx &=& 0 \label{eqn:trigorthc}
\end{eqnarray*}
for integers $n$ and $m$ where:
\begin{displaymath}
\delta_{nm} =  
\left\{
        \begin{array}{ll}
                1  & \mbox{if } n=m \\
                0 & \mbox{otherwise}
        \end{array}
\right.
\end{displaymath}
You may use the trigonometric identities:
\begin{eqnarray*}
\sin \alpha \sin \beta &=& \frac{1}{2} \{\cos(\alpha - \beta) - \cos(\alpha + \beta)\}\\
\cos \alpha \cos \beta &=& \frac{1}{2} \{\cos(\alpha - \beta) + \cos(\alpha + \beta)\}\\
\cos \alpha \sin \beta &=& \frac{1}{2} \{\sin(\alpha + \beta) - \sin(\alpha - \beta)\}.\\
\end{eqnarray*}


\end{document}





You covered linear algebra in PHY 104A.  See Chapter 2, Section 11 of {\bf Mathematical Methods in the Physical Sciences (Third Edition)}, Boas if you need a review.  Consider the real symmetric matrix:
\begin{displaymath}
  A =
  \begin{pmatrix}
    1 & 3 \\
    3 & 1 \\
  \end{pmatrix}
\end{displaymath}
(a) Find the eigenvalues of the matrix by finding the roots of the characteristic polynomial:
\begin{displaymath}
\det \left(A - \lambda I \right)
\end{displaymath}  
(b) Find the two eigenvectors $u$ and $v$, and normalize them so that:
\begin{displaymath}
u \cdot u = v \cdot v = 1
\end{displaymath}  
(c) Note that $A=A^T$.  Show that the two eigenvectors are orthogonal, i.e.:
\begin{displaymath}
u \cdot v = 0
\end{displaymath}  
(d) For the vector:
\begin{displaymath}
  x =
    \begin{pmatrix}
    2 \\
    3 \\
  \end{pmatrix}
\end{displaymath}  
Find the values $a$ and $b$ such that:
\begin{displaymath}
x = a u + b v
\end{displaymath}  









Show that the root-mean-square average 
\begin{displaymath}
V_{\rm rms} \equiv \sqrt{<V^2(t)>}
\end{displaymath}
for a sinusoidal AC current given by:
\begin{displaymath}
V(t) = V_0 \sin(\omega t)
\end{displaymath}
is:
\begin{displaymath}
V_{\rm rms} = V_0 / \sqrt{2}
\end{displaymath}

\end{document}
