\documentclass[12pt]{article}


\usepackage[dvips,letterpaper,margin=0.75in,bottom=0.5in]{geometry}
\usepackage{cite}
\usepackage{slashed}
\usepackage{graphicx}
\usepackage{amsmath}
\usepackage{amssymb}
\usepackage{braket}
\usepackage[american,fulldiode]{circuitikz}

\begin{document}
\newcommand{\ihbar}{\ensuremath{i \hbar}}
\newcommand{\dPsidt}{\ensuremath{ \frac{\partial \Psi}{\partial t} }}
\newcommand{\dPsidx}{\ensuremath{ \frac{\partial \Psi}{\partial x} }}
\newcommand{\ddPsidx}{\ensuremath{ \frac{\partial^2 \Psi}{\partial x^2} }}
\newcommand{\dPssdt}{\ensuremath{ \frac{\partial \Psi^*}{\partial t} }}
\newcommand{\dPssdx}{\ensuremath{ \frac{\partial \Psi^*}{\partial x} }}
\newcommand{\ddPssdx}{\ensuremath{ \frac{\partial^2 \Psi^*}{\partial x^2} }}

\newcommand{\dphidt}{\ensuremath{ \frac{d \phi}{dt} }}
\newcommand{\dpsidx}{\ensuremath{ \frac{d \psi}{dx} }}
\newcommand{\ddpsidx}{\ensuremath{ \frac{d^2 \psi}{dx^2} }}


\date{\vspace{-5ex}}

\title{Homework Assignment 6 \\ Scatter There Thy Cheerful Beams!}

\maketitle

\section*{Practice Problems}

These problems are graded on effort only.\\

\noindent
{\bf Griffiths: P2.23, P2.25, P2.58} \\
Hint for 2.23b:  use integration by parts to move the derivative off of $\theta(x)$.

\section*{Additional Problems}

\noindent
{\bf Problem 1:} 
Consider the vector space of 2x1 column vectors, for example the vectors $x$ and $y$:
$$x = \begin{pmatrix} a \\ b \end{pmatrix}, \hspace{2cm } y = \begin{pmatrix} c \\ d \end{pmatrix}$$
with transpose:
$$x^T = \begin{pmatrix} a & b \end{pmatrix}$$
We can create an inner product space by defining the inner product as:
$$ \braket{x|y} \; \equiv \; \left( x^* \right)^T y \; = \; a^* \, c + b^* \, d$$

\noindent
(A) Verify that this definition does satisfy the properties of an inner product:\\

\begin{tabular}{lll}
  {\bf I1} & $\forall x,y \in H$ & $\braket{x|y} = \braket{y|x}^*$\\
{\bf I2} & $\forall x,y,z \in H$and $\forall \alpha \in \mathbb{C}$ &
$\braket{x|\alpha y} = \alpha \braket{x|y}$\\
{\bf I3} & $\forall x,y,z \in H$ & $\braket{x+y|z} = \braket{x|z}+\braket{y|z}$\\
{\bf I4} & $\forall x \in H$ & $\braket{x|x} \geq 0$ \\
{\bf I5} & $\forall x \in H$ & $\braket{x|x}=0$ if and only if $x=0$ \\
\end{tabular}

\vskip 1cm

\noindent
By definition, operators return a new vector for a given vector.  In this finite-dimensional vector space, any linear operator $O$ can be represented as a 2x2 matrix:
$$O \, x = 
\begin{pmatrix} O_{11} & O_{12} \\ O_{21} & O_{22} \end{pmatrix}
\begin{pmatrix} a \\ b \end{pmatrix}
= 
\begin{pmatrix} O_{11} \, a + O_{12}\, b\\ O_{21} \, a + O_{22} \, b \end{pmatrix}
$$

\noindent
(B) Show that the hermetian adjoint of $O$ is given by:
$$O^{\dagger} = \left(O^*\right)^T = \begin{pmatrix} O_{11}^* & O_{21}^* \\ O_{12}^* & O_{22}^* \end{pmatrix}$$
from our definition:
$$\braket{x|O^\dagger y} = \braket{Ox | y}$$
Hint:  calculate:
$$\braket{Ox|y}$$
then do whatever it takes to bring the action over onto the $y$ instead, and then read off $O^\dagger$.
You may use the property of matrices $A$ and $B$ that:
$$(AB)^T = B^T A^T$$

\vskip 1cm

\noindent
An operator $U$ is unitary if 
$$U^\dagger U \, = \, \begin{pmatrix} 1 & 0 \\ 0 & 1 \end{pmatrix} \equiv I$$
Note that:
$$Ix=x$$
for any vector $x$.\\

\noindent
(C) Show that the rotation matrix:
$$R \equiv  \begin{pmatrix} \cos\theta & -\sin\theta \\ \sin\theta & \cos\theta \end{pmatrix}$$
is unitary.

\vskip 1cm

\noindent
(D) Show that for a unitary matrix $U$:
$$\braket{Ux|Uy} = \braket{x|y}$$


\newpage


\noindent
{\bf Problem 2:} 
In lecture we studied the scattering states (with $E>0$) of the delta-function potential:
$$V(x) = -\alpha \delta(x)$$
We found that the general solution:
$$\psi(x) = \begin{cases}
A e^{\displaystyle ikx} + B e^{\displaystyle -ikx} &  x\leq0 \\
F e^{\displaystyle ikx} + G e^{\displaystyle -ikx} &  x\geq0 \\
\end{cases}
$$
has boundary conditions:
$$F+G = A + B$$
and
$$F-G = A (1+2i\beta) - B (1-2i\beta) $$
where:
$$\beta = \frac{m \alpha}{\hbar^2 k}$$
Notice that the waves (with coefficients) $A$ and $G$ are incoming, while the waves $B$ and $F$ are outgoing.  They are connected by the scattering matrix $S$:
$$
\begin{pmatrix} B \\ F \end{pmatrix}
=
\begin{pmatrix} S_{11} & S_{12} \\ S_{21} & S_{22} \end{pmatrix}
\begin{pmatrix} A \\ G \end{pmatrix}
$$

\noindent
(A) Calculate the $S$-matrix from the boundary conditions.\\

\noindent
Hint:  Put $B$ and $F$ on the LHS of the boundary conditions and $A$ and $G$ on the RHS.  Eliminate $F$ and solve for $B$ in terms of $A$ and $G$.  Then read off:
$$B = S_{11} \, A + S_{12} \, G$$

\noindent
(B) Show that the $S$-matrix you calculated in (A) is unitary.\\

\noindent
(C) Calculate the probability current:
$$J(x) = \frac{i\hbar}{2m} \left(\psi\frac{d\psi^*}{dx} - \psi^*\frac{d\psi}{dx} \right)$$
for $x<0$.\\
Hint: this explodes a bit but you should get nice cancellation leaving you just two terms in your answer.\\

\noindent
(D) Also calculate the probability current $J(x)$ for $x>0$.\\

\noindent
Hint: do not calculate this again from scratch!  Use your results from (C) and substitution!

\vskip 1cm

\noindent
(E) For normalizable solutions, to the SE, we showed that:
$$J(a) = J(-a)$$
in the limit $a \to \infty$.  These are not normalizable solutions, but let's assume still that the probability current you calculated in (C) equals the probability current you calculated in (D).  Calculate a condition on $|A|^2$, $|B|^2$, $|F|^2$, and $|G|^2$ that results from this.

\noindent
(F) Define the outgoing waves $O$ and the incoming waves $I$ as:
$$O = \begin{pmatrix} B \\ F \end{pmatrix}, \hspace{2cm } I = \begin{pmatrix} A \\ G \end{pmatrix}$$
Calculate a condition on $|A|^2$, $|B|^2$, $|F|^2$, and $|G|^2$ from:
$$\braket{O|O} = \braket{I|I}$$
Compare to your condition in (E).\\

\noindent
(G) Show that:
$$\braket{O|O} = \braket{I|I}$$
implies that $S$ is unitary.\\[5pt]
Hint: use $O = SI$.

\vskip 1cm

\noindent
{\bf Problem 3:}
Consider the potential: 
$$V(x) = \begin{cases}
0   &  x < 0 \\
V_0 &  x \geq 0 \\
\end{cases}
$$
and assume $E>V_0$.\\

\noindent
(A) Show that the general solutions to the SE can be written as:
$$\psi(x) = \begin{cases}
A e^{\displaystyle ikx} + B e^{\displaystyle -ikx} &  x\leq0 \\
F e^{\displaystyle i\eta kx} + G e^{\displaystyle -i \eta kx} &  x\geq0 \\
\end{cases}$$
where:
$$\eta \equiv \sqrt{\frac{E-V_0}{E}}$$

\noindent
(B) Determine the boundary conditions at $x=0$ and use these conditions to determine the scattering matrix $S$ defined by:
$$
\begin{pmatrix} B \\ F \end{pmatrix}
=
\begin{pmatrix} S_{11} & S_{12} \\ S_{21} & S_{22} \end{pmatrix}
\begin{pmatrix} A \\ G \end{pmatrix}
$$\\

\noindent
(C) Is $S$ unitary? Is there anything unphysical about this potential which might explain this?\\

\noindent
(D) Calculate the reflection from the left:
$$R = \left. \frac{|B|^2}{|A|^2} \; \right\rvert_{G=0} $$

\noindent
(E) And transmission from the left by:

$$T = 1-R$$

\noindent
(F) Calculate:
$$\left. \frac{|F|^2}{|A|^2} \; \right\rvert_{G=0} $$
Is this the same as $T$?

\end{document}




