\documentclass[12pt]{article}

\usepackage[dvips,letterpaper,margin=0.75in,bottom=0.5in]{geometry}
\usepackage{cite}
\usepackage{slashed}
\usepackage{graphicx}
\usepackage{amsmath}
\usepackage{amssymb}
\usepackage{braket}
\usepackage[american,fulldiode]{circuitikz}

\begin{document}
\newcommand{\ihbar}{\ensuremath{i \hbar}}
\newcommand{\dPsidt}{\ensuremath{ \frac{\partial \Psi}{\partial t} }}
\newcommand{\dPsidx}{\ensuremath{ \frac{\partial \Psi}{\partial x} }}
\newcommand{\ddPsidx}{\ensuremath{ \frac{\partial^2 \Psi}{\partial x^2} }}
\newcommand{\dPssdt}{\ensuremath{ \frac{\partial \Psi^*}{\partial t} }}
\newcommand{\dPssdx}{\ensuremath{ \frac{\partial \Psi^*}{\partial x} }}
\newcommand{\ddPssdx}{\ensuremath{ \frac{\partial^2 \Psi^*}{\partial x^2} }}

\newcommand{\dphidt}{\ensuremath{ \frac{d \phi}{dt} }}
\newcommand{\dpsidx}{\ensuremath{ \frac{d \psi}{dx} }}
\newcommand{\ddpsidx}{\ensuremath{ \frac{d^2 \psi}{dx^2} }}


\date{\vspace{-5ex}}

\title{Homework Assignment 8 \\ ``What a long strange trips it's been''}

\maketitle

\section*{Practice Problems}

These problems are graded on effort only.\\

\noindent
{\bf Problem 1:} Reproduce the derivation of generalized Ehrenfest theorem (Eq. 3.73).\\[5pt]

\noindent
{\bf Griffiths: P3.18, P3.19, P3.25, P3.26} \\
  
\section*{Additional Problems}

\noindent
{\bf Problem 2:} The Parseval-Plancherel identity:
$$\int_{-\infty}^{+\infty} |\Psi(x)|^2 \, dx \; = \; \int_{-\infty}^{+\infty} |\widetilde{\Psi}(p)|^2 \, dp$$
shows that if a position-space wave function $\Psi(x)$ is normalized, then so is it's momentum-space wave function $\widetilde{\Psi}(p)$.\\[5pt]

\noindent
(A) Derive the Parseval-Plancherel identity by explicit integration.  {\bf Hint:  } start with the LHS and write out $\Psi(x)$ and $\Psi^*(x)$ in terms of their Fourier transforms.  Make sure you use two different dummy variables for the integral over momentum (e.g. dq and dp) .  At this point, you should have {\bf three} integrals: $dx$,$dp$, and $dq$.  Look carefully and note that only two factors involve $x$.  Move the integral $dx$ to include just those factors, and use the orthogonality of the complex exponentials:
$$\frac{1}{2\pi\hbar}\int_{-\infty}^{+\infty} \exp((q-p)x/\hbar) \, dx = \delta(q-p)$$
This kills the integral over $dx$, then use the resulting $\delta$-function to kill the integral of dq.\\[8pt]

\noindent
(B) Derive Parseval-Plancherel identity using the Dirac notation.  Start with:
$$\braket{\Psi|\Psi}$$
then insert the identity:
$$\int dx \ket{x}\bra{x} = 1$$
and identify:
$$\Psi(x) = \braket{x|\Psi}$$
Next, do the same thing but using the identity:
$$\int dp \ket{p}\bra{p} = 1$$
and identify:
$$\widetilde{\Psi}(p) = \braket{p|\Psi}$$

\end{document}




