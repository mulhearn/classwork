\documentclass[12pt]{article}


\usepackage[dvips,letterpaper,margin=0.75in,bottom=0.5in]{geometry}
\usepackage{cite}
\usepackage{slashed}
\usepackage{graphicx}
\usepackage{amsmath}
\usepackage{amssymb}
\usepackage{braket}
\usepackage[american,fulldiode]{circuitikz}

\begin{document}
\newcommand{\ihbar}{\ensuremath{i \hbar}}
\newcommand{\dPsidt}{\ensuremath{ \frac{\partial \Psi}{\partial t} }}
\newcommand{\dPsidx}{\ensuremath{ \frac{\partial \Psi}{\partial x} }}
\newcommand{\ddPsidx}{\ensuremath{ \frac{\partial^2 \Psi}{\partial x^2} }}
\newcommand{\dPssdt}{\ensuremath{ \frac{\partial \Psi^*}{\partial t} }}
\newcommand{\dPssdx}{\ensuremath{ \frac{\partial \Psi^*}{\partial x} }}
\newcommand{\ddPssdx}{\ensuremath{ \frac{\partial^2 \Psi^*}{\partial x^2} }}

\newcommand{\dphidt}{\ensuremath{ \frac{d \phi}{dt} }}
\newcommand{\dpsidx}{\ensuremath{ \frac{d \psi}{dx} }}
\newcommand{\ddpsidx}{\ensuremath{ \frac{d^2 \psi}{dx^2} }}


\date{\vspace{-5ex}}

\title{Practice Problems}

\maketitle

No solutions will be posted for these practice problems.. they are
intended to focus your mind on the material and you should review the
material and work these problems until you are fully confident in your
abilities!

\noindent
{\bf Problem 1:}
A particle is in an infinite square well potential of width $b$ for which we know that the allowed energies are:
$$E_n = \frac{n^2 \pi^2 \hbar^2}{2mb^2}$$
and the orthonormal stationary states are:
$$
\psi_n(x) = 
\begin{cases}    
   {\displaystyle \sqrt{\frac{2}{b}}\sin(k_n x)} & 0 {\displaystyle \leq x \leq b} \\[8pt]
   0 & {\rm otherwise} \\
\end{cases}   
$$
At time $t=0$, the particle is in a state where the wavefunction within the square well is:
$$\Psi(x,0) = A\left( \psi_1(x) + i\psi_2(x) + \psi_3(x) \right)$$
\noindent
(A) What is the value of the constant $A$ which correctly normalizes $\Psi(x,0)$, if we take it to be positive and real?\\[5pt]
(B) What is the probability that a measurement of the particles total energy at $t=0$ would yield the value $E_2$?
(C) What is the probability that a measurement of the particles total energy at $t=0$ would yield the value $9*E_1$?
(D) What is the probability that a measurement of the particles total energy at $t=0$ would yield the value $7*E_1$?
(E) What is the probability that a measurement of the particles total energy at $t=0$ would yield the value $16*E_1$?
(F) What is the expectation value for the total energy divided by E_1:
$$\frac{\hat{H}}{E_1}$$
(G) Write down the time-dependent wave function $\Psi(x,t)$, using $E_1$ when possible. 

\noindent
{\bf Problem 2:}
A particle is in a harmonic oscillator potential




in state:
$$\Psi(x,0) = A\left( \psi_3(x) + 2\psi_4(x) \right)$$

(A) What is the value of the constant $A$ which correctly normalizes $\Psi(x,0)$, if we take it to be positive and real?\\[5pt]

(B) Calculate the expectation value of $x$ using an algebraid trick involving ladder operators:













for
$$n=1,2,3,\dots$$
and the orthonormal stationary states are:
$$
\psi_n(x) = 
\begin{cases}    
   {\displaystyle \sqrt{\frac{2}{b}}\sin(k_n x)} & 0 {\displaystyle \leq x \leq b} \\[8pt]
   0 & {\rm otherwise} \\
\end{cases}   
$$
where
$$k_n =\frac{n\pi}{b}$$
Suppose that at $t=0$, within the well ($0 \leq x \leq b$) the wave function for the particle is given by:
$$\Psi(x,0) = A \left( \sin(2\pi x/b) + i\sin(3\pi x/b)\right)$$

\noindent
(A) What is the positive real value of $A$ that correctly normalizes $\Psi(x,0)$?\\[5pt]

\noindent
(B) What is the probability that a measurement of the total energy will give a value of $7E_1$?\\[5pt]

\noindent
(C) What is the probability that a measurement of the total energy will give a value of $9E_1$?\\[5pt]

\noindent
(D) What is the expectation value of the energy? (Answer as a factor times $E_1$)\\[5pt]

\noindent
(E) Write down the time-dependent wave function $\psi(x,t)$.  (You may use $E_1$ as a parameter in your answer if you like.)\\[5pt]

\newpage

\noindent
{\bf Problem 2:} A particle is in a harmonic oscillator potential.  We studied the ladder operators:
\begin{eqnarray}
\hat{a}_- &\equiv& \frac{1}{\sqrt{2}}\left(i \frac{\hat{p}}{p_0} + \frac{\hat{x}}{x_0}\right) \notag \\
\hat{a}_+ &\equiv& \frac{1}{\sqrt{2}}\left(-i \frac{\hat{p}}{p_0} + \frac{\hat{x}}{x_0} \right) \notag \\
\end{eqnarray}
where
$$x_0 \equiv \sqrt{\frac{\hbar}{m \omega}}$$
and
$$p_0 \equiv \frac{\hbar}{x_0} = \sqrt{\hbar m \omega}$$
These operators have the commutator:
$$\left[\hat{a}_-\,,\,\hat{a}_+\right] \; = \;  1$$
In the homework, we used the trick of writing:
\begin{eqnarray*}
\hat{x} &=& \frac{x_0}{\sqrt{2}} \left(\hat{a}_+ + \hat{a}_-\right)\\
\hat{p} &=& \frac{i\,p_0}{\sqrt{2}} \left(\hat{a}_+ - \hat{a}_-\right)\\
\end{eqnarray*}

\noindent
(A) Show that:
$$[\hat{x}, \hat{p}] = i\hbar$$
follows from the commutation relationship for $\hat{a}_+$ and $\hat{a}_-$


\end{document}




