\documentclass[12pt]{article}


\usepackage[dvips,letterpaper,margin=0.75in,bottom=0.5in]{geometry}
\usepackage{cite}
\usepackage{slashed}
\usepackage{graphicx}
\usepackage{amsmath}
\usepackage{amssymb}
\usepackage{braket}
\usepackage[american,fulldiode]{circuitikz}

\begin{document}
\newcommand{\ihbar}{\ensuremath{i \hbar}}
\newcommand{\dPsidt}{\ensuremath{ \frac{\partial \Psi}{\partial t} }}
\newcommand{\dPsidx}{\ensuremath{ \frac{\partial \Psi}{\partial x} }}
\newcommand{\ddPsidx}{\ensuremath{ \frac{\partial^2 \Psi}{\partial x^2} }}
\newcommand{\dPssdt}{\ensuremath{ \frac{\partial \Psi^*}{\partial t} }}
\newcommand{\dPssdx}{\ensuremath{ \frac{\partial \Psi^*}{\partial x} }}
\newcommand{\ddPssdx}{\ensuremath{ \frac{\partial^2 \Psi^*}{\partial x^2} }}

\newcommand{\dphidt}{\ensuremath{ \frac{d \phi}{dt} }}
\newcommand{\dpsidx}{\ensuremath{ \frac{d \psi}{dx} }}
\newcommand{\ddpsidx}{\ensuremath{ \frac{d^2 \psi}{dx^2} }}


\date{\vspace{-5ex}}

\title{Practice Problems}

\maketitle

No solutions will be posted for these practice problems.. they are intended to focus your mind on the material.\\[5pt]

\noindent
{\bf Problem 1:}
A particle is in an infinite square well potential of width $b$ for which we know that the allowed energies are:
$$E_n = \frac{n^2 \pi^2 \hbar^2}{2mb^2}$$
for
$$n=1,2,3,4,\ldots$$
and the orthonormal stationary states are:
$$
\psi_n(x) = 
\begin{cases}    
   {\displaystyle \sqrt{\frac{2}{b}}\sin(k_n x)} & 0 {\displaystyle \leq x \leq b} \\[8pt]
   0 & {\rm otherwise} \\
\end{cases}   
$$
where
\begin{eqnarray*}
k_n&=&\frac{n\pi}{b}
\end{eqnarray*}
At time $t=0$, the particle is in a state where the wavefunction within the square well is:
$$\Psi(x,0) = A\left( \psi_1(x) + i\psi_2(x) + \psi_3(x) \right)$$
\noindent
(A) What is the value of the constant $A$ which correctly normalizes $\Psi(x,0)$, if we take it to be positive and real?\\[5pt]
(B) What is the probability that a measurement of the particles total energy at $t=0$ would yield the value $E_2$?\\[5pt]
(C) What is the probability that a measurement of the particles total energy at $t=0$ would yield the value $9\,E_1$?\\[5pt]
(D) What is the probability that a measurement of the particles total energy at $t=0$ would yield the value $7\,E_1$?\\[5pt]
(E) What is the probability that a measurement of the particles total energy at $t=0$ would yield the value $16\,E_1$?\\[5pt]
(F) What is the expectation value for the total energy divided by $E_1$:
$$\frac{\braket{\hat{H}}}{E_1}$$
(G) Write down the time-dependent wave function $\Psi(x,t)$.\\[5pt]
(H) What is the probability that a measurement of the particles total energy at time $t$ would yield the value $9\,E_1$?\\[5pt]

\noindent
{\bf Problem 2:}
A particle is in a harmonic oscillator potential:
$$V(x) = \frac{1}{2}k x^2 = \frac{1}{2} m \omega^2 x^2$$
where:
\begin{equation}
E_n = \hbar \omega \left( n + \frac{1}{2} \right)
\end{equation}
for
$$n=0,1,2,3,\ldots$$
and the stationary states of the Hamiltonian are:
$$\psi_n(x)$$
This problem will use the ladder operators:
\begin{eqnarray}
\hat{a}_- &\equiv& \frac{1}{\sqrt{2}}\left(i \frac{\hat{p}}{p_0} + \frac{\hat{x}}{x_0}\right) \notag \\
\hat{a}_+ &\equiv& \frac{1}{\sqrt{2}}\left(-i \frac{\hat{p}}{p_0} + \frac{\hat{x}}{x_0} \right) \notag \\
\end{eqnarray}
where:
$$x_0 \equiv \sqrt{\frac{\hbar}{m \omega}}$$
and also:
$$p_0 \equiv \frac{\hbar}{x_0} = \sqrt{\hbar m \omega}$$
Suppose that at $t=0$, the state is:
$$\Psi(x,0) = A \left( \psi_2(x) + \psi_3(x)\right)$$
\noindent
(A) What is the positive real value of $A$ that correctly normalizes $\Psi(x,0)$?\\[5pt]

\noindent
(B) Using the ladder operators (instead of calculating integrals) calculate $\braket{x}$
at $t=0$.\\[5pt]

\noindent
(C) Write down the time dependent wave function.\\[5pt]

\noindent
(D) Using the ladder operators (instead of calculating integrals) calculate $\braket{x}$
at time $t$.\\[5pt]

\end{document}




