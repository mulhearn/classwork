\documentclass[12pt]{article}


\usepackage[dvips,letterpaper,margin=0.75in,bottom=0.5in]{geometry}
\usepackage{cite}
\usepackage{slashed}
\usepackage{graphicx}
\usepackage{amsmath}
\usepackage{braket}
\usepackage[american,fulldiode]{circuitikz}

\begin{document}
\ctikzset{bipoles/thickness=1}
\ctikzset{bipoles/length=.6cm}

\date{\vspace{-5ex}}

\title{Homework Assignment 2 \\ Review}

\maketitle

\section*{Practice Problems}

These problems are graded on effort only.\\

\noindent
{\bf Griffiths: P1.5, P1.8} \\

\noindent
{\bf Hint for P1.8:}  Let $f(x,t)$ be a solution to the S.E.:
\begin{displaymath}
i\hbar \, \frac{\partial f}{\partial t} \; = \; - \frac{\hbar^2}{2 m} \; \frac{\partial^2 f}{\partial x^2} \, + \, V \, f
\end{displaymath}
Define:
\begin{displaymath}
g(x,t) = f(x,t) \, \exp(-\frac{i V_0 t}{\hbar})
\end{displaymath}
Then calculate:
\begin{displaymath}
i\hbar \, \frac{\partial g}{\partial t} 
\end{displaymath}

\section*{Additional Problems}

\noindent
    {\bf Problem 1:} Consider the discrete probability distribution function $P(n)$ with:
\begin{eqnarray*}
  P(0) &=& \frac{1}{6}\\[8pt]
  P(3) &=& \frac{1}{3}\\[8pt]
  P(4) &=& \frac{1}{2}\\
\end{eqnarray*}
and $P(n) = 0$ for all other $n$.\\[5pt]
(a) Is $P(n)$ properly normalized?  Show your work.\\[5pt]
(b) Find the expectation value $\braket{n}$ of the random variable $n$.\\[5pt]
(b) Find the variance $\sigma^2$ of the random variable $n$.\\ 

\vskip 1cm
\noindent
{\bf Problem 2:} Suppose the wave function for a particle is:
\begin{displaymath}
  \Psi(x,t) =
  \begin{cases}    
    {\displaystyle \sqrt{\frac{\pi}{2a}}\, e^{it/t_0} \, \sqrt{\sin\left(\frac{\pi x}{a}\right)}} & {\displaystyle 0 \leq x \leq a} \\[10pt]
    {\displaystyle 0} & {\displaystyle {\rm otherwise}} \\
  \end{cases}
\end{displaymath}

\noindent
(a) Plot $|\Psi(x,t)|^2$ as a function of $x$.  Does the time $t$ matter? \\[5pt]
(b) Is $\Psi$ properly normalized?  Show how you determined this.\\[5pt]
(c) For any position $b$ what is the probability that you observe the particle with $x \leq b$?  (Hint: consider three cases $b<0$, $0 \leq b \leq a$, and $b > a$.)

\vskip 1cm
\noindent
{\bf Problem 3:} Define $P_{ab}(t)$ as the probability of measuring a particle in the range $a< x < b$, at time t.  Show that:
\begin{displaymath}
  \frac{dP_{ab}}{dt} = J(a, t) - J(b, t),
\end{displaymath}
where:
\begin{displaymath}
  J(x, t) \equiv \frac{i \hbar}{2m}
  \left( \Psi \frac{\partial \Psi^*}{\partial x} - \Psi^* \frac{\partial \Psi}{\partial x}\right)
\end{displaymath}
Hint: look closely at the section on Normalization in Chapter 1 of the {\em lecture notes}.

\vskip 1cm
\noindent
{\bf Problem 4 (Worth Double Credit):} Suppose we have two wave functions:

\begin{displaymath}
  \Psi_1(x,t) =
  \begin{cases}    
    {\displaystyle \sqrt{\frac{2}{a}} \, \sin\left(\frac{2 \pi x}{a}\right) \exp\left(-i \omega_1 t\right)} & {\displaystyle -\frac{a}{2} \leq x \leq \frac{a}{2}} \\[10pt]
    {\displaystyle 0} & {\displaystyle {\rm otherwise}} \\
  \end{cases}
\end{displaymath}

\begin{displaymath}
  \Psi_2(x,t) =
  \begin{cases}    
    {\displaystyle \sqrt{\frac{2}{a}} \, \sin\left(\frac{4 \pi x}{a}\right) \exp\left(-i \omega_2 t\right)} & {\displaystyle -\frac{a}{2} \leq x \leq \frac{a}{2}} \\[10pt]
    {\displaystyle 0} & {\displaystyle {\rm otherwise}} \\
  \end{cases}
\end{displaymath}

\noindent
(a) Calculate $|\Psi_1|^2$ for $-a/2 \leq x \leq a/2$.  Is it time dependent?\\[5pt]
(b) Calculate $|\Psi_2|^2$ for $-a/2 \leq x \leq a/2$.  Is it time dependent?\\[5pt]
Define the wave function $\Psi$ as:
\begin{displaymath}
\Psi = A \cdot \left( \Psi_1 + \Psi_2 \right)
\end{displaymath}
for some constant A.\\[5pt]
(c) Calculate $|\Psi|^2$ for $-a/2 \leq x \leq a/2$.  Is it time dependent?\\[5pt]
{\bf Hint:} Use HW1, Problem 2 to make short work of the remaining steps!\\[5pt]
(d) Show that $\Psi_1$ and $\Psi_2$ are properly normalized.\\[5pt]
(e) Find the positive real value A that properly normalizes $\Psi$.\\[5pt]
(f) Show that $\braket{x}$ is zero for $\Psi_1$ and $\Psi_2$.\\[5pt]
(g) Calculate $\braket{x}$ for $\Psi$.\\

\end{document}




