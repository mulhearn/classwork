\documentclass[12pt]{article}

\usepackage[dvips,letterpaper,margin=0.75in,bottom=0.5in]{geometry}
\usepackage{cite}
\usepackage{slashed}
\usepackage{graphicx}
\usepackage{amsmath}
\usepackage{amssymb}
\usepackage{braket}
\usepackage[american,fulldiode]{circuitikz}

\begin{document}
\newcommand{\ihbar}{\ensuremath{i \hbar}}
\newcommand{\dPsidt}{\ensuremath{ \frac{\partial \Psi}{\partial t} }}
\newcommand{\dPsidx}{\ensuremath{ \frac{\partial \Psi}{\partial x} }}
\newcommand{\ddPsidx}{\ensuremath{ \frac{\partial^2 \Psi}{\partial x^2} }}
\newcommand{\dPssdt}{\ensuremath{ \frac{\partial \Psi^*}{\partial t} }}
\newcommand{\dPssdx}{\ensuremath{ \frac{\partial \Psi^*}{\partial x} }}
\newcommand{\ddPssdx}{\ensuremath{ \frac{\partial^2 \Psi^*}{\partial x^2} }}

\newcommand{\dphidt}{\ensuremath{ \frac{d \phi}{dt} }}
\newcommand{\dpsidx}{\ensuremath{ \frac{d \psi}{dx} }}
\newcommand{\ddpsidx}{\ensuremath{ \frac{d^2 \psi}{dx^2} }}


\date{\vspace{-5ex}}

\title{Homework Assignment 8 \\ ``A spectrum of possibilities''}

\maketitle

\section*{Practice Problems}

These problems are graded on effort only.\\

\noindent
{\bf Griffiths: P3.16, P3.32, P3.33} \\
  
\section*{Additional Problems}

\noindent
{\bf Problem 1:}  Suppose that an observable $\hat{Q}$ has a discrete spectrum of eigenvalues 
$\{ \lambda_i \}$ with corresponding eigenvectors $\ket{\lambda_i}$ so that:
$$\hat{Q} \ket{\lambda_i} = \lambda_i \ket{\lambda_i}$$
and at $t=0$ the state vector is:
$$\ket{\Psi_0} = \sqrt{\frac{2}{5}}\ket{\lambda_1} + i\,\sqrt{\frac{2}{5}}\ket{\lambda_2} + k\,\ket{\lambda_3}$$
where $k$ is a positive real number.\\

\noindent
(A) What is the value of $k$?\\[8pt]
\noindent
(B) What is the expectation value of the observable $\hat{Q}$? (Write your answer in terms of $\lambda_1$, $\lambda_2$, and $\lambda_3$.)\\[8pt]
\noindent
(C) Suppose the spectrum is non-degenerate.  What is the probability that a measurement of the observable $\hat{Q}$ would result in the outcome $\lambda_1$?\\[8pt]
\noindent
(D) Suppose the measurement in (C) was made with outcome $\lambda_1$. What is the state vector 
$\ket{\Psi_1}$ immediately after the measurement?\\[8pt]
\noindent
(E) Suppose the spectrum is degenerate with $\lambda_1 = \lambda_2$ and $\lambda_1 \neq \lambda_3$ and the state vector is $\ket{\Psi_0}$ above.  What is the probability that a measurement of the observable $\hat{Q}$ would result in the outcome $\lambda_1$?\\[8pt]
\noindent
(F) Suppose the measurement in (E) was made with outcome $\lambda_1$. What is the state vector 
$\ket{\Psi_1}$ immediately after the measurement?\\[8pt]

\newpage

\noindent
{\bf Problem 2:}  Suppose that an observable $\hat{Q}$ has a discrete non-degenerate spectrum of eigenvalues $\{ \lambda_i \}$ for $i=1,2,3,\ldots$ with corresponding eigenvectors $\ket{\lambda_i}$ so that:
$$\hat{Q} \ket{\lambda_i} = \lambda_i \ket{\lambda_i}$$
Also suppose that the Hamiltonian $\hat{H}$ has a discrete non-generate spectrum of eigenvalues 
$\{ E_i \}$ for $i=1,2,3,\ldots$ with corresponding stationary states $\ket{E_i}$ so that:
$$\hat{H} \ket{E_i} = E_i \ket{E_i}$$
Last, suppose we know the inner products:
$$\braket{E_1|\lambda_1} = a, \hspace{1cm} \braket{E_2|\lambda_1} = b, \hspace{1cm} \braket{E_i|\lambda_1} = 0 \hspace{2cm} (i>2) $$ 
and:
$$\braket{E_1|\lambda_2} = c, \hspace{1cm} \braket{E_2|\lambda_2} = d, \hspace{1cm} \braket{E_i|\lambda_2} = 0 \hspace{2cm} (i>2) $$ 
\\[2pt]
\noindent
(A) Suppose at $t=0$, a measurement of $\hat{Q}$ has outcome $\lambda_1$.  What is the state vector at time $t=0$? (You may choose the overall phase factor to be something convenient, like 1) \\[8 pt]
\noindent
(B) Determine the time-dependent state vector $\ket{\Psi(t)}$ appropriate for later times.  Write your answer in terms of $a$, $b$, $E_i$, and the stationary state vectors $\ket{E_i}$. {\bf Hint:} You do not know the time-dependence of the $\ket{\lambda_i}$ but you do know the time-dependence of $\ket{E_i}$.  So write your answer from (A) in terms of $\ket{E_i}$, using Fourier's trick to determine each coefficient, then just slap on the time-dependence.\\[8pt]
\noindent
(C) Suppose the state vector from part (B) evolves until time $t$.  What is the probability that a measurement of $\hat{Q}$ will measure value $\lambda_2$? {\bf Hint:} use Fourier's trick!\\[8pt]

\noindent
{\bf Problem 3:}  There's a bit more integration in this problem, but the insights are totally worth it!  For a Gaussian PDF:
$$G(y;\sigma) \equiv \frac{1}{\sigma \sqrt{2\pi}}\exp\left(-\frac{y^2}{2\sigma^2}\right)$$
you may assume use the following Gaussian integrals without calculating them again:
\begin{eqnarray*}
\int_{-\infty}^{+\infty} G(y;\sigma) \, dy       &=& 1\\
\int_{-\infty}^{+\infty} y \, G(y;\sigma) \, dy     &=& 0\\
\int_{-\infty}^{+\infty} y^2 \, G(y; \sigma) \, dy  &=& \sigma^2\\
\end{eqnarray*}
Consider a stationary Gaussian wave packet:\\
$$\psi(x) = \left( \frac{1}{2\pi \sigma_x^2}\right)^{\frac{1}{4}} \exp\left(-\frac{x^2}{4\sigma_x^2}\right)$$
\\[5pt]
\noindent
(A) Using the Gaussian integrals above, confirm that $\braket{x^2} - \braket{x}^2 = \sigma_x^2$.  (If it were otherwise, the choice of parameter name $\sigma_x$ would have been very confusing!)\\[8pt]
(B) Calculate the ``momentum space wave function'' $\widetilde{\psi}(p)$ which is the Fourier transform of $\psi(x)$ but in terms of momentum $p=\hbar k$ instead of just $k$:
$$\widetilde{\psi}(p) \equiv \frac{1}{\sqrt{2 \pi \hbar}} \int_{-\infty}^{+\infty} \, \psi(x) \, e^{-ipx/\hbar} \, dx$$
{\bf Hint:} use the technique of completing the square (used previously) to calculate this integral.\\[8pt]
(C) Show that:
$$|\widetilde{\psi}(p)|^2 = G(p;\sigma_p)$$
for a suitable choice of $\sigma_p$.\\[8pt]
(D) Calculate the uncertainty product $\sigma_x \sigma_p$ and compare your result to the uncertainty principle. 
\end{document}




