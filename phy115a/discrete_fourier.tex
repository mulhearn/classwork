\documentclass[12pt]{book}

\usepackage[dvips,letterpaper,margin=0.75in,bottom=0.5in]{geometry}
\usepackage{cite}
\usepackage{slashed}
\usepackage{graphicx}
\usepackage{amsmath}
\usepackage{amssymb}
\usepackage{braket}
\begin{document}

\title{The Discrete Fourier Transform \\} 
\author{Michael Mulhearn}

\maketitle

\appendix
\setcounter{chapter}{6}
\chapter{The Discrete Fourier Transform}

\subsection{Fourier Series for Complex-Valued Functions}

Recall that a complex periodic function with period $a$ can be represented as a series:
\begin{equation}
\psi(x) = \sum_{n=-\infty}^{\infty}  c_n \, e_n(x)
\end{equation}
of complex exponential basis functions:
\begin{equation}
e_n(x) \equiv \sqrt{\frac{1}{a}} \exp\left( i \, \frac{2 \pi n x}{a}\right)
\end{equation}
For the inner product defined over one period\footnote{We've started at 0 to make our notation below cleaner}:
\begin{equation}
\braket{f|g} = \int_{0}^{a} f^*(x) g(x) dx
\end{equation}
the complex exponential basis functions are orthonormal:
\begin{equation}
\braket{e_n|e_m} = \delta_{nm}
\end{equation}
so the coefficients can be determined using Fourier's trick:
\begin{equation}
c_n = \braket{e_n|f}
\end{equation}

When doing numerical analysis, we encounter discrete data sets, so instead of the continuous variable $x$ in $[0,a]$ we have $N$ discrete values:
$$x_j \equiv \lambda \, j, \hspace{1cm} \lambda \equiv \frac{a}{N}, \hspace{1cm} j=0,1,2,\ldots,(N-1)$$
so instead of the continuous function $\Psi(x)$ we have:
$$\Psi_j \equiv \Psi(x_j)$$
so we have:
$$\Psi(x) \to \sum_{j=0}^{N-1}\delta(x-\lambda j) \, \Psi_j \, \lambda $$
so the Fourier coefficients are:
\begin{eqnarray*}
c_m &=& \braket{e_m|\Psi}\\
&=& \int_{0}^{a} \; \Psi(x) \; e_m^*(x) \, dx \\
&=& \lambda \sqrt{\frac{1}{a}} \sum_j  \Psi_j \exp\left( -i \, \frac{2 \pi m \lambda j}{a}\right)\\
\frac{c_m}{\sqrt{\lambda}}&=& \sqrt{\frac{1}{N}} \sum_j  \Psi_j \exp\left( -i \, \frac{2 \pi m j}{N}\right)\\
\end{eqnarray*}
And:

\begin{eqnarray*}
\psi(x) &=& \sum_{m=-\infty}^{\infty}  c_m \, e_m(x)\\
        &=& \sum_{m=-\infty}^{\infty}  \frac{c_m}{\sqrt{\lambda}} \,        
        \frac{1}{\sqrt{n}}\exp\left( -i \, \frac{2 \pi m j}{N}\right)\\
\end{eqnarray*}
We can clean this up nicely by noting that for
\begin{equation}
\braket{f|g} = \sum_{j=0}^{(n-1)} f^*(x_j) \, g(x_j) 
\end{equation}
with:
\begin{equation}
e_m(x) \equiv \sqrt{\frac{1}{N}} \exp\left( i \, \frac{2 \pi m j}{N}\right)
\end{equation}
we still have:
\begin{equation}
\braket{e_n|e_m} = \delta_{nm}
\end{equation}
and so:
\begin{equation}
c_m = \sqrt{\frac{1}{N}} \sum_j  \Psi_j \exp\left( -i \, \frac{2 \pi m j}{N}\right)
\end{equation}
which in scipy is just FFT of $\Psi$ using the option norm=``ortho''.





\end{document}





