\chapter{Differentiation and Projectile Motion}

\section{Introduction}

\section{Preparation}


Our codes are getting complicated enough that we will need to pay some
attention to variable scope, as illustrated here:
\begin{python}
i=1
j=2
k=3
def f(i,j):
    print(i,j,k)
f(i,j)
f(j,i)
\end{python}
Try to predict the output of this snippet before running it.  The
first three lines define integers $i$,$j$, and $k$. These have global
scope, which means they can be accessed from anywhere.  The function
\pyth{f(i,j)} has parameters $i$ and $j$ which have a scope limited to
the function $f$.  Even though they have the same name as the global
variables $i$ and $j$, they are independent quantities.  Within the
function $f$ the variable $i$ is the first parameter, and $j$ is the
second parameter.  Because they have the same name, the global
variables $i$ and $j$ are {\em shadowed} by the local parameters $i$
and $j$.  The global variable $k$ is not shadowed.

When \pyth{f(j,i)} the paramter $i$ is set to the global variable $j$,
and the parameter $k$ is set to the global variable $i$.  There is no
parameter $k$ only global variable $k$.  This is why the output from
the call is ``2 1 3''.\\

In this lab, we will be passing a function as an argument to another
function, as in this simple example:
\begin{python}
def show(f):
    print(f(2))
    print(f(3))    
    
def f(i):
    return i**2

def g(i):
    return i**3

show(f)
show(g)    
\end{python}
Here the \pyth{show} function takes another function \pyth{f} as an
argument.  Within the show function, the function \pyth{f} is called
using paranthesis just like any other function, as in \pyth{f(2)} We
define two additional functions, \pyth{f} which returns the square of
its argument, and \pyth{g} which returns the cube.  When
\pyth{show(f)} the output 4 and 9.  When \pyth{show(g)} the output 8
and 27.  Run the code as is, and also check what happens if you define
\pyth{g(i)} to require a second argument as in \pyth{g(i,j)}.

\section{Numerical Differentiation}

In lecture we derived the right derivative (aka foward derivative) formula
\begin{displaymath}
  f'(x) = \frac{f(x+h) - f(x)}{h} + \mathcal{O}(h)
\end{displaymath}
for numerically determining the derivative of the function $f$.  Remember we do not calculate the 
$\mathcal{O}(h)$ term, that indicates that the truncation error is of order $h$.
We also derived centered derivative formula:
\begin{displaymath}
f'(x) = \frac{f(x+h) - f(x-h)}{2h} + \mathcal{O}(h^2)
\end{displaymath}

To evaluate a derivative using any of these formulas, we need to
choose an appropriate value of $h$.  If $h$ is too large, the
truncation error (the amount the estimated value differs from the
actual value) will dominate.  If $h$ is too small, floating point
precision will likely be the limit.

\plot Implement the right derivative formula as \pyth{right(f, x, h)}
where $f$ is the function to be evaluated, $x$ is the location to
evaluate the derivative, and $h$ is the step size for the numerical
integration.  Check you code on several functions with known
derivatives, like this:
\begin{python}
def f(x):  # derivative 0
    return 2
def g(x):  # derivative 3
    return 3*x
def h(x):  # derivative 4x
    return 2*x**2

print(right(f,1,0.01))
print(right(g,1,0.01))
print(right(h,1,0.01))
\end{python}


\plot Implement the center derivative function as \pyth{center} and test it in the same manner as in the previous exercise for \pyth{right}.


\plot Compare the performance of \pyth{right} and \pyth{center} like this:
\begin{python}
def f(x):  # derivative 6x**2
    return 2*x**3

print("right:", right(f,1,0.1),   "center:", center(f,1,0.1))
print("right:", right(f,1,0.01),  "center:", center(f,1,0.01)) 
print("right:", right(f,1,0.001), "center:", center(f,1,0.001)) 
\end{python}
Recall that the truncation error goes as $h$ for the right derivative and as $h^2$ for the center derivative.  Are these results consistent with that expecation?


From now on, we will use the center derivative function only due to
its better performance.  We can plot the derivative a function like
this:
\begin{python}
def f(x):
    return 0.5*x**2

x = np.linspace(0,1,100)
y = center(f, x, 0.1)
plt.plot(x,ya,"-b")
plt.xlabel("x")
plt.ylabel("y")
plt.show()
\end{python}
Notice how the argument $x$ passed to the function \pyth{right(f,x,h)}
and then to \pyth{f(x)} is now a numpy array of 100 values from 0 to
1.  The derivative is now evalued at 100 places with a single call.

\plot Define $f(x) = x^3$.  Use your \pyth{center} function to evaluate it's derivative $f'(x)$ in the x range $[-2,2]$.  Plot both $f(x)$ and $f'(x)$ in that range (in the same plot) using different colors for each.  Add a legend and axis labels.


\plot Define $f(x) = sin(x)$.  Use your \pyth{center} function to evaluate it's derivative $f'(x)$ in the x range $[0,2\pi]$.  Plot $f(x)$, $f'(x)$, and $cos(x)$ in that range (all in the same plot) using different colors for each.  Add a legend and axis labels.

\section{Projectile Motion}

\plot Check you implementation against the following test values:
\begin{python}
print(np.around(euler(0.134, 0.659, 0.282, 0.662, 0.643, 0.900, 0.451),2))
print(np.around(euler(0.924, 0.959, 0.575, 0.299, 0.710, 0.699, 0.471),2))
\end{python}
and ensure you get the corret output:
\begin{verbatim}
[0.75 0.37 0.78 0.7 ]
[1.24 1.23 0.94 1.15]
\end{verbatim}

\begin{python}
tau = 2*np.pi
vi    = 20   # [m/s]
g     = 9.8  # [m/s^2]
theta = tau/8
dt = 0.01 # [s] 
x  = 0    # [m]
y  = 0    # [m]
vx = vi*np.cos(theta)
vy = vi*np.sin(theta)
\end{python}

\begin{python}
tjx = np.array([x])
tjy = np.array([y])
while(y>=0):
    x,y,vx,vy = euler(dt,x,y,vx,vy,0,-g)
    tjx = np.append(tjx,  x)
    tjy = np.append(tjy,  y)
    
plt.plot(tjx, tjy)
plt.xlabel("x [m]")
plt.xlabel("y [m]")
plt.show()
\end{python}

\plot Simulate projectile motion using your function \pyth{euler} as described above.

\plot Extend your simulation to record vx and vy at each step along with the x and y positions.  Take the mass of the projectile to be $m=0.145~\rm kg$ and plot the kinetic energy, potential energy, and total energy as a function of time.  To build an array containing the time of each step 
for plotting quanties against time, you can do:
\begin{verbatim}
t = np.arange(tjx.size)*dt
\end{verbatim}
Include a legend. The $x$ and $y$ axes have changed to time and energy, so make certain to change the axes labels.


\section{Projectile Motion with Drag}

We can model drag as a deceleration:
\begin{displaymath}
\vec{a} = -k |\vec{v}| \vec{v}
\end{displaymath}
where $k=0.00622~\rm m^-1$ for typical baseballs .

\plot Extend your simulation to include the effect of drag.  Plot the trajectory without drag and the trajectory with drag in the same plot.  Include a legend and (as always) label all axes.

\plot Plot the kinetic energy, potential energy, and total energy as a function	of time, just as before.  Is total energy conserved?

