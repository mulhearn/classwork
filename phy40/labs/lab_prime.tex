\chapter{The Quadratic Equation and Prime Numbers}

\section{Introduction}

In this lab, we will make more extensive use of conditional statements
to implement algorithms which solve the quadratic equation, identify
prime numbers, and add fractions.  An optional challenge problem,
The Lucky Number of Euler, explores how the quadratic equation can
generate prime numbers.

\section{Preparation}

This lab will rely on the material from Sections 1.2.1 to 1.2.4 of the
Scientific Python Lecture notes.  We'll now be making frequent use of
conditional statements:
\begin{python}
def compare(a,b):
    if (a==b):
        print("a equals b")
    elif (a<b):
        print ("a is less than b")
    else:
        print ("a is greater than b")
\end{python}
Notice that Python uses \pyth{==} for comparison.  You will get a
syntax error if you use \pyth{a=b} instead.

We'll also use of the modulo operator $\%$ extensively.  The modulo
operation $a\%b$ returns the remainder from the integer division
$a/b$.
\begin{python}
# is b a factor of a?
def isfactor(a,b):
    if (a%b == 0):
        return True;
    return False
\end{python}
Why does \pyth{a%b == 0} mean that b is a factor of a?\\

\newpage
\vskip 0.25cm
\plot Consider this verbose code snippet:
\begin{python}
for i in range(100):
    print("on index ", i)
\end{python}
Which prints the current index on every iteration.  Use the modulo
operator to modify the code so that it only prints the index every 10
iterations.  This is a classic trick!\\

We will also be using while loops, which repeat a block of code until a condition is met:
\begin{python}
count=0
while(count<10):
    print(count)
    count = count+1
\end{python}

\section{Quadratic Formula}

The Quadratic equation:
\begin{displaymath}
  ax^2 + bx + c = 0
\end{displaymath}
has solutions which are given by the quadratic formula
\begin{equation} \label{eqn:quad}
x = \frac{-b \pm \sqrt{b^2 - 4ac}}{2a}
\end{equation}
The number of unique real solutions depends on the quantity in the
square root, which is called the discriminant:
\begin{displaymath}
b^2-4ac
\end{displaymath}
If this is positive there are two real solutions, it it is one there
is one real solution, and if it is negative there are no real
solutions.  For now, let's assume that a, b, and c are all integers.

In this case, the solution to the quadratic equation can be calculated
as follows:
\vskip -0.25cm
\noindent
\begin{algorithm}
  D := $b^2 - 4ac$
  if (D=0):
     calculate the single solution from quadratic 
     print single solution
  if (D<0): 
     print no solutions
  if (D>0):
     calculate both solutions from quadratic formula
     print both solution
\end{algorithm}
\vskip 0.25cm

A test case with one real solution is:
\begin{displaymath}
(x-1)(x-1)= x^2 -2x + 1.
\end{displaymath}
A test case with two real solution is:
\begin{displaymath}
(x-1)(x+1)= x^2 - 1.
\end{displaymath}
A test case with zero real solutions is:
\begin{displaymath}
(x-i)(x+i)= x^2 + 1.
\end{displaymath}

\vskip 0.25cm
\plot Implement a function \pyth{quad(a,b,c)} which reports the solutions to the quadratic equation and verify it with the test cases shown.\\

\vskip 0.25cm
\plot Calculate three more test cases with integer solutions and use them to test your function more thoroughly. \\

\vskip 0.25cm
\plot (Optional) The condition that the discriminant is exactly zero
(\pyth{D==0}) is problematic when applied to floating point
numbers. (Why?) In example is for a=.1 b=.3 c =0.225 which should have
only one real solution.  Test you function with this test case.
Modify the conditionals in your function to account for floating point
precision and test it.\\

\section{Prime Numbers}

A prime number is a number that has two and only two factors: itself
and one.  One is not prime, but two is.  We can determine if a number
$a$ is prime as follows:
\vskip 0.25cm
\begin{algorithm}
  if (a<2):
     return false
  i := 2
  while ($i \leq \sqrt{a}$):   
     if (a%i=0):
        return false
     i := i + 1
  return true
\end{algorithm}
\vskip 0.25cm
Why is there no need to check for factors larger than $\sqrt{a}$?\\

\plot Implement a function \pyth{isprime(a)} which returns \pyth{True}
if the integer $a$ is prime and \pyth{False} if not.\\

Suppose that next we want to find the first $n$ prime numbers greater
than or equal to a number $A$.  We can simply check if $A$, $A+1$,
$A+2$, and so on are prime until we find the first $n$.  But we do not
know how many numbers we will have to check before finding $n$ that
are prime.  This is a case for a \pyth{while} loop.\\

\plot Find the first $n$ prime numbers greater than $A$ using a while
loop and your \pyth{isprime} function. Test it for $n=10$ and $A=0$,
then for $A=1000000000$.  Try that with paper and pencil!\\

Notice how we broke this problem of finding primes into two parts:
determining whether or not a number is prime or not, then testing and
counting prime numbers.  We thoroughly tested the first part before
using it in the second.  This is an essential approach to solving
computational problems: break complicated tasks down into smaller
tasks which can be tested separately.\\

\plot Implement a function which computes the fraction
\begin{displaymath}
\frac{n}{d} = \frac{a}{b} + \frac{c}{d}
\end{displaymath}
from integer inputs $a$,$b$,$c$ and $d$ and returns integers $n$ and
$d$.  Returning $n>d$ is allowed, but $n/d$ should be a simplified
fraction with a greatest common factor of one.  You can return two
integers as a Tuple, like this:
\begin{python}
  # function which	adds fractions
  def addfrac(a,b,c,d):
     n=d=0
     #your code...
     return n, d
     
  # calling function:
  n,d =addfrac(1,2,1,3)
  print("{0}/{1}".format(n,d))
\end{python}
For full credit, you must factorize (see what I did there?) this
problem into two parts: your \pyth{addfrac} funtion should call a
second function that does one well defined task.
\section{The Lucky Number of Euler}

Euler discovered that the formula:
\begin{displaymath}
k^2 + k + 41
\end{displaymath}
produces prime numbers for $0 \leq k \leq 39$.  Perhaps you can beat Euler at his own game!

Consider quadratics of the form:
\begin{displaymath}
k^2 + ak + b
\end{displaymath}
For each integer value of $a$ and $b$, there is a maximum number $n$ such that the quadratic formula produces prime numbers for $0 \leq k < n$\\

\plot (Optional) Find the the values of $a$ and $b$ which produce the largest number of prime numbers.  Restrict yourself to $|a| \leq 1000$ and $|b| \leq 1000$.\\ \vskip 1cm

\noindent
If you do complete this optional problem, then nice work, hot shot,
but remember that Euler found his without using a computer!

