\chapter{Matrices, Rotations and Parity}

\section{Introduction}

\section{Preparation}

\section{Rotations and Parity}
\begin{python}
def rotz(theta):
    cs = np.cos(theta)
    sn = np.sin(theta)
    R  = np.array([[cs, sn, 0],[-sn, cs, 0], [0, 0, 1]])
    R=np.around(R,decimals=15)+0
    return R
\end{python}

\begin{plot} With $\tau \equiv 2 \pi$ print the rotation matrix for $\theta = 0, \tau/4, \tau/2, 3 \tau/4, \tau$\
 corresponding to zero, quarter turn, half turn, 3/4 turn, and a full turn.\end{plot}



\begin{plot} Show that $R(\vec{a} \times \vec{b}) = (R\vec{a}) \times (R\vec{a})$.\end{plot}


\begin{plot} Show that $P(\vec{a} \times \vec{b}) \neq (P\vec{a}) \times (P\vec{a})$.\end{plot}

