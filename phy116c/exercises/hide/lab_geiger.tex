\chapter{Statistics of Radioactive Decays}


\section{Introduction}

\begin{figure}[htbp]
\begin{center}
 \includegraphics[width=0.55\textwidth]{figs/labs/geiger/source.jpg};
\caption{\label{fig:source} A sealed radioactive source.  A small amount of Cs-137 is contained within the small button shaped piece of plastic.  For your safety, the sources will be handled only by the TA.}
\end{center}
\end{figure}

In this lab, you will use a Geiger Counter to study the statistics of radioactive decays.

\section{Precautions}

\noindent
{\bf Precautions with the Geiger counter:}
\begin{itemize}
\item Leave the cable from the Geiger counter controller to the Geiger counter in place {\em at all times}.  This carries voltages of approximately 1000 volts.  If you leave the cable in place, nothing can be inadvertently plugged in (including fingers)
\item Leave the Geiger tube in its holder.  It has a thin front window which is easily broken.
\item Do not set the high voltage higher than 1000 volts.
\end{itemize}

\noindent
{\bf Precautions with the radioactive source:}
\begin{itemize}
\item See Fig.~\ref{fig:source} to familiarize yourself with what the sources look like.
\item Don't touch the source.
\item Leave the source in the tray at all times.  The TA will provide the sources and handle moving them from place to place.
\item Radiation falls off as $1/r^2$.  So minimize your time near sources and maximize your distance from them.
\end{itemize}


\section{The Geiger Counter}


\begin{figure}[htbp]
\begin{center}
\begin{tikzpicture}
    \node[anchor=south west,inner sep=0] (image) at (0,0,0) {\includegraphics[width=0.55\textwidth]{figs/labs/geiger/assembly.jpg}};

    \node[right](X) at (10.0,3.0) {Timer};
    \draw (X.west) -- (8.0,3.5);

    \node[right](X) at (10.0,5.0) {\parbox{3cm}{\flushleft High-Voltage and Counter}};
    \draw (X.west) -- (5.0,4.75);

    \node[left](X) at (0.0,4.5) {\parbox{2.5cm}{\flushright Geiger Tube Holder}};
    \draw[white,thick] (X.east) -- (1.25,5.0);
    \draw (X.east) -- (1.25,5.0);

    \node[left](X) at (0.0,3.0) {Source};
    \draw (X.east) -- (1.35,4.05);

\end{tikzpicture}
\caption{\label{fig:geigersetup} The Geiger Counter assembly.}
\end{center}
\end{figure}

To begin, you will familiarize yourself with the counter and timer
features of your Geiger counter assembly using the built-in test mode.
Your lab bench will already be prepared with a Geiger Counter assembly
as shown in Fig.~\ref{fig:geigersetup}.  Ensure that the high-voltage
(HV) is off by turning the knob labeled ``H.V. Adjust''
counter-clockwise all the way to zero.  Now put the Geiger counter
into test mode by flipping the left red switch to ``TEST''.  Flip the
right red switch to ``COUNT'' and you should see the counter display
begin incrementing.  Push the button on the front of your timer and
you should see the Timer turn on and off.  Leave the timer
incrementing.  Now flip right red switch to ``STOP'', and observe that
the both the counter and the timer stop simultaneously.  The knob on
left side of the old-school lab timer can be used to reset the time.
Keep turning the knob clockwise until the time reads 0.  Use the black
button on the Counter to reset the count to zero.

Flip the right switch to ``COUNT'' and then back to ``STOP'' when 10
seconds have passed.  During this time, the $60~\rm Hz$ test signal
should increment the counter close to 600 times.  Try this a few times
and make sure you can reliably count close to $600$ test pulses in a
10 second interval.  You should reset the count each time, but there
is no need to reset the timer.  Simply stop when the timer reaches the
next factor of ten.  Due to your reaction time, you may well stop at
one-to-two tenths of a second later.  This is OK, and will only add
less than a few percent error to your measurements over 10 second
intervals.

\section{High-Voltage Calibration}

When you are confident that you know how to operate the timer, switch
the left red switch to ``USE'' mode.  Ask the TA to provide you with a
sealed radioactive source in the second shelf from the top of your
Geiger tube holder.  Switch the right switch to ``COUNT'' mode.  With
the HV off, you should not see any pulses.  Turn the HV up until you
begin to see counts increment on the display, and continue to the next
interval of 50 volts (e.g. if it first starts incrementing at 730
volts, set the dial to 750 volts).  Count the number of events in a
ten second interval.

Repeat this measurement, twice for each voltage setting, in 50 volt
steps up to 1000 volts.  Do not exceed 1000 volts.

{\bf Plot 1: } Plot the rate (in Hz) as a function of high voltage.
You should see a plateau region (a leveling off) which indicates the
onset of the Geiger mode within the Geiger tube.  From your plot,
chose a high-voltage near the beginning of the Geiger mode, and set
the high-voltage to this calibrated value.

\begin{figure}[htbp]
\begin{center}
 \includegraphics[width=0.55\textwidth]{figs/labs/geiger/pulse.jpg};
\caption{\label{fig:geigerpulse} An example Geiger counter pulse.}
\end{center}
\end{figure}

Connect an oscilloscope to the output of the counter assembly (on the
back, labeled ``SCOPE'').  Adjust your scope to view the Geiger pulses
like that of Fig.~\ref{fig:geigerpulse}.  Note that the Geiger counter
output contains a DC component in addition to the AC pulse, so you
will want to use your scope in AC coupling mode which will remove the
DC component and allow you to see the pulse.  You will also want to
see the attenuation to 1X because you are not using an attenuating
probe.  The rate of the 




\section{Data Collection}

Even in today's world of massive amounts of automation, it is still
useful to know how to collect a small amount (up to a few hundred data
points) of data manually.  Often in the lab, you have one-off
measurements that you would like to make without investing in a lot of
automation.

In this section, you will collect data manually for about one hour.
Practice a routine with your lab partner that allows you to take and
record the data as fast as possible.  For instance, person A should
operate the counter, and person B should use the PC.  Person A turns
the counter on for ten seconds, turns it off, and says (quietly)
``OK''.  Person B records the value on the PC and says ``Go''.  Person
A resets the counter and continues.  Remember that there is no need to
reset the Timer each time, which would take too long, and which would
actually be counterproductive (if you consider the effect of a roughly
constant reaction time.) 

Practice your routine a few times, and make sure your count is near 1000 events in
a ten second interval.  Then record 200 data points.

When you have finished recording your data with the radioactive
source, ask your TA to remove the source and return it to the
radioactive locker.

Now record an additional 200 data points with no source, to measure
the background radiation rate.  You should record around 3 background
counts per 10 second interval.

\section{Analysis}

\begin{figure}[htbp]
\begin{center}
\begin{tabular}{cc}
\includegraphics[height=0.22\textheight]{figs/labs/geiger/background.pdf}
\includegraphics[height=0.22\textheight]{figs/labs/geiger/source.pdf}
\end{tabular}
\end{center}
\caption{\label{fig:geigeranalysis} Numerical simulation of the experiment
  for (a) background radiation only, and (b) radioactive source
  present.}
\end{figure}

Using Scientific Python, measure the mean and variance of your
collected background and source data.  Then produce histograms to
display your data as in Fig.~\ref{fig:geigeranalysis}.  For the
background data, plot the histogram for eleven bins: 0,1,2,...,10.
For the source data, plot about 20 bins covering a few hundred counts
around the mean value.

Compare your collected background data to a Poisson distribution,
appropriately normalized, with a mean set to the mean of your data.
Compare your collected source data to a Gaussian distribution,
appropriately normalized, with a mean set to the mean value of your
data, and sigma set to the square root of your mean.










