\documentclass[12pt,oneside]{book}

\usepackage[dvips,letterpaper,margin=0.75in,bottom=0.75in]{geometry}
\usepackage{cite}
\usepackage{slashed}
\usepackage{graphicx}
\usepackage{amsmath}
\usepackage{enumitem}
\usepackage{amsthm}
\usepackage{braket}
\theoremstyle{definition}

\usepackage[american,fulldiode]{circuitikz}
\tikzset{component/.style={draw,thick,circle,fill=white,minimum size =0.75cm,inner sep=0pt}}
\newtheorem{measurement}{Logbook Entry}[chapter]
\newtheorem{plot}{Jupyter Notebook}[chapter]

\begin{document}
\ctikzset{bipoles/thickness=1}
\ctikzset{bipoles/length=.6cm}

\title{Physics 116C Lab Manual}
\author{Michael Mulhearn}
\maketitle

\tableofcontents

\chapter{Arrays and Plotting}

\section{Introduction}

\section{Preparation}

This lab will rely on the material from Sections 1.4.1 to 1.4.2 and
1.5.1 to 1.5.2 of the Scientific Python Lecture notes.  This is the
first lab that relies on inline plotting, so make sure you are
starting your notebook with the ``line magic'':
\begin{python}
  %pylab inline
\end{python}
This will load the numpy library as np, the matplotlib.pyplot library
as plt, and setup the matplotlib backend to imbed plots in your
notebook.

A Numpy array is a grid of values.  Unlike Python lists, the elements
of a numpy array all have the same data type, which makes them much
more computionally efficient.  The numpy library provides a wide range
of analysis tools that are mostly centered on the numpy array type.
Numpy arrays can be constructed from a Python list:
\begin{python}
a = np.array([1.3,7.2,4.1,0.0])
b = np.array([[1,2],[3,4]])
print(a)
print(b)
print(np.shape(a))
print(np.shape(b))
\end{python}
or they can be constructed from numpy function designed for the purpose:
\begin{python}
a = np.linspace(0,1,11)
print(a)
b = np.arange(0,5,1)
print(b)
\end{python}

\section{Plotting with Scentific Python}

Basic plotting in Python requires two numpy arrives: one for the $x$
coordinates and one for the $y$ coordinates.  Consider the following
very simple plot:
\begin{python}
x = np.array([0.0, 1.0, 2.0, 3.0, 4.0,  5.0])
y = np.array([0.3, 3.2, 5.8, 9.0, 12.4, 14.7])
plot(x,y,"bo")
\end{python}
Here, the ``bo'' options specifies blue circles.  Now consider:
\begin{python}
x = np.linspace(0, 1, 100)
y = np.sin(np.pi*x)
plt.plot(x,y,"r-")
\end{python}
Here the ``r-'' option specifies red line.  Including 100 points (as
done here) results produces a smooth looking curve.

Now promise me that you will never make another plot without labeling
the $x$ and $y$ axes! Here's another example will all the bells and
whistles you need to make a professional looking plot:
\begin{python}
UPPER = 2
LOWER = 0
tau   = 2*np.pi
x = np.linspace(LOWER,UPPER,100)
s = np.sin(tau*x)
c = np.cos(tau*x)
plt.plot(x,s,"b-",label="sin")
plt.plot(x,c,"r-",label="cos")
plt.xlabel("x")
plt.ylabel("y")
plt.title("Two Periods of a Sine and Cosine")
plt.legend(frameon=False)
plt.show()
\end{python}
Make sure you understand all of the features demonstrated here:
\begin{itemize}
 \item Variables \pyth{UPPER} and \pyth{LOWER} located at the top of
   the snippet, allowing for easy adjustment of parameters that affect
   the plot.
 \item Use of \pyth{np.linspace} to define an array of x values, with
   plenty of them (100) to produce nice smooth curves.
 \item Creation of two different arrays of y values, one for sin and one for cos.
 \item Plotting the arrays of $x$ and $y$ values with \pyth{plt.plot} using the ``-'' option for a line and color blue(``b'') for sin and red(``r'') for cos. 
 \item Defining appropriate axis labels with \pyth{plt.xlabel} and \pyth{plt.ylabel}. 
 \item Adding a title with \pyth{plt.title}
 \item Creation of a legend using the {\tt label} optional argument to {\tt plt.plot} and the {plot.legend()} command.  Removing the frame with option \pyth{frameon=False}
\end{itemize}
It is written so concisely and intuitively, you might not even notice
what is going on with the line:
\begin{python}
s = np.sin(tau*x)  
\end{python}

Remember that $x$ here is a numpy array of 100 elements.  The
\pyth{tau*x} multiples every element of x by our value tau.  The
\pyth{np.sin(tau*x)} then takes the sine of each element.  The
resulting numpy array, also of 100 elements, is referenced by variable
s.  Each element of $s$ contains $\sin(\tau x)$ for the corresponding
element of the array $x$.  It takes some getting used to for
programmers used to explicitly writing for loops for things like this,
but ultimately, the fact that python handles so much of this
bookkeeping for us is what makes it a very fun language to work with.

\plot \begin{plot} \end{plot}
Plot the sinc function as a smooth line in the $x$ range from -5 to 5.  Add appropriate axes labels.  Include a legend identifying the sinc function.  For the line color, use any color other than red or blue.




\section{The Logistics Map}
The logistics map is the recurrence relation
\begin{displaymath}
x_{n+1} = r \, x_n \, (1 - x_n)
\end{displaymath}
with the variable $x$ between $0$ and $1$.  The variable $x$ can be
thought to represent the ratio of a population to its maximum possible
value.  The population increases due to birth and decreases due to
starvation as the population approaches it's maximum value ($x$ near
1).  This leads to the non-linear relationship that defines the
logistic map.  The mapping keeps the variable x between $0$ and $1$ as
long as the parameter r is in the range $[0,4]$.

The logistics map is frequently encountered as a simple example of a
chaotic system emerging from a simple non-linear system.  If we
consider the long term behavior of the population $x$ as a function of
the parameter $r$, as shown in Fig.~\ref{fig:logmap}, we see that for
values of $r$ less than $3$ the population approaches a single fixed
value.  At the value $r=3$ the non-linear system exhibits bifurcation
with the population oscillating between two values.  As $r$ increase,
further bifurcations occur at an ever increasing rate until the
systems exhibits chaotic behavior alternating with occasional returns
to stable oscillations.

\begin{figure}[htbp]
\begin{center}
\includegraphics[width=0.65\textwidth]{figs/plotting/bifurcation.png} 
\caption{Long term behavior of the logistics map.}
\label{fig:logmap}
\end{center}
\end{figure}

\begin{figure}[htbp]
\begin{center}
\includegraphics[width=0.85\textwidth]{figs/plotting/logmapstart.png} 
\caption{Modeling the logistics map.}
\label{fig:logmapstart}
\end{center}
\end{figure}

The long term behavior of the logistics map can be easily modeled in
Scientific Python.  A start is shown in Fig.~\ref{fig:logmapstart}
where you should understand:
\begin{itemize}
\item An array of $r$ values is defined.
\item An array of $x$ values of the same size as $r$ is defined and initialized to an arbitrary non-zero value (0.01).
\item Two example iterations of the logistic map are applied.
\item The next two iterations of the values of $x$ are plotted as function of $r$ on the same plot.
\end{itemize}

\noindent
%{\bf Plot 3:} 
\begin{plot} \end{plot} Reproduce the figure in Fig.~\ref{fig:logmap} by doing the following:
\begin{itemize}
\item Define two global variables {\tt ITER = 10} and {\tt PLOT = 5}.
\item Apply the logistics map {\tt ITER} times by using a for loop.
\item Apply the logistics map an additional {\tt PLOT} times, plotting the values of $x$ as a function of $r$, as in the example, each time.
\end{itemize}
You'll observe the long term behavior by increasing the value of {\tt
  ITER} to a large value, such as 10,000.  You'll see the full
dependence on $r$ by decreasing the step size in the initialization of
the numpy array $r$ to something like $0.001$.  You'll observe the
chaotic behavior by increasing the value of {\tt PLOT} to 100 or even
1000 iterations.  To make a prettier plot using finer points (once you
have a large number of points) you can reduce the size by adjusting
the {\tt s=10} parameter in the call to {\tt plt.scatter} to something
like {\tt s=0.0001}.


\chapter{The Monte Carlo Method}

\section{Introduction}

This lab, which will take two lab sessions to complete, introduces the
Monte Carlo method, an approach to solving a wide range of problems by
repeatedly drawing random numbers from a probability distribution.
You will produce a sequence of pseudorandom numbers.  You will use a
histogram to directly compare values of random variables to a
probability distribution function.  With these preliminaries in hand,
you will explore several widely used Monte Calro techniques: Monte
Carlo integration, the rejection method, and the transform method.
You will finish by looking at the evolution of entropy during
diffusion, as modeled by a random walk.

\section{Generating random numbers}

The Monte Carlo method relies on the generation of random numbers, so
we will start there.  The numbers we generate using computers are
actually ``pseudorandom'' numbers, because they are deterministically
obtained from an algorithm.  However, the algorithm is choosen so that
the numbers appear random for practical purposes.  This is no small
concern.  Much of the computational work in the early 1970's had to be
redone because of the widespread use of a deeply flawed pseudorandom
number generator called RANDU.

In this section, you will generate a pseudorandom number sequence
using the linear congruential method.  This sequence is determined
iteratively from the simple relationship:
\begin{displaymath}
  I_{n+1} = (a*I_{n} + c) \mod M
\end{displaymath}
Recall that $x \mod y$ (coded as {\tt x \% y} in python) is the remainder
after integer division $x//y$.  Each $I_n$ is called a seed, and the
initial seed $I_0$ must be provided e.g. by the user.  Notice that the
seeds are all integers in the range from 0 to $(M-1)$.  If we wish to
convert these seeds into a random variable $x$ in the range from 0 to
$L$, we simply use $x_n = L * I_n / M$.  As long as $M$ is much larger
than $L$, $x$ is approximately continous.

The algorithm works because the product $a*I_{n}$ is generally many
times larger than $M$, so the remainder is effectively a uniform
random number.  The effectiveness of this algorithm is highly
dependend on the choice of $a$,$c$, and $M$.  Choose poorly and you get
RANDU.  Choose wisely and you get the highly regarded algorithm of
Park and Miller.  We will do the latter and use $a=7^5$, $c=0$, and $M
= 2^{31}-1$.

\begin{samepage}
\begin{plot} \end{plot}
Generate a sequence of ten uniform random variables in the range
$[0,1]$ from the Park-Miller sequence, using an initial seed of one.
Check your code by testing that the generator returns a {\bf seed} of
1043618065 after 10000 calls.  Change the initial seed to a value of
your choice and report the first 10 random values.  If you like, round
to two decimal places using {\tt np.around} to tidy up your output.
\end{samepage}

\section{Visualizing distributions}

\begin{figure}[htbp]
 \begin{center}
 \begin{tabular}{cc}   
  \includegraphics[height=0.22\textheight]{figs/monte_carlo/flat2d.pdf} &
  \includegraphics[height=0.22\textheight]{figs/monte_carlo/flathist.pdf} \\
  (a) & (b) \\
 \end{tabular}
\caption{The (a) $x$ value of uniform random throws versus throw number and (b) corresponding histogram. }
\label{fig:flathist}
\end{center}
\end{figure}

\noindent
How can we verify that our Park-Miller random number generator
produces uniform random numbers in the range 0 to 1?  The associated
PDF is just $p(x)=1$, which we know how to plot, but how can we
compare this function to a sequence of numbers like $[0.21, 0.85,
  0.33, ...]$?  An initial attempt might look like
Fig.~\ref{fig:flathist}a, where we have simply plotted the $x$ value
of each throw versus the number of the throw.  Unfortunately, this
plot isn't particularly helpful.  If we zoomed in, we could determine
from the plot the $x$ value associated with each throw.  This is
simply too much information.

For interpreting a list of values as a distribution, there is only one
tool of choice: the histogram.  To histogram our data, we divide the
entire range $[0,1]$ into smaller ranges called {\em bins}.  Let's
start with 10 bins as an example. In this case, the first bin covers
the range from 0 to 0.1, or more precisely, the half-open interval
$\left[0,0.1\right)$ which includes $0$ and $0.099$, but not $0.1$.
  The second bin would have range $\left[0.1,0.2\right)$, the third
    bin would have range $\left[0.2,0.3\right)$ and so on, up to the
      last bin which would cover $\left[0.9,1.0\right]$.  To {\em
        fill} a histogram, you count the number of values that fall
      within the range of each bin.  So in our example, the value
      $0.21$ would add one to the count for the third bin, which has
      range $\left[0.2,0.3\right)$.  After filling, the histogram
        consists of a count associated with each bin range.

Fig.~\ref{fig:flathist}b shows a histogram filled with 1000 random
throws drawn from a uniform random number generator.  While we can now
see the shape of the distribution, we still don't have quite enough
information to answer the question, is this flat?  It is certainly not
perfectly flat!

The first feature we will need to add to the plot is the inclusion of
{\em error bars} to indicate the statistical uncertainty in our
histogram values.  Error bars are conventionally drawn with a size
equal to the standard deviation of the measured value, $\sigma$.  Each
histogram contains a {\em count} $n$.  As we will see in lecture, this
count is drawn from a Poisson distribution, and our best estimate for
the standard deviation $\sigma$ associated with a count $n$ is simply
$\sqrt{n}$.  So when drawing a histogram, the uncertainty in each bin
is simply the square root of the histogram value.  That is the beauty of
Poisson statistics!  If we have a count, we know the statistical
uncertainty.

\begin{figure}[htbp]
 \begin{center}
  \includegraphics[width=0.50\textwidth]{figs/monte_carlo/fancyhist.pdf}
  \includegraphics[width=0.75\textwidth]{figs/monte_carlo/fancyhist-code.png}
\caption{Histogram of data drawn from a flat distribution compared to prediction, with the code used to produce the plot.}
\label{fig:fancyhist}
\end{center}
\end{figure}

We'll also want to add the prediction to the plot, assuming a flat
distribution for the contents of each bin.  In this case, we generated
$N$ events and we have $N_{\rm BINS}$ histogram bins, which should
therefore each contain an equal share: $N / N_{\rm BINS}$.  The
resulting histogram, along with the code used to generate it, is shown
in Fig.~\ref{fig:fancyhist}.  With the prediction and errorbars
included in the plot, one can now see that these generated values are
indeed quite consistent with a flat prediction. All of the bins are
within nearly one-sigma. With this number of bins, it is not uncommon
to see a two-sigma descrepancy.

You will produce many histograms in this class, so you will need to
(eventually) understand every single line in this example code.  Take
the time to read through the documentation for the key functions like
{\tt np.histogram} and {\tt np.random.uniform}, available on the web
(see numpy.org or just search ``np.histrogram python'').  A big part
of learning to program effectively, is learning how to read and
understand software documentation correctly and efficiently.

There are a few important features to notice:
\begin{itemize}
 \item The $x$-values are contained in an {\tt np.array} filled with
   uniform random variables generated by calling the {\tt
     np.random.uniform} function.
 \item The function {\tt np.histogram} is used to calculate a histogram
   from these $x$ values.  The call requests {\tt NBINS=10} histogram
   bins, in range $[0,1]$.  Don't confuse the python tuple {\tt (0,1)}
   used to indicate this range as indicating an open interval... often
   the computing language differs significantly from math notation, as
   is the case here!
 \item The {\tt np.histogram} function returns two items we need:  a {\tt np.array} containing the count for each bin ({\tt hx}) and a {\tt np.array} of bin edges ({\tt bins})
 \item We want to plot the count over the center of each bin, not one of the edges, so we calculate the quantity {\tt cbins} which is an {\tt np.array} containing the center of each bin.  You'll use this trick a lot, so make sure you understand what it is doing!
 \item The uncertainty on each bin {\tt hunc} is calculated as the square root of the bin counts {\tt hx}.
 \item We use the somewhat poorly named {\tt np.errorbar} function to plot {\bf both} the histogram central value {\tt and} the errorbar in each bin.
 \item We draw the prediction as a straight line defined by two points defined by {\tt xp} and {\tt yp}.
\end{itemize}

\begin{plot} \end{plot}
Modify the example code to generate a histogram for uniform random
numbers generated from your Park-Miller sequence instead of ${\tt
  np.random.uniform}$.  Increase the number of events to {\tt
  N=10000}.  Increase the number of bins to {\tt NBINS=20}.  Does your
code appear to produce uniform random numbers?

To answer a question in your notebook, simply add a cell and answer the question as a comment (each line starting with {\tt \#}).

\section{Calculating the value of $\pi$}

\begin{figure}[htbp]
\begin{center}
\includegraphics[width=0.65\textwidth]{figs/monte_carlo/pitoss.jpg} 
\caption{Determining $\pi$ by throwing toothpicks.}
\label{fig:pitoss}
\end{center}
\end{figure}

\noindent
Hopefully 2021 will see the return of parties, so let's start by
examing a surefire way to be the life of the party: determining the
constant $\pi$ by throwing toothpicks!  The procedure is simple: you
cut a peice of paper to a width of four toothpicks, then draw two
vertical lines separated by the width of two tooth picks.  Take turns
tossing toothpicks, as in Fig.~\ref{fig:pitoss}.

From the geometry of the setup, it can be shown that the probability
that a toothpick which is entirely on the paper also crosses a line is
given by $1/\pi$.  Therefore, one can measure $\pi$ by counting the
total number of toothpicks that landed entirely on the page and
dividing by the number of those toothpicks that crossed a line.  This
is, in essence, the Monte Carlo method.

\begin{figure}[htbp]
\begin{center}
  \includegraphics[width=0.50\textwidth]{figs/monte_carlo/pimc.pdf}
  \includegraphics[width=0.75\textwidth]{figs/monte_carlo/pimc-code.png} 
\caption{Monte Carlo Determination $\pi$ .}
\label{fig:pimc}
\end{center}
\end{figure}

An easier Monte Carlo method to implement computationally is shown in
Fig.~\ref{fig:pimc} along with the code used to generate the plot.
The idea is to throw points uniformly in the unit square of area 1.
Much like in the toothpick example, the value of $\pi$ can be
determined by counting the number of generated points that also landed
within the unit circle.

The key features of the example code are:
\begin{itemize}
\item The $x$ and $y$ values are each contained in an {\tt np.array} filled with uniform random variables in $[0,1]$ by the {\tt np.random.uniform} function.
\item A mask {\tt inside} is created to indicate which points are inside the circle.  Recall that the mask is an {\tt np.array} of True or False values, with the same length as the $x$ and $y$ arrays.  For example {\tt x[inside]} is an np.array containing just the subset of {\tt x} which are inside the circle.  
\end{itemize}

\begin{plot} \end{plot}
Starting from the example code, determine the numerical value of $\pi$
using the Monte Carlo method.  The easiest way to obtain the count you
need is to apply the function {\tt np.sum} to an appropriate mask.
When counting a mask, each True is treated as a one, and each False is
treated as zero.  Work out the relationship between $\pi$ and the
fraction of events in the unit circle, and use your count to
numerically determine the value of $\pi$.  Increase the number of
generated events and confirm that your calculated value of $\pi$
approaches the known value.

\begin{plot} \end{plot}
This is an example of a binomial process, because points are either
inside or outside the circle. So we expect the number of events in the
circle to follow the binomial distribution with $\sigma^2 = n \epsilon
(1-\epsilon)$.  In this case, $n$ is the total number of generated
events and $\epsilon$ is the fraction that fall inside the unit
circle.  The statistical uncertainty on your measured value of $\pi$
works out to be:
\begin{displaymath}
\sigma_\pi = \sqrt{\frac{\pi \, (4-\pi)}{n}}
\end{displaymath}
where $n$ is the number of generated events.  Does your measured value
of $\pi$ agree with the known value within your statistical
uncertainty?

\section{Monte Carlo integration}

The Monte Carlo method can also be used to numerically integrate a
function.  Monte Carlo integration methods generally only outperform
deterministic methods when the number of dimensions is large, but we
can illustrate the method most easily in one dimension. In this
section, you'll use the Monte Carlo method to perform the integral:
\begin{displaymath}
  \int_0^\pi \sin^2 \theta \, d\theta
\end{displaymath}

To do so, you should make a copy of your solution from the previous section
and modify it in the following manner:
\begin{itemize}
 \item Instead of thowing $x$ in $[0,1]$, throw $\theta$ in $[0,\pi]$.  This means the area of the rectangle $A$ is now $\pi$ instead of 1.
 \item Count the number of throws that land below the integral $y < \sin^2 \theta$.
 \item Determine the area under the curve as the fraction of the throws under the curve times the total area of the rectangle $A$.
 \item The statistical uncertainty in this case is $\pi/(2\sqrt{n})$ where $n$ is the number of generated events.
\end{itemize}  

\begin{plot} \end{plot}
Use the Monte Carlo method to calculate the integral:
\begin{displaymath}
  \int_0^\pi \sin^2 \theta \, d\theta
\end{displaymath}
Make a plot similar to that of Fig.~\ref{fig:pimc} showing the thows
above the curve in red and below the curve in blue.  Calculate the
integral and statistical uncertainty and compare it to the value you obtain analytically.

\section{The Rejection method}

\begin{figure}[htbp]
\begin{center}
  \begin{tabular}{cc}
  \includegraphics[height=0.30\textheight]{figs/monte_carlo/rejectmc.pdf} &
  \includegraphics[height=0.30\textheight]{figs/monte_carlo/quadhist.pdf} \\
  (a) & (b) \\
 \end{tabular}
  \caption{Monte Carlo rejection method applied to $p(x) \propto
    x^2$. Uniformly generated points (a) are rejected (red) if they
    are above the PDF, and the $x$ values of points below the PDF
    (blue) are selected.  A histogram (b) of the selected $x$ values
    shows that they follow the PDF.}
  \label{fig:rejectmc}
\end{center}
\end{figure}

\noindent
We now know how to generate uniform random numbers, but suppose we
need a random variable thrown according to a non-uniform probability
distribution $p(x)$?  Fig.~\ref{fig:rejectmc} demonstrates one
approach, which closely follows the procedure for numerical
integration using the Monte Carlo technique.

The rejection method produces random variables in a range from 0 to $L$
according to any desired PDF $p(x)$.  Start by finding a value $Y$ which is at
least as large as the maximum value of $p(x)$ for $x$ in $[0,L]$.  Then:
\begin{itemize}
  \item Throw $x$ as a uniform random variable in range $[0, L]$.
  \item Throw $y$ as a uniform random variable in range $[0, Y]$.
  \item If $y > p(x)$ reject the $x$ value and start over, otherwise, use the $x$ value as one throw.
\end{itemize}
Repeat these steps as necessary until a sufficent number of $x$ values have been selected.

The rejection method works because the probability of an $x$ value
being selected is, by construction, proportional to $p(x)$.  Since the
$x$ values were initially chosen from a flat distribution, the
selected $x$ values will follow the $p(x)$ distribution.  You can
visualize this in Fig.~\ref{fig:rejectmc} which leaves very little
doubt that the $x$ values of the blue points will follow the PDF.
Notice that it isn't even necessary for $p(x)$ to be normalized for
this procedure to work: any function proportional to the PDF of interest will do.

To produce a smooth function such as the quadratic prediction of
Fig.~\ref{fig:rejectmc}b, make sure you use plenty of $x$ values
(around 100 at least), via {\tt np.arange} or {\tt np.linspace}, just
as you did in the Plotting lab.  When comparing a PDF to histogrammed
data (as you will do for the second plot below) you will need to
normalize it appropriately.  The number of throws we expect to find in
a bin with edges at $a$ and $b$ is given by
\begin{displaymath}
  N \cdot \int_a^b p(x) \, dx \; = N \cdot p(x^*) \cdot (b-a) 
\end{displaymath}
The integral is simply the probability that one throw ends up in the
range, which we scale by the total number of throws $N$.  The equality
holds for at least one $x^*$ in the range $[a,b]$ and $(b-a)$ is
simply the bin size.  Therefore, we can formulate a prediction from a
normalized PDF $p(x)$ to data from $N$ throws used to fill a histogram with
bin sizes $\Delta x$ as the smooth function resulting from:
\begin{displaymath}
N \cdot p(x) \cdot \Delta x
\end{displaymath}
This is a technique we will use over and over again, so make sure you
understand it!

\begin{plot} \end{plot}
Use the rejection method to generate random numbers in the region from [0,1] that follow a distribution $p(x) \propto x^2$.  You'll do the following:
\begin{itemize}
  \item Note that there is no need to normalize the PDF when using the rejection method, so use $p(x)=x^2$.
  \item In our $x$ range $[0,1]$, $p(x)$ has maximum value at $x=1$ so set $Y = p(1) = 1^2 = 1$. 
  \item Throw $x$ as a uniform random variable in the range $[0, 1]$.
  \item As $Y=1$, throw $y$ as a uniform random variable in range $[0, 1]$.
  \item If $y > x^2$ reject the $x$ value and try a new set of $x$ and $y$ values, otherwise, use the $x$ value as one throw.
\end{itemize}
Throw 1000 (unselected) $x$ values, and produce a plot like that of
Fig.~\ref{fig:rejectmc}a showing your selected points in blue, your
rejected points in red, and the selection function ($p(x) = x^2$).

\begin{samepage}
\begin{plot} \end{plot}
Increase the number of (unselected) $x$ values thrown to 10,000.
Count the number $N$ of selected $x$ values.  Generate a plot like that of
Fig.~\ref{fig:rejectmc}b comparing the distribution of your selected
$x$ values to the prediction, which in this case is given by:
\begin{displaymath}
N \cdot 3x^2 \cdot \Delta x
\end{displaymath}
Be careful to use the number of selected $x$ values for $N$, not the total thrown (10000) before rejection.
\end{samepage}

\section{The transformation method}

Suppose that you need to throw random variables according to an
exponential distribution $p(x) = \exp(-x)$.  This PDF is defined for
$[0,+\infty)$ and properly normalized across this range as you can verify:
\begin{displaymath}
  \int_0^{+\infty} \exp(-x) \, dx = 1
\end{displaymath}    
The first problem is that we can only generate uniform random
variables up to a finite value $L$, not $+\infty$.  But let's suppose we are
willing to work aound this by simply cutting off the PDF at some large
value, like say we won't produce values with $x>100$.

With this change, the rejection method will work in principle.  But it
still has a major shortcoming.  Since $p(x)$ has a maximum value of 1,
and $x$ ranges from 0 to 100, the rectangle we will be filling with
uniform random points has area 100.  But our PDF, even when
integrated to $+\infty$, only has area 1.  So less than one out of
every 100 points we throw will be selected.  Perhaps we can live with
this, but then what if we need to go out to $x=1000000$.  Now only one
out of every million points will be selected.  In many scenarios, the
rejection method becomes too computationally inefficient to be of any
practical value.

In these case, we can use the transformation method instead of the
rejection method.  The transformation method is premised on the fact
that for {\em any} normalized PDF, we must have
\begin{displaymath}
  p(x) \geq 0
\end{displaymath}
everywhere and
\begin{displaymath}
  \int_{-\infty}^{+\infty} p(x) \; dx = 1
\end{displaymath}
as long as we take care to set $p(x)=0$ outside our range for $x$.  It follows from these properties that for any value of $y$ in the range $[0,1]$ there is a unique largest $x$ value for which:
\begin{equation} \label{eqn:mctransform}
  \int_{-\infty}^{x} p(x) \; dx = y
\end{equation}
From the fundamental theorem of calculus, we see that:
\begin{displaymath}
  dy = p(x) \, dx
\end{displaymath}
If the variables $y$ are drawn from a uniform distribution with PDF $q(y)=1$, then we see that:
\begin{displaymath}
 \int_{y_1}^{y_2} q(y) \, dy = \int_{x_1}^{x_2}p(x) \, dx.
\end{displaymath}
for $x_i$ and $y_i$ related by Eqn.~\ref{eqn:mctransform}.  This shows
that while $y$ is a uniform random variable ($q(y)=1$), the
corresponding $x$ values will distributed according to the desired PDF
$p(x)$.

That provides the mathematical justification for the transformation
method, which starts by finding the inverse function $f^{-1}(y)$ for:
\begin{displaymath}
  y = f(x) = \int_{-\infty}^{x} p(x) \, dx
\end{displaymath}
Then the procedure is:
\begin{itemize}  
 \item Throw $y$ as a uniform random variable in $[0,1]$.
 \item Find $x = f^{-1}(y)$
\end{itemize}
The $x$ values determined in this way will be drawn from the $p(x)$ distribution.  

There is an intuitive explanation for why this works.  The $y$ value is
essentially a fraction of the probability integrated by the PDF.  In a
region of $x$ where $p(x)$ is relatively large, the integral is
changing rapidly and so a large range of $y$ values map to this region
of $x$-values.  In a region of $x$ where $p(x)$ is relatively small,
the integral is not changing rapidly and so a small range of $y$
values map to this region of $x$-values.

Let's see how this applies to our exponential function.  In this case we calculate:
\begin{displaymath}
y = f(x) = \int_0^x \exp(-x) \, dx = 1 - \exp(-x)
\end{displaymath}  
which we invert to find:
\begin{displaymath}
x = - \ln(1-y)
\end{displaymath}  
To determine values of the random variable $x$, we follow this procedure:
\begin{itemize}
 \item Throw a $y$ value flat in [0,1]
 \item Caculate $x = -\ln(1-y)$
\end{itemize}
Repeat to produce as many $x$ values as needed.  Notice that this
procedure gives one usable $x$ value for every random throw.

\begin{plot} \end{plot}
Use the transformation method as described to generate 10,000 values
of a random variable thrown from an exponential function.  Produce a
plot like that of of Fig.~\ref{fig:rejectmc}b comparing the
distribution of your generated events to the prediction for $p(x) =
\exp(-x)$.  Remember to properly normalize your prediction based on
the bin size and number of events thrown.

\newpage
\section{Particle diffusion}

\begin{figure}[htbp]
\begin{center}
  \includegraphics[width=0.80\textwidth]{figs/monte_carlo/diffusion.pdf}
  \caption{Simulation of the diffusion of a drop of particles at four different times.}
\label{fig:diffusion}
\end{center}
\end{figure}

\begin{figure}[htbp]
 \begin{center}
  \includegraphics[width=0.50\textwidth]{figs/monte_carlo/diffstart.pdf}
  \includegraphics[width=0.75\textwidth]{figs/monte_carlo/diffstart-code.png}
  \caption{Snapshot of the simulation at the start, along with the code used to produce it.}
\label{fig:diffstart}
\end{center}
\end{figure}


\noindent
In this section, we will model the diffusion of a drop of particles in
a medium, as in Fig.~\ref{fig:diffusion}, shown as snapshots at four
different times.  The starting point for the simulation, at $t=0$ is
shown in Fig.~\ref{fig:diffstart} along with the code used to produce
it.  The entire state of the system is contained in the arrays {\tt x}
and {\tt y} which contain the $x$ and $y$ positions of each particle.

The diffusion process is modeled by a random walk. During each update
(for one time step) the $x$ and $y$ values of each particle should be
randomly increased or decreased by an amount {\tt STEP=0.2} Any
particles that would leave the boundaries of the region $[-L,L]$ as a
result should be moved back into the region.  The numpy functions 
{\tt np.random.choice} and {\tt np.clip} are useful here.

Despite the symmetry of the random walk, the system clearly evolves by
diffusing outward over time.  This can be seen as a consequence of the
second law of thermodynamics.  Calculating the entropy from the
microscopic state of continuous particles is a bit tricky.  The
approach we will use is based on the Gibb's entropy.  We divide the
area into cells, and determine the fraction of the particles $f_i$ in
each cell $i$.  We calculate the entropy as:
\begin{displaymath}
  S = \sum_i f_i \ln f_i
\end{displaymath}
A python function which calculates the entropy in this manner:
\begin{tt}
\begin{verbatim}
from scipy import stats  
def entropy(x,y,l,sbins):
    h,xbins,ybins=np.histogram2d(x,y,bins=sbins,range=[[-l,l],[-l,l]])
    return stats.entropy(h.flatten())
\end{verbatim}
\end{tt}
The function takes as input parameters the position arrays {\tt x} and
{\tt y}, the boundary distance {\tt l} (set it to {\tt L} and the
number of bins in each dimension {\tt sbins} (set it to 20).  The
function returns the entropy of the current state of the system
described by {\tt x} and {\tt y}.

\begin{plot} \end{plot}
Starting from the example code, implement a random walk to model the
diffusion process, and plot four snap shops showing the evolution of
the system.

\begin{plot} \end{plot}
Calculate and record the entropy of the system as it evolves, and plot
the entropy as a function of time.


\input{lab_limits.tex}
\chapter{The Central Limit Theorem and Experimental Uncertainties}

%
% TODO:  Students were confused about how to handle bin position
% for plotting discrete data...  some clarification (text, figures) is needed.
%

\section{Introduction}

In this lab, you will produce a numerical demonstration of the central
limit theorem.  You will also model the propagation of uncertainties
and compare with the calculated uncertainties.


\section{Sampling Distributions}

\begin{figure}[htbp]
\begin{center}
\includegraphics[width=0.75\textwidth]{figs/uncertainties/step.png}\\
\end{center}
\caption{\label{fig:samplingstep}}
\end{figure}

\begin{figure}[htbp]
\begin{center}
\includegraphics[width=0.75\textwidth]{figs/uncertainties/gaussian.png}\\ 
\end{center}
\caption{\label{fig:samplinggauss}}
\end{figure}

Scientific python provides functions to draw random samples according
to various distributions.  In today's lab, we will draw samples
uniformly in the interval $[-1,1]$, as demonstrated in Fig.~\ref{fig:samplingstep}.   The line
\begin{verbatim}
r = np.random.uniform(low=-1.0, high=1.0, size=NEXP)
\end{verbatim}
creates a NumPy array {\tt r} which contains {\tt NEXP} entries, with
each entry chosen uniformly and randomly in the range from -1 to 1.
In the example, these events are displayed in a histogram.  When
plotting histograms with plenty of statistics (one million entries
here) and fine binning (60 bins here) it is usually preferable to use
lines instead of points with error bars for plotting the histograms,
as is done in this example.  Notice, however, that even with one
million events, there are still statistical fluctuations which prevent
the curve from being perfectly smooth.

In Fig.~\ref{fig:samplinggauss}, entries are instead drawn from the Gaussian distribution with the line:
\begin{verbatim}
r = np.random.normal(loc=5.0,scale=1.5,size=NEXP).
\end{verbatim}
The histogram is plotted with a logarithmic $y$ scale:
\begin{verbatim}
plt.semilogy()
\end{verbatim}
which results in the Gaussian distribution appearing as a parabola.  The histogram is compared to the Gaussian PDF appropriately normalized:
\begin{verbatim}
x = np.linspace(MIN,MAX,100)
y = NEXP*binsize*stats.norm.pdf(x,loc=MEAN,scale=SIGMA)
\end{verbatim}


\section{Demonstration of the Central Limit Theorem}

In this section, you'll show that average value of random variables
chosen uniformly from -1 to 1 approaches a Gaussian distribution,
consistent with the central limit theorem.  First create a 2-D array
of size {\tt NEXP} by {\tt NAVG} filled with uniform random values in
the interval from -1 to 1, as follows:
\begin{verbatim}
r = np.random.uniform(low=-1.0, high=1.0, size=(NEXP,NAVG))
\end{verbatim}
Then calculate averages values from {\tt NAVG} entries:
\begin{verbatim}
x = np.sum(r, axis=1)/float(NAVG)  
\end{verbatim}
From the Central Limit Theorem, we expect the entries in x to approach a Gaussian distribution.

{\bf Plot 1:} Set {\tt NEXP} to 1000000 for plenty of statistics.
Produce three different histograms with 40 bins covering the range
from -1.2 to 1.2 for three values {\tt NAVG}: 1,2, and 3.  Plot all
three histogram in the same graph with appropriate legend.

Your plot will show that already for three contributions to the average, the result looks quite Gaussian on a linear scale.  For more precise comparison, will use a log scale and compare to the PDF.

{\bf Plot 2:} Calculate {\tt NEXP}$=1000000$ average values {\tt x} for {\tt NAVG}$=10$.  Calculate the mean value of the entries in {\tt x} using the {\tt np.mean} function.  Calculate $\sigma$ for the entries in $x$ by taking the square root of the output from the {\tt np.var} (variance) function.  Produce a histograms with 20 bins covering the range
from -0.5 to 0.5 for the average values.  Compare with a Gaussian distribution, appropriately normalized, using your calculated values from the  mean and sigma.  Plot both the histogram and PDF on the same graph, including an appropriate legend.  Use a logarithmic $y$ axis.


\section{Propagation of Uncertainties}

\begin{figure}[htbp]
\begin{center}
\includegraphics[width=0.75\textwidth]{figs/uncertainties/addunc.pdf}\\
\end{center}
\caption{\label{fig:addunc} Simulation of many measurements of the quantity $c = a + b$. }
\end{figure}

Consider two measured values $a \pm \sigma_a$ and $b \pm \sigma_b$.  If we calculate the quantity $c = a + b$ or $c = a - b$, the uncertainty on the calculated value $c$ is given by:
\begin{displaymath}
\sigma_c = \sqrt{\sigma_a^2 + \sigma_b^2}.
\end{displaymath}
If instead, we calculate $c = a * b$ or $c = a/b$ the fractional uncertainty on $c$ is given by:
\begin{displaymath}
\frac{\sigma_c}{c} = \sqrt{\left(\frac{\sigma_a}{a}\right)^2 + \left(\frac{\sigma_b}{b}\right)^2}.
\end{displaymath}
In this section, you'll develop a numerical simulation for the
propagation of uncertainties under addition, subtraction,
multiplication, and division.  An example, for $c = a + b$ is shown in Fig.~\ref{fig:addunc}.

Simulate the measurement $a$ by drawing 100,000 random variables
sampled from the Gaussian distribution with mean $a$ and sigma
$\sigma_a$, and likewise for $b$.  Calculate the values of $c$ from
the $a$ and $b$ values.  Plot the results in histograms with 50 bins and an appropriate range, as in Fig.~\ref{fig:addunc}.  Calculate the mean and variance of the mean and variance of the $c$ values and compare to your expectation from the standard propagation of uncertainties.

{\bf Plot 3-6:}  Produce four plots simulating addition, subtraction, multiplication, and division, as in Fig.~\ref{fig:addunc}.  In each case, compare the measured variance of the $c$ values with your expectation.































\chapter{Simulation of an Ideal Gas}

\section{Introduction}

For an ideal gas composed of molecules with mass $m$ at temperature
$T$, the probability density for the component of velocity in the $x$
direction ($v_x$) is given by:
\begin{equation}
  \label{eqn:mbvx}
P(v_x) = \sqrt{\frac{m}{2 \pi k_{\rm B} T}} \exp\left(-\frac{m v_x^2}{2k_{\rm B} T}\right)
\end{equation}
where $k_{\rm B}$ is Boltzmann's constant.  Similary for the $y$ direction:
\begin{equation}
  \label{eqn:mbvx}
P(v_y) = \sqrt{\frac{m}{2 \pi k_{\rm B} T}} \exp\left(-\frac{m v_y^2}{2k_{\rm B} T}\right).
\end{equation}

For simplicity, we will be simulating a gas in two dimensions.  The infinitesimal probability associated with a velocity $(v_x, v_y)$ is given by: 
\begin{eqnarray*}
P(v_x, v_y) \, dv_x \, dv_y &=& P(v_x) \, dv_x \, P(v_y) \, dv_y \\
   &=& \frac{m}{2 \pi k_{\rm B} T} \exp\left(-\frac{m (v_x^2+v_y^2)}{2k_{\rm B} T}\right) \, dv_x \, dv_y \\
   &=& \frac{m v}{2 \pi k_{\rm B} T} \exp\left(-\frac{m v^2}{2k_{\rm B} T}\right) \, d\theta \, dv \\
\end{eqnarray*}
where we have changed to polar coordinates $v$ and $\theta$ in the usual manner with area differential $dv_x \, dv_y = v \, dv \, d\theta$.  This allows us to read off the probability density in polar coordintes:
\begin{equation*}
P(v, \theta) = \frac{m v}{2 \pi k_{\rm B} T} \exp\left(-\frac{m v^2}{2k_{\rm B} T}\right) 
\end{equation*}
Integrating over all possible directions $\theta$, we obtain:
\begin{eqnarray}
P(v) &=& \int_0^{2\pi} P(v,\theta) d\theta \nonumber \\
     &=& \int_0^{2\pi} \frac{m v}{2 \pi k_{\rm B} T} \exp\left(-\frac{m v^2}{2k_{\rm B} T}\right) \nonumber \\
P(v) &=& \frac{m v}{k_{\rm B} T} \exp \left(-\frac{m v^2}{2k_{\rm B} T}\right) \label{eqn:mbv}\\
\nonumber
\end{eqnarray}
which is the Maxwell-Boltzmann distribution for an ideal gas in two
dimensions.  This is the probability density for a gas molecule to have speed
$v$.

In this lab, we will create a simple numerical simulation of an ideal
gas and verify that the velocity of the gas follows the
Maxwell-Boltzmann distribution.

\section{System of Units}

Choosing an effective system of units is essential for building a
well-behaved numerical simulation.  Consider the Maxwell-Boltzmann
distribution, which involves the following SI values:
\begin{itemize}
\item Boltzmann's constant: $k_{\rm B} = 1.38 \times 10^{-23}~\rm J/K$
\item Molecular masses: e.g. $N_2$ with $m =  4.65 \times 10^{-26}~ \rm kg$.
\item Temperature: e.g. room temperature $T = 293~\rm K$.
\end{itemize}
The smallest number greater than zero that a computer can represent
with a single-precision floating point number is approximately
$10^{-38}$. Representing the SI value of Boltzmann's constant at
$10^{-23}$ uses a large fraction of this precision before we even begin
our calculation.  Numerical algorithms using floating point numbers
work best when the values involved in the calculation are near one.

It is usually best, therefore, to devise an alternate system of units
for any numerical simulation which keeps the values of variables of
interest as near one as possible.  We will call this the numerical
system of units.

To start, we choose a reference temperature near the temperature
we would like to simluate, say $T_0 = 293~\rm K$.  All temperatures in
the simulation will be in units of this reference temperature.  So a
temperature {\tt T=1.2} in the program will be $1.2 \, T_0 = 352~\rm K$
in SI units.  Our model also includes mass, so we choose a reference
mass near the mass of the molecules we will be simulating, say $M_0 =
4.65 \times 10^{-26}~ \rm kg$.  A mass {\tt m=2.1} in our program would have
an SI value value of $2.1 M_0 = 9.8 \times 10^{-26}~ \rm kg$.

The physics we will simulate involves Boltzmann's constant $k_{\rm B}$
which will have a value of one in our program.  This sets the
reference energy from our reference temeperature.  For example, an
energy {\tt kT = 3} in our program will have an SI value of $k_{\rm B}
T = 3~k_B T_0 = 1.21 \times 10^{-20}~J$.  The reference energy and
reference mass together define a reference velocity:
\begin{displaymath}
V_0 = \sqrt{\frac{k_b T_0}{M_0}} = 295~ \rm m/s.  
\end{displaymath}  

The only time the actual values choosen for the numerical system of
units are needed is if you need to convert inputs in SI units to the
numerical system of units, or convert the results of your simulation
to SI units.  In this lab, we will specify all inputs and report all
results using the numerical system of units.  {\bf So there is no need for
specific values such as $M_0 = 2.32 \times 10^{-25}~\rm kg$ to appear
anywhere in your program.}  If such values do appear, outside of
comments, you are certainly making a mistake!

\begin{figure}[htbp]
\begin{center}
\includegraphics[width=0.65\textwidth]{figs/maxwellboltzman/maxboltz.pdf} \\
\includegraphics[width=0.65\textwidth]{figs/maxwellboltzman/maxboltz-code.png} \\
\caption{The Maxwell-Boltzmann distribution using a system of units appropriate for a numerical simulation, along with the code used to produce the plot.}
\label{fig:mbdist}
\end{center}
\end{figure}

As an example, the Maxwell-Boltzmann distribution is plotted in
Fig.~\ref{fig:mbdist} alongside the code used to produce it.  Notice
how for {\tt kT} and {\tt m} near one, the typical velcities are also
near one.  This is sign of good numerical system of units.  Notice
also that Boltzmann's constant or any other small or large numbers in
SI units do not appear anywhere in the code. 

\section{Collision Model}

At the heart of your numerical simulation is the collision model.  It
is the collisions of molecules that will allow your simulated gas to
reach thermal equilibrium.  We will use the simple elastic collision
of identical mass particles, as illustrated in Fig.~\ref{fig:collcms}, as our collision model.  We consider particles a and b with velocities $\vec{v_a}$ and
$\vec{v_b}$ in the lab frame.  The velocity of particle a in the CMS frame before the collision is
\begin{displaymath}
\vec{u} = \frac{\vec{v_a} - \vec{v_b}}{2}.
\end{displaymath}
The collision rotates the velocity of particle a by the scattering angle $\theta$ so that the velocity $\vec{w}$ after the collision is
\begin{displaymath}
\begin{pmatrix}
w_x \\
w_y \\
\end{pmatrix}
  =
\begin{pmatrix}
\cos \theta  & \sin \theta \\
-\sin \theta  & \cos \theta \\
\end{pmatrix}
\,
\begin{pmatrix}
u_x \\
u_y \\
\end{pmatrix}
\end{displaymath}
In the lab frame, the velocity of molecule a changes by an amount:
\begin{displaymath}
\Delta \vec{v_a} = \vec{w} - \vec{u}
\end{displaymath}
and the velocity of molecule b changes by an amount:
\begin{displaymath}
\Delta \vec{v_b} = \vec{u} - \vec{w}
\end{displaymath}




\begin{figure}[htbp]
\begin{center}
\begin{tikzpicture}
\draw[->, line width=1.5, blue] (-3,0) -- (-0.1,0);
\draw[->, line width=1.5, blue] (3,0)  -- (0.1,0);
\draw[->, line width=1.5, red] (0,0) -- (3*0.50,3*0.86) coordinate(A);
\draw[->, line width=1.5, red] (0,0) -- (-3*0.50,-3*0.86) coordinate(B);
\node[right] at (0.5,0.5) {$\theta$};
\node[left] at (-3,0) {a};
\node[right] at (3,0) {b};
\node[above] at (A) {a};
\node[below] at (B) {b};
\node[above] at (-1.5,0) {$\vec{u}$};
\node[above] at (1.5,0) {$-\vec{u}$};
\node[left] at (0.8,1.5) {$\vec{w}$};
\node[left] at (-0.8,-1.4) {$-\vec{w}$};
\end{tikzpicture}
\caption{The collision model in the center-of-mass:  incoming molecule $a$ with velocity $\vec{u}$ collides with the incoming particle $b$ of identical mass with velocity $-\vec{u}$.  Particle $a$ is scattered by angle $\theta$ and leaves with velocity $\vec{w}$, while particle $b$ leaves with velocity $\vec{w}$.  The magnitude of the final and initial velocities are the same:  $|\vec{u}| = |\vec{w}|$.}
\label{fig:collcms}
\end{center}
\end{figure}

\section{Implementing the Collision Model}

Our Python implementation for the collision will be computed in terms
of the components of the velocity vectors of molecule a and molecule
b:
\begin{eqnarray*}
\vec{v_a} &=& 
\begin{pmatrix}
a_x \\
a_y \\
\end{pmatrix} \\
\vec{v_b} &=& 
\begin{pmatrix}
b_x \\
b_y \\
\end{pmatrix} \\
\end{eqnarray*}
We'll use the Python variable names {\tt ax}, {\tt ay}, {\tt bx}, and {\tt by} to refer to $a_x$,  $a_y$,  $b_x$, and $b_y$.  

First calculate the $x$ and $y$ component of $\vec{u}$ as:
\begin{eqnarray*}
u_x &\equiv& \frac{a_x - b_x}{2} \\
u_y &\equiv& \frac{a_y - b_y}{2} \\
\end{eqnarray*}
Then compute the $x$ and $y$ component to the change in velocity of particle a and particle b:
\begin{eqnarray*}
  \Delta a_x &=& (\cos\theta - 1) \, u_x + \sin\theta \, u_y \\
  \Delta a_y &=& (\cos\theta - 1) \, u_y - \sin\theta \,  u_x \\
  \Delta b_x &=& (1-\cos\theta) \, u_x - \sin\theta \, u_y \\
  \Delta b_y &=& (1-\cos\theta) \, u_y + \sin\theta \,  u_x \\
\end{eqnarray*}
Finally, update the $x$ and $y$ components of the particle velocities to their value after the collision:
\begin{eqnarray*}
  a_x &\to& a_x + \Delta a_x \\
  a_y &\to& a_y + \Delta a_y \\
  b_x &\to& b_x + \Delta b_x \\
  b_y &\to& b_y + \Delta b_y \\
\end{eqnarray*}

In essential technique for programming complicated task is dividing
complicated tasks into smaller tasks, and thoroughly testing the
smaller tasks.  You cannot program effectively until you master this
technique.  I've taught students programming for many years, and the
students that finish last are invariably the ones that rush to
complete their entire program and then try to test and debug it.  This
approach always fails because when you do not get the right answer,
and you won't, ever, on the first try, you have absolutely no idea
what part of a very long chain of calculations is not programmed
correctly.

\begin{figure}[htbp]
\begin{center}
\includegraphics[width=0.65\textwidth]{figs/maxwellboltzman/collide.png} \\
\caption{Collision function.}
\label{fig:collfunc}
\end{center}
\end{figure}


To use this approach in the lab, we'll be implementing the collision
algorithm as a function, exactly as in Fig.~\ref{fig:collfunc}.  This
function takes as input the velocity components {\tt ax}, {\tt ay},
{\tt bx}, {\tt by} as defined above plus the scattering angle {\tt
  theta}.  In Fig.~\ref{fig:collfunc}, the function simply returns the
velocity components unchanged.  You should modify the function to
implement the scattering algoirthm described above.

Normally at this point, you would have to devise your own test to
validate your code.  One technique, that would work here, is to
calculate a few examples and then compare your program output to what
you obtained with paper and pencil.  For this lab, I will provide some
specific example calculations for you to validate your collision function.

\begin{figure}[htbp]
\begin{center}
  \includegraphics[width=0.65\textwidth]{figs/maxwellboltzman/collx.png} \\
  \includegraphics[width=0.65\textwidth]{figs/maxwellboltzman/colly.png} \\
  \caption{Example collisions along the $x$ and $y$ axis.}
\label{fig:collxy}
\end{center}
\end{figure}

\begin{plot} \end{plot}  Implement the collision algorithm as a function as in Fig.~\ref{fig:collfunc} and test it using example collisions from Fig.~\ref{fig:collxy}.

When testing your code, start with easy, special cases, such as used in Fig.~\ref{fig:collfunc}.  This helps makes it clearer where the program is failing.  Once your code works on the simple cases, escalate to more complicated examples.

\begin{figure}[htbp]
\begin{center}
  \includegraphics[width=0.65\textwidth]{figs/maxwellboltzman/collxy.png} \\
  \includegraphics[width=0.65\textwidth]{figs/maxwellboltzman/collrand.png} \\
  \caption{More complicated example collisions.}
\label{fig:collcomp}
\end{center}
\end{figure}

\begin{plot} \end{plot}  Test your collision algorithm using the  example collisions from Fig.~\ref{fig:collcomp}.

\section{Initializing the Simulated Ideal Gas}

You will be modeling an ideal gas by direct Monte Carlo simulation of
{\tt NGAS} representative molecules.  We will use {\tt NGAS=1000}
intially, and you should use an even lower value while debugging.
We'll assume that the mass of each molecule in the gas is $M_0$,
or in the numerical system of units {\tt M=1}.

The state of your simulation will be completely contained in two numpy
arrays {\tt vx} and {\tt vy}, each of length {\tt NGAS}, which contain
the velocities of the particles in units of $V_0 = \sqrt{k_b T_0 /
  M_0}$.  Remember, the simulation uses a system of units that should
keep velocities near 1, so values such as 2.2, -3.1, 0.8, -0.01 are
all likely, and correspond to speeds up to several hundred meters per
second in SI units.  On the other hand, the presence of extremely
small values, like 5.3E-23, and extremely large values like 1.2E18 and
-8.2E28 are symptoms of bugs.

\begin{plot} Initialize both velocity arrays {\tt vx} and {\tt vy} of length {\tt NGAS}
with values choosen as uniform random variables in the range
$[-2,2]$. Fill two histograms, one with $v_x$ and one with $v_y$, with
an approriate range and 20 bins.  You should see that the velocities
are distributed uniformly (a flat distribution).  The distribution does not yet
resemble the Gaussian shape of Equation~\ref{eqn:mbvx} because it has not yet reached
thermal equilibrium.\end{plot}


\section{Collisions of an Ideal Gas}

To reach thermal equilibrium, you'll need to simulate collisions
betweens pairs of molecules in your gas.  For each collision, do the following:
\begin{itemize}
 \item Choose two molecules at random as particles $a$ and $b$. (See {\tt np.random.choice}.)
 \item Choose a random value $\theta$ uniformly in the range $[0,2\pi]$ (See {\tt np.random.uniform}.)
 \item Call your collision function with components of the velocity vectors for particles $a$ and $b$ and the scattering angle $\theta$.
 \item Update the velocity of particles $a$ and $b$ from the return value of your collision funciton
\end{itemize}   
For this model, you will need about 10 times as many collisions as gas
molecules in order to reach thermal equilibrium.

\begin{plot}  For {\tt NGAS=1000} simulate {\tt NCOLL = 10000} collisions as described above.
Fill two histograms, one with $v_x$ and one with $v_y$, with an
approriate range and 20 bins.  After reaching thermal equilibrium, the
distributions should resemble a Gaussian as predicted by Equation~\ref{eqn:mbvx}. \end{plot}

\section{Temperature of an Ideal Gas}

The temperature of the gas is related to the mean kinetic energy by:
\begin{equation}
\label{eqn:kt}
k_b T = m \, \frac{\braket{v_x^2} + \braket{v_y^2}}{2}  
\end{equation}  
You can estimate $\braket{v_x^2}$ from your simulation as {\tt np.mean(vx**2)}.

\begin{plot} Estimate $kT$ of the gas using Equation~\ref{eqn:kt} before and after simulating collisions.
The values should remain near the expected value 4/3.
\end{plot}

\section{The Maxwell-Boltzmann Distribution}

In this section, you'll reproduce the instructor plots of Fig.~\ref{fig:mbinst} using your own numerical simulation.

\begin{figure}[htbp]
\begin{center}
\includegraphics[width=0.85\textwidth]{figs/maxwellboltzman/maxboltz-instr.pdf} \\
\caption{Instructor plots.}
\label{fig:mbinst}
\end{center}
\end{figure}

\begin{plot}
  After your simulation reaches equilibrium, fill two histograms, one with $v_x$ and one with $v_y$, with an approriate range and 10 bins.  Compare with the prediction from Equation~\ref{eqn:mbvx}.  The results should resemble the right side of Fig.~\ref{fig:mbinst}, which were produced with {\tt NGAS=10000}.
\end{plot}

\begin{plot}
  After your simulation reaches equilibrium, fill a histograms with the magnitude of the velocity $v$, with an approriate range and 10 bins.  Compare with the prediction from Equation~\ref{eqn:mbv}.  The results should resemble the left side of Fig.~\ref{fig:mbinst}, which were produced with {\tt NGAS=10000}.
\end{plot}


  

\chapter{Curve Fitting}

\section{Introduction}

In this lab, you learn about curve fitting in Scientific Python.






















\end{document}



\input{lab_gas.tex}

\input{lab_fourier.tex}
\chapter{Simulating a Guitar String}

\section{Introduction}

This lab will introduce the Fourier Series and Transform.


%\section{Example Plots} 
%\begin{figure}[htbp]
%\begin{center}
%\includegraphics[width=0.65\textwidth]{figs/labs//plotting/plotting.png} 
%\caption{Sine function sampled at discrete values.}
%\label{fig:plotsin}
%\end{center}
%\end{figure}



\chapter{Curve Fitting}

\section{Introduction}

In this lab, you learn about curve fitting in Scientific Python.





















\chapter{The Speed of Light}

%\section{Pre-lab Calculations}
%
%\noindent
%1) What is the definition of the meter ? What is the exact value of the speed of light in vacuum?\\
%%The meter  is the length of the path travelled by light in vacuum during a time interval of 1/299792458 second. The exact value for the speed of light in vacuum is 299,792,458 metres per second.
%
%\noindent
%2) How long does it take a light to travel a length of 1m? Using the
%time you just calculated and assuming that the uncertainty in the
%length measurement is $1~\rm cm$, calculate the uncertainty in the
%speed of light you would obtain if you were to use this measurement.
%Using again the time you just calculated and assuming that the
%uncertainty in time measurement is $0.2~\rm ns$, calculate the
%uncertainty in the speed of light you would obtain if you were to use
%this measurement. Which uncertainty is larger? \\
%
%
%\noindent
%3) Light is slowed down in transparent media such as air, water and
%glass.  The ratio by which it is slowed is called the refractive index
%of the medium. Calculate this speed of light in air if the index of
%refraction is 1.0003. Calculate (in \%) how far off is speed of light
%in the air from the speed of light in vacuum? Assuming that in our
%setup we are aiming at few $\%$ accuracy is this correction relevant
%for us?\\

%\section{Safety}
%The laser used in this lab is of low power.  Even so, avoid pointing the laser directly into anyone's eye.

\section{Introduction}

In this lab, you will measure the speed of light in air by measuring
the time between sending and receiving a flash of light over a known
distance. 

The light signal is provided by a laser diode.  Like an
LED, the photons in the laser diode are the result of electrons and
holes recombining.  The laser diode produces simulated emission of
photons from population inversion of holes and electrons injected from
p-type and n-type semiconductors into an intermediate layer of
un-doped intrinsic semiconductor.  

\textbf{Safety: The laser used in this lab is of low power.  Even so, avoid pointing the laser directly into anyone's eye.}



\begin{figure}[htbp]
\begin{center}
\begin{tabular}{cc}
\includegraphics[width=0.8\textwidth]{figs/labs/c_air/cair_diodedriver}
\end{tabular}
\end{center}
\caption{\label{fig:clasercirc} Circuit diagram for the pulsed laser diode.}
\end{figure}

If laser light is produced continuously, we'll have no way to measure
a time difference between sending and receiving the pulse.  Instead,
we'll produce a brief pulse of laser light and a reference signal to
indicate the time at which the pulse was sent.  The circuit for pulsed
laser diode assembly we will be using is shown in
Fig.~\ref{fig:clasercirc}.  You will recognize the passive components
consisting of resistors, diodes, and capacitors.  The MAX7375 (square
symbol) is a silicon oscillator which produces a $1~\rm MHz$ square
wave function.  The 74C04 (triangle symbol) is technically an
inverter, but here they are used to simply produce two independent
copies of the square wave function output.  One copy of the square
wave function is sent to a BNC connector, as the time reference for
sending the laser pulse.  The other copy is send through a capacitor,
which acts as a high-pass filter, converting the step function into
very narrow positive and negative pulses.  The diode D1 rectifies this
AC signal, so that only the positive signal is used to drive the laser
diode, causing a brief pulse of laser light, in sync with the square
wave signal which will be available on the scope.

\begin{figure}[htbp]
\begin{center}
\begin{tabular}{cc}
\includegraphics[width=0.65\textwidth]{figs/labs/c_air/cair_detector}
\end{tabular}
\end{center}
\caption{\label{fig:cdetectorcirc} Circuit diagram for photodiode detector.}
\end{figure}

We'll detect the flash of light using a photo-diode.  The photo-diode
is placed in reverse bias, creating a depletion zone.  When photons
strike the depletion zone, they excite electrons to create
electron-hole pairs, which allows a current to flow.  The receiver
circuit is shown in Fig.~\ref{fig:cdetectorcirc}.  The photo-diode D1
is held under reverse bias by the externally supplied DC voltage.  The
current pulse created when the laser light reaches the photo-diode is
amplified by the SKYC5017 broadband amplifier.  All amplifiers have
limits to their bandwidth, but this amplifier is fast enough to handle
the brief laser pulse that we are sending.  The input connector for
this device has an internally generated DC voltage, so we use the
capacitor C5 to isolate this DC voltage from our circuit.  The
high-frequency AC signal pulse we wish to amplify will see this
capacitor as effectively a short-circuit.  All amplifiers require
external DC power, but this one is a bit peculiar in that the DC power
is supplied at the output pin.  This explains the use of inductors L1
and L2 and capacitor C6.  Remember inductors are a short-circuit to DC
and an open circuit to AC, where as capacitors are an open-circuit to
DC and a short-circuit to AC.  The DC supply is provided to the output
pin through the inductors, but the amplified AC output signal passes
through the capacitor.


\begin{calculation}  Calculate and record in your logbook how long does it take a light to travel a length of 1m? Using the
time you just calculated and assuming that the uncertainty in the length measurement is $1~\rm cm$, calculate and record the uncertainty in the
speed of light you would obtain if you were to use this measurement.
\end{calculation}

\begin{calculation} Using again the time you just calculated and assuming that the
uncertainty in time measurement is $0.2~\rm ns$, calculate the
uncertainty in the speed of light you would obtain if you were to use
this measurement. Which uncertainty is larger? 
\end{calculation}

\section{Experimental Setup}

\begin{figure}[htbp]
\begin{center}
\begin{tikzpicture}%[show background grid] %% Use grid for positioning, then turn off
    \node[anchor=south west,inner sep=0] (image) at (0,0,0) {\includegraphics[width=0.7\textwidth]{figs/labs/c_air/setup.jpg}};
    %\node[right](X) at (10.0,5.1) {\parbox{3cm}{\flushleft Trigger (yellow)}};
    \draw [dashed,red,thick] (4.55,5.4) -- (9.5,2.75);
    \draw [dashed,red,thick] (2.35,4.3) -- (9.5,2.75);
\end{tikzpicture}
\end{center}
\caption{\label{fig:csetup} Setup.}
\end{figure}

The setup is shown in Fig.~\ref{fig:csetup}.  You will use your
bench-top DC power supply to power the pulsed laser diode and the
photo-diode receiver assemblies.  The right most pair of output from
your supply provides a fixed $5~\rm V$ DC output, which you will use
to power the pulsed laser diode assembly (transmitter).  The
transmitter is housed in the green box. The reference signal for the
transmitter is output on the BNC connector, and should be connected to
your channel 1 of your scope, {\bf using a $50~\rm \Omega$ terminator.}

The photo-diode receiver circuit is housed in the red box.  It should
be powered at $10~\rm V$ from your bench-top DC power supply.  The
amplified signal output on the BNC connector should be connected to
channel 2 of your scope, {\bf using a $50~\rm \Omega$ terminator.}

%To suppress high-frequency noise, you can try installing RF chokes
%around your coaxial cable near the receiver and transmitter.  

\begin{figure}[htbp]
\begin{center}
\begin{tabular}{cc}
\includegraphics[width=0.7\textwidth]{figs/labs/c_air/tzero.jpg}
\end{tabular}
\end{center}
\caption{\label{fig:ctzero} Initially, point the laser directly into the receiver.}
\end{figure}

\begin{figure}[htbp]
\begin{center}
\begin{tabular}{cc}
\includegraphics[width=0.7\textwidth]{figs/labs/c_air/time_peak.jpg}
\end{tabular}
\end{center}
\caption{\label{fig:ctime} Measure the time interval from the receiver reference signal (yellow, offset by two divisions) crosses zero to the lowest point in the signal pulse (blue).}
\end{figure}

To test your setup and the equipment, point the laser directly into
the receiver as shown in Fig.~\ref{fig:ctzero}.  Use folded printer
paper to adjust the height and orientation of the boxes as needed.  

Make certain you have a scope with a bandwidth of 100 MHz.  The
receiver signal is the most intermittent, so to avoid seeing empty
wave-forms, you should always trigger on the receiver signal.  Set
both channels to AC coupling and make certain that the bandwidth limit
is off.  Set the time-scale to 5 nanoseconds.  Then measure the time
offset ($\Delta t$ at zero distance) using your scope, as shown in
Fig.~\ref{fig:ctime}.  Ideally, when making a timing measurement, you
should use a sharp edge.  This is why the reference signal is measured
on the rising edge.  The receiver signal, however, is quite noisy, so
the edge is easily distorted.  A slightly more reliable measurement
comes from the using the minimum of the pulse.  There is significant
variation of the pulse height, so it helps to set the trigger level to
only select the largest pulses (by moving the trigger threshold as far
toward the bottom of the screen as possible).  Because of timing
jitter, you want to acquire a single waveform to make the time
measurement, using the scopes run/stop or the single button.  At each
position, you should make at least three measurements.  Despite the
jitter, you should see that the time interval measurements are fairly
stable, varying by about 0.2 nanoseconds, the cursor resolution at the
five nanosecond scale.  
\begin{measurement} Record the results of
your time offset measurement and estimate it's uncertainty. 
\end{measurement}


\section{Speed of Light Measurement}

\begin{figure}[htbp]
\begin{center}
\begin{tabular}{cc}
\includegraphics[width=0.7\textwidth]{figs/labs/c_air/mirror.jpg}
\end{tabular}
\end{center}
\caption{\label{fig:mirror} Mirror held in a swivel-mount vise}
\end{figure}

Install a mirror in your swivel mount vise as shown in
Fig.~\ref{fig:mirror}. Move the transmitter to point the beam down the
long access of your lab bench.  Place the mirror approximately 20 cm
away from the transmitter as measured with a meter stick.  Using a
piece of white paper to track the beam spot, adjust the transmitter
until the beam is pointed at the mirror.  Place the receiver next to
the transmitter and pointed at the mirror.  Adjust the mirror until
the beam is directed back to the receiver. 

Once you have the beam pointed toward the receiver, orient your scope
display so that you can see it clearly from the mirror.  Set the
trigger to just below the noise level, and adjust the mirror until the
beam points into the receiver and a clear signal appears on your
scope.  Tighten the swivel mount to hold the mirror in place.  You may
find that the mirror moves slightly after you remove your hands and
the signal is lost.  If this is happening, try gradually tightening
and adjusting the mirror, or lightly tapping it while the swivel-mount
is tight.  Continue adjusting until you have a clear signal.  

\begin{measurement} Sketch the setup in your lab book. 
Each lab partner should make their own measurement of
the time difference between the transmitter and receiver signals, and
record all measurements in your logbooks.  Then each lab partner
should make a measurement of the total beam distance to the nearest
0.5 cm.  Record all measurements in your logbooks.  Estimate the
uncertainty on your measurement.  The resolution of the cursor time
measurement at the 5 ns scale is 0.2 ns.  The meter stick has a
resolution of about 0.5 cm.  If your measurements are consistent with
one another, you can use these resolutions as your uncertainties. 
If not calculate uncertainty on the mean as we learned in the class. 
From these measurements and the timing offset measured previously,
calculate the speed of light and an uncertainty. \end{measurement} 

\begin{calculation} How does your measured value compare to the known value for the speed of light? Record this observation in your log book. Remember you can express agreement in terms of "sigma". How you can improve the accuracy of your measured value? Hint: Use time offset measurement.  After correction how does your measured value compare to the the known value for the speed of light? \end{calculation}

 \textbf{This is a first sign-off point for the lab.}

\begin{measurement} Repeat the measurement for a starting position of the
mirror at 40 cm, 60 cm, 80 cm, 100 and 120 cm. 
\end{measurement}

\section{Analysis}

\begin{plot} Plot the data with the x-values populated by the
quantity with smaller uncertainty and y-values populated by the
quantity with the larger uncertainty. Include x and y-uncertainties in
the plot. Perform a straight line fit using {\tt curve{\_}fit}
function. Include y-uncertainties in the fit and be sure to set {\tt absolute{\_}sigma=True}.  From the fit values
calculate the speed of light together with its uncertainty. \end{plot}

A major source of systematic uncertainty in this measurement comes
from the timing measurement.  The transmitter reference has a nice
sharp edge, but the signal pulse is distorted by amplification.  As
distance traveled by the light pulse increases, the signal becomes
smaller, and the signal shape changes, which changes the measured time
interval.  This introduces non-linearity into the relationship between
the distance and your measured time.

\begin{plot} To estimate the size of this effect, you can simply set {\tt
  absolute{\_}sigma=False}.  This setting instructs the {\rm curve{\_}fit}
function to adjust the uncertainties until they are consistent with
the linear relationship assumed in the fit. 
\end{plot}
\begin{print}  You can then interpret
the parameter uncertainty as including both the statistical
uncertainty and the systematic uncertainty due to this non-linearity.  Print your result in the following form 
\begin{displaymath}
 \textbf{c}= \textit{your value here} \pm \textit{your value here} \,  ({\rm stat}) \pm\textit{ your value here} \, ({\rm syst})
\end{displaymath}
Assuming systematic and statistical uncertainties are considered to be independent, calculate the total experimental uncertainty as their sum in quadrature. 
\end{print}

\begin{print} 
How does your measured value compare to the known value for the speed of light?
\end{print}

\begin{print} 
What is the meaning of the intercept value different than zero? 
\end{print}

\textbf{This is a second sign-off point for the lab.}


%\chapter{Statistics of Radioactive Decays}


\section{Introduction}

\begin{figure}[htbp]
\begin{center}
 \includegraphics[width=0.55\textwidth]{figs/labs/geiger/source.jpg};
\caption{\label{fig:source} A sealed radioactive source.  A small amount of Cs-137 is contained within the small button shaped piece of plastic.  For your safety, the sources will be handled only by the TA.}
\end{center}
\end{figure}

In this lab, you will use a Geiger Counter to study the statistics of radioactive decays.

\section{Precautions}

\noindent
{\bf Precautions with the Geiger counter:}
\begin{itemize}
\item Leave the cable from the Geiger counter controller to the Geiger counter in place {\em at all times}.  This carries voltages of approximately 1000 volts.  If you leave the cable in place, nothing can be inadvertently plugged in (including fingers)
\item Leave the Geiger tube in its holder.  It has a thin front window which is easily broken.
\item Do not set the high voltage higher than 1000 volts.
\end{itemize}

\noindent
{\bf Precautions with the radioactive source:}
\begin{itemize}
\item See Fig.~\ref{fig:source} to familiarize yourself with what the sources look like.
\item Don't touch the source.
\item Leave the source in the tray at all times.  The TA will provide the sources and handle moving them from place to place.
\item Radiation falls off as $1/r^2$.  So minimize your time near sources and maximize your distance from them.
\end{itemize}


\section{The Geiger Counter}


\begin{figure}[htbp]
\begin{center}
\begin{tikzpicture}
    \node[anchor=south west,inner sep=0] (image) at (0,0,0) {\includegraphics[width=0.55\textwidth]{figs/labs/geiger/assembly.jpg}};

    \node[right](X) at (10.0,3.0) {Timer};
    \draw (X.west) -- (8.0,3.5);

    \node[right](X) at (10.0,5.0) {\parbox{3cm}{\flushleft High-Voltage and Counter}};
    \draw (X.west) -- (5.0,4.75);

    \node[left](X) at (0.0,4.5) {\parbox{2.5cm}{\flushright Geiger Tube Holder}};
    \draw[white,thick] (X.east) -- (1.25,5.0);
    \draw (X.east) -- (1.25,5.0);

    \node[left](X) at (0.0,3.0) {Source};
    \draw (X.east) -- (1.35,4.05);

\end{tikzpicture}
\caption{\label{fig:geigersetup} The Geiger Counter assembly.}
\end{center}
\end{figure}

To begin, you will familiarize yourself with the counter and timer
features of your Geiger counter assembly using the built-in test mode.
Your lab bench will already be prepared with a Geiger Counter assembly
as shown in Fig.~\ref{fig:geigersetup}.  Ensure that the high-voltage
(HV) is off by turning the knob labeled ``H.V. Adjust''
counter-clockwise all the way to zero.  Now put the Geiger counter
into test mode by flipping the left red switch to ``TEST''.  Flip the
right red switch to ``COUNT'' and you should see the counter display
begin incrementing.  Push the button on the front of your timer and
you should see the Timer turn on and off.  Leave the timer
incrementing.  Now flip right red switch to ``STOP'', and observe that
the both the counter and the timer stop simultaneously.  The knob on
left side of the old-school lab timer can be used to reset the time.
Keep turning the knob clockwise until the time reads 0.  Use the black
button on the Counter to reset the count to zero.

Flip the right switch to ``COUNT'' and then back to ``STOP'' when 10
seconds have passed.  During this time, the $60~\rm Hz$ test signal
should increment the counter close to 600 times.  Try this a few times
and make sure you can reliably count close to $600$ test pulses in a
10 second interval.  You should reset the count each time, but there
is no need to reset the timer.  Simply stop when the timer reaches the
next factor of ten.  Due to your reaction time, you may well stop at
one-to-two tenths of a second later.  This is OK, and will only add
less than a few percent error to your measurements over 10 second
intervals.

\section{High-Voltage Calibration}

When you are confident that you know how to operate the timer, switch
the left red switch to ``USE'' mode.  Ask the TA to provide you with a
sealed radioactive source in the second shelf from the top of your
Geiger tube holder.  Switch the right switch to ``COUNT'' mode.  With
the HV off, you should not see any pulses.  Turn the HV up until you
begin to see counts increment on the display, and continue to the next
interval of 50 volts (e.g. if it first starts incrementing at 730
volts, set the dial to 750 volts).  Count the number of events in a
ten second interval.

Repeat this measurement, twice for each voltage setting, in 50 volt
steps up to 1000 volts.  Do not exceed 1000 volts.

{\bf Plot 1: } Plot the rate (in Hz) as a function of high voltage.
You should see a plateau region (a leveling off) which indicates the
onset of the Geiger mode within the Geiger tube.  From your plot,
chose a high-voltage near the beginning of the Geiger mode, and set
the high-voltage to this calibrated value.

\begin{figure}[htbp]
\begin{center}
 \includegraphics[width=0.55\textwidth]{figs/labs/geiger/pulse.jpg};
\caption{\label{fig:geigerpulse} An example Geiger counter pulse.}
\end{center}
\end{figure}

Connect an oscilloscope to the output of the counter assembly (on the
back, labeled ``SCOPE'').  Adjust your scope to view the Geiger pulses
like that of Fig.~\ref{fig:geigerpulse}.  Note that the Geiger counter
output contains a DC component in addition to the AC pulse, so you
will want to use your scope in AC coupling mode which will remove the
DC component and allow you to see the pulse.  You will also want to
see the attenuation to 1X because you are not using an attenuating
probe.  The rate of the 




\section{Data Collection}

Even in today's world of massive amounts of automation, it is still
useful to know how to collect a small amount (up to a few hundred data
points) of data manually.  Often in the lab, you have one-off
measurements that you would like to make without investing in a lot of
automation.

In this section, you will collect data manually for about one hour.
Practice a routine with your lab partner that allows you to take and
record the data as fast as possible.  For instance, person A should
operate the counter, and person B should use the PC.  Person A turns
the counter on for ten seconds, turns it off, and says (quietly)
``OK''.  Person B records the value on the PC and says ``Go''.  Person
A resets the counter and continues.  Remember that there is no need to
reset the Timer each time, which would take too long, and which would
actually be counterproductive (if you consider the effect of a roughly
constant reaction time.) 

Practice your routine a few times, and make sure your count is near 1000 events in
a ten second interval.  Then record 200 data points.

When you have finished recording your data with the radioactive
source, ask your TA to remove the source and return it to the
radioactive locker.

Now record an additional 200 data points with no source, to measure
the background radiation rate.  You should record around 3 background
counts per 10 second interval.

\section{Analysis}

\begin{figure}[htbp]
\begin{center}
\begin{tabular}{cc}
\includegraphics[height=0.22\textheight]{figs/labs/geiger/background.pdf}
\includegraphics[height=0.22\textheight]{figs/labs/geiger/source.pdf}
\end{tabular}
\end{center}
\caption{\label{fig:geigeranalysis} Numerical simulation of the experiment
  for (a) background radiation only, and (b) radioactive source
  present.}
\end{figure}

Using Scientific Python, measure the mean and variance of your
collected background and source data.  Then produce histograms to
display your data as in Fig.~\ref{fig:geigeranalysis}.  For the
background data, plot the histogram for eleven bins: 0,1,2,...,10.
For the source data, plot about 20 bins covering a few hundred counts
around the mean value.

Compare your collected background data to a Poisson distribution,
appropriately normalized, with a mean set to the mean of your data.
Compare your collected source data to a Gaussian distribution,
appropriately normalized, with a mean set to the mean value of your
data, and sigma set to the square root of your mean.











%
\chapter{Measurement of Planck's Constant}

\section{Introduction}

In this lab, we will measure Planck's constant by measuring the
$V$-$I$ curves of three different colored light emitting diodes
(LEDs).  An LED is a particular type of diode for which the
recombination of electrons and holes produces photons, typically in
the visible light spectrum.  These diodes have an activation voltage given by:
\begin{equation} \label{eqn:va}
V_{\rm A} = \phi + \frac{hc}{e}\frac{1}{\lambda}
\end{equation}
where $\lambda$ is the wave-length of the light produced by the diode,
and $\phi$ is the contribution to the voltage drop due to other
effects in the $p-n$ junctions.  The diodes we are using have been
chosen to ensure that $\phi$ is approximately constant across all
three diodes.

The quantity
\begin{displaymath}
\frac{hc}{e}
\end{displaymath}
can therefore be determine from the slope of the activation voltage as a function of $1/\lambda$.

The 2018 redefinition of the SI is means that the quantity
\begin{displaymath}
hc = 1.23984193~\rm eV \mu m
\end{displaymath}
is technically now exactly known, because the values $h$, $c$, and $e$
are now taken as exact values which define the corresponding SI
units. Of course, it is still useful and fun to measure this quantity
ourselves in the lab.  Since we will also be measuring the speed of
light, we'll interpret this measurement as our determination of
Planck's constant.

\section{LED Model}

For the purpose of this experiment, we will model the LED as an ideal
diode with voltage drop equal to the activation voltage $V_{\rm A}$ of
Equation~\ref{eqn:va} plus a series resistance $R_{\rm LED}$.  As
shown in Fig.~\ref{fig:ledmodel}, the effect of this resistance is to
replace the vertical line at the activation voltage $V_A$ with a line
of slope $1/R_{\rm LED}$.

\begin{figure}[htbp]
\begin{center}
\includegraphics[height=0.3\textheight]{figs/labs/planck/model.pdf} \\
\end{center}
\caption{The LED model for $V_A=1.7~\rm V$ and $R_{\rm LED}=12~\Omega$}
\label{fig:ledmodel}
\end{figure}



\section{$V-I$ Curve from LED.}

\begin{figure}[htbp]
\begin{center}
\begin{circuitikz}[line width=1pt]
\draw (0,0) to[voltage source,bipoles/length=1.5cm,l=$V_1$] ++(0,4.0) to[short] ++(2.0,0) coordinate(C);
\draw (C) to[R, l_=$R_1$] ++(0,-2.0) coordinate(B) to[empty diode, l_=LED] ++(0,-2.0) coordinate(A) to[short] ++(-2.0,0.0);
\draw (A) to[short,*-] ++(1.5,0) to[short] ++(0,0.8) node[component]{V} to[short] ++(0,0.8) to[short,-*] ++(-1.5,0);
\draw (C) to[short,*-] ++(1.5,0) to[short] ++(0,-0.8) node[component]{V} to[short] ++(0,-0.8) to[short,-*] ++(-1.5,0);
\end{circuitikz} 
\end{center}
\caption{experimental setup.}
\label{fig:planck_setup}
\end{figure}

Construct the experimental setup shown in Fig.~\ref{fig:planck_setup}
using a $1~\rm k\Omega$ precision $1\%$ resistor for $R_1$, a green
LED, and using your bench-top DC supply to provide $V_1$, initially
set to $0~\rm V$.  Connect one voltmeter across the resistor $R_1$ and
another voltmeter across the LED.  When constructing your circuit,
make sure the LED can be easily removed and replaced.  Also keep in
mind that the longer lead is the positive terminal of the LED, i.e.,
the upper terminal of the LED as drawn in the
Fig.~\ref{fig:planck_setup}.  Set both voltmeters to the $20~\rm V$
setting.

Test your circuit first by turning up the voltage on your supply to
about $5~\rm V$ and checking that the LED lights up.  The voltmeter
across the resistor $R_1=1~\rm k\Omega$ is effectively measuring the
current in $mA$ as $1~{\rm V}/1~{\rm k\Omega} = 1~\rm mA$.  Do not
misunderstand this statement to mean you should set the multi-meter to
the current measurement: we measure the voltage, but from $Ohm's Law$
we know the current in resistor.  The voltmeter across the diode is
measuring the diode drop $V_{\rm D}$.

Take a series of measurements of $I$ and $V_D$ near target values of
$I = 0.5,1.0,2.0,4.0,6.0,8.0,10.0,12.0~\rm mA$ by adjusting the
voltage provided by your DC supply until the current $I$, as measured
with the voltmeter across $R_1$, is near the target value.  Remember
not to waste time fussing to make the measurement at exactly the
target value.  For instance, measuring at $I=2.16~\rm mA$ instead of
the target $I=2.0 ~\rm mA$ is perfectly acceptable.  Simply record the
actual measured value of $I$ next to target value, in addition to
measurement of $V_{\rm D}$.  

\begin{table}
\begin{center}
\caption{Instructor data from a red LED not used.}
\begin{tabular}{lll}
target $I$ (mA) & $I$ (mA) & $V_{\rm D}$ (V) \\
\hline
0.5  &  0.44  & 1.62 \\
1.0  &  0.95  & 1.66 \\
2.0  &  1.99  & 1.71 \\
4.0  &  4.15  & 1.77 \\
6.0  &  6.05  & 1.81 \\
8.0  &  8.09  & 1.85 \\
10.0 &  10.04 & 1.88 \\
12.0 &  12.02 & 1.91 \\ 
\end{tabular}
\end{center}
\end{table}

Repeat this measurement using a green and yellow LED.

\begin{table}[htbp]
\begin{center}
\caption{LEDs used in this experiment.}
\begin{tabular}{llll}
color & part no. & $\lambda$ (nm) & max current \\
%blue  & WP710A10QBC & 460 & 30 mA \\
green & WP710A10GT & 565 & 25 mA \\  
yellow & TLHY4405 & 581-594 & 30 mA \\ 
red & TLHR4405 & 612-625 & 30 mA \\ 
\end{tabular}
\end{center}
\end{table}

\section{Analysis}

\begin{figure}[htbp]
\begin{center}
\begin{tabular}{cc}
\includegraphics[width=0.45\textwidth]{figs/labs/planck/fit_diode.pdf} &
\includegraphics[width=0.45\textwidth]{figs/labs/planck/fit_vi.pdf} \\
(a) & (b) \\
\end{tabular}
\end{center}
\caption{Instructor fit for red LED.  The (a) real diode response
  departs from the simple linear model for $I \leq 2~\rm mA$, so the
  (b) fit for the linear response is performed for $I > 2~\rm mA$.
  Note that $V$ is taken as the $y$-axis for the purpose of fit,
  because the voltage uncertainties are the dominant uncertainty in
  the fit.  Notice also that the $0.02~\rm V$ uncertainty reported by
  the DMM specifications appears to be predominantly systematic.}
\label{fig:redfit}
\end{figure}

The V-I response for a red diode as measured by the instructor is
plotted in Fig~\ref{fig:redfit}a.  Notice that at large values for the
current ($I > 2~\rm mA$) the VI response is linear, indicating that it
is dominated by internal resistance of the diode.  As lower current
($I \leq 2~\rm mA$) the VI response is exponential, as expected for a
real diode.  We will fit our simple linear model for the diode
response only in the region where this approximation is valid, for
$I>2~\rm mA$.  As shown in Fig.~\ref{fig:redfit}b, perform a linear
fit only for the data with $I>2~\rm mA$.

With the DMM set to the $20~\rm V$ scale, the uncertainty on your
measured values for $V$ and $I$ is approximately $0.02~\rm V$ and
$0.02~\rm mA$ respectively.  Because the measured voltage range is
less than a volt, but the measured current range is around $10~\rm
mA$, it is the uncertainties on the voltage that dominate the
uncertainty on both the slope and intercept of the linear function.
We therefore choose to use $V$ as the $y$-axis and and $I$ as the
$x$-axis for the purposes of fitting, because the $\chi^2$ analysis of
the {\rm curve{\_}fit} function only considers uncertainties in the
$x$-axis.  There are straightforward ways to handle uncertainties in
both $x$ and $y$, but they add complexity which is best avoided if
possible.

For each LED, fit the $V$ versus $I$ data to a linear function, and
determine the best fit resistance and activation voltage, and the
uncertainties.  Assume the uncertainty on the measured voltages is
$0.02~\rm V$ for each data point and only consider $I > 2~\rm mA$ in
the fit.  Remember to set {\tt absolute(\_)sigma=True} in your fit so
that the fit uncertainties are based on the absolute uncertainties
without re-scaling.

\begin{figure}[htbp]
\begin{center}
\includegraphics[height=0.3\textheight]{figs/labs/planck/planck.pdf} \\
\end{center}
\caption{Instructor plot for determination of $hc$.}
\label{fig:planckfit}
\end{figure}

Plot the best-fit $V_A$ of each diode versus $1/\lambda$, and
determine the slope ($hc/e$) and its uncertainty from a linear
fit.  The instructor plot is shown in Fig.~\ref{fig:planckfit}.
Recall that $1~\rm eV$ is the change in potential energy of one
electron passing through $1~\rm V$ of potential energy, allowing you
to conveniently convert the slope in $\rm V \mu m$ to $\rm eV \mu m$ for
comparison with the established value $hc = 1.240~\rm eV \mu m$.

\section{Systematic Uncertainties}

In addition to the statistical uncertainties reported by the fit,
their are a number of systematic uncertainties.  For this lab, we'll
consider one obvious source of systematic uncertainty, which arises
from our treatment of the real diode response as a simple linear
function.

To determine the size of this systematic uncertainty, we measure the
effect of this assumption on our measured valued.  One simple way to
estimate this effect is to remove the requirement $I>2~\rm mA$ from
our analysis.  The difference between the measured value for $hc$ as
determined with and without the cut on $I>2~\rm mA$ can be interpreted
as a systematic uncertainty due to our simplistic model.

Repeat your analysis without the requirement $I > 2~\rm mA$ and take
this as the overall systematic uncertainty on your measurement.
















\chapter{Muon Lifetime}

\section{Introduction}

\begin{figure}[htbp]
\begin{center}
{\includegraphics[width=0.35\textwidth]{figs/labs/muon/cosmic_ray.jpg}}\\
\end{center}
\caption{\label{fig:cosmic}  Cosmic ray shower induced by a primary cosmic ray (typically protons) striking an atom in the upper atmosphere.}\end{figure}

The muon is a fundamental particle in the Standard Model of particle
physics.  It is essentially a heavier version of the electron.  Like
the electron, the muon has a corresponding anti-particle, the
anti-muon ($\mu^+$).  Muons are readily available for study because
they are produced as a result of showers that are induced by incident
cosmic rays that constantly bombard the earth.  Typical primary cosmic
rays are protons, and there collisions with nuclei in the upper
atmosphere produce mainly protons, neutrons, and pions.  The
subsequent decays of these particles produce electrons, neutrinos, and
the muons we will be studying.  The flux of muons at sea-level is
about 1 per ${\rm cm}^2$ per minute, and this population has a mean
kinetic energy of about $4~\rm GeV$.  Such muons are highly
penetrating: they pass quite readily through buildings.

The muon decays via the weak interaction, its most probable decays being:
\begin{eqnarray*}
\mu^- \to e^- + \bar{\nu}_e \nu_\mu,\\
\mu^+ \to e^+ + \bar{\nu}_\mu \nu_e.
\end{eqnarray*}
As fundamental particles, muons are indistinguishable from one
another, and therefore the decay rate for a population of $N$ muons
must be simply proportional to $N$:
\begin{displaymath}
\frac{dN}{dt} = -\frac{N}{\tau}
\end{displaymath}
The solution to this differential equation is:
\begin{displaymath}
N(t) = N(0) \exp(-t/\tau).
\end{displaymath}
In this lab, you will measure the lifetime of the muon, which has the value 
\begin{displaymath}
\tau_\mu = 2.1969811(22) \mu s
\end{displaymath}
in vacuum as reported by the Particle Data Group (PDG) with the
uncertainty in parenthesis.  Interactions with the scintillator
material in this experiment lead to a slightly different expectation
for the lifetime as discussed below. For this lab there are only jupyter notebook entries. 

Muons are produced at a typical height $15~\rm km$ above sea-level,
and so, in the earth frame, their transit time from the upper
atmosphere to our lab is therefore about $50~\rm \mu s$ or about 20
lifetimes.  The fact that we see an appreciable number of them at sea
level, given an upper limit on their production rate in the
atmosphere, is clear evidence for time dilation.

\section{Experimental Setup}

\begin{figure}[htbp]
\begin{center}
{\includegraphics[width=0.55\textwidth]{figs/labs/muon/setup.png}}\\
\end{center}
\caption{\label{fig:setup}  Experimental setup.}\end{figure}

The active component of the particle detector used in this experiment
is a polyvinyltoluene-based plastic scintillator in the shape of a
cylinder with a $15~\rm cm$ diameter and a $12.5~\rm cm$ height.  All
materials absorb energy due to the passage of ionizing radiation.
Scintillators are materials which re-emit a fraction of this energy as
visible light, typically in the blue to near UV range.  The light
yield is relatively low, so a sensitive photomultiplier tube is used
to amplify a modest number of photons into a large easily measurable
voltage.

\begin{figure}[htbp]
\begin{center}
{\includegraphics[width=0.65\textwidth]{figs/labs/muon/cosmics_raw.pdf}}\\
\end{center}
\caption{\label{fig:cosmics_raw}  Three days of cosmic ray data, plotted as the distribution in digitized time measurement (TDC counts).}
\end{figure}


Muons are therefore constantly passing through the scintillator,
depositing energy, and causing observable PMT pulses which are
recorded by the data acquisition system (DAQ).  Occasionally, however,
a relatively low-energy muon enters the scintillator and deposits all
of it's kinetic energy, coming to rest.  As an unstable particle, it
eventually decays into a highly-energetic electron and neutrinos.  The
electron deposits additional energy in the scintillator which is
observed as a second pulse.  The time interval between two consecutive
pulses is digitized using a time-to-digital (TDC) converter, which
converts analog time interval into a digital number.  In this case, we
use a 12-bit TDC, so the measured digital values are integers in the
range from 0 to $2^{12}-1 = 4095$.  To interpret these raw TDC counts
as a time value in microseconds, you will need to calibrate the TDC as
described below.


\begin{plot} Plot the three days of data that has already been collected for you. An example is shown in Fig.~\ref{fig:cosmics_raw}.  The data is available on the course
web site in the file {\tt cosmics.npy}. You can load the data
using the numpy {\tt load} command, taking care to handle the
directory path correctly.  As can be seen from the plot, most of the
recorded events have the maximum possible value 4095.  This is how the
DAQ records a single pulse, with no secondary pulse in the timing
window of the TDC.  This is what happens most of the time. \end{plot}

The remaining events, with a measured time below the maximum value, are from events were two pulses were detected within the timing window of the TDC.  This time
interval, the decay time of the muons in this population, will have a
time-dependence given by:
\begin{displaymath}
-\frac{dN}{dt} = \frac{N(0)}{\tau} \exp(-t/\tau)
\end{displaymath}
which you will use to extract the muon life time.  This is possible because the muon decay rate ($\lambda$)
and it's lifetime $\tau$ are simply recipocrals:
\begin{displaymath}
\lambda \equiv \frac{-\frac{dN}{dT}}{N(t)} = \frac{1}{\tau}.
\end{displaymath}
This exponential decay feature is clearly visible in the raw data as the downward sloping line on the right side of this semilog plot.

There is an additional source of background for two pulse events, which arises from the possibility for two muons to arrive independently within the time window of the TDC.   Sine the arrival time of the two muons is independent, this contribution is flat in time, and is clearly visible in the raw data on the left-hand side, where the exponential contribution becomes small.

\section{Calibration}

\begin{figure}[htbp]
\begin{center}
{\includegraphics[width=0.65\textwidth]{figs/labs/muon/pulser_4_raw.pdf}}\\
\end{center}
\caption{\label{fig:pulser_raw}  Calibration run with pulses spaced $4~\mu s$ apart.}
\end{figure}

To interpret the output of the TDC (digital TDC counts as integers in
the range from 0 to 4095) as physical times in units of microseconds,
a calibration is needed.  We are using a good TDC which we can assume has a linear response, meaning that the time interval and TDC counts are related as:
\begin{displaymath}
TDC = a \cdot \Delta t  + b 
\end{displaymath}

To determine the calibration parameters $a$ and $b$ we use an LED
pulser feature built into the detector.  The LED pulser feature
flashes light into the PMT in a controlled manner, so that two pulses
a distance $\delta t$ in time apart are sent repeatedly to the TDC.
An example of 1000 pulser events with $\Delta t = 4 \mu s$ is shown in
Fig.~\ref{fig:pulser_raw}.  A total of four pulser runs of 1000 events
each with $\Delta t = 1,2,4,8$ have already been collected.  The data
from these four calibration runs available on the course website with
names like {\tt pulser{\_}4.npy} for the $\Delta t = 4~\rm \mu s$ run.

\begin{plot} For each pulser run, compute the mean value of the recorded data and an uncertainty on this mean value. Plot these mean values, and their uncertainties, versus $\Delta t$.  Fit the linear response to obtain the parameters $a$ and $b$.  Plot the best fit linear function alongside the data. \end{plot}

\section{Muon Lifetime Analysis}

\begin{plot} With the calibration constants $a$ and $b$ convert the
TDC counts of the recorded cosmics data and plot the distribution of
times.  You should omit all events with RAW TDC count of 4095, as
these are single pulse events. \end{plot}

 \begin{plot}  Fit the distribution to an exponential
function plus a constant value for the flat (combinatoric) background.
Plot your fit function against the recorded data. \end{plot}

\begin{print} Report the statistical uncertainty on the fitted value for
the muon lifetime. \end{print}

\textbf{This is a sign-off point for the lab.}

\begin{print} In addition, there are several systematic
uncertainties associated with the experiment.  The largest is the
effect of material interactions on the lifetime of the muon, which is
about $5\%$.  Other systematic effects are smaller, and can be
neglected. Assuming systematic and statistical uncertainties are considered to be independent, calculate the total experimental uncertainty as their sum in quadrature. \end{print}

\begin{print} 
How does your measured value compare to the known value for the muon lifetime?
\end{print}

\begin{print} Print $\chi^2$ and reduced $\chi^2$ value. Is the fit reasonable based on the reduced $\chi^2$ value? Print your comment as well. \end{print}

%\section{Muon / Anti-Muon Ratio}

%Due to additional decay processes when captured by an atomic nucleus, the decay time of the negatively %charged muon in matter is lower than its free space value:
%\begin{eqnarray*}
%\tau_- &=& 2.043(3) \times 10^{-6} \mu s \\
%\tau_+ &=& 2.1969811(22) \times 10^{-6} \mu s
%\end{eqnarray*}
%where we have used the measured values in carbon, which should be close to value for our organic %scintillator.

%As discussed in class, you can determine the ratio $\rho$ of anti-muons to muons from the relation:
%\begin{displaymath}
%\rho = - \frac{\tau_+}{\tau_-} \cdot \frac{\tau_- - \tau}{\tau_+ - \tau}
%\end{displaymath}
%where $\tau$ is the measured valued of the muon lifetime.



\end{document}

% OTHER IDEAS:
% Fourier Transforms
% Monte Carlo techniques
% Speed of Light in Cable
% Measure e from shot noise
