\documentclass[12pt]{book}

\usepackage[dvips,letterpaper,margin=0.75in,bottom=0.5in]{geometry}
\usepackage{cite}
\usepackage{slashed}
\usepackage{graphicx}
\usepackage{amsmath}
\usepackage{amssymb}
\usepackage{braket}
\begin{document}



\title{PHY 130A \\ Lecture Notes: \\ 
QED \\
(Griffith's Chapter 7)}
\author{Michael Mulhearn}

\maketitle

\setcounter{chapter}{7}
\chapter{QED}

\section{Dirac Equation}

\section{Matrices of the Dirac Equation}

\section{Solutions to the Dirac Equation}


\section{A Remarkable Matrix}
Let's start by defining a matrix $S$ and exploring its properties:
$$S \equiv a_+ \; + \; a_- \, \gamma^0 \, \gamma^1$$
where:
$$a_\pm = \pm \sqrt{\frac{\gamma \pm 1}{2}}$$
with $\gamma$ the usual factor for a Lorentz transformation.  Let's take note right from the start the very useful identities:
\begin{eqnarray*}
a_{+}^2 - a_{-}^2 &=& 1 \\
a_{+}^2 + a_{-}^2 &=& \gamma \\
2 a_{+} a_{-} &=& -|\beta| \gamma \\
\end{eqnarray*}
from which we can see that we can construct every element of the a Lorentz Boost from $a_+$ and $a_-$.

For clarity in what follows, we will explicitly write out the 2x2 identity matrix 
$$I \equiv \begin{pmatrix} 
1 & 0 \\
0 & 1 \\
\end{pmatrix}
$$
even though this is very often left out of expressions when its existence can be inferred implicitly.  Recall also that $\sigma_1$ is the Pauli matrix associated with $x$:
$$\sigma_1 = \begin{pmatrix} 
0 & 1 \\ 
1 & 0 \\
\end{pmatrix}
$$
Now noting that:
$$\gamma^0 \gamma^1 = 
\begin{pmatrix} 
I &  0 \\ 
0 & -I \\
\end{pmatrix}
\;
\begin{pmatrix} 
0 & \sigma_1 \\ 
-\sigma_1 & 0 \\
\end{pmatrix}
=
\begin{pmatrix} 
0 & \sigma_1 \\ 
\sigma_1 & 0 \\
\end{pmatrix}
$$
so we can write $S$ as:
$$S = \begin{pmatrix} 
a_+ I & a_- \, \sigma_1 \\ 
a_- \, \sigma_1 & a_+ I \\
\end{pmatrix}
= \begin{pmatrix} 
a_+ & 0 & 0   & a_- \\
0 & a_+ & a_- & 0 \\
0 & a_- & a_+ & 0 \\
a_- & 0 & 0 & a_+ \\
\end{pmatrix}
$$
which is rather strikingly beautiful, if nothing else!  

Since $\sigma_1$ and $I$ commute, we can calculate the determinant of $S$ as if it were a 2x2 matrix:
$$\det(S) = \det((a_+ I)^2 - (a_-\sigma_1)^2) = \det((a_+) (I)^2 - (a_-)^2(\sigma_1)^2)$$
Recall that:
$$\sigma_1^2 = 
\begin{pmatrix} 
0 & 1 \\ 
1 & 0 \\
\end{pmatrix}
\,
\begin{pmatrix} 
0 & 1 \\ 
1 & 0 \\
\end{pmatrix}
=
\begin{pmatrix} 
1 & 0 \\ 
0 & 1 \\
\end{pmatrix}
=
I
$$
and so:
$$\det(S) = \det((a_+^2 - a_-^2) I) = (a_+^2 - a_-^2) \det(I) = 1 $$
This is non-zero, so $S$ has an inverse.  We can compute the inverse of $S$ as if it was a 2x2 matrix:
$$S^{-1} = \begin{pmatrix} 
a_+ I & -a_- \, \sigma_1 \\ 
-a_- \, \sigma_1 & a_+ I \\
\end{pmatrix}
$$
And a quick check shows that indeed:
$$S^{-1}S = 
\begin{pmatrix} 
a_+ I & -a_- \, \sigma_1 \\ 
-a_- \, \sigma_1 & a_+ I \\
\end{pmatrix}
\,
\begin{pmatrix} 
a_+ I & a_- \, \sigma_1 \\ 
a_- \, \sigma_1 & a_+ I \\
\end{pmatrix}
= 
\begin{pmatrix} 
a_+^2I^2-a_-^2\sigma_1^2 & 0 \\ 
0 & a_+^2I^2-a_-^2\sigma_1^2 \\ 
\end{pmatrix}
=
\begin{pmatrix} 
I & 0 \\ 
0 & I \\ 
\end{pmatrix}
$$
We can write this as:
$$S^{-1} = a_+ - a_- \gamma_0 \gamma_1$$
We are now well prepared to calculate the main result from this section, which is to calculate:
$$S^{-1} \gamma^u S$$
This is called a "similarity transformation" and appears frequently in linear algebra and group theory.  But for now you'll have to trust that the results will be useful to us.
We can work this out as:
\begin{eqnarray*}
S^{-1} \gamma^u S &=& (a_+ - a_- \gamma^0 \gamma^1) \gamma^u (a_+ - a_- \gamma^0 \gamma^1)\\
&=& a_+^2 \, \gamma^u - a_-^2 \, \gamma^0 \gamma^1 \gamma^u \gamma^0 \gamma^1
+ a_+ a_- (\gamma^u \gamma^0 \gamma^1 - \gamma^0 \gamma^1 \gamma^u)\\
\end{eqnarray*}
The fundamental property of the gamma matrices is written as the anticommutator:
$$\{\gamma^u, \gamma^w\} = 2 g^{uw}$$
which more practically means that:
$$(\gamma^0)^2 = 1, (\gamma^1)^2 = -1, (\gamma^2)^2 = -1, (\gamma^3)^2 = -1$$
and:
$$\gamma^u \gamma^w = -\gamma^w \gamma^u \hspace{2cm} (u \neq w)$$
where we see that anti-commuting is almost as good as commuting, you just have to keep track of the minus signs\footnote{When driving, anti-commuting is even better than commuting, as you tend to avoid the worst traffic.}. Using these properties alone, you can work out that:
$$\gamma^0\gamma^1\gamma^u\gamma^0\gamma^1 = \begin{cases}
- \gamma^u & u=0,1 \\
\gamma^u & u=2,3 \\
\end{cases}$$
and:
$$\gamma^u \gamma^0 \gamma^1 - \gamma^0 \gamma^1 \gamma^u 
= \begin{cases}
2\gamma^1 & u=0 \\
2\gamma^0 & u=1 \\
0 & u=2,3 \\ 
\end{cases}$$
by just working through each case explicitly.  So we can calculate:
$$S^{-1} \gamma^u S = 
\begin{cases}
(a_+^2 + a_-^2) \gamma^0 + 2 a_+ a_- \gamma^1 & u=0 \\
(a_+^2 + a_-^2) \gamma^1 + 2 a_+ a_- \gamma^0 & u=1 \\
(a_+^2 - a_-^2) \gamma^2                      & u=2 \\
(a_+^2 - a_-^2) \gamma^3                      & u=3 \\
\end{cases}$$
When we apply the identities above something magical occurs:
$$S^{-1} \gamma^u S = 
\begin{cases}
\gamma \gamma^0 - \gamma |\beta| \gamma^1 & u=0 \\
\gamma \gamma^1 - \gamma |\beta| \gamma^0 & u=1 \\
\gamma^2                      & u=2 \\
\gamma^3                      & u=3 \\
\end{cases}$$
which is just a Lorentz transformation as if $\gamma^u$ were a four vector.  Now of course $\gamma^u$ is {\bf not} a four vector, so there is more to this story, but for now we can write our most important result as:
$$S^{-1} \gamma^u S = \Lambda^u_w \gamma^w$$
Technically, we have only shown this for the particular Lorentz transformation $\Lambda^u_w$ with $\beta$ in the positive $x$-direction, but we can always change coordinates to arrange that this is so.

\subsection{Relativistic Transformation of Bi-Spinors}
We Dirac equation, derived consistent with special relativity, must be equivalent in all reference frames, so that:
$$i\hbar \gamma^u \delta_u \psi = mc \psi \Leftrightarrow 
i\hbar \gamma^u \widetilde{\delta}_u \widetilde{\psi} = mc \widetilde{\psi}$$
where $\psi$ and $\widetilde{\psi}$ are the bi-spinor in two frames related by a Lorentz transformation:
$$\widetilde{x}^u = \Lambda^u_w x^w$$
Since $\delta_u$ is a covariant vector, we have:
$$\widetilde{\delta}_u = M^w_u \delta_w$$
where $M = \Lambda^{-1}$.  Now suppose that the transformation from $\psi$ to $\widetilde{\psi}$ is given by a matrix $S$, so that:
$$\widetilde{\psi} = S \psi$$
then let's work out the properties of $S$ that would ensure the Dirac equation is frame invariant.  In this case:
\begin{eqnarray*}
i\hbar \gamma^u \widetilde{\delta}_u \widetilde{\psi} &=& mc \, \widetilde{\psi} \\[3pt]
i\hbar \gamma^u (M^w_u \delta_w) (S \psi) &=& mc \, S \psi \\[3pt]
i\hbar S^{-1} \gamma^u S M^w_u \delta_w \psi &=& mc \, \psi \\
\end{eqnarray*}
where we have taken care in the last line to maintain the order of matrices but have moved numbers like $M^w_u$ around at will.  Comparing to the Dirac equation:
$$i\hbar \gamma^w \delta_w \psi = mc \psi$$
we see that it is sufficient that:
$$S^{-1} \gamma^u S M^w_u = \gamma^w$$
or, since $M = \Lambda^{-1}$, equivalently:
$$S^{-1} \gamma^u S = \Lambda_w^u \gamma^w$$
In the previous section, we showed that the remarkable matrix 
$$S = a_+ + a_- \gamma^0 \gamma^1 = \begin{pmatrix} 
a_+ & a_- \, \sigma_1 \\ 
a_- \, \sigma_1 & a_+ \\
\end{pmatrix}
$$
has exactly this property.  So $S$ is the transformation matrix for the bi-spinor:
$$\widetilde{\psi} = S \psi$$

\section{Bilinear Covariants}

Suppose we want to construct a scalar (relativistic invariant) from the bi-spinor $\psi$. It would be quite reasonable to consider:
$$\psi^\dagger \psi = 
\begin{pmatrix}
\psi_1^* & \psi_2^* & \psi_3^* & \psi_4^* \\
\end{pmatrix}
\begin{pmatrix}
\psi_1 \\ \psi_2 \\ \psi_3 \\ \psi_4 \\
\end{pmatrix}
= |\psi_1|^2 + |\psi_2|^2+ |\psi_3|^2 + |\psi_4|^2
$$
We can check:
$$\widetilde{\psi}^\dagger \widetilde{\psi} = 
(S \psi)^\dagger (S \psi) = \psi^\dagger S^\dagger S \psi$$ 
this would be an invariant (scalar) if only $S^\dagger S = 1$ (i.e. if $S$ is unitary).  But alas, $S$ is not unitary.  A quick inspection shows $S^\dagger = S$ so:
$$S^\dagger S = S^2 = 
\begin{pmatrix}
a_+ & a_- \sigma_1 \\
a_- \sigma_1 & a_+ \\
\end{pmatrix}
\begin{pmatrix}
a_+ & a_- \sigma_1 \\
a_- \sigma_1 & a_+ \\
\end{pmatrix}
=
\begin{pmatrix}
a_+^2+a_-^2        & 2 a_+ a_- \sigma_1 \\
2 a_+ a_- \sigma_1 & a_+^2+a_-^2 \\
\end{pmatrix}
=
\begin{pmatrix}
\gamma        & -\gamma \beta \sigma_1 \\
-\gamma \beta \sigma_1 & \gamma \\
\end{pmatrix}
$$
which is not the identity matrix for $\beta > 0$.

However, it turns out that:
\begin{eqnarray*}
S^\dagger \gamma^0 S &=& 
\begin{pmatrix}
a_+^2+a_-^2        & 2 a_+ a_- \sigma_1 \\
2 a_+ a_- \sigma_1 & a_+^2+a_-^2 \\
\end{pmatrix}
\begin{pmatrix}
1        & 0 \\
0        & -1 \\
\end{pmatrix}
\begin{pmatrix}
a_+^2+a_-^2        & 2 a_+ a_- \sigma_1 \\
2 a_+ a_- \sigma_1 & a_+^2+a_-^2 \\
\end{pmatrix}\\[3pt]
&=& \begin{pmatrix}
a_+^2-a_-^2        & 0 \\
0                  & a_-^2-a_+^2 \\
\end{pmatrix}
= \begin{pmatrix}
1        & 0 \\
0        & -1 \\
\end{pmatrix}
\end{eqnarray*}
That is:
$$S^\dagger \gamma^0 S = \gamma^0$$
So now consider the quantity:
$\psi^\dagger \gamma^0 \psi$
we have:
$$\widetilde{\psi}^\dagger \gamma^0 \widetilde{\psi}
= (S \psi)^\dagger \gamma^0 (S \psi) 
= \psi^\dagger (S^\dagger \gamma^0 S) \psi
= \psi^\dagger \gamma^0 \psi
$$
which shows that this is indeed a relativistic invariant (scalar).  To clean up our notation, let's define the adjoint bi-spinor:
$$\overline{\psi} \equiv \psi^\dagger \gamma^0 $$
We've just seen that $\overline{\psi} \psi$ is a scalar.  Next let's see how $\overline{\psi} \gamma^u \psi$ transforms.  We have:
\begin{eqnarray*}
\widetilde{\overline{\psi} \gamma^u \psi} &=& 
\widetilde{\psi}^\dagger \gamma^0 \gamma^u \widetilde{\psi} \\
&=& \psi^\dagger S^\dagger \gamma^0 \gamma^u S \psi \\
\end{eqnarray*}
Now by inserting a factor of $1 = S S^{-1}$ we see that:
$$S^\dagger \gamma^0 \gamma^u S = (S^\dagger \gamma^0 S) (S^{-1}\gamma^u S)
= (\gamma^0) (\Lambda^u_w \gamma^w)$$
where we have used:
$$S^\dagger \gamma^0 S = \gamma^0$$
and
$$S^{-1} \gamma^u S = \Lambda^u_w \gamma^w$$.
Notice the different roles that $S^\dagger$ and $S^{-1}$ play here.  Returning to our original calculation we have now:
$$\widetilde{\overline{\psi} \gamma^u \psi} = \Lambda^u_w \overline{\psi} \gamma^w \psi$$
which shows that $\overline{\psi} \gamma^u \psi$ is a four-vector.












\section{Feynman Rules for QED without External Photons}

\end{document}




