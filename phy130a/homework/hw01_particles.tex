\documentclass[12pt]{article}

\usepackage[dvips,letterpaper,margin=0.75in,bottom=0.5in]{geometry}
\usepackage{cite}
\usepackage{slashed}
\usepackage{graphicx}
\usepackage{amsmath}
\usepackage{amssymb}
\usepackage{braket}
\usepackage[american,fulldiode]{circuitikz}

\begin{document}
\newcommand{\ihbar}{\ensuremath{i \hbar}}
\newcommand{\dPsidt}{\ensuremath{ \frac{\partial \Psi}{\partial t} }}
\newcommand{\dPsidx}{\ensuremath{ \frac{\partial \Psi}{\partial x} }}
\newcommand{\ddPsidx}{\ensuremath{ \frac{\partial^2 \Psi}{\partial x^2} }}
\newcommand{\dPssdt}{\ensuremath{ \frac{\partial \Psi^*}{\partial t} }}
\newcommand{\dPssdx}{\ensuremath{ \frac{\partial \Psi^*}{\partial x} }}
\newcommand{\ddPssdx}{\ensuremath{ \frac{\partial^2 \Psi^*}{\partial x^2} }}

\newcommand{\dphidt}{\ensuremath{ \frac{d \phi}{dt} }}
\newcommand{\dpsidx}{\ensuremath{ \frac{d \psi}{dx} }}
\newcommand{\ddpsidx}{\ensuremath{ \frac{d^2 \psi}{dx^2} }}


\date{\vspace{-5ex}}

\title{Homework Assignment 1 \\ ``Who ordered that?''}

\maketitle

\noindent
All problems are graded on effort only.\\

\noindent
{\bf Griffiths: P1.1, P1.3, P1.7, P1.8} \\
  
\noindent
{\bf Problem 1:}  In lecture we discussed conservation of baryon number ($A$), strangeness ($S$), muon number ($L_\mu$), electron number ($L_e$), charge ($q$), and energy ($E$).  For each process below, identify one of the quantities above that is not conserved.  Note that there are particle listings in the front matter of Griffiths.\\

\noindent
(A) $\pi^+ \; + \; p^+ \; \to \; K^+ \; + \; \Sigma^0$\\[5pt]
\noindent
(B) $\pi^- \; + \; p^+ \; \to \; \Lambda \; + \; \Sigma^0$\\[5pt]
\noindent
(C) $\pi^- \; \to \; \mu^- \; + \; \nu_\mu$\\[5pt]
\noindent
(D) $p^+ \; \to \; \Delta^{++} \; + \; \pi^-$\\[5pt]
\noindent
(E) $\mu^- \; \to \; e^- \; + \; \nu_\mu$\\[5pt]
\noindent
(F) $K^0 \; \to \; \pi^+ \; + \; \pi^-$\\[5pt]
\noindent


\noindent
{\bf Problem 2:}\\[5pt] 
(A) Which of the above processes is allowed as a weak process?\\[5pt]
(B) Transform one of the above processes to an allowed process using crossing symmetry and detailed balance.
\end{document}




